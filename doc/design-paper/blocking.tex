\documentclass{llncs}

\usepackage{url}
\usepackage{amsmath}
\usepackage{epsfig}

%\setlength{\textwidth}{5.9in}
%\setlength{\textheight}{8.4in}
%\setlength{\topmargin}{.5cm}
%\setlength{\oddsidemargin}{1cm}
%\setlength{\evensidemargin}{1cm}

\newenvironment{tightlist}{\begin{list}{$\bullet$}{
  \setlength{\itemsep}{0mm}
    \setlength{\parsep}{0mm}
    %  \setlength{\labelsep}{0mm}
    %  \setlength{\labelwidth}{0mm}
    %  \setlength{\topsep}{0mm}
    }}{\end{list}}

\begin{document}

\title{Design of a blocking-resistant anonymity system}

\author{}

\maketitle
\pagestyle{plain}

\begin{abstract}

Websites around the world are increasingly being blocked by
government-level firewalls. Many people use anonymizing networks like
Tor to contact sites without letting an attacker trace their activities,
and as an added benefit they are no longer affected by local censorship.
But if the attacker simply denies access to the Tor network itself,
blocked users can no longer benefit from the security Tor offers.

Here we describe a design that uses the current Tor network as a
building block to provide an anonymizing network that resists blocking
by government-level attackers.

\end{abstract}

\section{Introduction and Goals}

Websites like Wikipedia and Blogspot are increasingly being blocked by
government-level firewalls around the world.

China is the third largest user base for Tor clients~\cite{geoip-tor}.
Many people already want it, and the current Tor design is easy to block
(by blocking the directory authorities, by blocking all the server
IP addresses, or by filtering the signature of the Tor TLS handshake).

Now that we've got an overlay network, we're most of the way there in
terms of building a blocking-resistant tool.

And adding more different classes of users and goals to the Tor network
improves the anonymity for all Tor users~\cite{econymics,tor-weis06}.

\subsection{A single system that works for multiple blocked domains}

We want this to work for people in China, people in Iran, people in
Thailand, people in firewalled corporate networks, etc. The blocking
censor will be at different stages of the arms race in different places;
and likely the list of blocked addresses will be different in each
location too.


\section{Adversary assumptions}
\label{sec:adversary}

Three main network attacks by censors currently:

\begin{tightlist}
\item Block destination by string matches in TCP packets.

\item Block destination by IP address.

\item Intercept DNS requests.
\end{tightlist}

Assume the network firewall has very limited CPU per
user~\cite{clayton-pet2006}.

Assume that readers of blocked content will not be punished much
(relative to publishers).

Assume that while various different adversaries can coordinate and share
notes, there will be a significant time lag between one attacker learning
how to overcome a facet of our design and other attackers picking it up.

(Corollary: in the early stages of deployment, the insider threat isn't
as high of a risk.)

Assume that our users have control over their hardware and software -- no
spyware, no cameras watching their screen, etc.

Assume that the user will fetch a genuine version of Tor, rather than
one supplied by the adversary; see~\ref{subsec:trust-chain} for discussion
on helping the user confirm that he has a genuine version.

\section{Related schemes}

\subsection{public single-hop proxies}

Anonymizer and friends

\subsection{personal single-hop proxies}

Psiphon, circumventor, cgiproxy.

Simpler to deploy; might not require client-side software.

\subsection{break your sensitive strings into multiple tcp packets;
ignore RSTs}

\subsection{steganography}

infranet

\subsection{Internal caching networks}

Freenet is deployed inside China and caches outside content.

\subsection{Skype}

port-hopping. encryption. voice communications not so susceptible to
keystroke loggers (even graphical ones).

\section{Components of the current Tor design}

Anonymizing networks such as
Tor~\cite{tor-design}
aim to hide not only what is being said, but also who is
communicating with whom, which users are using which websites, and so on.
These systems have a broad range of users, including ordinary citizens
who want to avoid being profiled for targeted advertisements, corporations
who don't want to reveal information to their competitors, and law
enforcement and government intelligence agencies who need
to do operations on the Internet without being noticed.

Tor provides three security properties:
\begin{tightlist}
\item 1. A local observer can't learn, or influence, your destination.
\item 2. No single piece of the infrastructure can link you to your
destination.
\item 3. The destination, or somebody watching the destination,
can't learn your location.
\end{tightlist}

We care most clearly about property number 1. But when the arms race
progresses, property 2 will become important -- so the blocking adversary
can't learn user+destination pairs just by volunteering a relay. It's not so
clear to see that property 3 is important, but consider websites and
services that are pressured into treating clients from certain network
locations differently.

Other benefits:

\begin{tightlist}
\item Separates the role of relay from the role of exit node.

\item (Re)builds circuits automatically in the background, based on
whichever paths work.
\end{tightlist}

\subsection{Tor circuits}

can build arbitrary overlay paths given a set of descriptors~\cite{blossom}

\subsection{Tor directory servers}

central trusted locations that keep track of what Tor servers are
available and usable.

(threshold trust, so not quite so bad. See
Section~\ref{subsec:trust-chain} for details.)

\subsection{Tor user base}

Hundreds of thousands of users from around the world. Some with publically
reachable IP addresses.

\section{Why hasn't Tor been blocked yet?}

Hard to say. People think it's hard to block? Not enough users, or not
enough ordinary users? Nobody has been embarrassed by it yet? "Steam
valve"?

\section{Components of a blocking-resistant design}

Here we describe what we need to add to the current Tor design.

\subsection{Bridge relays}

Some Tor users on the free side of the network will opt to become
\emph{bridge relays}. They will relay a small amount of bandwidth into
the main Tor network, so they won't need to allow
exits.

They sign up on the bridge directory authorities (described below),
and they use Tor to publish their descriptor so an attacker observing
the bridge directory authority's network can't enumerate bridges.

...need to outline instructions for a Tor config that will publish
to an alternate directory authority, and for controller commands
that will do this cleanly.

\subsection{The bridge directory authority (BDA)}

They aggregate server descriptors just like the main authorities, and
answer all queries as usual, except they don't publish full directories
or network statuses.

So once you know a bridge relay's key, you can get the most recent
server descriptor for it.

Problem 1: need to figure out how to fetch some server statuses from the BDA
without fetching all statuses. A new URL to fetch I presume?

\subsection{Putting them together}

If a blocked user has a server descriptor for one working bridge relay,
then he can use it to make secure connections to the BDA to update his
knowledge about other bridge
relays, and he can make secure connections to the main Tor network
and directory servers to build circuits and connect to the rest of
the Internet.

So now we've reduced the problem from how to circumvent the firewall
for all transactions (and how to know that the pages you get have not
been modified by the local attacker) to how to learn about a working
bridge relay.

The following section describes ways to bootstrap knowledge of your first
bridge relay, and ways to maintain connectivity once you know a few
bridge relays. (See Section~\ref{later} for a discussion of exactly
what information is sufficient to characterize a bridge relay.)

\section{Discovering and maintaining working bridge relays}

Most government firewalls are not perfect. They allow connections to
Google cache or some open proxy servers, or they let file-sharing or
Skype or World-of-Warcraft connections through.
For users who can't use any of these techniques, hopefully they know
a friend who can -- for example, perhaps the friend already knows some
bridge relay addresses.
(If they can't get around it at all, then we can't help them -- they
should go meet more people.)

Thus they can reach the BDA. From here we either assume a social
network or other mechanism for learning IP:dirport or key fingerprints
as above, or we assume an account server that allows us to limit the
number of new bridge relays an external attacker can discover.

Going to be an arms race. Need a bag of tricks. Hard to say
which ones will work. Don't spend them all at once.

\subsection{Discovery based on social networks}

A token that can be exchanged at the BDA (assuming you
can reach it) for a new IP:dirport or server descriptor.

The account server

Users can establish reputations, perhaps based on social network
connectivity, perhaps based on not getting their bridge relays blocked,

(Lesson from designing reputation systems~\cite{p2p-econ}: easy to
reward good behavior, hard to punish bad behavior.

\subsection{How to give bridge addresses out}

Hold a fraction in reserve, in case our currently deployed tricks
all fail at once; so we can move to new approaches quickly.
(Bridges that sign up and don't get used yet will be sad; but this
is a transient problem -- if bridges are on by default, nobody will
mind not being used.)

Perhaps each bridge should be known by a single bridge directory
authority. This makes it easier to trace which users have learned about
it, so easier to blame or reward. It also makes things more brittle,
since loss of that authority means its bridges aren't advertised until
they switch, and means its bridge users are sad too.
(Need a slick hash algorithm that will map our identity key to a
bridge authority, in a way that's sticky even when we add bridge
directory authorities, but isn't sticky when our authority goes
away. Does this exist?)

Divide bridgets into buckets. You can learn only from the bucket your
IP address maps to.

\section{Security improvements}

\subsection{Minimum info required to describe a bridge}

There's another possible attack here: since we only learn an IP address
and port, a local attacker could intercept our directory request and
give us some other server descriptor. But notice that we don't need
strong authentication for the bridge relay. Since the Tor client will
ship with trusted keys for the bridge directory authority and the Tor
network directory authorities, the user can decide if the bridge relays
are lying to him or not.

Once the Tor client has fetched the server descriptor at least once,
it should remember the identity key fingerprint for that bridge relay.
If the bridge relay moves to a new IP address, the client can then
use the bridge directory authority to look up a fresh server descriptor
using this fingerprint.

\subsubsection{Scanning-resistance}

If it's trivial to verify that we're a bridge, and we run on a predictable
port, then it's conceivable our attacker would scan the whole Internet
looking for bridges. It would be nice to slow down this attack. It would
be even nicer to make it hard to learn whether we're a bridge without
first knowing some secret.

% XXX this para is in the wrong section
Could provide a password to the bridge user. He provides a nonced hash of
it or something when he connects. We'd need to give him an ID key for the
bridge too, and wait to present the password until we've TLSed, else the
adversary can pretend to be the bridge and MITM him to learn the password.


\subsection{Hiding Tor's network signatures}

The simplest format for communicating information about a bridge relay
is as an IP address and port for its directory cache. From there, the
user can ask the directory cache for an up-to-date copy of that bridge
relay's server descriptor, including its current circuit keys, the port
it uses for Tor connections, and so on.

However, connecting directly to the directory cache involves a plaintext
http request, so the censor could create a firewall signature for the
request and/or its response, thus preventing these connections. Therefore
we've modified the Tor protocol so that users can connect to the directory
cache via the main Tor port -- they establish a TLS connection with
the bridge as normal, and then send a Tor "begindir" relay cell to
establish a connection to its directory cache.

Predictable SSL ports:
We should encourage most servers to listen on port 443, which is
where SSL normally listens.
Is that all it will take, or should we set things up so some fraction
of them pick random ports? I can see that both helping and hurting.

Predictable TLS handshakes:
Right now Tor has some predictable strings in its TLS handshakes.
These can be removed; but should they be replaced with nothing, or
should we try to emulate some popular browser? In any case our
protocol demands a pair of certs on both sides -- how much will this
make Tor handshakes stand out?

\subsection{Anonymity issues from becoming a bridge relay}

You can actually harm your anonymity by relaying traffic in Tor.  This is
the same issue that ordinary Tor servers face. On the other hand, it
provides improved anonymity against some attacks too:

\begin{verbatim}
http://wiki.noreply.org/noreply/TheOnionRouter/TorFAQ#ServerAnonymity
\end{verbatim}



\section{Performance improvements}

\subsection{Fetch server descriptors just-in-time}

I guess we should encourage most places to do this, so blocked
users don't stand out.

\section{Other issues}

\subsection{How many bridge relays should you know about?}

If they're ordinary Tor users on cable modem or DSL, many of them will
disappear and/or move periodically. How many bridge relays should a
blockee know
about before he's likely to have at least one reachable at any given point?
How do we factor in a parameter for "speed that his bridges get discovered
and blocked"?

The related question is: if the bridge relays change IP addresses
periodically, how often does the bridge user need to "check in" in order
to keep from being cut out of the loop?

\subsection{How do we know if a bridge relay has been blocked?}

We need some mechanism for testing reachability from inside the
blocked area.

The easiest answer is for certain users inside the area to sign up as
testing relays, and then we can route through them and see if it works.

First problem is that different network areas block different net masks,
and it will likely be hard to know which users are in which areas. So
if a bridge relay isn't reachable, is that because of a network block
somewhere, because of a problem at the bridge relay, or just a temporary
outage?

Second problem is that if we pick random users to test random relays, the
adversary should sign up users on the inside, and enumerate the relays
we test. But it seems dangerous to just let people come forward and
declare that things are blocked for them, since they could be tricking
us. (This matters even moreso if our reputation system above relies on
whether things get blocked to punish or reward.)

Another answer is not to measure directly, but rather let the bridges
report whether they're being used. If they periodically report to their
bridge directory authority how much use they're seeing, the authority
can make smart decisions from there.

If they install a geoip database, they can periodically report to their
bridge directory authority which countries they're seeing use from. This
might help us to track which countries are making use of Ramp, and can
also let us learn about new steps the adversary has taken in the arms
race. (If the bridges don't want to install a whole geoip subsystem, they
can report samples of the /24 network for their users, and the authorities
can do the geoip work. This tradeoff has clear downsides though.)

Worry: adversary signs up a bunch of already-blocked bridges. If we're
stingy giving out bridges, users in that country won't get useful ones.
(Worse, we'll blame the users when the bridges report they're not
being used?)

Worry: the adversary could choose not to block bridges but just record
connections to them. So be it, I guess.

\subsection{Cablemodem users don't provide important websites}

...so our adversary could just block all DSL and cablemodem networks,
and for the most part only our bridge relays would be affected.

The first answer is to aim to get volunteers both from traditionally
``consumer'' networks and also from traditionally ``producer'' networks.

The second answer (not so good) would be to encourage more use of consumer
networks for popular and useful websites.

Other attack: China pressures Verizon to discourage its users from
running bridges.

\subsection{The trust chain}
\label{subsec:trust-chain}

Tor's ``public key infrastructure'' provides a chain of trust to
let users verify that they're actually talking to the right servers.
There are four pieces to this trust chain.

Firstly, when Tor clients are establishing circuits, at each step
they demand that the next Tor server in the path prove knowledge of
its private key~\cite{tor-design}. This step prevents the first node
in the path from just spoofing the rest of the path. Secondly, the
Tor directory authorities provide a signed list of servers along with
their public keys --- so unless the adversary can control a threshold
of directory authorities, he can't trick the Tor client into using other
Tor servers. Thirdly, the location and keys of the directory authorities,
in turn, is hard-coded in the Tor source code --- so as long as the user
got a genuine version of Tor, he can know that he is using the genuine
Tor network. And lastly, the source code and other packages are signed
with the GPG keys of the Tor developers, so users can confirm that they
did in fact download a genuine version of Tor.

But how can a user in an oppressed country know that he has the correct
key fingerprints for the developers? As with other security systems, it
ultimately comes down to human interaction. The keys are signed by dozens
of people around the world, and we have to hope that our users have met
enough people in the PGP web of trust~\cite{pgp-wot} that they can learn
the correct keys. For users that aren't connected to the global security
community, though, this question remains a critical weakness.

\subsection{Bridge users without Tor clients}

They could always open their socks proxy. This is bad though, firstly
because they learn the bridge users' destinations, and secondly because
we've learned that open socks proxies tend to attract abusive users who
have no idea they're using Tor.

\section{Future designs}

\subsection{Bridges inside the blocked network too}

Assuming actually crossing the firewall is the risky part of the
operation, can we have some bridge relays inside the blocked area too,
and more established users can use them as relays so they don't need to
communicate over the firewall directly at all? A simple example here is
to make new blocked users into internal bridges also -- so they sign up
on the BDA as part of doing their query, and we give out their addresses
rather than (or along with) the external bridge addresses. This design
is a lot trickier because it brings in the complexity of whether the
internal bridges will remain available, can maintain reachability with
the outside world, etc.

Hidden services as bridges. Hidden services as bridge directory authorities.

Make all Tor users become bridges if they're reachable -- needs more work
on usability first, but we're making progress.

\bibliographystyle{plain} \bibliography{tor-design}

\end{document}


