\documentclass{llncs}

\usepackage{url}
\usepackage{amsmath}
\usepackage{epsfig}

\setlength{\textwidth}{5.9in}
\setlength{\textheight}{8.4in}
\setlength{\topmargin}{.5cm}
\setlength{\oddsidemargin}{1cm}
\setlength{\evensidemargin}{1cm}

\newenvironment{tightlist}{\begin{list}{$\bullet$}{
  \setlength{\itemsep}{0mm}
    \setlength{\parsep}{0mm}
    %  \setlength{\labelsep}{0mm}
    %  \setlength{\labelwidth}{0mm}
    %  \setlength{\topsep}{0mm}
    }}{\end{list}}

\begin{document}

\title{Design of a blocking-resistant anonymity system\\DRAFT}

%\author{Roger Dingledine\inst{1} \and Nick Mathewson\inst{1}}
\author{Roger Dingledine \and Nick Mathewson}
\institute{The Free Haven Project\\
\email{\{arma,nickm\}@freehaven.net}}

\maketitle
\pagestyle{plain}

\begin{abstract}

Websites around the world are increasingly being blocked by
government-level firewalls. Many people use anonymizing networks like
Tor to contact sites without letting an attacker trace their activities,
and as an added benefit they are no longer affected by local censorship.
But if the attacker simply denies access to the Tor network itself,
blocked users can no longer benefit from the security Tor offers.

Here we describe a design that builds upon the current Tor network
to provide an anonymizing network that resists blocking
by government-level attackers.

\end{abstract}

\section{Introduction and Goals}

Anonymizing networks such as Tor~\cite{tor-design} bounce traffic around
a network of relays. They aim to hide not only what is being said, but
also who is communicating with whom, which users are using which websites,
and so on. These systems have a broad range of users, including ordinary
citizens who want to avoid being profiled for targeted advertisements,
corporations who don't want to reveal information to their competitors,
and law enforcement and government intelligence agencies who need to do
operations on the Internet without being noticed.

Historically, research on anonymizing systems has focused on a passive
attacker who monitors the user (call her Alice) and tries to discover her
activities, yet lets her reach any piece of the network. In more modern
threat models such as Tor's, the adversary is allowed to perform active
attacks such as modifying communications to trick Alice
into revealing her destination, or intercepting some connections
to run a man-in-the-middle attack. But these systems still assume that
Alice can eventually reach the anonymizing network.

An increasing number of users are using the Tor software
less for its anonymity properties than for its censorship
resistance properties---if they use Tor to access Internet sites like
Wikipedia
and Blogspot, they are no longer affected by local censorship
and firewall rules. In fact, an informal user study (described in
Appendix~\ref{app:geoip}) showed China as the third largest user base
for Tor clients, with perhaps ten thousand people accessing the Tor
network from China each day.

The current Tor design is easy to block if the attacker controls Alice's
connection to the Tor network---by blocking the directory authorities,
by blocking all the server IP addresses in the directory, or by filtering
based on the signature of the Tor TLS handshake. Here we describe a
design that builds upon the current Tor network to provide an anonymizing
network that also resists this blocking. Specifically,
Section~\ref{sec:adversary} discusses our threat model---that is,
the assumptions we make about our adversary; Section~\ref{sec:current-tor}
describes the components of the current Tor design and how they can be
leveraged for a new blocking-resistant design; Section~\ref{sec:related}
explains the features and drawbacks of the currently deployed solutions;
and ...

% The other motivation is for places where we're concerned they will
% try to enumerate a list of Tor users. So even if they're not blocking
% the Tor network, it may be smart to not be visible as connecting to it.

%And adding more different classes of users and goals to the Tor network
%improves the anonymity for all Tor users~\cite{econymics,usability:weis2006}.

% Adding use classes for countering blocking as well as anonymity has
% benefits too. Should add something about how providing undetected
% access to Tor would facilitate people talking to, e.g., govt. authorities
% about threats to public safety etc. in an environment where Tor use
% is not otherwise widespread and would make one stand out.

\section{Adversary assumptions}
\label{sec:adversary}

The history of blocking-resistance designs is littered with conflicting
assumptions about what adversaries to expect and what problems are
in the critical path to a solution. Here we try to enumerate our best
understanding of the current situation around the world.

In the traditional security style, we aim to describe a strong
attacker---if we can defend against this attacker, we inherit protection
against weaker attackers as well. After all, we want a general design
that will work for citizens of China, Iran, Thailand, and other censored
countries; for
whistleblowers in firewalled corporate network; and for people in
unanticipated oppressive situations. In fact, by designing with
a variety of adversaries in mind, we can take advantage of the fact that
adversaries will be in different stages of the arms race at each location,
so a server blocked in one locale can still be useful in others.

We assume there are three main network attacks in use by censors
currently~\cite{clayton:pet2006}:

\begin{tightlist}
\item Block a destination or type of traffic by automatically searching for
  certain strings or patterns in TCP packets.
\item Block a destination by manually listing its IP address at the
firewall.
\item Intercept DNS requests and give bogus responses for certain
destination hostnames.
\end{tightlist}

We assume the network firewall has limited CPU and memory per
connection~\cite{clayton:pet2006}. Against an adversary who carefully
examines the contents of every packet, we would need
some stronger mechanism such as steganography, which introduces its
own problems~\cite{active-wardens,tcpstego,bar}.

More broadly, we assume that the authorities are more likely to
block a given system as its popularity grows. That is, a system
used by only a few users will probably never be blocked, whereas a
well-publicized system with many users will receive much more scrutiny.

We assume that readers of blocked content are not in as much danger
as publishers. So far in places like China, the authorities mainly go
after people who publish materials and coordinate organized
movements~\cite{mackinnon}.
If they find that a user happens
to be reading a site that should be blocked, the typical response is
simply to block the site. Of course, even with an encrypted connection,
the adversary may be able to distinguish readers from publishers by
observing whether Alice is mostly downloading bytes or mostly uploading
them---we discuss this issue more in Section~\ref{subsec:upload-padding}.

We assume that while various different regimes can coordinate and share
notes, there will be a time lag between one attacker learning
how to overcome a facet of our design and other attackers picking it up.
Similarly, we assume that in the early stages of deployment the insider
threat isn't as high of a risk, because no attackers have put serious
effort into breaking the system yet.

We do not assume that government-level attackers are always uniform across
the country. For example, there is no single centralized place in China
that coordinates its specific censorship decisions and steps.

We assume that our users have control over their hardware and
software---they don't have any spyware installed, there are no
cameras watching their screens, etc. Unfortunately, in many situations
these threats are real~\cite{zuckerman-threatmodels}; yet
software-based security systems like ours are poorly equipped to handle
a user who is entirely observed and controlled by the adversary. See
Section~\ref{subsec:cafes-and-livecds} for more discussion of what little
we can do about this issue.

We assume that widespread access to the Internet is economically,
politically, and/or
socially valuable to the policymakers of each deployment country. After
all, if censorship
is more important than Internet access, the firewall administrators have
an easy job: they should simply block everything. The corollary to this
assumption is that we should design so that increased blocking of our
system results in increased economic damage or public outcry.

We assume that the user will be able to fetch a genuine
version of Tor, rather than one supplied by the adversary; see
Section~\ref{subsec:trust-chain} for discussion on helping the user
confirm that he has a genuine version and that he can connect to the
real Tor network.

\section{Components of the current Tor design}
\label{sec:current-tor}

Tor is popular and sees a lot of use. It's the largest anonymity
network of its kind.
Tor has attracted more than 800 volunteer-operated routers from around the
world.  Tor protects users by routing their traffic through a multiply
encrypted ``circuit'' built of a few randomly selected servers, each of which
can remove only a single layer of encryption.  Each server sees only the step
before it and the step after it in the circuit, and so no single server can
learn the connection between a user and her chosen communication partners.
In this section, we examine some of the reasons why Tor has become popular,
with particular emphasis to how we can take advantage of these properties
for a blocking-resistance design.

Tor aims to provide three security properties:
\begin{tightlist}
\item 1. A local network attacker can't learn, or influence, your
destination.
\item 2. No single router in the Tor network can link you to your
destination.
\item 3. The destination, or somebody watching the destination,
can't learn your location.
\end{tightlist}

For blocking-resistance, we care most clearly about the first
property. But as the arms race progresses, the second property
will become important---for example, to discourage an adversary
from volunteering a relay in order to learn that Alice is reading
or posting to certain websites. The third property helps keep users safe from
collaborating websites: consider websites and other Internet services 
that have been pressured
recently into revealing the identity of bloggers~\cite{arrested-bloggers}
or treating clients differently depending on their network
location~\cite{google-geolocation}.
% and cite{goodell-syverson06} once it's finalized.

The Tor design provides other features as well that are not typically
present in manual or ad hoc circumvention techniques.

First, the Tor directory authorities automatically aggregate, test,
and publish signed summaries of the available Tor routers. Tor clients
can fetch these summaries to learn which routers are available and
which routers are suitable for their needs. Directory information is cached
throughout the Tor network, so once clients have bootstrapped they never
need to interact with the authorities directly. (To tolerate a minority
of compromised directory authorities, we use a threshold trust scheme---
see Section~\ref{subsec:trust-chain} for details.)

Second, Tor clients can be configured to use any directory authorities
they want. They use the default authorities if no others are specified,
but it's easy to start a separate (or even overlapping) Tor network just
by running a different set of authorities and convincing users to prefer
a modified client. For example, we could launch a distinct Tor network
inside China; some users could even use an aggregate network made up of
both the main network and the China network. (But we should not be too
quick to create other Tor networks---part of Tor's anonymity comes from
users behaving like other users, and there are many unsolved anonymity
questions if different users know about different pieces of the network.)

Third, in addition to automatically learning from the chosen directories
which Tor routers are available and working, Tor takes care of building
paths through the network and rebuilding them as needed. So the user
never has to know how paths are chosen, never has to manually pick
working proxies, and so on. More generally, at its core the Tor protocol
is simply a tool that can build paths given a set of routers. Tor is
quite flexible about how it learns about the routers and how it chooses
the paths. Harvard's Blossom project~\cite{blossom-thesis} makes this
flexibility more concrete: Blossom makes use of Tor not for its security
properties but for its reachability properties. It runs a separate set
of directory authorities, its own set of Tor routers (called the Blossom
network), and uses Tor's flexible path-building to let users view Internet
resources from any point in the Blossom network.

Fourth, Tor separates the role of \emph{internal relay} from the
role of \emph{exit relay}. That is, some volunteers choose just to relay
traffic between Tor users and Tor routers, and others choose to also allow
connections to external Internet resources. Because we don't force all
volunteers to play both roles, we end up with more relays. This increased
diversity in turn is what gives Tor its security: the more options the
user has for her first hop, and the more options she has for her last hop,
the less likely it is that a given attacker will be watching both ends
of her circuit~\cite{tor-design}. As a bonus, because our design attracts
more internal relays that want to help out but don't want to deal with
being an exit relay, we end up with more options for the first hop---the
one most critical to being able to reach the Tor network.

Fifth, Tor is sustainable. Zero-Knowledge Systems offered the commercial
but now defunct Freedom Network~\cite{freedom21-security}, a design with
security comparable to Tor's, but its funding model relied on collecting
money from users to pay relay operators. Modern commercial proxy systems
similarly
need to keep collecting money to support their infrastructure. On the
other hand, Tor has built a self-sustaining community of volunteers who
donate their time and resources. This community trust is rooted in Tor's
open design: we tell the world exactly how Tor works, and we provide all
the source code. Users can decide for themselves, or pay any security
expert to decide, whether it is safe to use. Further, Tor's modularity
as described above, along with its open license, mean that its impact
will continue to grow.

Sixth, Tor has an established user base of hundreds of
thousands of people from around the world. This diversity of
users contributes to sustainability as above: Tor is used by
ordinary citizens, activists, corporations, law enforcement, and
even government and military users~\cite{tor-use-cases}, and they can
only achieve their security goals by blending together in the same
network~\cite{econymics,usability:weis2006}. This user base also provides
something else: hundreds of thousands of different and often-changing
addresses that we can leverage for our blocking-resistance design.

We discuss and adapt these components further in
Section~\ref{sec:bridges}. But first we examine the strengths and
weaknesses of other blocking-resistance approaches, so we can expand
our repertoire of building blocks and ideas.

\section{Current proxy solutions}
\label{sec:related}

Relay-based blocking-resistance schemes generally have two main
components: a relay component and a discovery component. The relay part
encompasses the process of establishing a connection, sending traffic
back and forth, and so on---everything that's done once the user knows
where she's going to connect. Discovery is the step before that: the
process of finding one or more usable relays.

For example, we can divide the pieces of Tor in the previous section
into the process of building paths and sending
traffic over them (relay) and the process of learning from the directory
servers about what routers are available (discovery).  With this distinction
in mind, we now examine several categories of relay-based schemes.

\subsection{Centrally-controlled shared proxies}

Existing commercial anonymity solutions (like Anonymizer.com) are based
on a set of single-hop proxies. In these systems, each user connects to
a single proxy, which then relays traffic between the user and her
destination. These public proxy
systems are typically characterized by two features: they control and
operate the proxies centrally, and many different users get assigned
to each proxy.

In terms of the relay component, single proxies provide weak security
compared to systems that distribute trust over multiple relays, since a
compromised proxy can trivially observe all of its users' actions, and
an eavesdropper only needs to watch a single proxy to perform timing
correlation attacks against all its users' traffic and thus learn where
everyone is connecting. Worse, all users
need to trust the proxy company to have good security itself as well as
to not reveal user activities.

On the other hand, single-hop proxies are easier to deploy, and they
can provide better performance than distributed-trust designs like Tor,
since traffic only goes through one relay. They're also more convenient
from the user's perspective---since users entirely trust the proxy,
they can just use their web browser directly.

Whether public proxy schemes are more or less scalable than Tor is
still up for debate: commercial anonymity systems can use some of their
revenue to provision more bandwidth as they grow, whereas volunteer-based
anonymity systems can attract thousands of fast relays to spread the load.

The discovery piece can take several forms. Most commercial anonymous
proxies have one or a handful of commonly known websites, and their users
log in to those websites and relay their traffic through them. When
these websites get blocked (generally soon after the company becomes
popular), if the company cares about users in the blocked areas, they
start renting lots of disparate IP addresses and rotating through them
as they get blocked. They notify their users of new addresses (by email,
for example). It's an arms race, since attackers can sign up to receive the
email too, but operators have one nice trick available to them: because they
have a list of paying subscribers, they can notify certain subscribers
about updates earlier than others.

Access control systems on the proxy let them provide service only to
users with certain characteristics, such as paying customers or people
from certain IP address ranges.

Discovery in the face of a government-level firewall is a complex and
unsolved
topic, and we're stuck in this same arms race ourselves; we explore it
in more detail in Section~\ref{sec:discovery}. But first we examine the
other end of the spectrum---getting volunteers to run the proxies,
and telling only a few people about each proxy.

\subsection{Independent personal proxies}

Personal proxies such as Circumventor~\cite{circumventor} and
CGIProxy~\cite{cgiproxy} use the same technology as the public ones as
far as the relay component goes, but they use a different strategy for
discovery. Rather than managing a few centralized proxies and constantly
getting new addresses for them as the old addresses are blocked, they
aim to have a large number of entirely independent proxies, each managing
its own (much smaller) set of users.

As the Circumventor site explains, ``You don't
actually install the Circumventor \emph{on} the computer that is blocked
from accessing Web sites. You, or a friend of yours, has to install the
Circumventor on some \emph{other} machine which is not censored.''

This tactic has great advantages in terms of blocking-resistance---recall
our assumption in Section~\ref{sec:adversary} that the attention
a system attracts from the attacker is proportional to its number of
users and level of publicity. If each proxy only has a few users, and
there is no central list of proxies, most of them will never get noticed by
the censors.

On the other hand, there's a huge scalability question that so far has
prevented these schemes from being widely useful: how does the fellow
in China find a person in Ohio who will run a Circumventor for him? In
some cases he may know and trust some people on the outside, but in many
cases he's just out of luck. Just as hard, how does a new volunteer in
Ohio find a person in China who needs it?

% another key feature of a proxy run by your uncle is that you
% self-censor, so you're unlikely to bring abuse complaints onto
% your uncle. self-censoring clearly has a downside too, though.

This challenge leads to a hybrid design---centrally-distributed
personal proxies---which we will investigate in more detail in
Section~\ref{sec:discovery}.

\subsection{Open proxies}

Yet another currently used approach to bypassing firewalls is to locate
open and misconfigured proxies on the Internet. A quick Google search
for ``open proxy list'' yields a wide variety of freely available lists
of HTTP, HTTPS, and SOCKS proxies. Many small companies have sprung up
providing more refined lists to paying customers.

There are some downsides to using these open proxies though. First,
the proxies are of widely varying quality in terms of bandwidth and
stability, and many of them are entirely unreachable. Second, unlike
networks of volunteers like Tor, the legality of routing traffic through
these proxies is questionable: it's widely believed that most of them
don't realize what they're offering, and probably wouldn't allow it if
they realized. Third, in many cases the connection to the proxy is
unencrypted, so firewalls that filter based on keywords in IP packets
will not be hindered. And last, many users are suspicious that some
open proxies are a little \emph{too} convenient: are they run by the
adversary, in which case they get to monitor all the user's requests
just as single-hop proxies can?

A distributed-trust design like Tor resolves each of these issues for
the relay component, but a constantly changing set of thousands of open
relays is clearly a useful idea for a discovery component. For example,
users might be able to make use of these proxies to bootstrap their
first introduction into the Tor network.

\subsection{JAP}

Stefan's WPES paper~\cite{koepsell:wpes2004} is probably the closest
related work, and is
the starting point for the design in this paper.

\subsection{steganography}

infranet

\subsection{break your sensitive strings into multiple tcp packets;
ignore RSTs}

\subsection{Internal caching networks}

Freenet is deployed inside China and caches outside content.

\subsection{Skype}

port-hopping. encryption. voice communications not so susceptible to
keystroke loggers (even graphical ones).


\subsection{Tor itself}

And last, we include Tor itself in the list of current solutions
to firewalls. Tens of thousands of people use Tor from countries that
routinely filter their Internet. Tor's website has been blocked in most
of them. But why hasn't the Tor network been blocked yet?

We have several theories. The first is the most straightforward: tens of
thousands of people are simply too few to matter. It may help that Tor is
perceived to be for experts only, and thus not worth attention yet. The
more subtle variant on this theory is that we've positioned Tor in the
public eye as a tool for retaining civil liberties in more free countries,
so perhaps blocking authorities don't view it as a threat. (We revisit
this idea when we consider whether and how to publicize a Tor variant
that improves blocking-resistance---see Section~\ref{subsec:publicity}
for more discussion.)

The broader explanation is that the maintainance of most government-level
filters is aimed at stopping widespread information flow and appearing to be
in control, not by the impossible goal of blocking all possible ways to bypass
censorship. Censors realize that there will always
be ways for a few people to get around the firewall, and as long as Tor
has not publically threatened their control, they see no urgent need to
block it yet.

We should recognize that we're \emph{already} in the arms race. These
constraints can give us insight into the priorities and capabilities of
our various attackers.

\section{The relay component of our blocking-resistant design}
\label{sec:bridges}

Section~\ref{sec:current-tor} describes many reasons why Tor is
well-suited as a building block in our context, but several changes will
allow the design to resist blocking better. The most critical changes are
to get more relay addresses, and to distribute them to users differently.

%We need to address three problems:
%- adapting the relay component of Tor so it resists blocking better.
%- Discovery.
%- Tor's network signature.

%Here we describe the new pieces we need to add to the current Tor design.

\subsection{Bridge relays}

Today, Tor servers operate on less than a thousand distinct IP addresses;
an adversary
could enumerate and block them all with little trouble.  To provide a
means of ingress to the network, we need a larger set of entry points, most
of which an adversary won't be able to enumerate easily.  Fortunately, we
have such a set: the Tor users.

Hundreds of thousands of people around the world use Tor. We can leverage
our already self-selected user base to produce a list of thousands of
often-changing IP addresses. Specifically, we can give them a little
button in the GUI that says ``Tor for Freedom'', and users who click
the button will turn into \emph{bridge relays} (or just \emph{bridges}
for short). They can rate limit relayed connections to 10 KB/s (almost
nothing for a broadband user in a free country, but plenty for a user
who otherwise has no access at all), and since they are just relaying
bytes back and forth between blocked users and the main Tor network, they
won't need to make any external connections to Internet sites. Because
of this separation of roles, and because we're making use of software
that the volunteers have already installed for their own use, we expect
our scheme to attract and maintain more volunteers than previous schemes.

As usual, there are new anonymity and security implications from running a
bridge relay, particularly from letting people relay traffic through your
Tor client; but we leave this discussion for Section~\ref{sec:security}.

%...need to outline instructions for a Tor config that will publish
%to an alternate directory authority, and for controller commands
%that will do this cleanly.

\subsection{The bridge directory authority}

How do the bridge relays advertise their existence to the world? We
introduce a second new component of the design: a specialized directory
authority that aggregates and tracks bridges. Bridge relays periodically
publish server descriptors (summaries of their keys, locations, etc,
signed by their long-term identity key), just like the relays in the
``main'' Tor network, but in this case they publish them only to the
bridge directory authorities.

The main difference between bridge authorities and the directory
authorities for the main Tor network is that the main authorities provide
a list of every known relay, but the bridge authorities only give
out a server descriptor if you already know its identity key. That is,
you can keep up-to-date on a bridge's location and other information
once you know about it, but you can't just grab a list of all the bridges.

The identity key, IP address, and directory port for each bridge
authority ship by default with the Tor software, so the bridge relays
can be confident they're publishing to the right location, and the
blocked users can establish an encrypted authenticated channel. See
Section~\ref{subsec:trust-chain} for more discussion of the public key
infrastructure and trust chain.

Bridges use Tor to publish their descriptors privately and securely,
so even an attacker monitoring the bridge directory authority's network
can't make a list of all the addresses contacting the authority.
Bridges may publish to only a subset of the
authorities, to limit the potential impact of an authority compromise.


%\subsection{A simple matter of engineering}
%
%Although we've described bridges and bridge authorities in simple terms
%above, some design modifications and features are needed in the Tor
%codebase to add them. We describe the four main changes here.
%
%Firstly, we need to get smarter about rate limiting:
%Bandwidth classes
%
%Secondly, while users can in fact configure which directory authorities
%they use, we need to add a new type of directory authority and teach
%bridges to fetch directory information from the main authorities while
%publishing server descriptors to the bridge authorities. We're most of
%the way there, since we can already specify attributes for directory
%authorities:
%add a separate flag named ``blocking''.
%
%Thirdly, need to build paths using bridges as the first
%hop. One more hole in the non-clique assumption.
%
%Lastly, since bridge authorities don't answer full network statuses,
%we need to add a new way for users to learn the current status for a
%single relay or a small set of relays---to answer such questions as
%``is it running?'' or ``is it behaving correctly?'' We describe in
%Section~\ref{subsec:enclave-dirs} a way for the bridge authority to
%publish this information without resorting to signing each answer
%individually.

\subsection{Putting them together}
\label{subsec:relay-together}

If a blocked user knows the identity keys of a set of bridge relays, and
he has correct address information for at least one of them, he can use
that one to make a secure connection to the bridge authority and update
his knowledge about the other bridge relays. He can also use it to make
secure connections to the main Tor network and directory servers, so he
can build circuits and connect to the rest of the Internet. All of these
updates happen in the background: from the blocked user's perspective,
he just accesses the Internet via his Tor client like always.

So now we've reduced the problem from how to circumvent the firewall
for all transactions (and how to know that the pages you get have not
been modified by the local attacker) to how to learn about a working
bridge relay.

There's another catch though. We need to make sure that the network
traffic we generate by simply connecting to a bridge relay doesn't stand
out too much.

%The following section describes ways to bootstrap knowledge of your first
%bridge relay, and ways to maintain connectivity once you know a few
%bridge relays.

% (See Section~\ref{subsec:first-bridge} for a discussion
%of exactly what information is sufficient to characterize a bridge relay.)



\section{Hiding Tor's network signatures}
\label{sec:network-signature}
\label{subsec:enclave-dirs}

Currently, Tor uses two protocols for its network communications. The
main protocol uses TLS for encrypted and authenticated communication
between Tor instances. The second protocol is standard HTTP, used for
fetching directory information. All Tor servers listen on their ``ORPort''
for TLS connections, and some of them opt to listen on their ``DirPort''
as well, to serve directory information. Tor servers choose whatever port
numbers they like; the server descriptor they publish to the directory
tells users where to connect.

One format for communicating address information about a bridge relay is
its IP address and DirPort. From there, the user can ask the bridge's
directory cache for an up-to-date copy of its server descriptor, and
learn its current circuit keys, its ORPort, and so on.

However, connecting directly to the directory cache involves a plaintext
HTTP request. A censor could create a network signature for the request
and/or its response, thus preventing these connections. To resolve this
vulnerability, we've modified the Tor protocol so that users can connect
to the directory cache via the main Tor port---they establish a TLS
connection with the bridge as normal, and then send a special ``begindir''
relay command to establish an internal connection to its directory cache.

Therefore a better way to summarize a bridge's address is by its IP
address and ORPort, so all communications between the client and the
bridge will use ordinary TLS. But there are other details that need
more investigation.

What port should bridges pick for their ORPort? We currently recommend
that they listen on port 443 (the default HTTPS port) if they want to
be most useful, because clients behind standard firewalls will have
the best chance to reach them. Is this the best choice in all cases,
or should we encourage some fraction of them pick random ports, or other
ports commonly permitted through firewalls like 53 (DNS) or 110
(POP)?  Or perhaps we should use other ports where TLS traffic is
expected, like 993 (IMAPS) or 995 (POP3S).  We need more research on our
potential users, and their current and anticipated firewall restrictions.

Furthermore, we need to look at the specifics of Tor's TLS handshake.
Right now Tor uses some predictable strings in its TLS handshakes. For
example, it sets the X.509 organizationName field to ``Tor'', and it puts
the Tor server's nickname in the certificate's commonName field. We
should tweak the handshake protocol so it doesn't rely on any unusual details
in the certificate, yet it remains secure; the certificate itself
should be made to resemble an ordinary HTTPS certificate.  We should also try
to make our advertised cipher-suites closer to what an ordinary web server
would support.

Tor's TLS handshake uses two-certificate chains: one certificate
contains the self-signed identity key for
the router, and the second contains a current TLS key, signed by the
identity key. We use these to authenticate that we're talking to the right
router, and to limit the impact of TLS-key exposure.  Most (though far from
all) consumer-oriented HTTPS services provide only a single certificate.
These extra certificates may help identify Tor's TLS handshake; instead,
bridges should consider using only a single TLS key certificate signed by
their identity key, and providing the full value of the identity key in an
early handshake cell.  More significantly, Tor currently has all clients
present certificates, so that clients are harder to distinguish from servers.
But in a blocking-resistance environment, clients should not present
certificates at all.

Last, what if the adversary starts observing the network traffic even
more closely? Even if our TLS handshake looks innocent, our traffic timing
and volume still look different than a user making a secure web connection
to his bank. The same techniques used in the growing trend to build tools
to recognize encrypted Bittorrent traffic
%~\cite{bt-traffic-shaping}
could be used to identify Tor communication and recognize bridge
relays. Rather than trying to look like encrypted web traffic, we may be
better off trying to blend with some other encrypted network protocol. The
first step is to compare typical network behavior for a Tor client to
typical network behavior for various other protocols. This statistical
cat-and-mouse game is made more complex by the fact that Tor transports a
variety of protocols, and we'll want to automatically handle web browsing
differently from, say, instant messaging.

% Tor cells are 512 bytes each. So TLS records will be roughly
% multiples of this size? How bad is this? -RD
% Look at ``Inferring the Source of Encrypted HTTP Connections''
% by Marc Liberatore and Brian Neil Levine (CCS 2006)
% They substantially flesh out the numbers for the  web fingerprinting
% attack. -PS
% Yes, but I meant detecting the signature of Tor traffic itself, not
% learning what websites we're going to. I wouldn't be surprised to
% learn that these are related problems, but it's not obvious to me. -RD

\subsection{Identity keys as part of addressing information}

We have described a way for the blocked user to bootstrap into the
network once he knows the IP address and ORPort of a bridge. What about
local spoofing attacks? That is, since we never learned an identity
key fingerprint for the bridge, a local attacker could intercept our
connection and pretend to be the bridge we had in mind. It turns out
that giving false information isn't that bad---since the Tor client
ships with trusted keys for the bridge directory authority and the Tor
network directory authorities, the user can learn whether he's being
given a real connection to the bridge authorities or not. (After all,
if the adversary intercepts every connection the user makes and gives
him a bad connection each time, there's nothing we can do.)

What about anonymity-breaking attacks from observing traffic, if the
blocked user doesn't start out knowing the identity key of his intended
bridge? The vulnerabilities aren't so bad in this case either---the
adversary could do similar attacks just by monitoring the network
traffic.
% cue paper by steven and george

Once the Tor client has fetched the bridge's server descriptor, it should
remember the identity key fingerprint for that bridge relay. Thus if
the bridge relay moves to a new IP address, the client can query the
bridge directory authority to look up a fresh server descriptor using
this fingerprint.

So we've shown that it's \emph{possible} to bootstrap into the network
just by learning the IP address and ORPort of a bridge, but are there
situations where it's more convenient or more secure to learn the bridge's
identity fingerprint as well as instead, while bootstrapping? We keep
that question in mind as we next investigate bootstrapping and discovery.

\section{Discovering working bridge relays}
\label{sec:discovery}

Tor's modular design means that we can develop a better relay component
independently of developing the discovery component. This modularity's
great promise is that we can pick any discovery approach we like; but the
unfortunate fact is that we have no magic bullet for discovery. We're
in the same arms race as all the other designs we described in
Section~\ref{sec:related}.

In this section we describe a variety of approaches to adding discovery
components for our design.

\subsection{Bootstrapping: finding your first bridge.}
\label{subsec:first-bridge}

In Section~\ref{subsec:relay-together}, we showed that a user who knows
a working bridge address can use it to reach the bridge authority and
to stay connected to the Tor network. But how do new users reach the
bridge authority in the first place? After all, the bridge authority
will be one of the first addresses that a censor blocks.

First, we should recognize that most government firewalls are not
perfect. That is, they may allow connections to Google cache or some
open proxy servers, or they let file-sharing traffic, Skype, instant
messaging, or World-of-Warcraft connections through. Different users will
have different mechanisms for bypassing the firewall initially. Second,
we should remember that most people don't operate in a vacuum; users will
hopefully know other people who are in other situations or have other
resources available. In the rest of this section we develop a toolkit
of different options and mechanisms, so that we can enable users in a
diverse set of contexts to bootstrap into the system.

(For users who can't use any of these techniques, hopefully they know
a friend who can---for example, perhaps the friend already knows some
bridge relay addresses. If they can't get around it at all, then we
can't help them---they should go meet more people or learn more about
the technology running the firewall in their area.)

By deploying all the schemes in the toolkit at once, we let bridges and
blocked users employ the discovery approach that is most appropriate
for their situation.

\subsection{Independent bridges, no central discovery}

The first design is simply to have no centralized discovery component at
all. Volunteers run bridges, and we assume they have some blocked users
in mind and communicate their address information to them out-of-band
(for example, through Gmail). This design allows for small personal
bridges that have only one or a handful of users in mind, but it can
also support an entire community of users. For example, Citizen Lab's
upcoming Psiphon single-hop proxy tool~\cite{psiphon} plans to use this
\emph{social network} approach as its discovery component.

There are several ways to do bootstrapping in this design. In the simple
case, the operator of the bridge informs each chosen user about his
bridge's address information and/or keys. A different approach involves
blocked users introducing new blocked users to the bridges they know.
That is, somebody in the blocked area can pass along a bridge's address to
somebody else they trust. This scheme brings in appealing but complex game
theoretic properties: the blocked user making the decision has an incentive
only to delegate to trustworthy people, since an adversary who learns
the bridge's address and filters it makes it unavailable for both of them.
Also, delegating known bridges to members of your social network can be
dangerous: an the adversary who can learn who knows which bridges may
be able to reconstruct the social network.

Note that a central set of bridge directory authorities can still be
compatible with a decentralized discovery process. That is, how users
first learn about bridges is entirely up to the bridges, but the process
of fetching up-to-date descriptors for them can still proceed as described
in Section~\ref{sec:bridges}. Of course, creating a central place that
knows about all the bridges may not be smart, especially if every other
piece of the system is decentralized. Further, if a user only knows
about one bridge and he loses track of it, it may be quite a hassle to
reach the bridge authority. We address these concerns next.

\subsection{Families of bridges, no central discovery}

Because the blocked users are running our software too, we have many
opportunities to improve usability or robustness. Our second design builds
on the first by encouraging volunteers to run several bridges at once
(or coordinate with other bridge volunteers), such that some
of the bridges are likely to be available at any given time.

The blocked user's Tor client would periodically fetch an updated set of
recommended bridges from any of the working bridges. Now the client can
learn new additions to the bridge pool, and can expire abandoned bridges
or bridges that the adversary has blocked, without the user ever needing
to care. To simplify maintenance of the community's bridge pool, each
community could run its own bridge directory authority---reachable via
the available bridges, and also mirrored at each bridge.

\subsection{Public bridges with central discovery}

What about people who want to volunteer as bridges but don't know any
suitable blocked users? What about people who are blocked but don't
know anybody on the outside? Here we describe how to make use of these
\emph{public bridges} in a way that still makes it hard for the attacker
to learn all of them.

The basic idea is to divide public bridges into a set of pools based on
identity key. Each pool corresponds to a \emph{distribution strategy}:
an approach to distributing its bridge addresses to users. Each strategy
is designed to exercise a different scarce resource or property of
the user.

How do we divide bridges between these strategy pools such that they're
evenly distributed and the allocation is hard to influence or predict,
but also in a way that's amenable to creating more strategies later
on without reshuffling all the pools? We assign a given bridge
to a strategy pool by hashing the bridge's identity key along with a
secret that only the bridge authority knows: the first $n$ bits of this
hash dictate the strategy pool number, where $n$ is a parameter that
describes how many strategy pools we want at this point. We choose $n=3$
to start, so we divide bridges between 8 pools; but as we later invent
new distribution strategies, we can increment $n$ to split the 8 into
16. Since a bridge can't predict the next bit in its hash, it can't
anticipate which identity key will correspond to a certain new pool
when the pools are split. Further, since the bridge authority doesn't
provide any feedback to the bridge about which strategy pool it's in,
an adversary who signs up bridges with the goal of filling a certain
pool~\cite{casc-rep} will be hindered.

% This algorithm is not ideal. When we split pools, each existing
% pool is cut in half, where half the bridges remain with the
% old distribution policy, and half will be under what the new one
% is. So the new distribution policy inherits a bunch of blocked
% bridges if the old policy was too loose, or a bunch of unblocked
% bridges if its policy was still secure. -RD
%
% I think it should be more chordlike.
% Bridges are allocated to wherever on the ring which is divided
% into arcs (buckets).
% If a bucket gets too full, you can just split it.
% More on this below. -PFS

The first distribution strategy (used for the first pool) publishes bridge
addresses in a time-release fashion. The bridge authority divides the
available bridges into partitions, and each partition is deterministically
available only in certain time windows. That is, over the course of a
given time slot (say, an hour), each requestor is given a random bridge
from within that partition. When the next time slot arrives, a new set
of bridges from the pool are available for discovery. Thus some bridge
address is always available when a new
user arrives, but to learn about all bridges the attacker needs to fetch
all new addresses at every new time slot. By varying the length of the
time slots, we can make it harder for the attacker to guess when to check
back. We expect these bridges will be the first to be blocked, but they'll
help the system bootstrap until they \emph{do} get blocked. Further,
remember that we're dealing with different blocking regimes around the
world that will progress at different rates---so this pool will still
be useful to some users even as the arms races progress.

The second distribution strategy publishes bridge addresses based on the IP
address of the requesting user. Specifically, the bridge authority will
divide the available bridges in the pool into a bunch of partitions
(as in the first distribution scheme), hash the requestor's IP address
with a secret of its own (as in the above allocation scheme for creating
pools), and give the requestor a random bridge from the appropriate
partition. To raise the bar, we should discard the last octet of the
IP address before inputting it to the hash function, so an attacker
who only controls a single ``/24'' network only counts as one user. A
large attacker like China will still be able to control many addresses,
but the hassle of establishing connections from each network (or spoofing
TCP connections) may still slow them down. Similarly, as a special case,
we should treat IP addresses that are Tor exit nodes as all being on
the same network.

The third strategy combines the time-based and location-based
strategies to further constrain and rate-limit the available bridge
addresses. Specifically, the bridge address provided in a given time
slot to a given network location is deterministic within the partition,
rather than chosen randomly each time from the partition. Thus, repeated
requests during that time slot from a given network are given the same
bridge address as the first request.

The fourth strategy is based on Circumventor's discovery strategy.
The Circumventor project, realizing that its adoption will remain limited
if it has no central coordination mechanism, has started a mailing list to
distribute new proxy addresses every few days. From experimentation it
seems they have concluded that sending updates every three or four days
is sufficient to stay ahead of the current attackers.

The fifth strategy provides an alternative approach to a mailing list:
users provide an email address and receive an automated response
listing an available bridge address. We could limit one response per
email address. To further rate limit queries, we could require a CAPTCHA
solution
%~\cite{captcha}
in each case too. In fact, we wouldn't need to
implement the CAPTCHA on our side: if we only deliver bridge addresses
to Yahoo or GMail addresses, we can leverage the rate-limiting schemes
that other parties already impose for account creation.

The sixth strategy ties in the social network design with public
bridges and a reputation system. We pick some seeds---trusted people in
blocked areas---and give them each a few dozen bridge addresses and a few
\emph{delegation tokens}. We run a website next to the bridge authority,
where users can log in (they connect via Tor, and they don't need to
provide actual identities, just persistent pseudonyms). Users can delegate
trust to other people they know by giving them a token, which can be
exchanged for a new account on the website. Accounts in ``good standing''
then accrue new bridge addresses and new tokens. As usual, reputation
schemes bring in a host of new complexities~\cite{rep-anon}: how do we
decide that an account is in good standing? We could tie reputation
to whether the bridges they're told about have been blocked---see
Section~\ref{subsec:geoip} below for initial thoughts on how to discover
whether bridges have been blocked. We could track reputation between
accounts (if you delegate to somebody who screws up, it impacts you too),
or we could use blinded delegation tokens~\cite{chaum-blind} to prevent
the website from mapping the seeds' social network. We put off deeper
discussion of the social network reputation strategy for future work.

Pools seven and eight are held in reserve, in case our currently deployed
tricks all fail at once and the adversary blocks all those bridges---so
we can adapt and move to new approaches quickly, and have some bridges
immediately available for the new schemes. New strategies might be based
on some other scarce resource, such as relaying traffic for others or
other proof of energy spent. (We might also worry about the incentives
for bridges that sign up and get allocated to the reserve pools: will they
be unhappy that they're not being used? But this is a transient problem:
if Tor users are bridges by default, nobody will mind not being used yet.
See also Section~\ref{subsec:incentives}.)

%Is it useful to load balance which bridges are handed out? The above
%pool concept makes some bridges wildly popular and others less so.
%But I guess that's the point.

\subsection{Public bridges with coordinated discovery}

We presented the above discovery strategies in the context of a single
bridge directory authority, but in practice we will want to distribute the
operations over several bridge authorities---a single point of failure
or attack is a bad move. The first answer is to run several independent
bridge directory authorities, and bridges gravitate to one based on
their identity key. The better answer would be some federation of bridge
authorities that work together to provide redundancy but don't introduce
new security issues. We could even imagine designs where the bridge
authorities have encrypted versions of the bridge's server descriptors,
and the users learn a decryption key that they keep private when they
first hear about the bridge---this way the bridge authorities would not
be able to learn the IP address of the bridges.

We leave this design question for future work.

\subsection{Assessing whether bridges are useful}

Learning whether a bridge is useful is important in the bridge authority's
decision to include it in responses to blocked users. For example, if
we end up with a list of thousands of bridges and only a few dozen of
them are reachable right now, most blocked users will not end up knowing
about working bridges.

There are three components for assessing how useful a bridge is. First,
is it reachable from the public Internet? Second, what proportion of
the time is it available? Third, is it blocked in certain jurisdictions?

The first component can be tested just as we test reachability of
ordinary Tor servers. Specifically, the bridges do a self-test---connect
to themselves via the Tor network---before they are willing to
publish their descriptor, to make sure they're not obviously broken or
misconfigured. Once the bridges publish, the bridge authority also tests
reachability to make sure they're not confused or outright lying.

The second component can be measured and tracked by the bridge authority.
By doing periodic reachability tests, we can get a sense of how often the
bridge is available. More complex tests will involve bandwidth-intensive
checks to force the bridge to commit resources in order to be counted as
available. We need to evaluate how the relationship of uptime percentage
should weigh into our choice of which bridges to advertise. We leave
this to future work.

The third component is perhaps the trickiest: with many different
adversaries out there, how do we keep track of which adversaries have
blocked which bridges, and how do we learn about new blocks as they
occur? We examine this problem next.

\subsection{How do we know if a bridge relay has been blocked?}
\label{subsec:geoip}

There are two main mechanisms for testing whether bridges are reachable
from inside each blocked area: active testing via users, and passive
testing via bridges.

In the case of active testing, certain users inside each area
sign up as testing relays. The bridge authorities can then use a
Blossom-like~\cite{blossom-thesis} system to build circuits through them
to each bridge and see if it can establish the connection. But how do
we pick the users? If we ask random users to do the testing (or if we
solicit volunteers from the users), the adversary should sign up so he
can enumerate the bridges we test. Indeed, even if we hand-select our
testers, the adversary might still discover their location and monitor
their network activity to learn bridge addresses.

Another answer is not to measure directly, but rather let the bridges
report whether they're being used.
%If they periodically report to their
%bridge directory authority how much use they're seeing, perhaps the
%authority can make smart decisions from there.
Specifically, bridges should install a GeoIP database such as the public
IP-To-Country list~\cite{ip-to-country}, and then periodically report to the
bridge authorities which countries they're seeing use from. This data
would help us track which countries are making use of the bridge design,
and can also let us learn about new steps the adversary has taken in
the arms race. (The compressed GeoIP database is only several hundred
kilobytes, and we could even automate the update process by serving it
from the bridge authorities.)
More analysis of this passive reachability
testing design is needed to resolve its many edge cases: for example,
if a bridge stops seeing use from a certain area, does that mean the
bridge is blocked or does that mean those users are asleep?

There are many more problems with the general concept of detecting whether
bridges are blocked. First, different pieces of the Internet are blocked
in different ways, and the actual firewall jurisdictions do not match
country borders. Our bridge scheme could help us map out the topology
of the censored Internet, but this is a huge task. More generally,
if a bridge relay isn't reachable, is that because of a network block
somewhere, because of a problem at the bridge relay, or just a temporary
outage somewhere in between? And last, an attacker could poison our
bridge database by signing up already-blocked bridges. In this case,
if we're stingy giving out bridge addresses, users in that country won't
learn working bridges.

All of these issues are made more complex when we try to integrate either
active or passive testing into our social network reputation system above.
Since in that case we punish or reward users based on whether bridges
get blocked, the adversary has new attacks to trick or bog down the
reputation tracking.

Clearly more analysis is required. The eventual solution will probably
involve a combination of passive measurement via GeoIP and active
measurement from trusted testers.  More generally, we can use the passive
feedback mechanism to track usage of the bridge network as a whole---which
would let us respond to attacks and adapt the design, and it would also
let the general public track the progress of the project.

%Worry: the adversary could choose not to block bridges but just record
%connections to them. So be it, I guess.

\subsection{Advantages of deploying all solutions at once}

For once we're not in the position of the defender: we don't have to
defend against every possible filtering scheme, we just have to defend
against at least one.

adversary has to guess how to allocate his resources

(nick, want to write this section?)

%\subsection{Remaining unsorted notes}

%In the first subsection we describe how to find a first bridge.

%Going to be an arms race. Need a bag of tricks. Hard to say
%which ones will work. Don't spend them all at once.

%Some techniques are sufficient to get us an IP address and a port,
%and others can get us IP:port:key. Lay out some plausible options
%for how users can bootstrap into learning their first bridge.

%\section{The account / reputation system}
%\section{Social networks with directory-side support}
%\label{sec:accounts}

%One answer is to measure based on whether the bridge addresses
%we give it end up blocked. But how do we decide if they get blocked?

%Perhaps each bridge should be known by a single bridge directory
%authority. This makes it easier to trace which users have learned about
%it, so easier to blame or reward. It also makes things more brittle,
%since loss of that authority means its bridges aren't advertised until
%they switch, and means its bridge users are sad too.
%(Need a slick hash algorithm that will map our identity key to a
%bridge authority, in a way that's sticky even when we add bridge
%directory authorities, but isn't sticky when our authority goes
%away. Does this exist?)

%\subsection{Discovery based on social networks}

%A token that can be exchanged at the bridge authority (assuming you
%can reach it) for a new bridge address.

%The account server runs as a Tor controller for the bridge authority.

%Users can establish reputations, perhaps based on social network
%connectivity, perhaps based on not getting their bridge relays blocked,

%Probably the most critical lesson learned in past work on reputation
%systems in privacy-oriented environments~\cite{rep-anon} is the need for
%verifiable transactions. That is, the entity computing and advertising
%reputations for participants needs to actually learn in a convincing
%way that a given transaction was successful or unsuccessful.

%(Lesson from designing reputation systems~\cite{rep-anon}: easy to
%reward good behavior, hard to punish bad behavior.

\section{Security considerations}
\label{sec:security}

\subsection{Possession of Tor in oppressed areas}

Many people speculate that installing and using a Tor client in areas with
particularly extreme firewalls is a high risk---and the risk increases
as the firewall gets more restrictive. This is probably true, but there's
a counter pressure as well: as the firewall gets more restrictive, more
ordinary people use Tor for more mainstream activities, such as learning
about Wall Street prices or looking at pictures of women's ankles. So
if the restrictive firewall pushes up the number of Tor users, then the
``typical'' Tor user becomes more mainstream.

Hard to say which of these pressures will ultimately win out.

Nick, want to rewrite/elaborate on this section?

\subsection{Observers can tell who is publishing and who is reading}
\label{subsec:upload-padding}

Should bridge users sometimes send bursts of long-range drop cells?

\subsection{Anonymity effects from acting as a bridge relay}

Against some attacks, relaying traffic for others can improve anonymity. The
simplest example is an attacker who owns a small number of Tor servers. He
will see a connection from the bridge, but he won't be able to know
whether the connection originated there or was relayed from somebody else.

There are some cases where it doesn't seem to help: if an attacker can
watch all of the bridge's incoming and outgoing traffic, then it's easy
to learn which connections were relayed and which started there. (In this
case he still doesn't know the final destinations unless he is watching
them too, but in this case bridges are no better off than if they were
an ordinary client.)

There are also some potential downsides to running a bridge. First, while
we try to make it hard to enumerate all bridges, it's still possible to
learn about some of them, and for some people just the fact that they're
running one might signal to an attacker that they place a high value
on their anonymity. Second, there are some more esoteric attacks on Tor
relays that are not as well-understood or well-tested---for example, an
attacker may be able to ``observe'' whether the bridge is sending traffic
even if he can't actually watch its network, by relaying traffic through
it and noticing changes in traffic timing~\cite{attack-tor-oak05}. On
the other hand, it may be that limiting the bandwidth the bridge is
willing to relay will allow this sort of attacker to determine if it's
being used as a bridge but not whether it is adding traffic of its own.

It is an open research question whether the benefits outweigh the risks. A
lot of the decision rests on which attacks the users are most worried
about. For most users, we don't think running a bridge relay will be
that damaging.

Need to examine how entry guards fit in. If the blocked user doesn't use
the bridge's entry guards, then the bridge doesn't gain as much cover
benefit. If he does, first how does that actually work, and second is
it turtles all the way down (need to use the guard's guards, ...)?

\subsection{Trusting local hardware: Internet cafes and LiveCDs}
\label{subsec:cafes-and-livecds}

Assuming that users have their own trusted hardware is not
always reasonable.

For Internet cafe Windows computers that let you attach your own USB key,
a USB-based Tor image would be smart. There's Torpark, and hopefully
there will be more thoroughly analyzed options down the road. Worries
about hardware or
software keyloggers and other spyware---and physical surveillance.

If the system lets you boot from a CD or from a USB key, you can gain
a bit more security by bringing a privacy LiveCD with you. Hardware
keyloggers and physical surveillance still a worry. LiveCDs also useful
if it's your own hardware, since it's easier to avoid leaving breadcrumbs
everywhere.

\subsection{Forward compatibility and retiring bridge authorities}

Eventually we'll want to change the identity key and/or location
of a bridge authority. How do we do this mostly cleanly?

\subsection{The trust chain}
\label{subsec:trust-chain}

Tor's ``public key infrastructure'' provides a chain of trust to
let users verify that they're actually talking to the right servers.
There are four pieces to this trust chain.

First, when Tor clients are establishing circuits, at each step
they demand that the next Tor server in the path prove knowledge of
its private key~\cite{tor-design}. This step prevents the first node
in the path from just spoofing the rest of the path. Second, the
Tor directory authorities provide a signed list of servers along with
their public keys---so unless the adversary can control a threshold
of directory authorities, he can't trick the Tor client into using other
Tor servers. Third, the location and keys of the directory authorities,
in turn, is hard-coded in the Tor source code---so as long as the user
got a genuine version of Tor, he can know that he is using the genuine
Tor network. And last, the source code and other packages are signed
with the GPG keys of the Tor developers, so users can confirm that they
did in fact download a genuine version of Tor.

But how can a user in an oppressed country know that he has the correct
key fingerprints for the developers? As with other security systems, it
ultimately comes down to human interaction. The keys are signed by dozens
of people around the world, and we have to hope that our users have met
enough people in the PGP web of trust
%~\cite{pgp-wot}
that they can learn
the correct keys. For users that aren't connected to the global security
community, though, this question remains a critical weakness.

% XXX make clearer the trust chain step for bridge directory authorities

\subsection{Security through obscurity: publishing our design}

Many other schemes like dynaweb use the typical arms race strategy of
not publishing their plans. Our goal here is to produce a design---a
framework---that can be public and still secure. Where's the tradeoff?

\section{Performance improvements}
\label{sec:performance}

\subsection{Fetch server descriptors just-in-time}

I guess we should encourage most places to do this, so blocked
users don't stand out.


network-status and directory optimizations. caching better. partitioning
issues?

\section{Maintaining reachability}

\subsection{How many bridge relays should you know about?}

If they're ordinary Tor users on cable modem or DSL, many of them will
disappear and/or move periodically. How many bridge relays should a
blockee know
about before he's likely to have at least one reachable at any given point?
How do we factor in a parameter for "speed that his bridges get discovered
and blocked"?

The related question is: if the bridge relays change IP addresses
periodically, how often does the bridge user need to "check in" in order
to keep from being cut out of the loop?

Families of bridges: give out 4 or 8 at once, bound together.

\subsection{Cablemodem users don't provide important websites}
\label{subsec:block-cable}

...so our adversary could just block all DSL and cablemodem networks,
and for the most part only our bridge relays would be affected.

The first answer is to aim to get volunteers both from traditionally
``consumer'' networks and also from traditionally ``producer'' networks.

The second answer (not so good) would be to encourage more use of consumer
networks for popular and useful websites.  (But P2P exists; minor websites
exist; gaming exists; IM exists; ...)

Other attack: China pressures Verizon to discourage its users from
running bridges.

\subsection{Scanning-resistance}

If it's trivial to verify that we're a bridge, and we run on a predictable
port, then it's conceivable our attacker would scan the whole Internet
looking for bridges. (In fact, he can just scan likely networks like
cablemodem and DSL services---see Section~\ref{block-cable} for a related
attack.) It would be nice to slow down this attack. It would
be even nicer to make it hard to learn whether we're a bridge without
first knowing some secret.

Password protecting the bridges.
Could provide a password to the bridge user. He provides a nonced hash of
it or something when he connects. We'd need to give him an ID key for the
bridge too, and wait to present the password until we've TLSed, else the
adversary can pretend to be the bridge and MITM him to learn the password.

We could some kind of ID-based knocking protocol, or we could act like an
unconfigured HTTPS server if treated like one.

We can assume that the attacker can easily recognize https connections
to unknown servers. It can then attempt to connect to them and block
connections to servers that seem suspicious. It may be that password
protected web sites will not be suspicious in general, in which case
that may be the easiest way to give controlled access to the bridge.
If such sites that have no other overt features are automatically
blocked when detected, then we may need to be more subtle.
Possibilities include serving an innocuous web page if a TLS encrypted
request is received without the authorization needed to access the Tor
network and only responding to a requested access to the Tor network
of proper authentication is given. If an unauthenticated request to
access the Tor network is sent, the bridge should respond as if
it has received a message it does not understand (as would be the
case were it not a bridge).


\subsection{How to motivate people to run bridge relays}
\label{subsec:incentives}

One of the traditional ways to get people to run software that benefits
others is to give them motivation to install it themselves.  An often
suggested approach is to install it as a stunning screensaver so everybody
will be pleased to run it. We take a similar approach here, by leveraging
the fact that these users are already interested in protecting their
own Internet traffic, so they will install and run the software.

Make all Tor users become bridges if they're reachable---needs more work
on usability first, but we're making progress.

Also, we can make a snazzy network graph with Vidalia that emphasizes
the connections the bridge user is currently relaying. (Minor anonymity
implications, but hey.) (In many cases there won't be much activity,
so this may backfire. Or it may be better suited to full-fledged Tor
servers.)

% Also consider everybody-a-server. Many of the scalability questions
% are easier when you're talking about making everybody a bridge.

\subsection{What if the clients can't install software?}

[this section should probably move to the related work section,
or just disappear entirely.]

Bridge users without Tor software

Bridge relays could always open their socks proxy. This is bad though,
first
because bridges learn the bridge users' destinations, and second because
we've learned that open socks proxies tend to attract abusive users who
have no idea they're using Tor.

Bridges could require passwords in the socks handshake (not supported
by most software including Firefox). Or they could run web proxies
that require authentication and then pass the requests into Tor. This
approach is probably a good way to help bootstrap the Psiphon network,
if one of its barriers to deployment is a lack of volunteers willing
to exit directly to websites. But it clearly drops some of the nice
anonymity and security features Tor provides.

A hybrid approach where the user gets his anonymity from Tor but his
software-less use from a web proxy running on a trusted machine on the
free side.

\subsection{Publicity attracts attention}
\label{subsec:publicity}

Many people working on this field want to publicize the existence
and extent of censorship concurrently with the deployment of their
circumvention software. The easy reason for this two-pronged push is
to attract volunteers for running proxies in their systems; but in many
cases their main goal is not to build the software, but rather to educate
the world about the censorship. The media also tries to do its part by
broadcasting the existence of each new circumvention system.

But at the same time, this publicity attracts the attention of the
censors. We can slow down the arms race by not attracting as much
attention, and just spreading by word of mouth. If our goal is to
establish a solid social network of bridges and bridge users before
the adversary gets involved, does this attention tradeoff work to our
advantage?

\subsection{The Tor website: how to get the software}



\section{Future designs}

\subsection{Bridges inside the blocked network too}

Assuming actually crossing the firewall is the risky part of the
operation, can we have some bridge relays inside the blocked area too,
and more established users can use them as relays so they don't need to
communicate over the firewall directly at all? A simple example here is
to make new blocked users into internal bridges also---so they sign up
on the BDA as part of doing their query, and we give out their addresses
rather than (or along with) the external bridge addresses. This design
is a lot trickier because it brings in the complexity of whether the
internal bridges will remain available, can maintain reachability with
the outside world, etc.

Hidden services as bridges. Hidden services as bridge directory authorities.

\section{Conclusion}

a technical solution won't solve the whole problem. after all, china's
firewall is *socially* very successful, even if technologies exist to
get around it.

but having a strong technical solution is still useful as a piece of the
puzzle.

\bibliographystyle{plain} \bibliography{tor-design}

\appendix

\section{Counting Tor users by country}
\label{app:geoip}

\end{document}

ship geoip db to bridges. they look up users who tls to them in the db,
and upload a signed list of countries and number-of-users each day. the
bridge authority aggregates them and publishes stats.

bridge relays have buddies
they ask a user to test the reachability of their buddy.
leaks O(1) bridges, but not O(n).

we should not be blockable by ordinary cisco censorship features.
that is, if they want to block our new design, they will need to
add a feature to block exactly this.
strategically speaking, this may come in handy.

Bridges come in clumps of 4 or 8 or whatever. If you know one bridge
in a clump, the authority will tell you the rest. Now bridges can
ask users to test reachability of their buddies.

Giving out clumps helps with dynamic IP addresses too. Whether it
should be 4 or 8 depends on our churn.

the account server. let's call it a database, it doesn't have to
be a thing that human interacts with.

so how do we reward people for being good?

\subsubsection{Public Bridges with Coordinated Discovery}

****Pretty much this whole subsubsection will probably need to be
deferred until ``later'' and moved to after end document, but I'm leaving
it here for now in case useful.******

Rather than be entirely centralized, we can have a coordinated
collection of bridge authorities, analogous to how Tor network
directory authorities now work.

Key components
``Authorities'' will distribute caches of what they know to overlapping
collections of nodes so that no one node is owned by one authority.
Also so that it is impossible to DoS info maintained by one authority
simply by making requests to it.

Where a bridge gets assigned is not predictable by the bridge?

If authorities don't know the IP addresses of the bridges they
are responsible for, they can't abuse that info (or be attacked for
having it). But, they also can't, e.g., control being sent massive
lists of nodes that were never good. This raises another question.
We generally decry use of IP address for location, etc. but we
need to do that to limit the introduction of functional but useless
IP addresses because, e.g., they are in China and the adversary
owns massive chunks of the IP space there.

We don't want an arbitrary someone to be able to contact the
authorities and say an IP address is bad because it would be easy
for an adversary to take down all the suspicious bridges
even if they provide good cover websites, etc. Only the bridge
itself and/or the directory authority can declare a bridge blocked
from somewhere.


9. Bridge directories must not simply be a handful of nodes that
provide the list of bridges. They must flood or otherwise distribute
information out to other Tor nodes as mirrors. That way it becomes
difficult for censors to flood the bridge directory servers with
requests, effectively denying access for others. But, there's lots of
churn and a much larger size than Tor directories.  We are forced to
handle the directory scaling problem here much sooner than for the
network in general. Authorities can pass their bridge directories
(and policy info) to some moderate number of unidentified Tor nodes.
Anyone contacting one of those nodes can get bridge info. the nodes
must remain somewhat synched to prevent the adversary from abusing,
e.g., a timed release policy or the distribution to those nodes must
be resilient even if they are not coordinating.

I think some kind of DHT like scheme would work here. A Tor node is
assigned a chunk of the directory.  Lookups in the directory should be
via hashes of keys (fingerprints) and that should determine the Tor
nodes responsible. Ordinary directories can publish lists of Tor nodes
responsible for fingerprint ranges.  Clients looking to update info on
some bridge will make a Tor connection to one of the nodes responsible
for that address.  Instead of shutting down a circuit after getting
info on one address, extend it to another that is responsible for that
address (the node from which you are extending knows you are doing so
anyway). Keep going.  This way you can amortize the Tor connection.

10. We need some way to give new identity keys out to those who need
them without letting those get immediately blocked by authorities. One
way is to give a fingerprint that gets you more fingerprints, as
already described. These are meted out/updated periodically but allow
us to keep track of which sources are compromised: if a distribution
fingerprint repeatedly leads to quickly blocked bridges, it should be
suspect, dropped, etc. Since we're using hashes, there shouldn't be a
correlation with bridge directory mirrors, bridges, portions of the
network observed, etc. It should just be that the authorities know
about that key that leads to new addresses.

This last point is very much like the issues in the valet nodes paper,
which is essentially about blocking resistance wrt exiting the Tor network,
while this paper is concerned with blocking the entering to the Tor network.
In fact the tickets used to connect to the IPo (Introduction Point),
could serve as an example, except that instead of authorizing
a connection to the Hidden Service, it's authorizing the downloading
of more fingerprints.

Also, the fingerprints can follow the hash(q + '1' + cookie) scheme of
that paper (where q = hash(PK + salt) gave the q.onion address).  This
allows us to control and track which fingerprint was causing problems.

Note that, unlike many settings, the reputation problem should not be
hard here. If a bridge says it is blocked, then it might as well be.
If an adversary can say that the bridge is blocked wrt
$\mathit{censor}_i$, then it might as well be, since
$\mathit{censor}_i$ can presumably then block that bridge if it so
chooses.

11. How much damage can the adversary do by running nodes in the Tor
network and watching for bridge nodes connecting to it?  (This is
analogous to an Introduction Point watching for Valet Nodes connecting
to it.) What percentage of the network do you need to own to do how
much damage. Here the entry-guard design comes in helpfully.  So we
need to have bridges use entry-guards, but (cf. 3 above) not use
bridges as entry-guards. Here's a serious tradeoff (again akin to the
ratio of valets to IPos) the more bridges/client the worse the
anonymity of that client. The fewer bridges/client the worse the 
blocking resistance of that client.



