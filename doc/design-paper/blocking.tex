\documentclass{llncs}

\usepackage{url}
\usepackage{amsmath}
\usepackage{epsfig}

%\setlength{\textwidth}{5.9in}
%\setlength{\textheight}{8.4in}
%\setlength{\topmargin}{.5cm}
%\setlength{\oddsidemargin}{1cm}
%\setlength{\evensidemargin}{1cm}

\newenvironment{tightlist}{\begin{list}{$\bullet$}{
  \setlength{\itemsep}{0mm}
    \setlength{\parsep}{0mm}
    %  \setlength{\labelsep}{0mm}
    %  \setlength{\labelwidth}{0mm}
    %  \setlength{\topsep}{0mm}
    }}{\end{list}}

\begin{document}

\title{Design of a blocking-resistant anonymity system}

\author{Roger Dingledine\inst{1} \and
Nick Mathewson\inst{1}}
\institute{The Free Haven Project \email{<\{arma,nickm\}@freehaven.net>}}

\maketitle
\pagestyle{plain}

\begin{abstract}

Websites around the world are increasingly being blocked by
government-level firewalls. Many people use anonymizing networks like
Tor to contact sites without letting an attacker trace their activities,
and as an added benefit they are no longer affected by local censorship.
But if the attacker simply denies access to the Tor network itself,
blocked users can no longer benefit from the security Tor offers.

Here we describe a design that builds upon the current Tor network
to provide an anonymizing network that resists blocking
by government-level attackers.

\end{abstract}

\section{Introduction and Goals}

Anonymizing networks such as Tor~\cite{tor-design} bounce traffic around
a network of relays. They aim to hide not only what is being said, but
also who is communicating with whom, which users are using which websites,
and so on. These systems have a broad range of users, including ordinary
citizens who want to avoid being profiled for targeted advertisements,
corporations who don't want to reveal information to their competitors,
and law enforcement and government intelligence agencies who need to do
operations on the Internet without being noticed.

Historically, research on anonymizing systems has assumed a passive
attacker who monitors the user (named Alice) and tries to discover her
activities, yet lets her reach any piece of the network. In more modern
threat models such as Tor's, the adversary is allowed to perform active
attacks such as modifying communications in hopes of tricking Alice
into revealing her destination, or intercepting some of her connections
to run a man-in-the-middle attack. But these systems still assume that
Alice can eventually reach the anonymizing network.

An increasing number of users are making use of the Tor software not
so much for its anonymity properties but for its censorship resistance
properties -- if they access Internet sites like Wikipedia and Blogspot
via Tor, they are no longer affected by local censorship and firewall
rules. In fact, an informal user study showed China as the third largest
user base for Tor clients~\cite{geoip-tor}, with tens of thousands of
people accessing the Tor network from China each day.

The current Tor design is easy to block if the attacker controls Alice's
connection to the Tor network -- by blocking the directory authorities,
by blocking all the server IP addresses in the directory, or by filtering
based on the signature of the Tor TLS handshake. Here we describe a
design that builds upon the current Tor network to provide an anonymizing
network that also resists this blocking.

%And adding more different classes of users and goals to the Tor network
%improves the anonymity for all Tor users~\cite{econymics,tor-weis06}.

\section{Adversary assumptions}
\label{sec:adversary}

The history of blocking-resistance designs is littered with all sorts
of conflicting assumptions about what adversaries to expect and what
problems are in the critical path to a solution. Here we try to enumerate
our best understanding of the current situation around the world.

In the traditional security style, we aim to describe a strong attacker
-- if we can defend against it, we inherit protection against weaker
attackers as well. After all, we want a general design that will
work for people in China, people in Iran, people in Thailand, people
in firewalled corporate networks who can't get out to whistleblow,
and people in whatever the next oppressive situation is. In fact, by
designing with a variety of adversaries in mind, we can actually take
advantage of the fact that adversaries will be in different stages of
the arms race at each location.

We assume there are three main network attacks by censors
currently~\cite{clayton:pet2006}:

\begin{tightlist}
\item Block destination by automatically searching for certain strings
in TCP packets.
\item Block destination by manually listing its IP address at the
firewall.
\item Intercept DNS requests and give bogus responses for certain
destination hostnames.
\end{tightlist}

We assume the network firewall has very limited CPU per
connection~\cite{clayton:pet2006}. Against an adversary who spends
hours looking through the contents of each packet, we would need
some stronger mechanism such as steganography, which introduces its
own problems~\cite{active-wardens,foo,bar}.

We assume that readers of blocked content will not be punished much,
relative to publishers. So far in places like China, the authorities
mainly go after people who publish materials and coordinate organized
movements against the state. If they find that a user happens to be
reading a site that should be blocked, the typical response is simply
to block the site. Of course, even with an encrypted connection,
the adversary can observe whether Alice is mostly downloading
bytes or mostly uploading them -- we discuss this issue more in
Section~\ref{subsec:upload-padding}.

We assume that while various different adversaries can coordinate and share
notes, there will be a significant time lag between one attacker learning
how to overcome a facet of our design and other attackers picking it up.
(Corollary: in the early stages of deployment, the insider threat isn't
as high of a risk.)

We assume that our users have control over their hardware and
software -- they don't have any spyware installed, there are no
cameras watching their screen, etc. Unfortunately, in many situations
these attackers are very real~\cite{zuckerman-threatmodels}; yet
software-based security systems like ours are poorly equipped to handle
a user who is entirely observed and controlled by the adversary. See
Section~\ref{subsec:cafes-and-livecds} for more discussion of what little
we can do about this issue.

We assume that the user will fetch a genuine version of Tor, rather than
one supplied by the adversary; see Section~\ref{subsec:trust-chain}
for discussion on helping the user confirm that he has a genuine version
and that he can connected to the real Tor network.

\section{Related schemes}

\subsection{public single-hop proxies}

Anonymizer and friends

\subsection{personal single-hop proxies}

Psiphon, circumventor, cgiproxy.

Simpler to deploy; might not require client-side software.

\subsection{JAP}

Stefan's WPES paper is probably the closest related work, and is
the starting point for the design in this paper.

\subsection{break your sensitive strings into multiple tcp packets;
ignore RSTs}

\subsection{steganography}

infranet

\subsection{Internal caching networks}

Freenet is deployed inside China and caches outside content.

\subsection{Skype}

port-hopping. encryption. voice communications not so susceptible to
keystroke loggers (even graphical ones).

\section{Components of the current Tor design}

Tor provides three security properties:
\begin{tightlist}
\item 1. A local observer can't learn, or influence, your destination.
\item 2. No single piece of the infrastructure can link you to your
destination.
\item 3. The destination, or somebody watching the destination,
can't learn your location.
\end{tightlist}

We care most clearly about property number 1. But when the arms race
progresses, property 2 will become important -- so the blocking adversary
can't learn user+destination pairs just by volunteering a relay. It's not so
clear to see that property 3 is important, but consider websites and
services that are pressured into treating clients from certain network
locations differently.

Other benefits:

\begin{tightlist}
\item Separates the role of relay from the role of exit node.

\item (Re)builds circuits automatically in the background, based on
whichever paths work.
\end{tightlist}

\subsection{Tor circuits}

can build arbitrary overlay paths given a set of descriptors~\cite{blossom}

\subsection{Tor directory servers}

central trusted locations that keep track of what Tor servers are
available and usable.

(threshold trust, so not quite so bad. See
Section~\ref{subsec:trust-chain} for details.)

\subsection{Tor user base}

Hundreds of thousands of users from around the world. Some with publically
reachable IP addresses.

\section{Why hasn't Tor been blocked yet?}

Hard to say. People think it's hard to block? Not enough users, or not
enough ordinary users? Nobody has been embarrassed by it yet? "Steam
valve"?




\section{Components of a blocking-resistant design}

Here we describe the new pieces we need to add to the current Tor design.

\subsection{Bridge relays}

Some Tor users on the free side of the network will opt to become
\emph{bridge relays}. They will relay a small amount of bandwidth into
the main Tor network, so they won't need to allow exits.

They sign up on the bridge directory authorities (described below),
and they use Tor to publish their descriptor so an attacker observing
the bridge directory authority's network can't enumerate bridges.

...need to outline instructions for a Tor config that will publish
to an alternate directory authority, and for controller commands
that will do this cleanly.

\subsection{The bridge directory authority (BDA)}

They aggregate server descriptors just like the main authorities, and
answer all queries as usual, except they don't publish full directories
or network statuses.

So once you know a bridge relay's key, you can get the most recent
server descriptor for it.

Since bridge authorities don't answer full network statuses, we
need to add a new way for users to learn the current status for a
single relay or a small set of relays -- to answer such questions as
``is it running?''  or ``is it behaving correctly?'' We describe in
Section~\ref{subsec:enclave-dirs} a way for the bridge authority to
publish this information without resorting to signing each answer
individually.

\subsection{Putting them together}

If a blocked user has address information for one working bridge relay,
then he can use it to make secure connections to the BDA to update his
knowledge about other bridge
relays, and he can make secure connections to the main Tor network
and directory servers to build circuits and connect to the rest of
the Internet.

So now we've reduced the problem from how to circumvent the firewall
for all transactions (and how to know that the pages you get have not
been modified by the local attacker) to how to learn about a working
bridge relay.

The following section describes ways to bootstrap knowledge of your first
bridge relay, and ways to maintain connectivity once you know a few
bridge relays. (See Section~\ref{later} for a discussion of exactly
what information is sufficient to characterize a bridge relay.)

\section{Discovering and maintaining working bridge relays}

Most government firewalls are not perfect. They allow connections to
Google cache or some open proxy servers, or they let file-sharing or
Skype or World-of-Warcraft connections through.
For users who can't use any of these techniques, hopefully they know
a friend who can -- for example, perhaps the friend already knows some
bridge relay addresses.
(If they can't get around it at all, then we can't help them -- they
should go meet more people.)

Thus they can reach the BDA. From here we either assume a social
network or other mechanism for learning IP:dirport or key fingerprints
as above, or we assume an account server that allows us to limit the
number of new bridge relays an external attacker can discover.

Going to be an arms race. Need a bag of tricks. Hard to say
which ones will work. Don't spend them all at once.

\subsection{Discovery based on social networks}

A token that can be exchanged at the BDA (assuming you
can reach it) for a new IP:dirport or server descriptor.

The account server

runs as a Tor controller for the bridge authority

Users can establish reputations, perhaps based on social network
connectivity, perhaps based on not getting their bridge relays blocked,

(Lesson from designing reputation systems~\cite{p2p-econ}: easy to
reward good behavior, hard to punish bad behavior.

\subsection{How to allocate bridge addresses to users}

Hold a fraction in reserve, in case our currently deployed tricks
all fail at once -- so we can move to new approaches quickly.
(Bridges that sign up and don't get used yet will be sad; but this
is a transient problem -- if bridges are on by default, nobody will
mind not being used.)

Perhaps each bridge should be known by a single bridge directory
authority. This makes it easier to trace which users have learned about
it, so easier to blame or reward. It also makes things more brittle,
since loss of that authority means its bridges aren't advertised until
they switch, and means its bridge users are sad too.
(Need a slick hash algorithm that will map our identity key to a
bridge authority, in a way that's sticky even when we add bridge
directory authorities, but isn't sticky when our authority goes
away. Does this exist?)

Divide bridges into buckets based on their identity key.
[Design question: need an algorithm to deterministically map a bridge's
identity key into a category that isn't too gameable. Take a keyed
hash of the identity key plus a secret the bridge authority keeps?
An adversary signing up bridges won't easily be able to learn what
category he's been put in, so it's slow to attack.]

One portion of the bridges is the public bucket. If you ask the
bridge account server for a public bridge, it will give you a random
one of these. We expect they'll be the first to be blocked, but they'll
help the system bootstrap until it *does* get blocked, and remember that
we're dealing with different blocking regimes around the world that will
progress at different rates.

The generalization of the public bucket is a bucket based on the bridge
user's IP address: you can learn a random entry only from the subbucket
your IP address (actually, your /24) maps to.

Another portion of the bridges can be sectioned off to be given out in
a time-release basis. The bucket is partitioned into pieces which are
deterministically available only in certain time windows.

And of course another portion is made available for the social network
design above.


Is it useful to load balance which bridges are handed out? The above
bucket concept makes some bridges wildly popular and others less so.
But I guess that's the point.

\section{Security improvements}

\subsection{Hiding Tor's network signatures}
\label{subsec:enclave-dirs}

The simplest format for communicating information about a bridge relay
is as an IP address and port for its directory cache. From there, the
user can ask the directory cache for an up-to-date copy of that bridge
relay's server descriptor, to learn its current circuit keys, the port
it uses for Tor connections, and so on.

However, connecting directly to the directory cache involves a plaintext
http request, so the censor could create a network signature for the
request and/or its response, thus preventing these connections. Therefore
we've modified the Tor protocol so that users can connect to the directory
cache via the main Tor port -- they establish a TLS connection with
the bridge as normal, and then send a Tor "begindir" relay cell to
establish a connection to its directory cache.

Predictable SSL ports:
We should encourage most servers to listen on port 443, which is
where SSL normally listens.
Is that all it will take, or should we set things up so some fraction
of them pick random ports? I can see that both helping and hurting.

Predictable TLS handshakes:
Right now Tor has some predictable strings in its TLS handshakes.
These can be removed; but should they be replaced with nothing, or
should we try to emulate some popular browser? In any case our
protocol demands a pair of certs on both sides -- how much will this
make Tor handshakes stand out?

\subsection{Minimum info required to describe a bridge}

In the previous subsection, we described a way for the bridge user
to bootstrap into the network just by knowing the IP address and
Tor port of a bridge. What about local spoofing attacks? That is,
since we never learned an identity key fingerprint for the bridge,
a local attacker could intercept our connection and pretend to be
the bridge we had in mind. It turns out that giving false information
isn't that bad -- since the Tor client ships with trusted keys for the
bridge directory authority and the Tor network directory authorities,
the user can learn whether he's being given a real connection to the
bridge authorities or not. (If the adversary intercepts every connection
the user makes and gives him a bad connection each time, there's nothing
we can do.)

What about anonymity-breaking attacks from observing traffic? Not so bad
either, since the adversary could do the same attacks just by monitoring
the network traffic.

Once the Tor client has fetched the bridge's server descriptor at least
once, he should remember the identity key fingerprint for that bridge
relay. Thus if the bridge relay moves to a new IP address, the client
can then query the bridge directory authority to look up a fresh server
descriptor using this fingerprint.

So we've shown that it's \emph{possible} to bootstrap into the network
just by learning the IP address and port of a bridge, but are there
situations where it's more convenient or more secure to learn its
identity fingerprint at the beginning too? We discuss that question
more in Section~\ref{sec:bootstrapping}, but first we introduce more
security topics.

\subsection{Scanning-resistance}

If it's trivial to verify that we're a bridge, and we run on a predictable
port, then it's conceivable our attacker would scan the whole Internet
looking for bridges. (In fact, he can just scan likely networks like
cablemodem and DSL services -- see Section~\ref{block-cable} for a related
attack.) It would be nice to slow down this attack. It would
be even nicer to make it hard to learn whether we're a bridge without
first knowing some secret.

\subsection{Password protecting the bridges}

Could provide a password to the bridge user. He provides a nonced hash of
it or something when he connects. We'd need to give him an ID key for the
bridge too, and wait to present the password until we've TLSed, else the
adversary can pretend to be the bridge and MITM him to learn the password.




\subsection{Observers can tell who is publishing and who is reading}
\label{subsec:upload-padding}

Should bridge users sometimes send bursts of long-range drop cells?


\subsection{Anonymity effects from becoming a bridge relay}

Against some attacks, becoming a bridge relay can improve anonymity. The
simplest example is an attacker who owns a small number of Tor servers. He
will see a connection from the bridge, but he won't be able to know
whether the connection originated there or was relayed from somebody else.

There are some cases where it doesn't seem to help: if an attacker can
watch all of the bridge's incoming and outgoing traffic, then it's easy
to learn which connections were relayed and which started there. (In this
case he still doesn't know the final destinations unless he is watching
them too, but in this case bridges are no better off than if they were
an ordinary client.)

There are also some potential downsides to running a bridge. First, while
we try to make it hard to enumerate all bridges, it's still possible to
learn about some of them, and for some people just the fact that they're
running one might signal to an attacker that they place a high value
on their anonymity. Second, there are some more esoteric attacks on Tor
relays that are not as well-understood or well-tested -- for example, an
attacker may be able to ``observe'' whether the bridge is sending traffic
even if he can't actually watch its network, by relaying traffic through
it and noticing changes in traffic timing~\cite{attack-tor-oak05}. On
the other hand, it may be that limiting the bandwidth the bridge is
willing to relay will allow this sort of attacker to determine if it's
being used as a bridge but not whether it is adding traffic of its own.

It is an open research question whether the benefits outweigh the risks. A
lot of the decision rests on which the attacks users are most worried
about. For most users, we don't think running a bridge relay will be
that damaging.

\subsection{Trusting local hardware: Internet cafes and LiveCDs}
\label{subsec:cafes-and-livecds}

Assuming that users have their own trusted hardware is not
always reasonable.

For Internet cafe Windows computers that let you attach your own USB key,
a USB-based Tor image would be smart. There's Torpark, and hopefully
there will be more options down the road. Worries about hardware or
software keyloggers and other spyware -- and physical surveillance.

If the system lets you boot from a CD or from a USB key, you can gain
a bit more security by bringing a privacy LiveCD with you. Hardware
keyloggers and physical surveillance still a worry. LiveCDs also useful
if it's your own hardware, since it's easier to avoid leaving breadcrumbs
everywhere.

\subsection{Forward compatibility and retiring bridge authorities}

Eventually we'll want to change the identity key and/or location
of a bridge authority. How do we do this mostly cleanly?


\section{Performance improvements}

\subsection{Fetch server descriptors just-in-time}

I guess we should encourage most places to do this, so blocked
users don't stand out.

\section{Other issues}

\subsection{How many bridge relays should you know about?}

If they're ordinary Tor users on cable modem or DSL, many of them will
disappear and/or move periodically. How many bridge relays should a
blockee know
about before he's likely to have at least one reachable at any given point?
How do we factor in a parameter for "speed that his bridges get discovered
and blocked"?

The related question is: if the bridge relays change IP addresses
periodically, how often does the bridge user need to "check in" in order
to keep from being cut out of the loop?

\subsection{How do we know if a bridge relay has been blocked?}

We need some mechanism for testing reachability from inside the
blocked area.

The easiest answer is for certain users inside the area to sign up as
testing relays, and then we can route through them and see if it works.

First problem is that different network areas block different net masks,
and it will likely be hard to know which users are in which areas. So
if a bridge relay isn't reachable, is that because of a network block
somewhere, because of a problem at the bridge relay, or just a temporary
outage?

Second problem is that if we pick random users to test random relays, the
adversary should sign up users on the inside, and enumerate the relays
we test. But it seems dangerous to just let people come forward and
declare that things are blocked for them, since they could be tricking
us. (This matters even moreso if our reputation system above relies on
whether things get blocked to punish or reward.)

Another answer is not to measure directly, but rather let the bridges
report whether they're being used. If they periodically report to their
bridge directory authority how much use they're seeing, the authority
can make smart decisions from there.

If they install a geoip database, they can periodically report to their
bridge directory authority which countries they're seeing use from. This
might help us to track which countries are making use of Ramp, and can
also let us learn about new steps the adversary has taken in the arms
race. (If the bridges don't want to install a whole geoip subsystem, they
can report samples of the /24 network for their users, and the authorities
can do the geoip work. This tradeoff has clear downsides though.)

Worry: adversary signs up a bunch of already-blocked bridges. If we're
stingy giving out bridges, users in that country won't get useful ones.
(Worse, we'll blame the users when the bridges report they're not
being used?)

Worry: the adversary could choose not to block bridges but just record
connections to them. So be it, I guess.

\subsection{How to learn how well the whole idea is working}

We need some feedback mechanism to learn how much use the bridge network
as a whole is actually seeing. Part of the reason for this is so we can
respond and adapt the design; part is because the funders expect to see
progress reports.

The above geoip-based approach to detecting blocked bridges gives us a
solution though.

\subsection{Cablemodem users don't provide important websites}
\label{subsec:block-cable}

...so our adversary could just block all DSL and cablemodem networks,
and for the most part only our bridge relays would be affected.

The first answer is to aim to get volunteers both from traditionally
``consumer'' networks and also from traditionally ``producer'' networks.

The second answer (not so good) would be to encourage more use of consumer
networks for popular and useful websites.

Other attack: China pressures Verizon to discourage its users from
running bridges.

\subsection{The trust chain}
\label{subsec:trust-chain}

Tor's ``public key infrastructure'' provides a chain of trust to
let users verify that they're actually talking to the right servers.
There are four pieces to this trust chain.

Firstly, when Tor clients are establishing circuits, at each step
they demand that the next Tor server in the path prove knowledge of
its private key~\cite{tor-design}. This step prevents the first node
in the path from just spoofing the rest of the path. Secondly, the
Tor directory authorities provide a signed list of servers along with
their public keys --- so unless the adversary can control a threshold
of directory authorities, he can't trick the Tor client into using other
Tor servers. Thirdly, the location and keys of the directory authorities,
in turn, is hard-coded in the Tor source code --- so as long as the user
got a genuine version of Tor, he can know that he is using the genuine
Tor network. And lastly, the source code and other packages are signed
with the GPG keys of the Tor developers, so users can confirm that they
did in fact download a genuine version of Tor.

But how can a user in an oppressed country know that he has the correct
key fingerprints for the developers? As with other security systems, it
ultimately comes down to human interaction. The keys are signed by dozens
of people around the world, and we have to hope that our users have met
enough people in the PGP web of trust~\cite{pgp-wot} that they can learn
the correct keys. For users that aren't connected to the global security
community, though, this question remains a critical weakness.

% XXX make clearer the trust chain step for bridge directory authorities

\subsection{How to motivate people to run bridge relays}

One of the traditional ways to get people to run software that benefits
others is to give them motivation to install it themselves.  An often
suggested approach is to install it as a stunning screensaver so everybody
will be pleased to run it. We take a similar approach here, by leveraging
the fact that these users are already interested in protecting their
own Internet traffic, so they will install and run the software.

Make all Tor users become bridges if they're reachable -- needs more work
on usability first, but we're making progress.

Also, we can make a snazzy network graph with Vidalia that emphasizes
the connections the bridge user is currently relaying. (Minor anonymity
implications, but hey.) (In many cases there won't be much activity,
so this may backfire. Or it may be better suited to full-fledged Tor
servers.)

\subsection{What if the clients can't install software?}

Bridge users without Tor clients

Bridge relays could always open their socks proxy. This is bad though,
firstly
because they learn the bridge users' destinations, and secondly because
we've learned that open socks proxies tend to attract abusive users who
have no idea they're using Tor.

Bridges could require passwords in the socks handshake (not supported
by most software including Firefox). Or they could run web proxies
that require authentication and then pass the requests into Tor. This
approach is probably a good way to help bootstrap the Psiphon network,
if one of its barriers to deployment is a lack of volunteers willing
to exit directly to websites. But it clearly drops some of the nice
anonymity features Tor provides.

\section{Future designs}

\subsection{Bridges inside the blocked network too}

Assuming actually crossing the firewall is the risky part of the
operation, can we have some bridge relays inside the blocked area too,
and more established users can use them as relays so they don't need to
communicate over the firewall directly at all? A simple example here is
to make new blocked users into internal bridges also -- so they sign up
on the BDA as part of doing their query, and we give out their addresses
rather than (or along with) the external bridge addresses. This design
is a lot trickier because it brings in the complexity of whether the
internal bridges will remain available, can maintain reachability with
the outside world, etc.

Hidden services as bridges. Hidden services as bridge directory authorities.

\bibliographystyle{plain} \bibliography{tor-design}

\end{document}


