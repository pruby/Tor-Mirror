\documentclass{llncs}

\usepackage{url}
\usepackage{amsmath}
\usepackage{epsfig}

%\setlength{\textwidth}{5.9in}
%\setlength{\textheight}{8.4in}
%\setlength{\topmargin}{.5cm}
%\setlength{\oddsidemargin}{1cm}
%\setlength{\evensidemargin}{1cm}

\newenvironment{tightlist}{\begin{list}{$\bullet$}{
  \setlength{\itemsep}{0mm}
    \setlength{\parsep}{0mm}
    %  \setlength{\labelsep}{0mm}
    %  \setlength{\labelwidth}{0mm}
    %  \setlength{\topsep}{0mm}
    }}{\end{list}}

\begin{document}

\title{Design of a blocking-resistant anonymity system}

%\author{Roger Dingledine\inst{1} \and Nick Mathewson\inst{1}}
\author{Roger Dingledine \and Nick Mathewson}
\institute{The Free Haven Project\\
\email{\{arma,nickm\}@freehaven.net}}

\maketitle
\pagestyle{plain}

\begin{abstract}

Websites around the world are increasingly being blocked by
government-level firewalls. Many people use anonymizing networks like
Tor to contact sites without letting an attacker trace their activities,
and as an added benefit they are no longer affected by local censorship.
But if the attacker simply denies access to the Tor network itself,
blocked users can no longer benefit from the security Tor offers.

Here we describe a design that builds upon the current Tor network
to provide an anonymizing network that resists blocking
by government-level attackers.

\end{abstract}

\section{Introduction and Goals}

Anonymizing networks such as Tor~\cite{tor-design} bounce traffic around
a network of relays. They aim to hide not only what is being said, but
also who is communicating with whom, which users are using which websites,
and so on. These systems have a broad range of users, including ordinary
citizens who want to avoid being profiled for targeted advertisements,
corporations who don't want to reveal information to their competitors,
and law enforcement and government intelligence agencies who need to do
operations on the Internet without being noticed.

Historically, research on anonymizing systems has assumed a passive
attacker who monitors the user (call her Alice) and tries to discover her
activities, yet lets her reach any piece of the network. In more modern
threat models such as Tor's, the adversary is allowed to perform active
attacks such as modifying communications in hopes of tricking Alice
into revealing her destination, or intercepting some of her connections
to run a man-in-the-middle attack. But these systems still assume that
Alice can eventually reach the anonymizing network.

An increasing number of users are making use of the Tor software
not so much for its anonymity properties but for its censorship
resistance properties -- if they access Internet sites like Wikipedia
and Blogspot via Tor, they are no longer affected by local censorship
and firewall rules. In fact, an informal user study (described in
Appendix~\ref{app:geoip}) showed China as the third largest user base
for Tor clients, with perhaps ten thousand people accessing the Tor
network from China each day.

The current Tor design is easy to block if the attacker controls Alice's
connection to the Tor network --- by blocking the directory authorities,
by blocking all the server IP addresses in the directory, or by filtering
based on the signature of the Tor TLS handshake. Here we describe a
design that builds upon the current Tor network to provide an anonymizing
network that also resists this blocking. Specifically,
Section~\ref{sec:adversary} discusses our threat model --- that is,
the assumptions we make about our adversary; Section~\ref{sec:current-tor}
describes the components of the current Tor design and how they can be
leveraged for a new blocking-resistant design; Section~\ref{sec:related}
explains the features and drawbacks of the currently deployed solutions;
and ...

%And adding more different classes of users and goals to the Tor network
%improves the anonymity for all Tor users~\cite{econymics,usability:weis2006}.

\section{Adversary assumptions}
\label{sec:adversary}

The history of blocking-resistance designs is littered with conflicting
assumptions about what adversaries to expect and what problems are
in the critical path to a solution. Here we try to enumerate our best
understanding of the current situation around the world.

In the traditional security style, we aim to describe a strong attacker
--- if we can defend against this attacker, we inherit protection
against weaker attackers as well. After all, we want a general design
that will work for people in China, people in Iran, people in Thailand,
whistleblowers in firewalled corporate networks, and people in whatever
turns out to be the next oppressive situation. In fact, by designing with
a variety of adversaries in mind, we can take advantage of the fact that
adversaries will be in different stages of the arms race at each location.

We assume there are three main network attacks in use by censors
currently~\cite{clayton:pet2006}:

\begin{tightlist}
\item Block destination by automatically searching for certain strings
in TCP packets.
\item Block destination by manually listing its IP address at the
firewall.
\item Intercept DNS requests and give bogus responses for certain
destination hostnames.
\end{tightlist}

We assume the network firewall has very limited CPU per
connection~\cite{clayton:pet2006}. Against an adversary who spends
hours looking through the contents of each packet, we would need
some stronger mechanism such as steganography, which introduces its
own problems~\cite{active-wardens,tcpstego,bar}.

More broadly, we assume that the chance that the authorities try to
block a given system grows as its popularity grows. That is, a system
used by only a few users will probably never be blocked, whereas a
well-publicized system with many users will receive much more scrutiny.

We assume that readers of blocked content are not in as much danger
as publishers. So far in places like China, the authorities mainly go
after people who publish materials and coordinate organized movements
against the state~\cite{mackinnon}. If they find that a user happens
to be reading a site that should be blocked, the typical response is
simply to block the site. Of course, even with an encrypted connection,
the adversary may be able to distinguish readers from publishers by
observing whether Alice is mostly downloading bytes or mostly uploading
them --- we discuss this issue more in Section~\ref{subsec:upload-padding}.

We assume that while various different regimes can coordinate and share
notes, there will be a significant time lag between one attacker learning
how to overcome a facet of our design and other attackers picking it up.
Similarly, we assume that in the early stages of deployment the insider
threat isn't as high of a risk, because no attackers have put serious
effort into breaking the system yet.

We assume that government-level attackers are not always uniform across
the country. For example, there is no single centralized place in China
that coordinates its censorship decisions and steps.

We assume that our users have control over their hardware and
software --- they don't have any spyware installed, there are no
cameras watching their screen, etc. Unfortunately, in many situations
these threats are very real~\cite{zuckerman-threatmodels}; yet
software-based security systems like ours are poorly equipped to handle
a user who is entirely observed and controlled by the adversary. See
Section~\ref{subsec:cafes-and-livecds} for more discussion of what little
we can do about this issue.

We assume that widespread access to the Internet is economically and/or
socially valuable in each deployment country. After all, if censorship
is more important than Internet access, the firewall administrators have
an easy job: they should simply block everything. The corollary to this
assumption is that we should design so that increased blocking of our
system results in increased economic damage or public outcry.

We assume that the user will be able to fetch a genuine
version of Tor, rather than one supplied by the adversary; see
Section~\ref{subsec:trust-chain} for discussion on helping the user
confirm that he has a genuine version and that he can connect to the
real Tor network.

\section{Components of the current Tor design}
\label{sec:current-tor}

Tor is popular and sees a lot of use. It's the largest anonymity
network of its kind.
Tor has attracted more than 800 routers from around the world.
A few sentences about how Tor works.
In this section, we examine some of the reasons why Tor has taken off,
with particular emphasis to how we can take advantage of these properties
for a blocking-resistance design.

Tor aims to provide three security properties:
\begin{tightlist}
\item 1. A local network attacker can't learn, or influence, your
destination.
\item 2. No single router in the Tor network can link you to your
destination.
\item 3. The destination, or somebody watching the destination,
can't learn your location.
\end{tightlist}

For blocking-resistance, we care most clearly about the first
property. But as the arms race progresses, the second property
will become important --- for example, to discourage an adversary
from volunteering a relay in order to learn that Alice is reading
or posting to certain websites. The third property is not so clearly
important in this context, but we believe it will turn out to be helpful:
consider websites and other Internet services that have been pressured
recently into treating clients differently depending on their network
location~\cite{google-geolocation}.
% and cite{goodell-syverson06} once it's finalized.

The Tor design provides other features as well over manual or ad
hoc circumvention techniques.

Firstly, the Tor directory authorities automatically aggregate, test,
and publish signed summaries of the available Tor routers. Tor clients
can fetch these summaries to learn which routers are available and
which routers have desired properties. Directory information is cached
throughout the Tor network, so once clients have bootstrapped they never
need to interact with the authorities directly. (To tolerate a minority
of compromised directory authorities, we use a threshold trust scheme ---
see Section~\ref{subsec:trust-chain} for details.)

Secondly, Tor clients can be configured to use any directory authorities
they want. They use the default authorities if no others are specified,
but it's easy to start a separate (or even overlapping) Tor network just
by running a different set of authorities and convincing users to prefer
a modified client. For example, we could launch a distinct Tor network
inside China; some users could even use an aggregate network made up of
both the main network and the China network. But we should not be too
quick to create other Tor networks --- part of Tor's anonymity comes from
users behaving like other users, and there are many unsolved anonymity
questions if different users know about different pieces of the network.

Thirdly, in addition to automatically learning from the chosen directories
which Tor routers are available and working, Tor takes care of building
paths through the network and rebuilding them as needed. So the user
never has to know how paths are chosen, never has to manually pick
working proxies, and so on. More generally, at its core the Tor protocol
is simply a tool that can build paths given a set of routers. Tor is
quite flexible about how it learns about the routers and how it chooses
the paths. Harvard's Blossom project~\cite{blossom-thesis} makes this
flexibility more concrete: Blossom makes use of Tor not for its security
properties but for its reachability properties. It runs a separate set
of directory authorities, its own set of Tor routers (called the Blossom
network), and uses Tor's flexible path-building to let users view Internet
resources from any point in the Blossom network.

Fourthly, Tor separates the role of \emph{internal relay} from the
role of \emph{exit relay}. That is, some volunteers choose just to relay
traffic between Tor users and Tor routers, and others choose to also allow
connections to external Internet resources. Because we don't force all
volunteers to play both roles, we end up with more relays. This increased
diversity in turn is what gives Tor its security: the more options the
user has for her first hop, and the more options she has for her last hop,
the less likely it is that a given attacker will be watching both ends
of her circuit~\cite{tor-design}. As a bonus, because our design attracts
more internal relays that want to help out but don't want to deal with
being an exit relay, we end up with more options for the first hop ---
the one most critical to being able to reach the Tor network.

Fifthly, Tor is sustainable. Zero-Knowledge Systems offered the commercial
but now-defunct Freedom Network~\cite{freedom21-security}, a design with
security comparable to Tor's, but its funding model relied on collecting
money from users to pay relays. Modern commercial proxy systems similarly
need to keep collecting money to support their infrastructure. On the
other hand, Tor has built a self-sustaining community of volunteers who
donate their time and resources. This community trust is rooted in Tor's
open design: we tell the world exactly how Tor works, and we provide all
the source code. Users can decide for themselves, or pay any security
expert to decide, whether it is safe to use. Further, Tor's modularity
as described above, along with its open license, mean that its impact
will continue to grow.

Sixthly, Tor has an established user base of hundreds of
thousands of people from around the world. This diversity of
users contributes to sustainability as above: Tor is used by
ordinary citizens, activists, corporations, law enforcement, and
even governments and militaries~\cite{tor-use-cases}, and they can
only achieve their security goals by blending together in the same
network~\cite{econymics,usability:weis2006}. This user base also provides
something else: hundreds of thousands of different and often-changing
addresses that we can leverage for our blocking-resistance design.

We discuss and adapt these components further in
Section~\ref{sec:bridges}. But first we examine the strengths and
weaknesses of other blocking-resistance approaches, so we can expand
our repertoire of building blocks and ideas.

\section{Current proxy solutions}
\label{sec:related}

Relay-based blocking-resistance schemes generally have two main
components: a relay component and a discovery component. The relay part
encompasses the process of establishing a connection, sending traffic
back and forth, and so on --- everything that's done once the user knows
where he's going to connect. Discovery is the step before that: the
process of finding one or more usable relays.

For example, we described several pieces of Tor in the previous section,
but we can divide them into the process of building paths and sending
traffic over them (relay) and the process of learning from the directory
servers about what routers are available (discovery). With this distinction
in mind, we now examine several categories of relay-based schemes.

\subsection{Centrally-controlled shared proxies}

Existing commercial anonymity solutions (like Anonymizer.com) are based
on a set of single-hop proxies. In these systems, each user connects to
a single proxy, which then relays the user's traffic. These public proxy
systems are typically characterized by two features: they control and
operate the proxies centrally, and many different users get assigned
to each proxy.

In terms of the relay component, single proxies provide weak security
compared to systems that distribute trust over multiple relays, since a
compromised proxy can trivially observe all of its users' actions, and
an eavesdropper only needs to watch a single proxy to perform timing
correlation attacks against all its users' traffic. Worse, all users
need to trust the proxy company to have good security itself as well as
to not reveal user activities.

On the other hand, single-hop proxies are easier to deploy, and they
can provide better performance than distributed-trust designs like Tor,
since traffic only goes through one relay. They're also more convenient
from the user's perspective --- since users entirely trust the proxy,
they can just use their web browser directly.

Whether public proxy schemes are more or less scalable than Tor is
still up for debate: commercial anonymity systems can use some of their
revenue to provision more bandwidth as they grow, whereas volunteer-based
anonymity systems can attract thousands of fast relays to spread the load.

The discovery piece can take several forms. Most commercial anonymous
proxies have one or a handful of commonly known websites, and their users
log in to those websites and relay their traffic through them. When
these websites get blocked (generally soon after the company becomes
popular), if the company cares about users in the blocked areas, they
start renting lots of disparate IP addresses and rotating through them
as they get blocked. They notify their users of new addresses by email,
for example. It's an arms race, since attackers can sign up to receive the
email too, but they have one nice trick available to them: because they
have a list of paying subscribers, they can notify certain subscribers
about updates earlier than others.

Access control systems on the proxy let them provide service only to
users with certain characteristics, such as paying customers or people
from certain IP address ranges.

Discovery in the face of a government-level firewall is a complex and
unsolved
topic, and we're stuck in this same arms race ourselves; we explore it
in more detail in Section~\ref{sec:discovery}. But first we examine the
other end of the spectrum --- getting volunteers to run the proxies,
and telling only a few people about each proxy.

\subsection{Independent personal proxies}

Personal proxies such as Circumventor~\cite{circumventor} and
CGIProxy~\cite{cgiproxy} use the same technology as the public ones as
far as the relay component goes, but they use a different strategy for
discovery. Rather than managing a few centralized proxies and constantly
getting new addresses for them as the old addresses are blocked, they
aim to have a large number of entirely independent proxies, each managing
its own (much smaller) set of users.

As the Circumventor site~\cite{circumventor} explains, ``You don't
actually install the Circumventor \emph{on} the computer that is blocked
from accessing Web sites. You, or a friend of yours, has to install the
Circumventor on some \emph{other} machine which is not censored.''

This tactic has great advantages in terms of blocking-resistance ---
recall our assumption in Section~\ref{sec:adversary} that the attention
a system attracts from the attacker is proportional to its number of
users and level of publicity. If each proxy only has a few users, and
there is no central list of proxies, most of them will never get noticed.

On the other hand, there's a huge scalability question that so far has
prevented these schemes from being widely useful: how does the fellow
in China find a person in Ohio who will run a Circumventor for him? In
some cases he may know and trust some people on the outside, but in many
cases he's just out of luck. Just as hard, how does a new volunteer in
Ohio find a person in China who needs it?

%discovery is also hard because the hosts keep vanishing if they're
%on dynamic ip. But not so bad, since they can use dyndns addresses.

This challenge leads to a hybrid design --- centrally-distributed
personal proxies --- which we will investigate in more detail in
Section~\ref{sec:discovery}.

\subsection{Open proxies}

Yet another currently used approach to bypassing firewalls is to locate
open and misconfigured proxies on the Internet. A quick Google search
for ``open proxy list'' yields a wide variety of freely available lists
of HTTP, HTTPS, and SOCKS proxies. Many small companies have sprung up
providing more refined lists to paying customers.

There are some downsides to using these oen proxies though. Firstly,
the proxies are of widely varying quality in terms of bandwidth and
stability, and many of them are entirely unreachable. Secondly, unlike
networks of volunteers like Tor, the legality of routing traffic through
these proxies is questionable: it's widely believed that most of them
don't realize what they're offering, and probably wouldn't allow it if
they realized. Thirdly, in many cases the connection to the proxy is
unencrypted, so firewalls that filter based on keywords in IP packets
will not be hindered. And lastly, many users are suspicious that some
open proxies are a little \emph{too} convenient: are they run by the
adversary, in which case they get to monitor all the user's requests
just as single-hop proxies can?

A distributed-trust design like Tor resolves each of these issues for
the relay component, but a constantly changing set of thousands of open
relays is clearly a useful idea for a discovery component. For example,
users might be able to make use of these proxies to bootstrap their
first introduction into the Tor network.

\subsection{JAP}

Stefan's WPES paper~\cite{koepsell:wpes2004} is probably the closest
related work, and is
the starting point for the design in this paper.

\subsection{steganography}

infranet

\subsection{break your sensitive strings into multiple tcp packets;
ignore RSTs}

\subsection{Internal caching networks}

Freenet is deployed inside China and caches outside content.

\subsection{Skype}

port-hopping. encryption. voice communications not so susceptible to
keystroke loggers (even graphical ones).


\subsection{Tor itself}

And lastly, we include Tor itself in the list of current solutions
to firewalls. Tens of thousands of people use Tor from countries that
routinely filter their Internet. Tor's website has been blocked in most
of them. But why hasn't the Tor network been blocked yet?

We have several theories. The first is the most straightforward: tens of
thousands of people are simply too few to matter. It may help that Tor is
perceived to be for experts only, and thus not worth attention yet. The
more subtle variant on this theory is that we've positioned Tor in the
public eye as a tool for retaining civil liberties in more free countries,
so perhaps blocking authorities don't view it as a threat. (We revisit
this idea when we consider whether and how to publicize a Tor variant
that improves blocking-resistance --- see Section~\ref{subsec:publicity}
for more discussion.)

The broader explanation is that most government-level filters are not
created by people setting out to block all possible ways to bypass
them. They're created by people who want to do a good enough job that
they can still appear in control. They realize that there will always
be ways for a few people to get around the firewall, and as long as Tor
has not publically threatened their control, they see no urgent need to
block it yet.

We should recognize that we're \emph{already} in the arms race. These
constraints can give us insight into the priorities and capabilities of
our various attackers.

\section{The relay component of our blocking-resistant design}
\label{sec:bridges}

Section~\ref{sec:current-tor} describes many reasons why Tor is
well-suited as a building block in our context, but several changes will
allow the design to resist blocking better. The most critical changes are
to get more relay addresses, and to distribute them to users differently.

%We need to address three problems:
%- adapting the relay component of Tor so it resists blocking better.
%- Discovery.
%- Tor's network signature.

%Here we describe the new pieces we need to add to the current Tor design.

\subsection{Bridge relays}

Hundreds of thousands of people around the world use Tor. We can leverage
our already self-selected user base to produce a list of thousands of
often-changing IP addresses. Specifically, we can give them a little
button in the GUI that says ``Tor for Freedom'', and users who click
the button will turn into \emph{bridge relays}, or just \emph{bridges}
for short. They can rate limit relayed connections to 10 KB/s (almost
nothing for a broadband user in a free country, but plenty for a user
who otherwise has no access at all), and since they are just relaying
bytes back and forth between blocked users and the main Tor network, they
won't need to make any external connections to Internet sites. Because
of this separation of roles, and because we're making use of software
that the volunteers have already installed for their own use, we expect
our scheme to attract and maintain more volunteers than previous schemes.

As usual, there are new anonymity and security implications from running a
bridge relay, particularly from letting people relay traffic through your
Tor client; but we leave this discussion for Section~\ref{sec:security}.

%...need to outline instructions for a Tor config that will publish
%to an alternate directory authority, and for controller commands
%that will do this cleanly.

\subsection{The bridge directory authority}

How do the bridge relays advertise their existence to the world? We
introduce a second new component of the design: a specialized directory
authority that aggregates and tracks bridges. Bridge relays periodically
publish server descriptors (summaries of their keys, locations, etc,
signed by their long-term identity key), just like the relays in the
``main'' Tor network, but in this case they publish them only to the
bridge directory authorities.

The main difference between bridge authorities and the directory
authorities for the main Tor network is that the main authorities provide
out a list of every known relay, but the bridge authorities only give
out a server descriptor if you already know its identity key. That is,
you can keep up-to-date on a bridge's location and other information
once you know about it, but you can't just grab a list of all the bridges.

The identity keys, IP address, and directory port for the bridge
authorities ship by default with the Tor software, so the bridge relays
can be confident they're publishing to the right location, and the
blocked users can establish an encrypted authenticated channel. See
Section~\ref{subsec:trust-chain} for more discussion of the public key
infrastructure and trust chain.

Bridges use Tor to publish their descriptors privately and securely,
so even an attacker monitoring the bridge directory authority's network
can't make a list of all the addresses contacting the authority and
track them that way.

%\subsection{A simple matter of engineering}
%
%Although we've described bridges and bridge authorities in simple terms
%above, some design modifications and features are needed in the Tor
%codebase to add them. We describe the four main changes here.
%
%Firstly, we need to get smarter about rate limiting:
%Bandwidth classes
%
%Secondly, while users can in fact configure which directory authorities
%they use, we need to add a new type of directory authority and teach
%bridges to fetch directory information from the main authorities while
%publishing server descriptors to the bridge authorities. We're most of
%the way there, since we can already specify attributes for directory
%authorities:
%add a separate flag named ``blocking''.
%
%Thirdly, need to build paths using bridges as the first
%hop. One more hole in the non-clique assumption.
%
%Lastly, since bridge authorities don't answer full network statuses,
%we need to add a new way for users to learn the current status for a
%single relay or a small set of relays --- to answer such questions as
%``is it running?'' or ``is it behaving correctly?'' We describe in
%Section~\ref{subsec:enclave-dirs} a way for the bridge authority to
%publish this information without resorting to signing each answer
%individually.

\subsection{Putting them together}
\label{subsec:relay-together}

If a blocked user knows the identity keys of a set of bridge relays, and
he has correct address information for at least one of them, he can use
that one to make a secure connection to the bridge authority and update
his knowledge about the other bridge relays. He can also use it to make
secure connections to the main Tor network and directory servers, so he
can build circuits and connect to the rest of the Internet. All of these
updates happen in the background: from the blocked user's perspective,
he just accesses the Internet via his Tor client like always.

So now we've reduced the problem from how to circumvent the firewall
for all transactions (and how to know that the pages you get have not
been modified by the local attacker) to how to learn about a working
bridge relay.

There's another catch though. We need to make sure that the network
traffic we generate by simply connecting to a bridge relay doesn't stand
out too much.

%The following section describes ways to bootstrap knowledge of your first
%bridge relay, and ways to maintain connectivity once you know a few
%bridge relays.

% (See Section~\ref{subsec:first-bridge} for a discussion
%of exactly what information is sufficient to characterize a bridge relay.)

\section{Hiding Tor's network signatures}
\label{sec:network-signature}
\label{subsec:enclave-dirs}

Currently, Tor uses two protocols for its network communications. The
main protocol uses TLS for encrypted and authenticated communication
between Tor instances. The second protocol is standard HTTP, used for
fetching directory information. All Tor servers listen on their ``ORPort''
for TLS connections, and some of them opt to listen on their ``DirPort''
as well, to serve directory information. Tor servers choose whatever port
numbers they like; the server descriptor they publish to the directory
tells users where to connect.

One format for communicating address information about a bridge relay is
its IP address and DirPort. From there, the user can ask the bridge's
directory cache for an up-to-date copy of its server descriptor, and
learn its current circuit keys, its ORPort, and so on.

However, connecting directly to the directory cache involves a plaintext
HTTP request. A censor could create a network signature for the request
and/or its response, thus preventing these connections. To resolve this
vulnerability, we've modified the Tor protocol so that users can connect
to the directory cache via the main Tor port --- they establish a TLS
connection with the bridge as normal, and then send a special ``begindir''
relay command to establish an internal connection to its directory cache.

Therefore a better way to summarize a bridge's address is by its IP
address and ORPort, so all communications between the client and the
bridge will use ordinary TLS. But there are other details that need
more investigation.

What port should bridges pick for their ORPort? We currently recommend
that they listen on port 443 (the default HTTPS port) if they want to
be most useful, because clients behind standard firewalls will have
the best chance to reach them. Is this the best choice in all cases,
or should we encourage some fraction of them pick random ports, or other
ports commonly permitted through firewalls like 53 (DNS) or 110
(POP)? We need
more research on our potential users, and their current and anticipated
firewall restrictions.

Furthermore, we need to look at the specifics of Tor's TLS handshake.
Right now Tor uses some predictable strings in its TLS handshakes. For
example, it sets the X.509 organizationName field to ``Tor'', and it puts
the Tor server's nickname in the certificate's commonName field. We
should tweak the handshake protocol so it doesn't rely on any details
in the certificate headers, yet it remains secure. Should we replace
it with blank entries for each field, or should we research the common
values that Firefox and Internet Explorer use and try to imitate those?

Worse, Tor's TLS handshake involves sending two certificates in each
direction: one certificate contains the self-signed identity key for
the router, and the second contains the current link key, signed by the
identity key. We use these to authenticate that we're talking to the right
router, and also to establish perfect forward secrecy for that link.
How much will these extra certificates make Tor's TLS handshake stand
out? We have to work on normalizing our appearance not just in terms
of the fields used in each certificate, but also in the number of
certificates we present for each side.
% Nick, I need help with the above paragraph. What are the two certs
% for really, and how much work would it be to start acting like a normal
% browser? -RD

Lastly, what if the adversary starts observing the network traffic even
more closely? Even if our TLS handshake looks innocent, our traffic timing
and volume still look different than a user making a secure web connection
to his bank. The same techniques used in the growing trend to build tools
to recognize encrypted Bittorrent traffic~\cite{bt-traffic-shaping}
could be used to identify Tor communication and recognize bridge
relays. Rather than trying to look like encrypted web traffic, we may be
better off trying to blend with some other encrypted network protocol. The
first step is to compare typical network behavior for a Tor client to
typical network behavior for various other protocols. This statistical
cat-and-mouse game is made more complex by the fact that Tor transports a
variety of protocols, and we'll want to automatically handle web browsing
differently from, say, instant messaging.

\subsection{Identity keys as part of addressing information}

We have described a way for the blocked user to bootstrap into the
network once he knows the IP address and ORPort of a bridge. What about
local spoofing attacks? That is, since we never learned an identity
key fingerprint for the bridge, a local attacker could intercept our
connection and pretend to be the bridge we had in mind. It turns out
that giving false information isn't that bad --- since the Tor client
ships with trusted keys for the bridge directory authority and the Tor
network directory authorities, the user can learn whether he's being
given a real connection to the bridge authorities or not. (After all,
if the adversary intercepts every connection the user makes and gives
him a bad connection each time, there's nothing we can do.)

What about anonymity-breaking attacks from observing traffic, if the
blocked user doesn't start out knowing the identity key of his intended
bridge? The vulnerabilities aren't so bad in this case either ---
the adversary could do similar attacks just by monitoring the network
traffic.
% cue paper by steven and george

Once the Tor client has fetched the bridge's server descriptor, it should
remember the identity key fingerprint for that bridge relay. Thus if
the bridge relay moves to a new IP address, the client can query the
bridge directory authority to look up a fresh server descriptor using
this fingerprint.

So we've shown that it's \emph{possible} to bootstrap into the network
just by learning the IP address and ORPort of a bridge, but are there
situations where it's more convenient or more secure to learn the bridge's
identity fingerprint as well as instead, while bootstrapping? We keep
that question in mind as we next investigate bootstrapping and discovery.

\section{Discovering and maintaining working bridge relays}
\label{sec:discovery}

Tor's modular design means that we can develop a better relay component
independently of developing the discovery component. This modularity's
great promise is that we can pick any discovery approach we like; but the
unfortunate fact is that we have no magic bullet for discovery. We're
in the same arms race as all the other designs we described in
Section~\ref{sec:related}.

In this section we describe four approaches to adding discovery
components for our design, in order of increasing complexity. Note that
we can deploy all four schemes at once --- bridges and blocked users can
use the discovery approach that is most appropriate for their situation.

\subsection{Independent bridges, no central discovery}

The first design is simply to have no centralized discovery component at
all. Volunteers run bridges, and we assume they have some blocked users
in mind and communicate their address information to them out-of-band
(for example, through gmail). This design allows for small personal
bridges that have only one or a handful of users in mind, but it can
also support an entire community of users. For example, Citizen Lab's
upcoming Psiphon single-hop proxy tool~\cite{psiphon} plans to use this
\emph{social network} approach as its discovery component.

There are some variations on bootstrapping in this design. In the simple
case, the operator of the bridge informs each chosen user about his
bridge's address information and/or keys. Another approach involves
blocked users introducing new blocked users to the bridges they know.
That is, somebody in the blocked area can pass along a bridge's address to
somebody else they trust. This scheme brings in appealing but complex game
theory properties: the blocked user making the decision has an incentive
only to delegate to trustworthy people, since an adversary who learns
the bridge's address and filters it makes it unavailable for both of them.

Note that a central set of bridge directory authorities can still be
compatible with a decentralized discovery process. That is, how users
first learn about bridges is entirely up to the bridges, but the process
of fetching up-to-date descriptors for them can still proceed as described
in Section~\ref{sec:bridges}. Of course, creating a central place that
knows about all the bridges may not be smart, especially if every other
piece of the system is decentralized. Further, if a user only knows
about one bridge and he loses track of it, it may be quite a hassle to
reach the bridge authority. We address these concerns next.

\subsection{Families of bridges, no central discovery}

Because the blocked users are running our software too, we have many
opportunities to improve usability or robustness. Our second design builds
on the first by encouraging volunteers to run several bridges at once
(or coordinate with other bridge volunteers), such that some fraction
of the bridges are likely to be available at any given time.

The blocked user's Tor client could periodically fetch an updated set of
recommended bridges from any of the working bridges. Now the client can
learn new additions to the bridge pool, and can expire abandoned bridges
or bridges that the adversary has blocked, without the user ever needing
to care. To simplify maintenance of the community's bridge pool, rather
than mirroring all of the information at each bridge, each community
could instead run its own bridge directory authority (accessed via the
available bridges),

\subsection{Social networks with directory-side support}

In the above designs, 

- social network scheme, with accounts and stuff.



- public proxies. given out like circumventors. or all sorts of other rate limiting ways.




In the first subsection we describe how to find a first bridge.

Thus they can reach the BDA. From here we either assume a social
network or other mechanism for learning IP:dirport or key fingerprints
as above, or we assume an account server that allows us to limit the
number of new bridge relays an external attacker can discover.

Going to be an arms race. Need a bag of tricks. Hard to say
which ones will work. Don't spend them all at once.

\subsection{Bootstrapping: finding your first bridge}
\label{subsec:first-bridge}

Most government firewalls are not perfect. They allow connections to
Google cache or some open proxy servers, or they let file-sharing or
Skype or World-of-Warcraft connections through.
For users who can't use any of these techniques, hopefully they know
a friend who can --- for example, perhaps the friend already knows some
bridge relay addresses.
(If they can't get around it at all, then we can't help them --- they
should go meet more people.)

Some techniques are sufficient to get us an IP address and a port,
and others can get us IP:port:key. Lay out some plausible options
for how users can bootstrap into learning their first bridge.

Round one:

- the bridge authority server will hand some out.

- get one from your friend.

- send us mail with a unique account, and get an automated answer.

- 

Round two:

- social network thing

attack: adversary can reconstruct your social network by learning who
knows which bridges.

\subsection{Centrally-distributed personal proxies}

Circumventor, realizing that its adoption will remain limited if would-be
users can't connect with volunteers, has started a mailing list to
distribute new proxy addresses every few days. From experimentation
it seems they have concluded that sending updates every 3 or 4 days is
sufficient to stay ahead of the current attackers.

If there are many volunteer proxies and many interested users, a central
watering hole to connect them is a natural solution. On the other hand,
at first glance it appears that we've inherited the \emph{bad} parts of
each of the above designs: not only do we have to attract many volunteer
proxies, but the users also need to get to a single site that is sure
to be blocked.

There are two reasons why we're in better shape. Firstly, the users don't
actually need to reach the watering hole directly: it can respond to
email, for example. Secondly, 

% In fact, the JAP
%project~\cite{web-mix,koepsell:wpes2004} suggested an alternative approach
%to a mailing list: new users email a central address and get an automated
%response listing a proxy for them.
% While the exact details of the
%proposal are still to be worked out, the idea of giving out



\subsection{Discovery based on social networks}

A token that can be exchanged at the BDA (assuming you
can reach it) for a new IP:dirport or server descriptor.

The account server

runs as a Tor controller for the bridge authority

Users can establish reputations, perhaps based on social network
connectivity, perhaps based on not getting their bridge relays blocked,

Probably the most critical lesson learned in past work on reputation
systems in privacy-oriented environments~\cite{rep-anon} is the need for
verifiable transactions. That is, the entity computing and advertising
reputations for participants needs to actually learn in a convincing
way that a given transaction was successful or unsuccessful.

(Lesson from designing reputation systems~\cite{rep-anon}: easy to
reward good behavior, hard to punish bad behavior.

\subsection{How to allocate bridge addresses to users}

Hold a fraction in reserve, in case our currently deployed tricks
all fail at once --- so we can move to new approaches quickly.
(Bridges that sign up and don't get used yet will be sad; but this
is a transient problem --- if bridges are on by default, nobody will
mind not being used.)

Perhaps each bridge should be known by a single bridge directory
authority. This makes it easier to trace which users have learned about
it, so easier to blame or reward. It also makes things more brittle,
since loss of that authority means its bridges aren't advertised until
they switch, and means its bridge users are sad too.
(Need a slick hash algorithm that will map our identity key to a
bridge authority, in a way that's sticky even when we add bridge
directory authorities, but isn't sticky when our authority goes
away. Does this exist?)

Divide bridges into buckets based on their identity key.
[Design question: need an algorithm to deterministically map a bridge's
identity key into a category that isn't too gameable. Take a keyed
hash of the identity key plus a secret the bridge authority keeps?
An adversary signing up bridges won't easily be able to learn what
category he's been put in, so it's slow to attack.]

One portion of the bridges is the public bucket. If you ask the
bridge account server for a public bridge, it will give you a random
one of these. We expect they'll be the first to be blocked, but they'll
help the system bootstrap until it *does* get blocked, and remember that
we're dealing with different blocking regimes around the world that will
progress at different rates.

The generalization of the public bucket is a bucket based on the bridge
user's IP address: you can learn a random entry only from the subbucket
your IP address (actually, your /24) maps to.

Another portion of the bridges can be sectioned off to be given out in
a time-release basis. The bucket is partitioned into pieces which are
deterministically available only in certain time windows.

And of course another portion is made available for the social network
design above.

Captchas.

Is it useful to load balance which bridges are handed out? The above
bucket concept makes some bridges wildly popular and others less so.
But I guess that's the point.

\subsection{How do we know if a bridge relay has been blocked?}

We need some mechanism for testing reachability from inside the
blocked area.

The easiest answer is for certain users inside the area to sign up as
testing relays, and then we can route through them and see if it works.

First problem is that different network areas block different net masks,
and it will likely be hard to know which users are in which areas. So
if a bridge relay isn't reachable, is that because of a network block
somewhere, because of a problem at the bridge relay, or just a temporary
outage?

Second problem is that if we pick random users to test random relays, the
adversary should sign up users on the inside, and enumerate the relays
we test. But it seems dangerous to just let people come forward and
declare that things are blocked for them, since they could be tricking
us. (This matters even moreso if our reputation system above relies on
whether things get blocked to punish or reward.)

Another answer is not to measure directly, but rather let the bridges
report whether they're being used. If they periodically report to their
bridge directory authority how much use they're seeing, the authority
can make smart decisions from there.

If they install a geoip database, they can periodically report to their
bridge directory authority which countries they're seeing use from. This
might help us to track which countries are making use of Ramp, and can
also let us learn about new steps the adversary has taken in the arms
race. (If the bridges don't want to install a whole geoip subsystem, they
can report samples of the /24 network for their users, and the authorities
can do the geoip work. This tradeoff has clear downsides though.)

Worry: adversary signs up a bunch of already-blocked bridges. If we're
stingy giving out bridges, users in that country won't get useful ones.
(Worse, we'll blame the users when the bridges report they're not
being used?)

Worry: the adversary could choose not to block bridges but just record
connections to them. So be it, I guess.

\subsection{How to learn how well the whole idea is working}

We need some feedback mechanism to learn how much use the bridge network
as a whole is actually seeing. Part of the reason for this is so we can
respond and adapt the design; part is because the funders expect to see
progress reports.

The above geoip-based approach to detecting blocked bridges gives us a
solution though.


\section{Security considerations}
\label{sec:security}

\subsection{Observers can tell who is publishing and who is reading}
\label{subsec:upload-padding}

Should bridge users sometimes send bursts of long-range drop cells?

\subsection{Anonymity effects from acting as a bridge relay}

Against some attacks, relaying traffic for others can improve anonymity. The
simplest example is an attacker who owns a small number of Tor servers. He
will see a connection from the bridge, but he won't be able to know
whether the connection originated there or was relayed from somebody else.

There are some cases where it doesn't seem to help: if an attacker can
watch all of the bridge's incoming and outgoing traffic, then it's easy
to learn which connections were relayed and which started there. (In this
case he still doesn't know the final destinations unless he is watching
them too, but in this case bridges are no better off than if they were
an ordinary client.)

There are also some potential downsides to running a bridge. First, while
we try to make it hard to enumerate all bridges, it's still possible to
learn about some of them, and for some people just the fact that they're
running one might signal to an attacker that they place a high value
on their anonymity. Second, there are some more esoteric attacks on Tor
relays that are not as well-understood or well-tested --- for example, an
attacker may be able to ``observe'' whether the bridge is sending traffic
even if he can't actually watch its network, by relaying traffic through
it and noticing changes in traffic timing~\cite{attack-tor-oak05}. On
the other hand, it may be that limiting the bandwidth the bridge is
willing to relay will allow this sort of attacker to determine if it's
being used as a bridge but not whether it is adding traffic of its own.

It is an open research question whether the benefits outweigh the risks. A
lot of the decision rests on which attacks the users are most worried
about. For most users, we don't think running a bridge relay will be
that damaging.

\subsection{Trusting local hardware: Internet cafes and LiveCDs}
\label{subsec:cafes-and-livecds}

Assuming that users have their own trusted hardware is not
always reasonable.

For Internet cafe Windows computers that let you attach your own USB key,
a USB-based Tor image would be smart. There's Torpark, and hopefully
there will be more thoroughly analyzed options down the road. Worries
about hardware or
software keyloggers and other spyware --- and physical surveillance.

If the system lets you boot from a CD or from a USB key, you can gain
a bit more security by bringing a privacy LiveCD with you. Hardware
keyloggers and physical surveillance still a worry. LiveCDs also useful
if it's your own hardware, since it's easier to avoid leaving breadcrumbs
everywhere.

\subsection{Forward compatibility and retiring bridge authorities}

Eventually we'll want to change the identity key and/or location
of a bridge authority. How do we do this mostly cleanly?

\subsection{The trust chain}
\label{subsec:trust-chain}

Tor's ``public key infrastructure'' provides a chain of trust to
let users verify that they're actually talking to the right servers.
There are four pieces to this trust chain.

Firstly, when Tor clients are establishing circuits, at each step
they demand that the next Tor server in the path prove knowledge of
its private key~\cite{tor-design}. This step prevents the first node
in the path from just spoofing the rest of the path. Secondly, the
Tor directory authorities provide a signed list of servers along with
their public keys --- so unless the adversary can control a threshold
of directory authorities, he can't trick the Tor client into using other
Tor servers. Thirdly, the location and keys of the directory authorities,
in turn, is hard-coded in the Tor source code --- so as long as the user
got a genuine version of Tor, he can know that he is using the genuine
Tor network. And lastly, the source code and other packages are signed
with the GPG keys of the Tor developers, so users can confirm that they
did in fact download a genuine version of Tor.

But how can a user in an oppressed country know that he has the correct
key fingerprints for the developers? As with other security systems, it
ultimately comes down to human interaction. The keys are signed by dozens
of people around the world, and we have to hope that our users have met
enough people in the PGP web of trust~\cite{pgp-wot} that they can learn
the correct keys. For users that aren't connected to the global security
community, though, this question remains a critical weakness.

% XXX make clearer the trust chain step for bridge directory authorities

\subsection{Security through obscurity: publishing our design}

Many other schemes like dynaweb use the typical arms race strategy of
not publishing their plans. Our goal here is to produce a design ---
a framework --- that can be public and still secure. Where's the tradeoff?

\section{Performance improvements}
\label{sec:performance}

\subsection{Fetch server descriptors just-in-time}

I guess we should encourage most places to do this, so blocked
users don't stand out.


network-status and directory optimizations. caching better. partitioning
issues?

\section{Maintaining reachability}

\subsection{How many bridge relays should you know about?}

If they're ordinary Tor users on cable modem or DSL, many of them will
disappear and/or move periodically. How many bridge relays should a
blockee know
about before he's likely to have at least one reachable at any given point?
How do we factor in a parameter for "speed that his bridges get discovered
and blocked"?

The related question is: if the bridge relays change IP addresses
periodically, how often does the bridge user need to "check in" in order
to keep from being cut out of the loop?

\subsection{Cablemodem users don't provide important websites}
\label{subsec:block-cable}

...so our adversary could just block all DSL and cablemodem networks,
and for the most part only our bridge relays would be affected.

The first answer is to aim to get volunteers both from traditionally
``consumer'' networks and also from traditionally ``producer'' networks.

The second answer (not so good) would be to encourage more use of consumer
networks for popular and useful websites.

Other attack: China pressures Verizon to discourage its users from
running bridges.

\subsection{Scanning-resistance}

If it's trivial to verify that we're a bridge, and we run on a predictable
port, then it's conceivable our attacker would scan the whole Internet
looking for bridges. (In fact, he can just scan likely networks like
cablemodem and DSL services --- see Section~\ref{block-cable} for a related
attack.) It would be nice to slow down this attack. It would
be even nicer to make it hard to learn whether we're a bridge without
first knowing some secret.

Password protecting the bridges.
Could provide a password to the bridge user. He provides a nonced hash of
it or something when he connects. We'd need to give him an ID key for the
bridge too, and wait to present the password until we've TLSed, else the
adversary can pretend to be the bridge and MITM him to learn the password.

\subsection{How to motivate people to run bridge relays}

One of the traditional ways to get people to run software that benefits
others is to give them motivation to install it themselves.  An often
suggested approach is to install it as a stunning screensaver so everybody
will be pleased to run it. We take a similar approach here, by leveraging
the fact that these users are already interested in protecting their
own Internet traffic, so they will install and run the software.

Make all Tor users become bridges if they're reachable -- needs more work
on usability first, but we're making progress.

Also, we can make a snazzy network graph with Vidalia that emphasizes
the connections the bridge user is currently relaying. (Minor anonymity
implications, but hey.) (In many cases there won't be much activity,
so this may backfire. Or it may be better suited to full-fledged Tor
servers.)

\subsection{What if the clients can't install software?}

Bridge users without Tor clients

Bridge relays could always open their socks proxy. This is bad though,
firstly
because bridges learn the bridge users' destinations, and secondly because
we've learned that open socks proxies tend to attract abusive users who
have no idea they're using Tor.

Bridges could require passwords in the socks handshake (not supported
by most software including Firefox). Or they could run web proxies
that require authentication and then pass the requests into Tor. This
approach is probably a good way to help bootstrap the Psiphon network,
if one of its barriers to deployment is a lack of volunteers willing
to exit directly to websites. But it clearly drops some of the nice
anonymity and security features Tor provides.

\subsection{Publicity attracts attention}
\label{subsec:publicity}

Many people working on this field want to publicize the existence
and extent of censorship concurrently with the deployment of their
circumvention software. The easy reason for this two-pronged push is
to attract volunteers for running proxies in their systems; but in many
cases their main goal is not to build the software, but rather to educate
the world about the censorship. The media also tries to do its part by
broadcasting the existence of each new circumvention system.

But at the same time, this publicity attracts the attention of the
censors. We can slow down the arms race by not attracting as much
attention, and just spreading by word of mouth. If our goal is to
establish a solid social network of bridges and bridge users before
the adversary gets involved, does this attention tradeoff work to our
advantage?

\subsection{The Tor website: how to get the software}



\section{Future designs}

\subsection{Bridges inside the blocked network too}

Assuming actually crossing the firewall is the risky part of the
operation, can we have some bridge relays inside the blocked area too,
and more established users can use them as relays so they don't need to
communicate over the firewall directly at all? A simple example here is
to make new blocked users into internal bridges also -- so they sign up
on the BDA as part of doing their query, and we give out their addresses
rather than (or along with) the external bridge addresses. This design
is a lot trickier because it brings in the complexity of whether the
internal bridges will remain available, can maintain reachability with
the outside world, etc.

Hidden services as bridges. Hidden services as bridge directory authorities.

\section{Conclusion}

\bibliographystyle{plain} \bibliography{tor-design}

\appendix

\section{Counting Tor users by country}
\label{app:geoip}

\end{document}

ship geoip db to bridges. they look up users who tls to them in the db,
and upload a signed list of countries and number-of-users each day. the
bridge authority aggregates them and publishes stats.

bridge relays have buddies
they ask a user to test the reachability of their buddy.
leaks O(1) bridges, but not O(n).

we should not be blockable by ordinary cisco censorship features.
that is, if they want to block our new design, they will need to
add a feature to block exactly this.
strategically speaking, this may come in handy.

hash identity key + secret that bridge authority knows. start
out dividing into 2^n buckets, where n starts at 0, and we choose
which bucket you're in based on the first n bits of the hash.

Bridges come in clumps of 4 or 8 or whatever. If you know one bridge
in a clump, the authority will tell you the rest. Now bridges can
ask users to test reachability of their buddies.

Giving out clumps helps with dynamic IP addresses too. Whether it
should be 4 or 8 depends on our churn.

the account server. let's call it a database, it doesn't have to
be a thing that human interacts with.

rate limiting mechanisms:
energy spent. captchas. relaying traffic for others?
send us \$10, we'll give you an account

so how do we reward people for being good?

