\documentclass{llncs}

\usepackage{url}
\usepackage{amsmath}
\usepackage{epsfig}

\newenvironment{tightlist}{\begin{list}{$\bullet$}{
  \setlength{\itemsep}{0mm}
    \setlength{\parsep}{0mm}
    %  \setlength{\labelsep}{0mm}
    %  \setlength{\labelwidth}{0mm}
    %  \setlength{\topsep}{0mm}
    }}{\end{list}}

\begin{document}

\title{Challenges in practical low-latency stream anonymity (DRAFT)}

\author{Roger Dingledine and Nick Mathewson}
\institute{The Free Haven Project\\
\email{\{arma,nickm\}@freehaven.net}}

\maketitle
\pagestyle{empty}

\begin{abstract}
foo
\end{abstract}

\section{Introduction}

Tor is a low-latency anonymous communication overlay network designed
to be practical and usable for protecting TCP streams over the
Internet~\cite{tor-design}. We have been operating a publicly deployed
Tor network since October 2003 that has grown to over a hundred volunteer
nodes and carries over 70 megabits per second of average traffic.

Tor has a weaker threat model than many anonymity designs in the
literature, because we aim primarily to provide a
practical and useful network. Given that fixed assumption, we then
provide as much anonymity as we can. In particular, because we
want to support interactive communications, we fall prey to a variety
of intra-network~\cite{danezis-oakland,flow-correlation04,bar} and
end-to-end~\cite{danezis-pet2004,SS03} anonymity breaking attacks.

Tor's defense lies in having a diverse enough network that its adversaries
are unlikely to be in the right places to attack both ends of a user's
stream. Specifically,
Tor aims to resist observers and insiders by distributing each transaction
over several nodes in the network.  This ``distributed trust'' approach
means the Tor network can be safely operated and used by a wide variety
of mutually distrustful users, providing more sustainability and security
than previous attempts at anonymizing networks.
The Tor network has a broad range of users, including ordinary citizens
who want to avoid being profiled for targeted advertisements, corporations
who don't want to reveal information to their competitors, and law
enforcement and government intelligence agencies who need
to do operations on the Internet without being noticed.

Tor research and development has been funded by the U.S. Navy, for use
in securing government
communications, and also by the Electronic Frontier Foundation, for use
in maintaining civil liberties for ordinary citizens online. The Tor
protocol is one of the leading choices
to be the anonymizing layer in the European Union's PRIME directive to
help maintain privacy in Europe. The University of Dresden in Germany
has integrated an independent implementation of the Tor protocol into
their popular Java Anon Proxy anonymizing client. This wide variety of
interests helps maintain both the stability and the security of the
network.

While~\cite{tor-design} gives an overall view of the Tor design and goals,
this paper describes the policy and technical issues that Tor faces are
we continue deployment. We aim to lay a research agenda for others to
help in addressing these issues. Section~\ref{sec:what-is-tor} gives an
overview of the Tor
design and ours goals. We go on in Section~\ref{sec:related} to describe
Tor's context in the anonymity space. Sections~\ref{sec:crossroads-policy}
and~\ref{sec:crossroads-technical} describe the practical challenges,
both policy and technical respectively, that stand in the way of moving
from a practical useful network to a practical useful anonymous network.

\section{What Is Tor}
\label{sec:what-is-tor}

Here we give a basic overview of the Tor design and its properties. For
details on the design, assumptions, and security arguments, we refer
the reader to~\cite{tor-design}.

\subsection{Distributed trust: safety in numbers}

Tor provides \emph{forward privacy}, so that users can connect to
Internet sites without revealing their logical or physical locations
to those sites or to observers.  It also provides \emph{location-hidden
services}, so that critical servers can support authorized users without
giving adversaries an effective vector for physical or online attacks.
The design provides this protection even when a portion of its own
infrastructure is controlled by an adversary.

To create a private network pathway with Tor, the user's software (client)
incrementally builds a \emph{circuit} of encrypted connections through
servers on the network. The circuit is extended one hop at a time, and
each server along the way knows only which server gave it data and which
server it is giving data to. No individual server ever knows the complete
path that a data packet has taken. The client negotiates a separate set
of encryption keys for each hop along the circuit to ensure that each
hop can't trace these connections as they pass through.

Once a circuit has been established, many kinds of data can be exchanged
and several different sorts of software applications can be deployed over
the Tor network. Because each server sees no more than one hop in the
circuit, neither an eavesdropper nor a compromised server can use traffic
analysis to link the connection's source and destination. Tor only works
for TCP streams and can be used by any application with SOCKS support.

For efficiency, the Tor software uses the same circuit for connections
that happen within the same minute or so. Later requests are given a new
circuit, to prevent long-term linkability between different actions by
a single user.

Tor also makes it possible for users to hide their locations while
offering various kinds of services, such as web publishing or an instant
messaging server. Using Tor ``rendezvous points'', other Tor users can
connect to these hidden services, each without knowing the other's network
identity.
%This hidden service functionality could allow Tor users to
%set up a website where people publish material without worrying about
%censorship. Nobody would be able to determine who was offering the site,
%and nobody who offered the site would know who was posting to it.

tor works for tcp on socks (see section \ref{subsec:tcp-vs-ip}). it
only anonymizes the channel, so you need application-level scrubbers
like privoxy.

Tor differs from other deployed systems for traffic analysis resistance
in its security and flexibility.  Mix networks such as
Mixmaster~\cite{mixmaster} or its successor Mixminion~\cite{minion-design}
gain the highest degrees of anonymity at the expense of introducing highly
variable delays, thus making them unsuitable for applications such as web
browsing that require quick response times.  Commercial single-hop proxies
such as {\url{anonymizer.com}} present a single point of failure, where
a single compromise can expose all users' traffic, and a single-point
eavesdropper can perform traffic analysis on the entire network.
Also, their proprietary implementations place any infrastucture that
depends on these single-hop solutions at the mercy of their providers'
financial health as well as network security.

No organization can achieve this security on its own.  If a single
corporation or government agency were to build a private network to
protect its operations, any connections entering or leaving that network
would be obviously linkable to the controlling organization.  The members
and operations of that agency would be easier, not harder, to distinguish.

Instead, to protect our networks from traffic analysis, we must
collaboratively blend the traffic from many organizations and private
citizens, so that an eavesdropper can't tell which users are which,
and who is looking for what information.  By bringing more users onto
the network, all users become more secure \cite{econymics}.

Naturally, organizations will not want to depend on others for their
security.  If most participating providers are reliable, Tor tolerates
some hostile infiltration of the network.  For maximum protection,
the Tor design includes an enclave approach that lets data be encrypted
(and authenticated) end-to-end, so high-sensitivity users can be sure it
hasn't been read or modified.  This even works for Internet services that
don't have built-in encryption and authentication, such as unencrypted
HTTP or chat, and it requires no modification of those services to do so.

weasel's graph of \# nodes and of bandwidth, ideally from week 0.

Tor doesn't try to provide steg (but see Sec \ref{china}), or
the other non-goals listed in tor-design.

\section{Tor's position in the anonymity field}
\label{sec:related}

There are many other classes of systems: single-hop proxies, open proxies,
jap, mixminion, flash mixes, freenet, i2p, mute/ants/etc, tarzan,
morphmix, freedom. Give brief descriptions and brief characterizations
of how we differ. This is not the breakthrough stuff and we only have
a page or two for it.

have a serious discussion of morphmix's assumptions, since they would
seem to be the direct competition. in fact tor is a flexible architecture
that would encompass morphmix, and they're nearly identical except for
path selection and node discovery. and the trust system morphmix has
seems overkill (and/or insecure) based on the threat model we've picked.

\section{Threat model}

discuss $\frac{c^2}{n^2}$, except how in practice the chance of owning
the last hop is not $c/n$ since that doesn't take the destination (website)
into account. so in cases where the adversary does not also control the
final destination we're in good shape, but if he *does* then we'd be better
off with a system that lets each hop choose a path.

in practice tor's threat model is based entirely on the goal of dispersal
and diversity. george and steven describe an attack \cite{draft} that
lets them determine the nodes used in a circuit; yet they can't identify
alice or bob through this attack. so it's really just the endpoints that
remain secure. and the enclave model seems particularly threatened by
this, since this attack lets us identify endpoints when they're servers.
see \ref{subsec:helper-nodes} for discussion of some ways to address this
issue.

see \ref{subsec:routing-zones} for discussion of larger
adversaries and our dispersal goals.

\section{Crossroads: Policy issues}
\label{sec:crossroads-policy}

Many of the issues the Tor project needs to address are not just a
matter of system design or technology development. In particular, the
Tor project's \emph{image} with respect to its users and the rest of
the Internet impacts the security it can provide.

As an example to motivate this section, some U.S.~Department of Enery
penetration testing engineers are tasked with compromising DoE computers
from the outside. They only have a limited number of ISPs from which to
launch their attacks, and they found that the defenders were recognizing
attacks because they came from the same IP space. These engineers wanted
to use Tor to hide their tracks. First, from a technical standpoint,
Tor does not support the variety of IP packets they would like to use in
such attacks (see Section \ref{subsec:ip-vs-tcp}). But aside from this,
we also decided that it would probably be poor precedent to encourage
such use -- even legal use that improves national security -- and managed
to dissuade them.

With this image issue in mind, here we discuss the Tor user base and
Tor's interaction with other services on the Internet.



\subsection{Usability}

Usability: fc03 paper was great, except the lower latency you are the
less useful it seems it is.
A Tor gui, how jap's gui is nice but does not reflect the security
they provide.
Public perception, and thus advertising, is a security parameter.


\subsection{Image, usability, and sustainability}

Image: substantial non-infringing uses. Image is a security parameter,
since it impacts user base and perceived sustainability.

Sustainability. Previous attempts have been commercial which we think
adds a lot of unnecessary complexity and accountability. Freedom didn't
collect enough money to pay its servers; JAP bandwidth is supported by
continued money, and they periodically ask what they will do when it
dries up.

good uses are kept private, bad uses are publicized. not good.

\subsection{Tor and file-sharing}

Bittorrent and dmca. Should we add an IDS to autodetect protocols and
snipe them?

because only at the exit is it evident what port or protocol a given
tor stream is, you can't choose not to carry file-sharing traffic.

hibernation vs rate-limiting: do we want diversity or throughput? i
think we're shifting back to wanting diversity.

\subsection{Tor and blacklists}

Takedowns and efnet abuse and wikipedia complaints and irc
networks.

It was long expected that, alongside Tor's legitimate users, it would also
attract troublemakers who exploited Tor in order to abuse services on the
Internet.  Our initial answer to this situation was to use ``exit policies''
to allow individual Tor servers to block access to specific IP/port ranges.
This approach was meant to make operators more willing to run Tor by allowing
them to prevent their servers from being used for abusing particular
services.  For example, all Tor servers currently block SMTP (port 25), in
order to avoid being used to send spam.

This approach is useful, but is insufficient for two reasons.  First, since
it is not possible to force all ORs to block access to any given service,
many of those services try to block Tor instead.  More broadly, while being
blockable is important to being good netizens, we would like to encourage
services to allow anonymous access; services should not need to decide
between blocking legitimate anonymous use and allowing unlimited abuse.

This is potentially a bigger problem than it may appear. 
On the one hand, if people want to refuse connections from you on
their servers it would seem that they should be allowed to.  But, a
possible major problem with the blocking of Tor is that it's not just
the decision of the individual server administrator whose deciding if
he wants to post to wikipedia from his Tor node address or allow
people to read wikipedia anonymously through his Tor node. If e.g.,
s/he comes through a campus or corporate NAT, then the decision must
be to have the entire population behind it able to have a Tor exit
node or write access to wikipedia. This is a loss for both of us (Tor
and wikipedia). We don't want to compete for (or divvy up) the NAT
protected entities of the world.

(A related problem is that many IP blacklists are not terribly fine-grained.
No current IP blacklist, for example, allow a service provider to blacklist
only those Tor servers that allow access to a specific IP or port, even
though this information is readily available.  One IP blacklist even bans
every class C network that contains a Tor server, and recommends banning SMTP
from these networks even though Tor does not allow SMTP at all.)

Problems of abuse occur mainly with services such as IRC networks and
Wikipedia, which rely on IP-blocking to ban abusive users.  While at first
blush this practice might seem to depend on the anachronistic assumption that
each IP is an identifier for a single user, it is actually more reasonable in
practice: it assumes that non-proxy IPs are a costly resource, and that an
abuser can not change IPs at will.  By blocking IPs which are used by Tor
servers, open proxies, and service abusers, these systems hope to make
ongoing abuse difficult.  Although the system is imperfect, it works
tolerably well for them in practice.

But of course, we would prefer that legitimate anonymous users be able to
access abuse-prone services.  One conceivable approach would be to require
would-be IRC users, for instance, to register accounts if they wanted to
access the IRC network from Tor.  But in practise, this would not
significantly impede abuse if creating new accounts were easily automatable;
this is why services use IP blocking.  In order to deter abuse, pseudonymous
identities need to impose a significant switching cost in resources or human
time.

Once approach, similar to that taken by Freedom, would be to bootstrap some
non-anonymous costly identification mechanism to allow access to a
blind-signature pseudonym protocol.  This would effectively create costly
pseudonyms, which services could require in order to allow anonymous access.
This approach has difficulties in practise, however:
\begin{tightlist}
\item Unlike Freedom, Tor is not a commercial service.  Therefore, it would
  be a shame to require payment in order to make Tor useful, or to make
  non-paying users second-class citizens.
\item It is hard to think of an underlying resource that would actually work.
  We could use IP addresses, but that's the problem, isn't it?
\item Managing single sign-on services is not considered a well-solved
  problem in practice.  If Microsoft can't get universal acceptance for
  passport, why do we think that a Tor-specific solution would do any good?
\item Even if we came up with a perfect authentication system for our needs,
  there's no guarantee that any service would actually start using it.  It
  would require a nonzero effort for them to support it, and it might just
  be less hassle for them to block tor anyway.
\end{tightlist}

Squishy IP based ``authentication'' and ``authorization'' is a reality
we must contend with. We should say something more about the analogy
with SSNs.



\subsection{Other}

Tor's scope: How much should Tor aim to do? Applications that leak
data: we can say they're not our problem, but they're somebody's problem.
Also, the more widely deployed Tor becomes, the more people who need a
deployed overlay network tell us they'd like to use us if only we added
the following more features. For example, Blossom \cite{blossom} and
random community wireless projects both want source-routable overlay
networks for their own purposes. Fortunately, our modular design separates
routing from node discovery; so we could implement Morphmix in Tor just
by implementing the Morphmix-specific node discovery and path selection
pieces. On the other hand, we could easily get distracted building a
general-purpose overlay library, and we're only a few developers.

Should we allow revocation of anonymity if a threshold of
servers want to?

Logging. Making logs not revealing. A happy coincidence that verbose
logging is our \#2 performance bottleneck. Is there a way to detect
modified servers, or to have them volunteer the information that they're
logging verbosely? Would that actually solve any attacks?

\section{Crossroads: Scaling and Design choices}
\label{sec:crossroads-design}

\subsection{Transporting the stream vs transporting the packets}

We periodically run into ex ZKS employees who tell us that the process of
anonymizing IPs should ``obviously'' be done at the IP layer. Here are
the issues that need to be resolved before we'll be ready to switch Tor
over to arbitrary IP traffic.

\begin{enumerate}
\setlength{\itemsep}{0mm}
\setlength{\parsep}{0mm}
\item \emph{IP packets reveal OS characteristics.} We still need to do
IP-level packet normalization, to stop things like IP fingerprinting
\cite{ip-fingerprinting}. There exist libraries \cite{ip-normalizing}
that can help with this.
\item \emph{Application-level streams still need scrubbing.} We still need
Tor to be easy to integrate with user-level application-specific proxies
such as Privoxy. So it's not just a matter of capturing packets and
anonymizing them at the IP layer.
\item \emph{Certain protocols will still leak information.} For example,
DNS requests destined for my local DNS servers need to be rewritten
to be delivered to some other unlinkable DNS server. This requires
understanding the protocols we are transporting.
\item \emph{The crypto is unspecified.} First we need a block-level encryption
approach that can provide security despite
packet loss and out-of-order delivery. Freedom allegedly had one, but it was
never publicly specified, and we believe it's likely vulnerable to tagging
attacks \cite{tor-design}. Also, TLS over UDP is not implemented or even
specified, though some early work has begun on that \cite{ben-tls-udp}.
\item \emph{We'll still need to tune network parameters}. Since the above
encryption system will likely need sequence numbers and maybe more to do
replay detection, handle duplicate frames, etc, we will be reimplementing
some subset of TCP anyway to manage throughput, congestion control, etc.
\item \emph{Exit policies for arbitrary IP packets mean building a secure
IDS.}  Our server operators tell us that exit policies are one of
the main reasons they're willing to run Tor over previous attempts
at anonymizing networks.  Adding an IDS to handle exit policies would
increase the security complexity of Tor, and would likely not work anyway,
as evidenced by the entire field of IDS and counter-IDS papers. Many
potential abuse issues are resolved by the fact that Tor only transports
valid TCP streams (as opposed to arbitrary IP including malformed packets
and IP floods), so exit policies become even \emph{more} important as
we become able to transport IP packets. We also need a way to compactly
characterize the exit policies and let clients parse them to decide
which nodes will allow which packets to exit.
\item \emph{The Tor-internal name spaces would need to be redesigned.} We
support hidden service \tt{.onion} addresses, and other special addresses
like \tt{.exit} (see Section \ref{subsec:}), by intercepting the addresses
when they are passed to the Tor client.
\end{enumerate}

This list is discouragingly long right now, but we recognize that it
would be good to investigate each of these items in further depth and to
understand which are actual roadblocks and which are easier to resolve
than we think. We certainly wouldn't mind if Tor one day is able to
transport a greater variety of protocols.

\subsection{Mid-latency}

Mid-latency. Can we do traffic shape to get any defense against George's
PET2004 paper? Will padding or long-range dummies do anything then? Will
it kill the user base or can we get both approaches to play well together?

explain what mid-latency is. propose a single network where users of
varying latency goals can combine.

Note that in practice as the network is growing and we accept cable
modem and dsl nodes, and nodes in other continents, we're *already*
looking at many-second delays for some transactions. The engineering
required to get this lower is going to be extremely hard. It's worth
considering how hard it would be to accept the fixed (higher) latency
and improve the protection we get from it.

% can somebody besides arma flesh this section out?

%\subsection{The DNS problem in practice}

\subsection{Measuring performance and capacity}

How to measure performance without letting people selectively deny service
by distinguishing pings. Heck, just how to measure performance at all. In
practice people have funny firewalls that don't match up to their exit
policies and Tor doesn't deal.

Network investigation: Is all this bandwidth publishing thing a good idea?
How can we collect stats better? Note weasel's smokeping, at
http://seppia.noreply.org/cgi-bin/smokeping.cgi?target=Tor
which probably gives george and steven enough info to break tor?

\subsection{Plausible deniability}

Does running a server help you or harm you? George's Oakland attack.
Plausible deniability -- without even running your traffic through Tor! We
have to pick the path length so adversary can't distinguish client from
server (how many hops is good?).

\subsection{Helper nodes}

When does fixing your entry or exit node help you?
Helper nodes in the literature don't deal with churn, and
especially active attacks to induce churn.

Do general DoS attacks have anonymity implications? See e.g. Adam
Back's IH paper, but I think there's more to be pointed out here.

\subsection{Location-hidden services}

Survivable services are new in practice, yes? Hidden services seem
less hidden than we'd like, since they stay in one place and get used
a lot. They're the epitome of the need for helper nodes. This means
that using Tor as a building block for Free Haven is going to be really
hard. Also, they're brittle in terms of intersection and observation
attacks. Would be nice to have hot-swap services, but hard to design.

\subsection{Trust and discovery}

The published Tor design adopted a deliberately simplistic design for
authorizing new nodes and informing clients about servers and their status.
In the early Tor designs, all ORs periodically uploaded a signed description
of their locations, keys, and capabilities to each of several well-known {\it
  directory servers}.  These directory servers constructed a signed summary
of all known ORs (a ``directory''), and a signed statement of which ORs they
believed to be operational at any given time (a ``network status'').  Clients
periodically downloaded a directory in order to learn the latest ORs and
keys, and more frequently downloaded a network status to learn which ORs are
likely to be running.  ORs also operate as directory caches, in order to
lighten the bandwidth on the authoritative directory servers.

In order to prevent Sybil attacks (wherein an adversary signs up many
purportedly independent servers in order to increase her chances of observing
a stream as it enters and leaves the network), the early Tor directory design
required the operators of the authoritative directory servers to manually
approve new ORs.  Unapproved ORs were included in the directory, but clients
did not use them at the start or end of their circuits.  In practice,
directory administrators performed little actual verification, and tended to
approve any OR whose operator could compose a coherent email.  This procedure
may have prevented trivial automated Sybil attacks, but would do little
against a clever attacker.

There are a number of flaws in this system that need to be addressed as we
move forward.  They include:
\begin{tightlist}
\item Each directory server represents an independent point of failure; if
  any one were compromised, it could immediately compromise all of its users
  by recommending only compromised ORs.
\item The more servers appear join the network, the more unreasonable it
  becomes to expect clients to know about them all.  Directories
  become unfeasibly large, and downloading the list of servers becomes
  burdonsome.
\item The validation scheme may do as much harm as it does good.  It is not
  only incapable of preventing clever attackers from mounting Sybil attacks,
  but may deter server operators from joining the network.  (For instance, if
  they expect the validation process to be difficult, or if they do not share
  any languages in common with the directory server operators.)
\end{tightlist}

We could try to move the system in several directions, depending on our
choice of threat model and requirements.  If we did not need to increase
network capacity in order to support more users, there would be no reason not
to adopt even stricter validation requirements, and reduce the number of
servers in the network to a trusted minimum.  But since we want Tor to work
for as many users as it can, we need XXXXX

In order to address the first two issues, it seems wise to move to a system
including a number of semi-trusted directory servers, no one of which can
compromise a user on its own.  Ultimately, of course, we cannot escape the
problem of a first introducer: since most users will run Tor in whatever
configuration the software ships with, the Tor distribution itself will
remain a potential single point of failure so long as it includes the seed
keys for directory servers, a list of directory servers, or any other means
to learn which servers are on the network.  But omitting this information
from the Tor distribution would only delegate the trust problem to the
individual users, most of whom are presumably less informed about how to make
trust decisions than the Tor developers.

%Network discovery, sybil, node admission, scaling. It seems that the code
%will ship with something and that's our trust root. We could try to get
%people to build a web of trust, but no. Where we go from here depends
%on what threats we have in mind. Really decentralized if your threat is
%RIAA; less so if threat is to application data or individuals or...


Game theory for helper nodes: if Alice offers a hidden service on a
server (enclave model), and nobody ever uses helper nodes, then against
George+Steven's attack she's totally nailed. If only Alice uses a helper
node, then she's still identified as the source of the data. If everybody
uses a helper node (including Alice), then the attack identifies the
helper node and also Alice, and knows which one is which. If everybody
uses a helper node (but not Alice), then the attacker figures the real
source was a client that is using Alice as a helper node. [How's my
logic here?]

people are using hidden services as a poor man's vpn and firewall-buster.
rather than playing with dyndns and trying to pierce holes in their
firewall (say, so they can ssh in from the outside), they run a hidden
service on the inside and then rendezvous with that hidden service
externally.

in practice, sites like bloggers without borders (www.b19s.org) are
running tor servers but more important are advertising a hidden-service
address on their front page. doing this can provide increased robustness
if they used the dual-IP approach we describe in tor-design, but in
practice they do it to a) increase visibility of the tor project and their
support for privacy, and b) to offer a way for their users, using vanilla
software, to get end-to-end encryption and end-to-end authentication to
their website.


\section{Crossroads: Scaling}
%\label{sec:crossroads-scaling}
%P2P + anonymity issues:

Tor is running today with hundreds of servers and tens of thousands of
users, but it will certainly not scale to millions.

Scaling Tor involves three main challenges.  First is safe server
discovery, both bootstrapping -- how a Tor client can robustly find an
initial server list -- and ongoing -- how a Tor client can learn about
a fair sample of honest servers and not let the adversary control his
circuits (see Section x).  Second is detecting and handling the speed
and reliability of the variety of servers we must use if we want to
accept many servers (see Section y).
Since the speed and reliability of a circuit is limited by its worst link,
we must learn to track and predict performance.  Finally, in order to get
a large set of servers in the first place, we must address incentives
for users to carry traffic for others (see Section incentives).

\subsection{Incentives}

There are three behaviors we need to encourage for each server: relaying
traffic; providing good throughput and reliability while doing it;
and allowing traffic to exit the network from that server.

We encourage these behaviors through \emph{indirect} incentives, that
is, designing the system and educating users in such a way that users
with certain goals will choose to relay traffic.  In practice, the
main incentive for running a Tor server is social benefit: volunteers
altruistically donate their bandwidth and time.  We also keep public
rankings of the throughput and reliability of servers, much like
seti@home.  We further explain to users that they can get \emph{better
security} by operating a server, because they get plausible deniability
(indeed, they may not need to route their own traffic through Tor at all
-- blending directly with other traffic exiting Tor may be sufficient
protection for them), and because they can use their own Tor server
as entry or exit point and be confident it's not run by the adversary.
Finally, we can improve the usability and feature set of the software:
rate limiting support and easy packaging decrease the hassle of
maintaining a server, and our configurable exit policies allow each
operator to advertise a policy describing the hosts and ports to which
he feels comfortable connecting.

Beyond these, however, there is also a need for \emph{direct} incentives:
providing payment or other resources in return for high-quality service.
Paying actual money is problematic: decentralized e-cash systems are
not yet practical, and a centralized collection system not only reduces
robustness, but also has failed in the past (the history of commercial
anonymizing networks is littered with failed attempts).  A more promising
option is to use a tit-for-tat incentive scheme: provide better service
to nodes that have provided good service to you.

Unfortunately, such an approach introduces new anonymity problems.
Does the incentive system enable the adversary to attract more traffic by
performing well? Typically a user who chooses evenly from all options is
most resistant to an adversary targetting him, but that approach prevents
us from handling heterogeneous servers \cite{casc-rep}.
When a server (call him Steve) performs well for Alice, does Steve gain
reputation with the entire system, or just with Alice? If the entire
system, how does Alice tell everybody about her experience in a way that
prevents her from lying about it yet still protects her identity? If
Steve's behavior only affects Alice's behavior, does this allow Steve to
selectively perform only for Alice, and then break her anonymity later
when somebody (presumably Alice) routes through his node?

These are difficult and open questions, yet choosing not to scale means
leaving most users to a less secure network or no anonymizing network
at all.  We will start with a simplified approach to the tit-for-tat
incentive scheme based on two rules: (1) each node should measure the
service it receives from adjacent nodes, and provide service relative to
the received service, but (2) when a node is making decisions that affect
its own security (e.g. when building a circuit for its own application
connections), it should choose evenly from a sufficiently large set of
nodes that meet some minimum service threshold.  This approach allows us
to discourage bad service without opening Alice up as much to attacks.

%XXX rewrite the above so it sounds less like a grant proposal and
%more like a "if somebody were to try to solve this, maybe this is a
%good first step".

%We should implement the above incentive scheme in the
%deployed Tor network, in conjunction with our plans to add the necessary
%associated scalability mechanisms.  We will do experiments (simulated
%and/or real) to determine how much the incentive system improves
%efficiency over baseline, and also to determine how far we are from
%optimal efficiency (what we could get if we ignored the anonymity goals).

\subsection{Peer-to-peer / practical issues}

Making use of servers with little bandwidth. How to handle hammering by
certain applications.

Handling servers that are far away from the rest of the network, e.g. on
the continents that aren't North America and Europe. High latency,
often high packet loss.

Running Tor servers behind NATs, behind great-firewalls-of-China, etc.
Restricted routes. How to propagate to everybody the topology? BGP
style doesn't work because we don't want just *one* path. Point to
Geoff's stuff.

\subsection{ISP-class adversaries}

Routing-zones. It seems that our threat model comes down to diversity and
dispersal. But hard for Alice to know how to act. Many questions remain.

\subsection{The China problem}

We have lots of users in Iran and similar (we stopped
logging, so it's hard to know now, but many Persian sites on how to use
Tor), and they seem to be doing ok. But the China problem is bigger. Cite
Stefan's paper, and talk about how we need to route through clients,
and we maybe we should start with a time-release IP publishing system +
advogato based reputation system, to bound the number of IPs leaked to the
adversary.

\cite{infranet}
\cite{koepsell-wpes2004}
\cite{advogato}
\cite{berkman}

\subsection{Non-clique topologies}

Because of its threat model that is substantially weaker than high
latency mixnets, Tor is actually in a potentially better position to
scale at least initially. The issues for scaling include how many
neighbors can nodes support and how many users (alternatively how much
application traffic capacity) can the network handle for each new node
that comes into the network. This depends on many things, most notably
the traffic capacity of the new nodes.  We can observe, however, that
adding a tor node of any feasible bandwidth will increase the traffic
capacity of the network. This means that, as a first step to scaling,
we can focus on the interconnectivity of the nodes, followed by
directories, discovery, etc.

By reducing the connectivity of the network we increase the total
number of nodes that the network can contain. Anonymity implications
of restricted routes for mix networks has already been explored by
Danezis~\cite{danezis-pets03}.  That paper explicitly considered only
traffic analysis resistance provided by the network and sidestepped
questions of traffic confirmation resistance. But, Tor is designed
only to resist traffic confirmation. For this and other reasons, we
cannot simply adopt his mixnet results to onion routing networks.  If
an attacker gains minimal increase in the likelyhood of compromising
the endpoints of a Tor circuit through a sparse network (vs.\ a clique
on the same node set), then the restriction will have had minimal
impact on the anonymity provided by that network.

As Danezis noted, what is wanted is an expander graph, i.e., a graph
in which any subgraph of nodes is likely to have lots of nodes as
neighbors. For Tor we can be a bit more specific. As long as most
(non-enclave) circuits have three nodes, then ideally any pair of nodes
should be linked to every node in the network with high probability.

I need to work out some numbers here: Consider networks of 100,
200, 500, and 1000 nodes with this property. Figure out the savings
in connectivity in each case. Consider also reducing the probability.
Something to do tomorrow.

Need to tell some story a la the FC02 paper about assigning the
links in the graph. Also tomorrow or so.

This approach does not take different node bandwidth into account. We
could consider a clique of high bandwidth/high reliability nodes that
is connected to all nodes in the network. All circuits would then go
through this `backbone'. This simplifies many issues but makes the
expected minimum path length four. On the other hand, it is not
likely that there will be substantial increase in network latency
given that the added hop will always be between high bandwidth nodes.

Directories need not be too much more of a problem. They can list the
Top tier nodes, then for each of those, to which nodes they are
connected.  For non-enclave purposes, it is enough to download the top
tier list and a few of those below it.  Lots of threat issues here,
can address them with witness connections or other means. (E.g., does
it make sense to favor the nodes that are listed by more than one node
at the top?)

Been making this too hard. Save elegant answers for another venue.
Just assume 50 node clique (center).  Assume these can each handle 125
connections to other nodes. Assume everyone else connects to 3 nodes
in the center and anyone out of the center that they want to. All
3-node paths choose a center node for their second hop. Then the
network easily scales to c. 1300 nodes with commensurate increase in
bandwidth. Distribute the center hardwired to new nodes or publicize.
Let directories tell about other nodes in the network.  50-50 that
path goes whatever-center-center.


\section{The Future}
\label{sec:conclusion}


\bibliographystyle{plain} \bibliography{tor-design}

\end{document}

