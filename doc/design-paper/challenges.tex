\documentclass{llncs}

\usepackage{url}
\usepackage{amsmath}
\usepackage{epsfig}

\newenvironment{tightlist}{\begin{list}{$\bullet$}{
  \setlength{\itemsep}{0mm}
    \setlength{\parsep}{0mm}
    %  \setlength{\labelsep}{0mm}
    %  \setlength{\labelwidth}{0mm}
    %  \setlength{\topsep}{0mm}
    }}{\end{list}}

\begin{document}

\title{Challenges in practical low-latency stream anonymity (DRAFT)}

\author{Roger Dingledine and Nick Mathewson}
\institute{The Free Haven Project\\
\email{\{arma,nickm\}@freehaven.net}}

\maketitle
\pagestyle{empty}

\begin{abstract}

We describe our experiences with deploying Tor, a low-latency anonymous
communication system that has been funded both by the U.S.~Navy
and also by the Electronic Frontier Foundation.

Because of its simplified threat model, Tor does not aim to defend
against many of the attacks in the literature.

We describe both policy issues that have come up from operating the
network and technical challenges in building a more sustainable and
scalable network.

\end{abstract}

\section{Introduction}

Tor is a low-latency anonymous communication overlay network designed
to be practical and usable for protecting TCP streams over the
Internet~\cite{tor-design}. We have been operating a publicly deployed
Tor network since October 2003 that has grown to over a hundred volunteer
nodes and sometimes as much as 80 megabits of average traffic per second.

Tor has a weaker threat model than many anonymity designs in the
literature, because our foremost goal is to deploy a
practical and useful network for interactive (low-latency) communications.
Subject to this restriction, we try to
provide as much anonymity as we can. In particular, because we
support interactive communications without impractically expensive padding,
we fall prey to a variety
of intra-network~\cite{back01,attack-tor-oak05,flow-correlation04} and
end-to-end~\cite{danezis-pet2004,SS03} anonymity-breaking attacks.

Tor is secure so long as adversaries are unable to
observe connections as they both enter and leave the Tor network.
Therefore, Tor's defense lies in having a diverse enough set of servers
that most real-world
adversaries are unlikely to be in the right places to attack users.
Specifically,
Tor aims to resist observers and insiders by distributing each transaction
over several nodes in the network.  This ``distributed trust'' approach
means the Tor network can be safely operated and used by a wide variety
of mutually distrustful users, providing more sustainability and security
than some previous attempts at anonymizing networks.
The Tor network has a broad range of users, including ordinary citizens
concerned about their privacy, corporations
who don't want to reveal information to their competitors, and law
enforcement and government intelligence agencies who need
to do operations on the Internet without being noticed.

Tor research and development has been funded by the U.S.~Navy, for use
in securing government
communications, and also by the Electronic Frontier Foundation, for use
in maintaining civil liberties for ordinary citizens online. The Tor
protocol is one of the leading choices
to be the anonymizing layer in the European Union's PRIME directive to
help maintain privacy in Europe. The University of Dresden in Germany
has integrated an independent implementation of the Tor protocol into
their popular Java Anon Proxy anonymizing client. This wide variety of
interests helps maintain both the stability and the security of the
network.

%awk
Tor's principal research strategy, in attempting to deploy a network that is
practical, useful, and anonymous, has been to insist, when trade-offs arise
between these properties, on remaining useful enough to attract many users,
and practical enough to support them.  Subject to these
constraints, we aim to maximize anonymity.  This is not the only possible
direction in anonymity research: designs exist that provide more anonymity
than Tor at the expense of significantly increased resource requirements, or
decreased flexibility in application support (typically because of increased
latency).  Such research does not typically abandon aspirations towards
deployability or utility, but instead tries to maximize deployability and
utility subject to a certain degree of inherent anonymity (inherent because
usability and practicality affect usage which affects the actual anonymity
provided by the network \cite{back01,econymics}). We believe that these
approaches can be promising and useful, but that by focusing on deploying a
usable system in the wild, Tor helps us experiment with the actual parameters
of what makes a system ``practical'' for volunteer operators and ``useful''
for home users, and helps illuminate undernoticed issues which any deployed
volunteer anonymity network will need to address. 

While~\cite{tor-design} gives an overall view of the Tor design and goals,
this paper describes the policy and technical issues that Tor faces as
we continue deployment. Rather than trying to provide complete solutions
to every problem here, we lay out the assumptions and constraints
that we have observed through deploying Tor in the wild. In doing so, we
aim to create a research agenda for others to
help in addressing these issues. Section~\ref{sec:what-is-tor} gives an
overview of the Tor
design and ours goals. Sections~\ref{sec:crossroads-policy}
and~\ref{sec:crossroads-design} go on to describe the practical challenges,
both policy and technical respectively, that stand in the way of moving
from a practical useful network to a practical useful anonymous network.

%\section{What Is Tor}
\section{Distributed trust: safety in numbers}
\label{sec:what-is-tor}

Here we give a basic overview of the Tor design and its properties. For
details on the design, assumptions, and security arguments, we refer
the reader to the Tor design paper~\cite{tor-design}.

Tor provides \emph{forward privacy}, so that users can connect to
Internet sites without revealing their logical or physical locations
to those sites or to observers.  It also provides \emph{location-hidden
services}, so that critical servers can support authorized users without
giving adversaries an effective vector for physical or online attacks.
The design provides this protection even when a portion of its own
infrastructure is controlled by an adversary.

To create a private network pathway with Tor, the user's software (client)
incrementally builds a \emph{circuit} of encrypted connections through
servers on the network. The circuit is extended one hop at a time, and
each server along the way knows only which server gave it data and which
server it is giving data to. No individual server ever knows the complete
path that a data packet has taken. The client negotiates a separate set
of encryption keys for each hop along the circuit to ensure that each
hop can't trace these connections as they pass through.

Once a circuit has been established, many kinds of data can be exchanged
and several different sorts of software applications can be deployed over
the Tor network. Because each server sees no more than one hop in the
circuit, neither an eavesdropper nor a compromised server can use traffic
analysis to link the connection's source and destination. Tor only works
for TCP streams and can be used by any application with SOCKS support.

For efficiency, the Tor software uses the same circuit for connections
that happen within the same minute or so. Later requests are given a new
circuit, to prevent long-term linkability between different actions by
a single user.

Tor also makes it possible for users to hide their locations while
offering various kinds of services, such as web publishing or an instant
messaging server. Using Tor ``rendezvous points'', other Tor users can
connect to these hidden services, each without knowing the other's network
identity.

Tor attempts to anonymize the transport layer, not the application layer, so
application protocols that include personally identifying information need
additional application-level scrubbing proxies, such as
Privoxy~\cite{privoxy} for HTTP.  Furthermore, Tor does not permit arbitrary
IP packets; it only anonymizes TCP and DNS, and only supports connections via
SOCKS (see Section~\ref{subsec:tcp-vs-ip}).

Tor differs from other deployed systems for traffic analysis resistance
in its security and flexibility.  Mix networks such as
Mixmaster~\cite{mixmaster-spec} or its successor Mixminion~\cite{minion-design}
gain the highest degrees of anonymity at the expense of introducing highly
variable delays, thus making them unsuitable for applications such as web
browsing that require quick response times.  Commercial single-hop
proxies~\cite{anonymizer} present a single point of failure, where
a single compromise can expose all users' traffic, and a single-point
eavesdropper can perform traffic analysis on the entire network.
Also, their proprietary implementations place any infrastucture that
depends on these single-hop solutions at the mercy of their providers'
financial health as well as network security.

No organization can achieve this security on its own.  If a single
corporation or government agency were to build a private network to
protect its operations, any connections entering or leaving that network
would be obviously linkable to the controlling organization.  The members
and operations of that agency would be easier, not harder, to distinguish.

Instead, to protect our networks from traffic analysis, we must
collaboratively blend the traffic from many organizations and private
citizens, so that an eavesdropper can't tell which users are which,
and who is looking for what information.  By bringing more users onto
the network, all users become more secure~\cite{econymics}.

Naturally, organizations will not want to depend on others for their
security.  If most participating providers are reliable, Tor tolerates
some hostile infiltration of the network.  For maximum protection,
the Tor design includes an enclave approach that lets data be encrypted
(and authenticated) end-to-end, so high-sensitivity users can be sure it
hasn't been read or modified.  This even works for Internet services that
don't have built-in encryption and authentication, such as unencrypted
HTTP or chat, and it requires no modification of those services to do so.

As of January 2005, the Tor network has grown to around a hundred servers
on four continents, with a total capacity exceeding 1Gbit/s. Appendix A
shows a graph of the number of working servers over time, as well as a
graph of the number of bytes being handled by the network over time. At
this point the network is sufficiently diverse for further development
and testing; but of course we always encourage and welcome new servers
to join the network.

%Tor doesn't try to provide steg (but see Section~\ref{subsec:china}), or
%the other non-goals listed in tor-design.

Tor is not the only anonymity system that aims to be practical and useful.
Commercial single-hop proxies~\cite{anonymizer}, as well as unsecured
open proxies around the Internet, can provide good
performance and some security against a weaker attacker. Dresden's Java
Anon Proxy~\cite{web-mix} provides similar functionality to Tor but only
handles web browsing rather than arbitrary TCP\@. Also, JAP's network
topology uses cascades (fixed routes through the network); since without
end-to-end padding it is just as vulnerable as Tor to end-to-end timing
attacks, its dispersal properties are therefore worse than Tor's.
%Some peer-to-peer file-sharing overlay networks such as
%Freenet~\cite{freenet} and Mute~\cite{mute}
Zero-Knowledge Systems' commercial Freedom
network~\cite{freedom21-security} was even more flexible than Tor in
that it could transport arbitrary IP packets, and it also supported
pseudonymous access rather than just anonymous access; but it had
a different approach to sustainability (collecting money from users
and paying ISPs to run servers), and has shut down due to financial
load.  Finally, more scalable designs like Tarzan~\cite{tarzan:ccs02} and
MorphMix~\cite{morphmix:fc04} have been proposed in the literature, but
have not yet been fielded. We direct the interested reader to Section
2 of~\cite{tor-design} for a more indepth review of related work.

%six-four. crowds. i2p.


have a serious discussion of morphmix's assumptions, since they would
seem to be the direct competition. in fact tor is a flexible architecture
that would encompass morphmix, and they're nearly identical except for
path selection and node discovery. and the trust system morphmix has
seems overkill (and/or insecure) based on the threat model we've picked.
% this para should probably move to the scalability / directory system. -RD

\section{Threat model}
\label{sec:threat-model}

Tor does not attempt to defend against a global observer.  Any adversary who
can see a user's connection to the Tor network, and who can see the
corresponding connection as it exits the Tor network, can use timing
correlation to confirm the user's chosen
communication partners.  Defeating this attack would seem to require
introducing a prohibitive degree of traffic padding between the user and the
network, or introducing an unacceptable degree of latency (but see
Section \ref{subsec:mid-latency}).
And, it is not clear that padding works at all if we assume a
minimally active adversary that merely modifies the timing of packets
to or from the user. Thus, Tor only attempts to defend against
external observers who cannot observe both sides of a user's
connection.

Against internal attackers, who sign up Tor servers, the situation is more
complicated.  In the simplest case, if an adversary has compromised $c$ of
$n$ servers on the Tor network, then the adversary will be able to compromise
a random circuit with probability $\frac{c^2}{n^2}$ (since the circuit
initiator chooses hops randomly).  But there are
complicating factors:
\begin{tightlist}
\item If the user continues to build random circuits over time, an adversary
  is pretty certain to see a statistical sample of the user's traffic, and
  thereby can build an increasingly accurate profile of her behavior.  (See
  \ref{subsec:helper-nodes} for possible solutions.)
\item If an adversary controls a popular service outside of the Tor network,
  he can be certain of observing all connections to that service; he
  therefore will trace connections to that service with probability
  $\frac{c}{n}$.
\item Users do not in fact choose servers with uniform probability; they
  favor servers with high bandwidth, and exit servers that permit connections
  to their favorite services.
\end{tightlist}

%discuss $\frac{c^2}{n^2}$, except how in practice the chance of owning
%the last hop is not $c/n$ since that doesn't take the destination (website)
%into account. so in cases where the adversary does not also control the
%final destination we're in good shape, but if he *does* then we'd be better
%off with a system that lets each hop choose a path.
%
%Isn't it more accurate to say ``If the adversary _always_ controls the final
% dest, we would be just as well off with such as system.'' ?  If not, why
% not? -nm
% Sure. In fact, better off, since they seem to scale more easily. -rd

In practice Tor's threat model is based entirely on the goal of
dispersal and diversity. Murdoch and Danezis describe an attack
\cite{attack-tor-oak05} that lets an attacker determine the nodes used
in a circuit; yet s/he cannot identify the initiator or responder,
e.g., client or web server, through this attack. So the endpoints
remain secure, which is the goal. On the other hand we can imagine an
adversary that could attack or set up observation of all connections
to an arbitrary Tor node in only a few minutes.  If such an adversary
were to exist, s/he could use this probing to remotely identify a node
for further attack.  Also, the enclave model seems particularly
threatened by this attack, since it identifies endpoints when they're
also nodes in the Tor network: see Section~\ref{subsec:helper-nodes}
for discussion of some ways to address this issue.

[*****Suppose an adversary with active access to the responder traffic
wants to keep a circuit alive long enough to attack an identified
node. Could s/he do this without the overt cooperation of the client
proxy? More immediately, someone could identify nodes in this way and
if in their jurisdiction, immediately get a subpoena (if they even
need one) and tell the node operator(s) that she must retain all the
active circuit data she now has at that moment.  That \emph{can} be
done in real time.********** We should say something about this
here or later in the paper -pfs]

see \ref{subsec:routing-zones} for discussion of larger
adversaries and our dispersal goals.

[this section will get written once the rest of the paper is farther along]

\section{Crossroads: Policy issues}
\label{sec:crossroads-policy}

Many of the issues the Tor project needs to address are not just a
matter of system design or technology development. In particular, the
Tor project's \emph{image} with respect to its users and the rest of
the Internet impacts the security it can provide.

As an example to motivate this section, some U.S.~Department of Enery
penetration testing engineers are tasked with compromising DoE computers
from the outside. They only have a limited number of ISPs from which to
launch their attacks, and they found that the defenders were recognizing
attacks because they came from the same IP space. These engineers wanted
to use Tor to hide their tracks. First, from a technical standpoint,
Tor does not support the variety of IP packets one would like to use in
such attacks (see Section~\ref{subsec:tcp-vs-ip}). But aside from this,
we also decided that it would probably be poor precedent to encourage
such use---even legal use that improves national security---and managed
to dissuade them.

With this image issue in mind, this section discusses the Tor user base and
Tor's interaction with other services on the Internet.

\subsection{Image and security}

A growing field of papers argue that usability for anonymity systems
contributes directly to their security, because how usable the system
is impacts the possible anonymity set~\cite{back01,econymics}. Or
conversely, an unusable system attracts few users and thus can't provide
much anonymity.

This phenomenon has a second-order effect: knowing this, users should
choose which anonymity system to use based in part on how usable
\emph{others} will find it, in order to get the protection of a larger
anonymity set. Thus we might replace the adage ``usability is a security
parameter''~\cite{back01} with a new one: ``perceived usability is a
security parameter.'' From here we can better understand the effects
of publicity and advertising on security: the more convincing your
advertising, the more likely people will believe you have users, and thus
the more users you will attract. Perversely, over-hyped systems (if they
are not too broken) may be a better choice than modestly promoted ones,
if the hype attracts more users~\cite{usability-network-effect}.

So it follows that we should come up with ways to accurately communicate
the available security levels to the user, so she can make informed
decisions. Dresden's JAP project aims to do this, by including a
comforting `anonymity meter' dial in the software's graphical interface,
giving the user an impression of the level of protection for her current
traffic.

However, there's a catch. For users to share the same anonymity set,
they need to act like each other. An attacker who can distinguish
a given user's traffic from the rest of the traffic will not be
distracted by other users on the network. For high-latency systems like
Mixminion, where the threat model is based on mixing messages with each
other, there's an arms race between end-to-end statistical attacks and
counter-strategies~\cite{statistical-disclosure,minion-design,e2e-traffic,trickle02}.
But for low-latency systems like Tor, end-to-end \emph{traffic
correlation} attacks~\cite{danezis-pet2004,SS03,defensive-dropping}
allow an attacker who watches or controls both ends of a communication
to use statistics to match packet timing and volume, quickly linking
the initiator to her destination. This is why Tor's threat model is
based on preventing the adversary from observing both the initiator and
the responder.

Like Tor, the current JAP implementation does not pad connections
(apart from using small fixed-size cells for transport). In fact,
its cascade-based network toplogy may be even more vulnerable to these
attacks, because the network has fewer endpoints. JAP was born out of
the ISDN mix design~\cite{isdn-mixes}, where padding made sense because
every user had a fixed bandwidth allocation, but in its current context
as a general Internet web anonymizer, adding sufficient padding to JAP
would be prohibitively expensive.\footnote{Even if they could find and
maintain extra funding to run higher-capacity nodes, our experience with
users suggests that many users would not accept the increased per-user
bandwidth requirements, leading to an overall much smaller user base. But
see Section \ref{subsec:mid-latency}.} Therefore, since under this threat
model the number of concurrent users does not seem to have much impact
on the anonymity provided, we suggest that JAP's anonymity meter is not
correctly communicating security levels to its users.

% because more users don't help anonymity much, we need to rely more
% on other incentive schemes, both policy-based (see sec x) and
% technically enforced (see sec y)

On the other hand, while the number of active concurrent users may not
matter as much as we'd like, it still helps to have some other users
who use the network. We investigate this issue in the next section.

\subsection{Reputability}

Another factor impacting the network's security is its reputability:
the perception of its social value based on its current user base. If I'm
the only user who has ever downloaded the software, it might be socially
accepted, but I'm not getting much anonymity. Add a thousand animal rights
activists, and I'm anonymous, but everyone thinks I'm a bambi lover (or
NRA member if you prefer a contrasting example). Add a thousand
random citizens (cancer survivors, privacy enthusiasts, and so on)
and now I'm harder to profile.

The more cancer survivors on Tor, the better for the human rights
activists. The more script kiddies, the worse for the normal users. Thus,
reputability is an anonymity issue for two reasons. First, it impacts
the sustainability of the network: a network that's always about to be
shut down has difficulty attracting and keeping users, so its anonymity
set suffers. Second, a disreputable network attracts the attention of
powerful attackers who may not mind revealing the identities of all the
users to uncover a few bad ones.

While people therefore have an incentive for the network to be used for
``more reputable'' activities than their own, there are still tradeoffs
involved when it comes to anonymity. To follow the above example, a
network used entirely by cancer survivors might welcome some animal rights
activists onto the network, though of course they'd prefer a wider
variety of users.

Reputability becomes even more tricky in the case of privacy networks,
since the good uses of the network (such as publishing by journalists in
dangerous countries) are typically kept private, whereas network abuses
or other problems tend to be more widely publicized.

The impact of public perception on security is especially important
during the bootstrapping phase of the network, where the first few
widely publicized uses of the network can dictate the types of users it
attracts next.

%% "outside of academia, jap has just lost, permanently".  (That is,
%% even though the crime detection issues are resolved and are unlikely
%% to go down the same way again, public perception has not been kind.)

\subsection{Sustainability and incentives}
One of the (arguably) unsolved problems in low-latency anonymity designs is
how to keep the servers running.  Zero-Knowledge Systems's Freedom network
depended on paying third parties to run its servers; the JAP project's
bandwidth is dependent on grants from ???? to pay for its bandwidth and
administrative expenses.  In Tor, bandwidth and administrative costs are
distributed across the volunteers who run Tor nodes, so at least we have
reason to think that the Tor network could survive without continued research
funding.\footnote{It also helps that Tor is implemented with free and open
  source software that can be maintained by anybody with the ability and
  inclination.}  But why are these volunteers running nodes, and what can we
do to encourage more volunteers to do so?

We have not surveyed Tor operators to learn why they are running ORs, but
from the information they have provided, it seems that many of them run Tor
nodes for reasons of personal interest in privacy issues.  It is possible
that others are running Tor for anonymity reasons, but of course they are
hardly likely to tell us if they are.

Significantly, Tor's threat model changes the anonymity incentives for running
a server.  In a high-latency mix network, users can receive additional
anonymity by running their own server, since doing so obscures when they are
injecting messages into the network.  But in Tor, anybody observing a Tor
server can tell when the server is generating traffic that corresponds to
none of its incoming traffic, and therefore originating traffic itself.
Still, anonymity and privacy incentives do remain for server operators:
\begin{tightlist}
\item Against a hostile website, running a Tor exit node can provide a degree
  of ``deniaibility'' for traffic that originates at that exit node.  For
  example, it is likely in practise that HTTP requests from a Tor server's IP
  will be assumed to have left the Tor network.
\item Local Tor entry and exit servers allow users on a network to run in an
  `enclave' configuration.  [XXXX say more]
\end{tightlist}

First, we try to make the costs of running a Tor server easily minimized.
Since Tor is run by volunteers, the most crucial software usability issue is
usability by operators: when an operator leaves, the network becomes less
usable by everybody.  To keep operators pleased, we must try to keep Tor's
resource and administrative demands as low as possible. [XXXX say mroe]

Because of ISP billing structures, many Tor operators have underused capacity
that they are willing to donate to the network, at no additional monetary
cost to them.  Features to limit bandwidth have been essential to adoption.
Also useful has been a ``hibernation'' feature that allows a server that
wants to provide high bandwidth, but no more than a certain amount in a
giving billing cycle, to become dormant once its bandwidth is exhausted, and
to reawaken at a random offset into the next billing cycle.  This feature has
interesting policy implications, however; see
section~\ref{subsec:bandwidth-and-usability} below.

[XXXX say more.  Why else would you run a server? What else can we do/do we
  already do to make running a server more attractive?]

\subsection{Bandwidth and usability}
\label{subsec:bandwidth-and-usability}
Once users have configured their applications to work with Tor, the largest
remaining usability issue is bandwidth.  When websites ``feel slow,'' users
begin to suffer.

Clients currently try to build their connections through servers that they
guess will have enough bandwidth.  But even if capacity is allocated
optimally, it seems unlikely that the current network architecture will have
enough capacity to provide every user with as much bandwidth as she would
receive if she weren't using Tor, unless far more servers join the network
(see above).

Limited capacity does not destroy the network, however.  Instead, usage tends
towards an equilibrium: when performance suffers, users who value performance
over anonymity tend to leave the system, thus freeing capacity until the
remaining users on the network are exactly those willing to use that capacity
there is.

XXXX hibernation vs rate-limiting: do we want diversity or throughput? i
think we're shifting back to wanting diversity.

\subsection{Tor and file-sharing}
One potentially problematical area with deploying Tor has been our response
to file-sharing applications.  These applications make up an enormous
fraction of the traffic on the Internet today, and provide two challenges to
any anonymizing network: their intensive bandwidth requirement, and the
degree to which they are associated (correctly or not) with copyright
violation.

As noted above, high-bandwidth protocols can make the network unresponsive,
but tend to be somewhat self-correcting.  Issues of copyright violation,
however, are more interesting.  Typical exit node operators want to help
people achieve privacy and anonymous speech, not to help people (say) host
Vin Diesel movies for illegal download; and typical ISPs would rather not
deal with customers who incur them the overhead of getting menacing letters
from the MPAA.  While it is quite likely that the operators are doing nothing
illegal, many ISPs have policies of dropping users who get repeated legal
threats regardless of the merits of those threats, and many operators would
prefer to avoid receiving legal threats even if those threats have little
merit.  So when the letters arrive, operators are likely to face
pressure to block filesharing applications entirely, in order to avoid the
hassle.

But blocking filesharing would not necessarily be easy; most popular
protocols have evolved to run on a variety of non-standard ports in order to
get around other port-based bans.  Thus, exit node operators who wanted to
block filesharing would have to find some way to integrate Tor with a
protocol-aware exit filter.  This could be a technically expensive
undertaking, and one with poor prospects: it is unlikely that Tor exit nodes
would succeed where so many institutional firewalls have failed.  Another
possibility for sensitive operators is to run a very restrictive server that
only permits exit connections to a very restricted range of ports which are
not frequently associated with file sharing.  There are increasingly few such
ports.

For the moment, it seems that Tor's bandwidth issues have rendered it
unattractive for bulk file-sharing traffic; this may continue to be so in the
future.  Nevertheless, Tor will likely remain attractive for limited use in
filesharing protocols that have separate control and data channels.

[xxxx We should say more -- but what?  That we'll see a similar
  equilibriating effect as with bandwidth, where sensitive ops switch to
  middleman, and we become less useful for filesharing, so the filesharing
  people back off, so we get more ops since there's less filesharing, so the
  filesharers come back, etc.]

in practice, plausible deniability is hypothetical and doesn't seem very
convincing. if ISPs find the activity antisocial, they don't care *why*
your computer is doing that behavior.

\subsection{Tor and blacklists}

Takedowns and efnet abuse and wikipedia complaints and irc
networks.

It was long expected that, alongside Tor's legitimate users, it would also
attract troublemakers who exploited Tor in order to abuse services on the
Internet.  Our initial answer to this situation was to use ``exit policies''
to allow individual Tor servers to block access to specific IP/port ranges.
This approach was meant to make operators more willing to run Tor by allowing
them to prevent their servers from being used for abusing particular
services.  For example, all Tor servers currently block SMTP (port 25), in
order to avoid being used to send spam.

This approach is useful, but is insufficient for two reasons.  First, since
it is not possible to force all ORs to block access to any given service,
many of those services try to block Tor instead.  More broadly, while being
blockable is important to being good netizens, we would like to encourage
services to allow anonymous access; services should not need to decide
between blocking legitimate anonymous use and allowing unlimited abuse.

This is potentially a bigger problem than it may appear. 
On the one hand, if people want to refuse connections from you on
their servers it would seem that they should be allowed to.  But, a
possible major problem with the blocking of Tor is that it's not just
the decision of the individual server administrator whose deciding if
he wants to post to Wikipedia from his Tor node address or allow
people to read Wikipedia anonymously through his Tor node. (Wikipedia
has blocked all posting from all Tor nodes based in IP address.) If e.g.,
s/he comes through a campus or corporate NAT, then the decision must
be to have the entire population behind it able to have a Tor exit
node or to have write access to Wikipedia. This is a loss for both of us (Tor
and Wikipedia). We don't want to compete for (or divvy up) the NAT
protected entities of the world.

(A related problem is that many IP blacklists are not terribly fine-grained.
No current IP blacklist, for example, allow a service provider to blacklist
only those Tor servers that allow access to a specific IP or port, even
though this information is readily available.  One IP blacklist even bans
every class C network that contains a Tor server, and recommends banning SMTP
from these networks even though Tor does not allow SMTP at all.)
[****Since this is stupid and we oppose it, shouldn't we name names here -pfs]


Problems of abuse occur mainly with services such as IRC networks and
Wikipedia, which rely on IP blocking to ban abusive users.  While at first
blush this practice might seem to depend on the anachronistic assumption that
each IP is an identifier for a single user, it is actually more reasonable in
practice: it assumes that non-proxy IPs are a costly resource, and that an
abuser can not change IPs at will.  By blocking IPs which are used by Tor
servers, open proxies, and service abusers, these systems hope to make
ongoing abuse difficult.  Although the system is imperfect, it works
tolerably well for them in practice.

But of course, we would prefer that legitimate anonymous users be able to
access abuse-prone services.  One conceivable approach would be to require
would-be IRC users, for instance, to register accounts if they wanted to
access the IRC network from Tor.  But in practise, this would not
significantly impede abuse if creating new accounts were easily automatable;
% XXX captcha
this is why services use IP blocking.  In order to deter abuse, pseudonymous
identities need to impose a significant switching cost in resources or human
time.

One approach, similar to that taken by Freedom, would be to bootstrap some
non-anonymous costly identification mechanism to allow access to a
blind-signature pseudonym protocol.  This would effectively create costly
pseudonyms, which services could require in order to allow anonymous access.
This approach has difficulties in practise, however:
\begin{tightlist}
\item Unlike Freedom, Tor is not a commercial service.  Therefore, it would
  be a shame to require payment in order to make Tor useful, or to make
  non-paying users second-class citizens.
\item It is hard to think of an underlying resource that would actually work.
  We could use IP addresses, but that's the problem, isn't it?
\item Managing single sign-on services is not considered a well-solved
  problem in practice.  If Microsoft can't get universal acceptance for
  Passport, why do we think that a Tor-specific solution would do any good?
\item Even if we came up with a perfect authentication system for our needs,
  there's no guarantee that any service would actually start using it.  It
  would require a nonzero effort for them to support it, and it might just
  be less hassle for them to block tor anyway.
\end{tightlist}

The use of squishy IP-based ``authentication'' and ``authorization''
has not broken down even to the level that SSNs used for these
purposes have in commercial and public record contexts. Externalities
and misplaced incentives cause a continued focus on fighting identity
theft by protecting SSNs rather than developing better authentication
and incentive schemes \cite{price-privacy}. Similarly we can expect a
continued use of identification by IP number as long as there is no
workable alternative.

%Fortunately, our modular design separates
%routing from node discovery; so we could implement Morphmix in Tor just
%by implementing the Morphmix-specific node discovery and path selection
%pieces.

\section{Crossroads: Scaling and Design choices}
\label{sec:crossroads-design}

\subsection{Transporting the stream vs transporting the packets}
\label{subsec:stream-vs-packet}
\label{subsec:tcp-vs-ip}

We periodically run into ex ZKS employees who tell us that the process of
anonymizing IPs should ``obviously'' be done at the IP layer. Here are
the issues that need to be resolved before we'll be ready to switch Tor
over to arbitrary IP traffic.

\begin{enumerate}
\setlength{\itemsep}{0mm}
\setlength{\parsep}{0mm}
\item \emph{IP packets reveal OS characteristics.} We still need to do
IP-level packet normalization, to stop things like IP fingerprinting
attacks. There likely exist libraries that can help with this.
\item \emph{Application-level streams still need scrubbing.} We still need
Tor to be easy to integrate with user-level application-specific proxies
such as Privoxy. So it's not just a matter of capturing packets and
anonymizing them at the IP layer.
\item \emph{Certain protocols will still leak information.} For example,
DNS requests destined for my local DNS servers need to be rewritten
to be delivered to some other unlinkable DNS server. This requires
understanding the protocols we are transporting.
\item \emph{The crypto is unspecified.} First we need a block-level encryption
approach that can provide security despite
packet loss and out-of-order delivery. Freedom allegedly had one, but it was
never publicly specified. %, and we believe it's likely vulnerable to tagging
%attacks \cite{tor-design}.
Also, TLS over UDP is not implemented or even
specified, though some early work has begun on that~\cite{dtls}.
\item \emph{We'll still need to tune network parameters}. Since the above
encryption system will likely need sequence numbers (and maybe more) to do
replay detection, handle duplicate frames, etc, we will be reimplementing
some subset of TCP anyway.
\item \emph{Exit policies for arbitrary IP packets mean building a secure
IDS.}  Our server operators tell us that exit policies are one of
the main reasons they're willing to run Tor.
Adding an Intrusion Detection System to handle exit policies would
increase the security complexity of Tor, and would likely not work anyway,
as evidenced by the entire field of IDS and counter-IDS papers. Many
potential abuse issues are resolved by the fact that Tor only transports
valid TCP streams (as opposed to arbitrary IP including malformed packets
and IP floods), so exit policies become even \emph{more} important as
we become able to transport IP packets. We also need a way to compactly
characterize the exit policies and let clients parse them to decide
which nodes will allow which packets to exit.
\item \emph{The Tor-internal name spaces would need to be redesigned.} We
support hidden service {\tt{.onion}} addresses, and other special addresses
like {\tt{.exit}} (see Section~\ref{subsec:}), by intercepting the addresses
when they are passed to the Tor client.
\end{enumerate}

This list is discouragingly long right now, but we recognize that it
would be good to investigate each of these items in further depth and to
understand which are actual roadblocks and which are easier to resolve
than we think. We certainly wouldn't mind if Tor one day is able to
transport a greater variety of protocols.

\subsection{Mid-latency}
\label{subsec:mid-latency}

Though Tor has always been designed to be practical and usable first
with as much anonymity as can be built in subject to those goals, we
have contemplated that users might need resistance to at least simple
traffic correlation attacks.  Higher-latency mix-networks resist these
attacks by introducing variability into message arrival times in order to
suppress timing correlation.  Thus, it seems worthwhile to consider the
whether we can improving Tor's anonymity by introducing batching and delaying
strategies to the Tor messages to prevent observers from linking incoming and
outgoing traffic.

Before we consider the engineering issues involved in the approach, of
course, we first need to study whether it can genuinely make users more
anonymous.  Research on end-to-end traffic analysis on higher-latency mix
networks~\cite{e2e-traffic} indicates that as timing variance decreases,
timing correlation attacks require increasingly less data; it might be the
case that Tor can't resist timing attacks for longer than a few minutes
without increasing message delays to an unusable degree.  Conversely, if Tor
can remain usable and slow timing attacks by even a matter of hours, this
would represent a significant improvement in practical anonymity: protecting
short-duration, once-off activities against a global observer is better than
protecting no activities at all.  In order to answer this question, we might
try to adapt the techniques of~\cite{e2e-traffic} to a lower-latency mix
network, where instead of sending uncorrelated messages, users send batches
of cells in temporally clustered connections.

Once the anonymity questions are answered, we need to consider usability.  If
the latency could be kept to two or three times its current overhead, this
might be acceptable to most Tor users. However, it might also destroy much of
the user base, and it is difficult to know in advance.  Note also that in
practice, as the network grows to incorporate more DSL and cable-modem nodes,
and more nodes in various continents, this alone will \emph{already} cause
many-second delays for some transactions.  Reducing this latency will be
hard, so perhaps it's worth considering whether accepting this higher latency
can improve the anonymity we provide.  Also, it could be possible to
run a mid-latency option over the Tor network for those
users either willing to experiment or in need of more
anonymity.  This would allow us to experiment with both
the anonymity provided and the interest on the part of users.

Adding a mid-latency option should not require significant fundamental
change to the Tor client or server design; circuits could be labeled as
low- or mid- latency as they are constructed. Low-latency traffic
would be processed as now, while cells on on circuits that are mid-latency
would be sent in uniform-size chunks at synchronized intervals.  (Traffic
already moves through the Tor network in fixed-sized cells; this would
increase the granularity.)  If servers forward these chunks in roughly
synchronous  fashion, it will increase the similarity of data stream timing
signatures. By experimenting with the granularity of data chunks and
of synchronization we can attempt once again to optimize for both
usability and anonymity. Unlike in \cite{sync-batching}, it may be
impractical to synchronize on network batches by dropping chunks from
a batch that arrive late at a given node---unless Tor moves away from
stream processing to a more loss-tolerant paradigm (cf.\
Section~\ref{subsec:tcp-vs-ip}). Instead, batch timing would be obscured by
synchronizing batches at the link level, and there would
be no direct attempt to synchronize all batches
entering the Tor network at the same time.
%Alternatively, if end-to-end traffic correlation is the
%concern, there is little point in mixing.
%   Why not?? -NM
It might also be feasible to
pad chunks to uniform size as is done now for cells; if this is link
padding rather than end-to-end, then it will take less overhead,
especially in bursty environments.
% This is another way in which it
%would be fairly practical to set up a mid-latency option within the
%existing Tor network.
Other padding regimens might supplement the
mid-latency option; however, we should continue the caution with which
we have always approached padding lest the overhead cost us too much
performance or too many volunteers.

The distinction between traffic correlation and traffic analysis is
not as cut and dried as we might wish. In \cite{hintz-pet02} it was
shown that if data volumes of various popular
responder destinations are catalogued, it may not be necessary to
observe both ends of a stream to learn a source-destination link.
This should be fairly effective without simultaneously observing both
ends of the connection. However, it is still essentially confirming
suspected communicants where the responder suspects are ``stored'' rather
than observed at the same time as the client.
Similarly latencies of going through various routes can be
catalogued~\cite{back01} to connect endpoints.
This is likely to entail high variability and massive storage since
% XXX hintz-pet02 just looked at data volumes of the sites. this
% doesn't require much variability or storage. I think it works
% quite well actually. Also, \cite{kesdogan:pet2002} takes the
% attack another level further, to narrow down where you could be
% based on an intersection attack on subpages in a website. -RD
%
% I was trying to be terse and simultaneously referring to both the
% Hintz stuff and the Back et al. stuff from Info Hiding 01. I've
% separated the two and added the references. -PFS
routes through the network to each site will be random even if they
have relatively unique latency characteristics. So the do
not seem an immediate practical threat. Further along similar lines,
the same paper suggested a ``clogging attack''. A version of this
was demonstrated to be practical in
\cite{attack-tor-oak05}. There it was shown that an outside attacker can
trace a stream through the Tor network while a stream is still active
simply by observing the latency of his own traffic sent through
various Tor nodes. These attacks are especially significant since they
counter previous results that running one's own onion router protects
better than using the network from the outside. The attacks do not
show the client address, only the first server within the Tor network,
making helper nodes all the more worthy of exploration for enclave
protection. Setting up a mid-latency subnet as described above would
be another significant step to evaluating resistance to such attacks.

The attacks in \cite{attack-tor-oak05} are also dependent on
cooperation of the responding application or the ability to modify or
monitor the responder stream, in order of decreasing attack
effectiveness.  So, another way to slow some of these attacks
would be to cache responses at exit servers where possible, as it is with
DNS lookups and cacheable HTTP responses.  Caching would, however,
create threats of its own.
%To be
%useful, such caches would need to be distributed to any likely exit
%nodes of recurred requests for the same data.
%   Even local caches could be useful, I think. -NM
Aside from the logistic
difficulties and overhead, caches would  constitute a
record of destinations and data visited by Tor users.  While
limited to network insiders, given the need for wide distribution
they could serve as useful data to an attacker deciding which locations
to target for confirmation. A way to counter this distribution
threat might be to only cache at certain semitrusted helper nodes.

%Does that cacheing discussion belong in low-latency?

\subsection{Application support: SOCKS and beyond}

Tor supports the SOCKS protocol, which provides a standardized interface for
generic TCP proxies.  Unfortunately, this is not a complete solution for
many applications and platforms:
\begin{tightlist}
\item Many applications do not support SOCKS. To support such applications,
  it's necessary to replace the networking system calls with SOCKS-aware
  versions, or to run a local SOCKS tunnel and convince the applications to
  connect to localhost.  Neither of these tasks is easy for the average user,
  even with good instructions.
\item Even when applications do use SOCKS, they often make DNS requests
  themselves.  (The various versions of the SOCKS protocol include some where
  the application tells the proxy an IP address, and some where it sends a
  hostname.)  By connecting to the DNS sever directly, the application breaks
  the user's anonymity and advertises where it is about to connect.
\end{tightlist}

So in order to actually provide good anonymity, we need to make sure that
users have a practical way to use Tor anonymously.  Possibilities include
writing wrappers for applications to anonymize them automatically; improving
the applications' support for SOCKS; writing libraries to help application
writers use Tor properly; and implementing a local DNS proxy to reroute DNS
requests to Tor so that applications can simply point their DNS resolvers at
localhost and continue to use SOCKS for data only.

\subsection{Measuring performance and capacity}
One of the paradoxes with engineering an anonymity network is that we'd like
to learn as much as we can about how traffic flows so we can improve the
network, but we want to prevent others from learning how traffic flows in
order to trace users' connections through the network.  Furthermore, many
mechanisms that help Tor run efficiently (such as having clients choose servers
based on their capacities) require measurements about the network.

Currently, servers record their bandwidth use in 15-minute intervals and
include this information in the descriptors they upload to the directory.
They also try to deduce their own available bandwidth, on the basis of how
much traffic they have been able to transfer recently, and upload this
information as well.

This is, of course, eminantly cheatable.  A malicious server can get a
disproportionate amount of traffic simply by claiming to have more bandiwdth
than it does.  But better mechanisms have their problems.  If bandwidth data
is to be measured rather than self-reported, it is usually possible for
servers to selectively provide better service for the measuring party, or
sabotage the measured value of other servers.  Complex solutions for
mix networks have been proposed, but do not address the issues
completely~\cite{mix-acc,casc-rep}.

Even without the possibility of cheating, network measurement is
non-trivial.  It is far from unusual for one observer's view of a server's
latency or bandwidth to disagree wildly with another's.  Furthermore, it is
unclear whether total bandwidth is really the right measure; perhaps clients
should be considering servers on the basis of unused bandwidth instead, or
perhaps observed throughput.
% XXXX say more here?

%How to measure performance without letting people selectively deny service
%by distinguishing pings. Heck, just how to measure performance at all. In
%practice people have funny firewalls that don't match up to their exit
%policies and Tor doesn't deal.

%Network investigation: Is all this bandwidth publishing thing a good idea?
%How can we collect stats better? Note weasel's smokeping, at
%http://seppia.noreply.org/cgi-bin/smokeping.cgi?target=Tor
%which probably gives george and steven enough info to break tor?

Even if we can collect and use this network information effectively, we need
to make sure that it is not more useful to attackers than to us.  While it
seems plausible that bandwidth data alone is not enough to reveal
sender-recipient connections under most circumstances, it could certainly
reveal the path taken by large traffic flows under low-usage circumstances.

\subsection{Running a Tor server, path length, and helper nodes}

It has been thought for some time that the best anonymity protection
comes from running your own onion router~\cite{or-pet00,tor-design}.
(In fact, in Onion Routing's first design, this was the only option
possible~\cite{or-ih96}.) The first design also had a fixed path
length of five nodes. Middle Onion Routing involved much analysis
(mostly unpublished) of route selection algorithms and path length
algorithms to combine efficiency with unpredictability in routes.
Since, unlike Crowds, nodes in a route cannot all know the ultimate
destination of an application connection, it was generally not
considered significant if a node could determine via latency that it
was second in the route. But if one followed Tor's three node default
path length, an enclave-to-enclave communication (in which two of the
ORs were at each enclave) would be completely compromised by the
middle node. Thus for enclave-to-enclave communication, four is the fewest
number of nodes that preserves the $\frac{c^2}{n^2}$ degree of protection
in any setting.

The Murdoch-Danezis attack, however, shows that simply adding to the
path length may not protect usage of an enclave protecting OR\@.  A
hostile web server can determine all of the nodes in a three node Tor
path. The attack only identifies that a node is on the route, not
where. For example, if all of the nodes on the route were enclave
nodes, the attack would not identify which of the two not directly
visible to the attacker was the source.  Thus, there remains an
element of plausible deniability that is preserved for enclave nodes.
However, Tor has always sought to be stronger than plausible
deniability. Our assumption is that users of the network are concerned
about being identified by an adversary, not with being proven guilty
beyond any reasonable doubt. Still it is something, and may be desired
in some settings.

It is reasonable to think that this attack can be easily extended to
longer paths should those be used; nonetheless there may be some
advantage to random path length. If the number of nodes is unknown,
then the adversary would need to send streams to all the nodes in the
network and analyze the resulting latency from them to be reasonably
certain that it has not missed the first node in the circuit. Also,
the attack does not identify the order of nodes in a route, so the
longer the route, the greater the uncertainty about which node might
be first. It may be possible to extend the attack to learn the route
node order, but it is not clear that this is practically feasible.

Another way to reduce the threats to both enclaves and simple Tor
clients is to have helper nodes. Helper nodes were introduced
in~\cite{wright03} as a suggested means of protecting the identity
of the initiator of a communication in various anonymity protocols.
The idea is to use a single trusted node as the first one you go to,
that way an attacker cannot ever attack the first nodes you connect
to and do some form of intersection attack. This will not affect the
Danezis-Murdoch attack at all.

We have to pick the path length so adversary can't distinguish client from
server (how many hops is good?).

\subsection{Helper nodes}
\label{subsec:helper-nodes}

Tor can only provide anonymity against an attacker if that attacker can't
monitor the user's entry and exit on the Tor network.  But since Tor
currently chooses entry and exit points randomly and changes them frequently,
a patient attacker who controls a single entry and a single exit is sure to
eventually break some circuits of frequent users who consider those servers.
(We assume that users are as concerned about statistical profiling as about
the anonymity any particular connection.  That is, it is almost as bad to
leak the fact that Alice {\it sometimes} talks to Bob as it is to leak the times
when Alice is {\it actually} talking to Bob.)


One solution to this problem is to use ``helper nodes''~\cite{wright02,wright03}---to
have each client choose a few fixed servers for critical positions in her
circuits.  That is, Alice might choose some server H1 as her preferred
entry, so that unless the attacker happens to control or observe her
connection to H1, her circuits will remain anonymous.  If H1 is compromised,
Alice is vunerable as before.  But now, at least, she has a chance of
not being profiled.

(Choosing fixed exit nodes is less useful, since the connection from the exit
node to Alice's destination will be seen not only by the exit but by the
destination.  Even if Alice chooses a good fixed exit node, she may
nevertheless connect to a hostile website.)

There are still obstacles remaining before helper nodes can be implemented.
For one, the litereature does not describe how to choose helpers from a list
of servers that changes over time.  If Alice is forced to choose a new entry
helper every $d$ days, she can expect to choose a compromised server around
every $dc/n$ days.  Worse, an attacker with the ability to DoS servers could
force their users to switch helper nodes more frequently.

%Do general DoS attacks have anonymity implications? See e.g. Adam
%Back's IH paper, but I think there's more to be pointed out here. -RD
% Not sure what you want to say here. -NM

%Game theory for helper nodes: if Alice offers a hidden service on a
%server (enclave model), and nobody ever uses helper nodes, then against
%George+Steven's attack she's totally nailed. If only Alice uses a helper
%node, then she's still identified as the source of the data. If everybody
%uses a helper node (including Alice), then the attack identifies the
%helper node and also Alice, and knows which one is which. If everybody
%uses a helper node (but not Alice), then the attacker figures the real
%source was a client that is using Alice as a helper node. [How's my
%logic here?] -RD
%
% Not sure about the logic.  For the attack to work with helper nodes, the
%attacker needs to guess that Alice is running the hidden service, right?
%Otherwise, how can he know to measure her traffic specifically? -NM

%point to routing-zones section re: helper nodes to defend against
%big stuff.

\subsection{Location-hidden services}

While most of the discussions about have been about forward anonymity
with Tor, it also provides support for \emph{rendezvous points}, which
let users provide TCP services to other Tor users without revealing
their location. Since this feature is relatively recent, we describe here
a couple of our early observations from its deployment.

First, our implementation of hidden services seems less hidden than we'd
like, since they are configured on a single client and get used over
and over---particularly because an external adversary can induce them to
produce traffic. They seem the ideal use case for our above discussion
of helper nodes. This insecurity means that they may not be suitable as
a building block for Free Haven~\cite{freehaven-berk} or other anonymous
publishing systems that aim to provide long-term security.
%Also, they're brittle in terms of intersection and observation attacks.

\emph{Hot-swap} hidden services, where more than one location can
provide the service and loss of any one location does not imply a
change in service, would help foil intersection and observation attacks
where an adversary monitors availability of a hidden service and also
monitors whether certain users or servers are online. However, the design
challenges in providing these services without otherwise compromising
the hidden service's anonymity remain an open problem.

In practice, hidden services are used for more than just providing private
access to a web server or IRC server. People are using hidden services
as a poor man's VPN and firewall-buster. Many people want to be able
to connect to the computers in their private network via secure shell,
and rather than playing with dyndns and trying to pierce holes in their
firewall, they run a hidden service on the inside and then rendezvous
with that hidden service externally.

Also, sites like Bloggers Without Borders (www.b19s.org) are advertising
a hidden-service address on their front page. Doing this can provide
increased robustness if they use the dual-IP approach we describe in
tor-design, but in practice they do it firstly to increase visibility
of the tor project and their support for privacy, and secondly to offer
a way for their users, using unmodified software, to get end-to-end
encryption and end-to-end authentication to their website.

\subsection{Trust and discovery}

[arma will edit this and expand/retract it]

The published Tor design adopted a deliberately simplistic design for
authorizing new nodes and informing clients about servers and their status.
In the early Tor designs, all ORs periodically uploaded a signed description
of their locations, keys, and capabilities to each of several well-known {\it
  directory servers}.  These directory servers constructed a signed summary
of all known ORs (a ``directory''), and a signed statement of which ORs they
believed to be operational at any given time (a ``network status'').  Clients
periodically downloaded a directory in order to learn the latest ORs and
keys, and more frequently downloaded a network status to learn which ORs are
likely to be running.  ORs also operate as directory caches, in order to
lighten the bandwidth on the authoritative directory servers.

In order to prevent Sybil attacks (wherein an adversary signs up many
purportedly independent servers in order to increase her chances of observing
a stream as it enters and leaves the network), the early Tor directory design
required the operators of the authoritative directory servers to manually
approve new ORs.  Unapproved ORs were included in the directory, but clients
did not use them at the start or end of their circuits.  In practice,
directory administrators performed little actual verification, and tended to
approve any OR whose operator could compose a coherent email.  This procedure
may have prevented trivial automated Sybil attacks, but would do little
against a clever attacker.

There are a number of flaws in this system that need to be addressed as we
move forward.  They include:
\begin{tightlist}
\item Each directory server represents an independent point of failure; if
  any one were compromised, it could immediately compromise all of its users
  by recommending only compromised ORs.
\item The more servers appear join the network, the more unreasonable it
  becomes to expect clients to know about them all.  Directories
  become unfeasibly large, and downloading the list of servers becomes
  burdonsome.
\item The validation scheme may do as much harm as it does good.  It is not
  only incapable of preventing clever attackers from mounting Sybil attacks,
  but may deter server operators from joining the network.  (For instance, if
  they expect the validation process to be difficult, or if they do not share
  any languages in common with the directory server operators.)
\end{tightlist}

We could try to move the system in several directions, depending on our
choice of threat model and requirements.  If we did not need to increase
network capacity in order to support more users, there would be no reason not
to adopt even stricter validation requirements, and reduce the number of
servers in the network to a trusted minimum.  But since we want Tor to work
for as many users as it can, we need XXXXX

In order to address the first two issues, it seems wise to move to a system
including a number of semi-trusted directory servers, no one of which can
compromise a user on its own.  Ultimately, of course, we cannot escape the
problem of a first introducer: since most users will run Tor in whatever
configuration the software ships with, the Tor distribution itself will
remain a potential single point of failure so long as it includes the seed
keys for directory servers, a list of directory servers, or any other means
to learn which servers are on the network.  But omitting this information
from the Tor distribution would only delegate the trust problem to the
individual users, most of whom are presumably less informed about how to make
trust decisions than the Tor developers.

%Network discovery, sybil, node admission, scaling. It seems that the code
%will ship with something and that's our trust root. We could try to get
%people to build a web of trust, but no. Where we go from here depends
%on what threats we have in mind. Really decentralized if your threat is
%RIAA; less so if threat is to application data or individuals or...

\section{Crossroads: Scaling}
%\label{sec:crossroads-scaling}
%P2P + anonymity issues:

Tor is running today with hundreds of servers and tens of thousands of
users, but it will certainly not scale to millions.

Scaling Tor involves three main challenges.  First is safe server
discovery, both bootstrapping -- how a Tor client can robustly find an
initial server list -- and ongoing -- how a Tor client can learn about
a fair sample of honest servers and not let the adversary control his
circuits (see Section x).  Second is detecting and handling the speed
and reliability of the variety of servers we must use if we want to
accept many servers (see Section y).
Since the speed and reliability of a circuit is limited by its worst link,
we must learn to track and predict performance.  Finally, in order to get
a large set of servers in the first place, we must address incentives
for users to carry traffic for others (see Section incentives).

\subsection{Incentives by Design}

[nick will try to make this section shorter and more to the point.]

[most of the technical incentive schemes in the literature introduce
anonymity issues which we don't understand yet, and we seem to be doing
ok without them]

There are three behaviors we need to encourage for each server: relaying
traffic; providing good throughput and reliability while doing it;
and allowing traffic to exit the network from that server.

We encourage these behaviors through \emph{indirect} incentives, that
is, designing the system and educating users in such a way that users
with certain goals will choose to relay traffic.  In practice, the
main incentive for running a Tor server is social benefit: volunteers
altruistically donate their bandwidth and time.  We also keep public
rankings of the throughput and reliability of servers, much like
seti@home.  We further explain to users that they can get \emph{better
security} by operating a server, because they get plausible deniability
(indeed, they may not need to route their own traffic through Tor at all
-- blending directly with other traffic exiting Tor may be sufficient
protection for them), and because they can use their own Tor server
as entry or exit point and be confident it's not run by the adversary.
Finally, we can improve the usability and feature set of the software:
rate limiting support and easy packaging decrease the hassle of
maintaining a server, and our configurable exit policies allow each
operator to advertise a policy describing the hosts and ports to which
he feels comfortable connecting.

Beyond these, however, there is also a need for \emph{direct} incentives:
providing payment or other resources in return for high-quality service.
Paying actual money is problematic: decentralized e-cash systems are
not yet practical, and a centralized collection system not only reduces
robustness, but also has failed in the past (the history of commercial
anonymizing networks is littered with failed attempts).  A more promising
option is to use a tit-for-tat incentive scheme: provide better service
to nodes that have provided good service to you.

Unfortunately, such an approach introduces new anonymity problems.
Does the incentive system enable the adversary to attract more traffic by
performing well? Typically a user who chooses evenly from all options is
most resistant to an adversary targetting him, but that approach prevents
us from handling heterogeneous servers \cite{casc-rep}.
When a server (call him Steve) performs well for Alice, does Steve gain
reputation with the entire system, or just with Alice? If the entire
system, how does Alice tell everybody about her experience in a way that
prevents her from lying about it yet still protects her identity? If
Steve's behavior only affects Alice's behavior, does this allow Steve to
selectively perform only for Alice, and then break her anonymity later
when somebody (presumably Alice) routes through his node?

These are difficult and open questions, yet choosing not to scale means
leaving most users to a less secure network or no anonymizing network
at all.  We will start with a simplified approach to the tit-for-tat
incentive scheme based on two rules: (1) each node should measure the
service it receives from adjacent nodes, and provide service relative to
the received service, but (2) when a node is making decisions that affect
its own security (e.g. when building a circuit for its own application
connections), it should choose evenly from a sufficiently large set of
nodes that meet some minimum service threshold.  This approach allows us
to discourage bad service without opening Alice up as much to attacks.

%XXX rewrite the above so it sounds less like a grant proposal and
%more like a "if somebody were to try to solve this, maybe this is a
%good first step".

%We should implement the above incentive scheme in the
%deployed Tor network, in conjunction with our plans to add the necessary
%associated scalability mechanisms.  We will do experiments (simulated
%and/or real) to determine how much the incentive system improves
%efficiency over baseline, and also to determine how far we are from
%optimal efficiency (what we could get if we ignored the anonymity goals).

\subsection{Peer-to-peer / practical issues}

[leave this section for now, and make sure things here are covered
elsewhere. then remove it.]

Making use of servers with little bandwidth. How to handle hammering by
certain applications.

Handling servers that are far away from the rest of the network, e.g. on
the continents that aren't North America and Europe. High latency,
often high packet loss.

Running Tor servers behind NATs, behind great-firewalls-of-China, etc.
Restricted routes. How to propagate to everybody the topology? BGP
style doesn't work because we don't want just *one* path. Point to
Geoff's stuff.

\subsection{Location diversity and ISP-class adversaries}
\label{subsec:routing-zones}

Anonymity networks have long relied on diversity of node location for
protection against attacks---typically an adversary who can observe a
larger fraction of the network can launch a more effective attack. One
way to achieve dispersal involves growing the network so a given adversary
sees less. Alternately, we can arrange the topology so traffic can enter
or exit at many places (for example, by using a free-route network
like Tor rather than a cascade network like JAP). Lastly, we can use
distributed trust to spread each transaction over multiple jurisdictions.
But how do we decide whether two nodes are in related locations?

Feamster and Dingledine defined a \emph{location diversity} metric
in \cite{feamster:wpes2004}, and began investigating a variant of location
diversity based on the fact that the Internet is divided into thousands of
independently operated networks called {\em autonomous systems} (ASes).
The key insight from their paper is that while we typically think of a
connection as going directly from the Tor client to her first Tor node,
actually it traverses many different ASes on each hop. An adversary at
any of these ASes can monitor or influence traffic. Specifically, given
plausible initiators and recipients and path random path selection,
some ASes in the simulation were able to observe 10\% to 30\% of the
transactions (that is, learn both the origin and the destination) on
the deployed Tor network (33 nodes as of June 2004).

The paper concludes that for best protection against the AS-level
adversary, nodes should be in ASes that have the most links to other ASes:
Tier-1 ISPs such as AT\&T and Abovenet. Further, a given transaction
is safest when it starts or ends in a Tier-1 ISP. Therefore, assuming
initiator and responder are both in the U.S., it actually \emph{hurts}
our location diversity to add far-flung nodes in continents like Asia
or South America.

Many open questions remain. First, it will be an immense engineering
challenge to get an entire BGP routing table to each Tor client, or at
least summarize it sufficiently. Without a local copy, clients won't be
able to safely predict what ASes will be traversed on the various paths
through the Tor network to the final destination. Tarzan~\cite{tarzan:ccs02}
and MorphMix~\cite{morphmix:fc04} suggest that we compare IP prefixes to
determine location diversity; but the above paper showed that in practice
many of the Mixmaster nodes that share a single AS have entirely different
IP prefixes. When the network has scaled to thousands of nodes, does IP
prefix comparison become a more useful approximation?
%
Second, can take advantage of caching certain content at the exit nodes, to
limit the number of requests that need to leave the network at all.
what about taking advantage of caches like akamai's or googles? what
about treating them as adversaries?
%
Third, if we follow the paper's recommendations and tailor path selection
to avoid choosing endpoints in similar locations, how much are we hurting
anonymity against larger real-world adversaries who can take advantage
of knowing our algorithm?
%
Lastly, can we use this knowledge to figure out which gaps in our network
would most improve our robustness to this class of attack, and go recruit
new servers with those ASes in mind?

Tor's security relies in large part on the dispersal properties of its
network. We need to be more aware of the anonymity properties of various
approaches we can make better design decisions in the future.

\subsection{The China problem}
\label{subsec:china}

Citizens in a variety of countries, such as most recently China and
Iran, are periodically blocked from accessing various sites outside
their country. These users try to find any tools available to allow
them to get-around these firewalls. Some anonymity networks, such as
Six-Four~\cite{six-four}, are designed specifically with this goal in
mind; others like the Anonymizer~\cite{anonymizer} are paid by sponsors
such as Voice of America to set up a network to encourage Internet
freedom. Even though Tor wasn't
designed with ubiquitous access to the network in mind, thousands of
users across the world are trying to use it for exactly this purpose.
% Academic and NGO organizations, peacefire, \cite{berkman}, etc

Anti-censorship networks hoping to bridge country-level blocks face
a variety of challenges. One of these is that they need to find enough
exit nodes---servers on the `free' side that are willing to relay
arbitrary traffic from users to their final destinations. Anonymizing
networks including Tor are well-suited to this task, since we have
already gathered a set of exit nodes that are willing to tolerate some
political heat.

The other main challenge is to distribute a list of reachable relays
to the users inside the country, and give them software to use them,
without letting the authorities also enumerate this list and block each
relay. Anonymizer solves this by buying lots of seemingly-unrelated IP
addresses (or having them donated), abandoning old addresses as they are
`used up', and telling a few users about the new ones. Distributed
anonymizing networks again have an advantage here, in that we already
have tens of thousands of separate IP addresses whose users might
volunteer to provide this service since they've already installed and use
the software for their own privacy~\cite{koepsell:wpes2004}. Because
the Tor protocol separates routing from network discovery (see Section
\ref{do-we-discuss-this?}), volunteers could configure their Tor clients
to generate server descriptors and send them to a special directory
server that gives them out to dissidents who need to get around blocks.

Of course, this still doesn't prevent the adversary
from enumerating all the volunteer relays and blocking them preemptively.
Perhaps a tiered-trust system could be built where a few individuals are
given relays' locations, and they recommend other individuals by telling them
those addresses, thus providing a built-in incentive to avoid letting the
adversary intercept them. Max-flow trust algorithms~\cite{advogato}
might help to bound the number of IP addresses leaked to the adversary. Groups
like the W3C are looking into using Tor as a component in an overall system to
help address censorship; we wish them luck.

%\cite{infranet}

\subsection{Non-clique topologies}

Tor's comparatively  weak model makes it easier to scale than other mix net
designs.  High-latency mix networks need to avoid partitioning attacks, where
network splits prevent users of the separate partitions from providing cover
for each other.  In Tor, however, we assume that the adversary cannot
cheaply observe nodes at will, so even if the network becomes split, the
users do not necessarily receive much less protection.
Thus, a simple possibility when the scale of a Tor network
exceeds some size is to simply split it. Care could be taken in
allocating which nodes go to which network along the lines of
\cite{casc-rep} to insure that collaborating hostile nodes are not
able to gain any advantage in network splitting that they do not
already have in joining a network.

% Describe these attacks; many people will not have read the paper!
The attacks in \cite{attack-tor-oak05} show that certain types of
brute force attacks are in fact feasible; however they make the
above point stronger not weaker. The attacks do not appear to be
significantly more difficult to mount against a network that is
twice the size. Also, they only identify the Tor nodes used in a
circuit, not the client. Finally note that even if the network is split,
a client does not need to use just one of the two resulting networks.
Alice could use either of them, and it would not be difficult to make
the Tor client able to access several such network on a per circuit
basis. More analysis is needed; we simply note here that splitting
a Tor network is an easy way to achieve moderate scalability and that
it does not necessarily have the same implications as splitting a mixnet.

Alternatively, we can try to scale a single Tor network.  Some issues for
scaling include restricting the number of sockets and the amount of bandwidth
used by each server.  The number of sockets is determined by the network's
connectivity and the number of users, while bandwidth capacity is determined
by the total bandwidth of servers on the network.  The simplest solution to
bandwidth capacity is to add more servers, since adding a tor node of any
feasible bandwidth will increase the traffic capacity of the network.  So as
a first step to scaling, we should focus on making the network tolerate more
servers, by reducing the interconnectivity of the nodes; later we can reduce
overhead associated withy directories, discovery, and so on.

By reducing the connectivity of the network we increase the total number of
nodes that the network can contain. Danezis~\cite{danezis-pets03} considers
the anonymity implications of restricting routes on mix networks, and
recommends an approach based on expander graphs (where any subgraph is likely
to have many neighbors).  It is not immediately clear that this approach will
extend to Tor, which has a weaker threat model but higher performance
requirements than the network considered.  Instead of analyzing the
probability of an attacker's viewing whole paths, we will need to examine the
attacker's likelihood of compromising the endpoints of a Tor circuit through
a sparse network.

% Nick edits these next 2 grafs.

To make matters simpler, Tor may not need an expander graph per se: it
may be enough to have a single subnet that is highly connected.  As an
example, assume fifty nodes of relatively high traffic capacity.  This
\emph{center} forms are a clique.  Assume each center node can each
handle 200 connections to other nodes (including the other ones in the
center). Assume every noncenter node connects to three nodes in the
center and anyone out of the center that they want to.  Then the
network easily scales to c. 2500 nodes with commensurate increase in
bandwidth. There are many open questions: how directory information
is distributed (presumably information about the center nodes could
be given to any new nodes with their codebase), whether center nodes
will need to function as a `backbone', etc. As above the point is
that this would create problems for the expected anonymity for a mixnet,
but for an onion routing network where anonymity derives largely from
the edges, it may be feasible.

Another point is that we already have a non-clique topology.
Individuals can set up and run Tor nodes without informing the
directory servers. This will allow, e.g., dissident groups to run a
local Tor network of such nodes that connects to the public Tor
network. This network is hidden behind the Tor network and its
only visible connection to Tor at those points where it connects.
As far as the public network is concerned or anyone observing it,
they are running clients.

\section{The Future}
\label{sec:conclusion}

we should put random thoughts here until there are enough for a
conclusion.

will our sustainability approach work? we'll see.

Applications that leak data: we can say they're not our problem, but
they're somebody's problem.
The more widely deployed Tor becomes, the more people who need a
deployed overlay network tell us they'd like to use us if only we added
the following more features.

"These are difficult and open questions, yet choosing not to solve them
means leaving most users to a less secure network or no anonymizing
network at all."

\bibliographystyle{plain} \bibliography{tor-design}

\clearpage
\appendix

\begin{figure}[t]
%\unitlength=1in
\centering
%\begin{picture}(6.0,2.0)
%\put(3,1){\makebox(0,0)[c]{\epsfig{figure=graphnodes,width=6in}}}
%\end{picture}
\mbox{\epsfig{figure=graphnodes,width=5in}}
\caption{Number of servers over time. Lowest line is number of exit
nodes that allow connections to port 80. Middle line is total number of
verified (registered) servers. The line above that represents servers
that are not yet registered.}
\label{fig:graphnodes}
\end{figure}

\begin{figure}[t]
\centering
\mbox{\epsfig{figure=graphtraffic,width=5in}}
\caption{The sum of traffic reported by each server over time. The bottom
pair show average throughput, and the top pair represent the largest 15
minute burst in each 4 hour period.}
\label{fig:graphtraffic}
\end{figure}

\end{document}

