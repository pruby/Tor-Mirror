\documentclass{llncs}

\usepackage{url}
\usepackage{amsmath}
\usepackage{epsfig}

\setlength{\textwidth}{5.9in}
\setlength{\textheight}{8.4in}
\setlength{\topmargin}{.5cm}
\setlength{\oddsidemargin}{1cm}
\setlength{\evensidemargin}{1cm}

\newenvironment{tightlist}{\begin{list}{$\bullet$}{
  \setlength{\itemsep}{0mm}
    \setlength{\parsep}{0mm}
    %  \setlength{\labelsep}{0mm}
    %  \setlength{\labelwidth}{0mm}
    %  \setlength{\topsep}{0mm}
    }}{\end{list}}

\begin{document}

\title{Challenges in deploying low-latency anonymity (DRAFT)}

\author{Roger Dingledine\inst{1} \and
Nick Mathewson\inst{1} \and
Paul Syverson\inst{2}}
\institute{The Free Haven Project \email{<\{arma,nickm\}@freehaven.net>} \and
Naval Research Laboratory \email{<syverson@itd.nrl.navy.mil>}}

\maketitle
\pagestyle{plain}

\begin{abstract}
  There are many unexpected or unexpectedly difficult obstacles to
  deploying anonymous communications.  Drawing on our experiences deploying
  Tor (the second-generation onion routing network), we describe social
  challenges and technical issues that must be faced
  in building, deploying, and sustaining a scalable, distributed, low-latency
  anonymity network.
\end{abstract}

\section{Introduction}
% Your network is not practical unless it is sustainable and distributed.
Anonymous communication is full of surprises.  This paper discusses some
unexpected challenges arising from our experiences deploying Tor, a
low-latency general-purpose anonymous communication system.  We will discuss
some of the difficulties we have experienced and how we have met them (or how
we plan to meet them, if we know).  We also discuss some less
troublesome open problems that we must nevertheless eventually address.
%We will describe both those future challenges that we intend to explore and
%those that we have decided not to explore and why.

Tor is an overlay network for anonymizing TCP streams over the
Internet~\cite{tor-design}.  It addresses limitations in earlier Onion
Routing designs~\cite{or-ih96,or-jsac98,or-discex00,or-pet00} by adding
perfect forward secrecy, congestion control, directory servers, data
integrity, configurable exit policies, and location-hidden services using
rendezvous points.  Tor works on the real-world Internet, requires no special
privileges or kernel modifications, requires little synchronization or
coordination between nodes, and provides a reasonable trade-off between
anonymity, usability, and efficiency.

We deployed the public Tor network in October 2003; since then it has
grown to over a hundred volunteer-operated nodes
and as much as 80 megabits of
average traffic per second.  Tor's research strategy has focused on deploying
a network to as many users as possible; thus, we have resisted designs that
would compromise deployability by imposing high resource demands on node
operators, and designs that would compromise usability by imposing
unacceptable restrictions on which applications we support.  Although this
strategy has
drawbacks (including a weakened threat model, as discussed below), it has
made it possible for Tor to serve many thousands of users and attract
funding from diverse sources whose goals range from security on a
national scale down to individual liberties.

In~\cite{tor-design} we gave an overall view of Tor's
design and goals.  Here we describe some policy, social, and technical
issues that we face as we continue deployment.
Rather than providing complete solutions to every problem, we
instead lay out the challenges and constraints that we have observed while
deploying Tor.  In doing so, we aim to provide a research agenda
of general interest to projects attempting to build
and deploy practical, usable anonymity networks in the wild.

%While the Tor design paper~\cite{tor-design} gives an overall view its
%design and goals,
%this paper describes the policy and technical issues that Tor faces as
%we continue deployment. Rather than trying to provide complete solutions
%to every problem here, we lay out the assumptions and constraints
%that we have observed through deploying Tor in the wild. In doing so, we
%aim to create a research agenda for others to
%help in addressing these issues.
% Section~\ref{sec:what-is-tor} gives an
%overview of the Tor
%design and ours goals. Sections~\ref{sec:crossroads-policy}
%and~\ref{sec:crossroads-design} go on to describe the practical challenges,
%both policy and technical respectively,
%that stand in the way of moving
%from a practical useful network to a practical useful anonymous network.

%\section{What Is Tor}
\section{Background}
Here we give a basic overview of the Tor design and its properties, and
compare Tor to other low-latency anonymity designs.

\subsection{Tor, threat models, and distributed trust}
\label{sec:what-is-tor}

%Here we give a basic overview of the Tor design and its properties. For
%details on the design, assumptions, and security arguments, we refer
%the reader to the Tor design paper~\cite{tor-design}.

Tor provides \emph{forward privacy}, so that users can connect to
Internet sites without revealing their logical or physical locations
to those sites or to observers.  It also provides \emph{location-hidden
services}, so that servers can support authorized users without
giving an effective vector for physical or online attackers.
Tor provides these protections even when a portion of its
infrastructure is compromised.

To connect to a remote server via Tor, the client software learns a signed
list of Tor nodes from one of several central \emph{directory servers}, and
incrementally creates a private pathway or \emph{circuit} of encrypted
connections through authenticated Tor nodes on the network, negotiating a
separate set of encryption keys for each hop along the circuit.  The circuit
is extended one node at a time, and each node along the way knows only the
immediately previous and following nodes in the circuit, so no individual Tor
node knows the complete path that each fixed-sized data packet (or
\emph{cell}) will take.
%Because each node sees no more than one hop in the
%circuit,
Thus, neither an eavesdropper nor a compromised node can
see both the connection's source and destination.  Later requests use a new
circuit, to complicate long-term linkability between different actions by
a single user.

Tor also helps servers hide their locations while
providing services such as web publishing or instant
messaging.  Using ``rendezvous points'', other Tor users can
connect to these authenticated hidden services, neither one learning the
other's network identity.

Tor attempts to anonymize the transport layer, not the application layer.
This approach is useful for applications such as SSH
where authenticated communication is desired. However, when anonymity from
those with whom we communicate is desired,
application protocols that include personally identifying information need
additional application-level scrubbing proxies, such as
Privoxy~\cite{privoxy} for HTTP\@.  Furthermore, Tor does not relay arbitrary
IP packets; it only anonymizes TCP streams and DNS requests
%, and only supports
%connections via SOCKS
(but see Section~\ref{subsec:tcp-vs-ip}).

Most node operators do not want to allow arbitrary TCP traffic. % to leave
%their server.
To address this, Tor provides \emph{exit policies} so
each exit node can block the IP addresses and ports it is unwilling to allow.
Tor nodes advertise their exit policies to the directory servers, so that
client can tell which nodes will support their connections.

As of January 2005, the Tor network has grown to around a hundred nodes
on four continents, with a total capacity exceeding 1Gbit/s. Appendix A
shows a graph of the number of working nodes over time, as well as a
graph of the number of bytes being handled by the network over time.
The network is now sufficiently diverse for further development
and testing; but of course we always encourage new nodes
to join.

Tor research and development has been funded by ONR and DARPA
for use in securing government
communications, and by the Electronic Frontier Foundation for use
in maintaining civil liberties for ordinary citizens online. The Tor
protocol is one of the leading choices
for the anonymizing layer in the European Union's PRIME directive to
help maintain privacy in Europe.
The AN.ON project in Germany
has integrated an independent implementation of the Tor protocol into
their popular Java Anon Proxy anonymizing client.
% This wide variety of
%interests helps maintain both the stability and the security of the
%network.

\medskip
\noindent
{\bf Threat models and design philosophy.}
The ideal Tor network would be practical, useful and anonymous. When
trade-offs arise between these properties, Tor's research strategy has been
to remain useful enough to attract many users,
and practical enough to support them.  Only subject to these
constraints do we try to maximize
anonymity.\footnote{This is not the only possible
direction in anonymity research: designs exist that provide more anonymity
than Tor at the expense of significantly increased resource requirements, or
decreased flexibility in application support (typically because of increased
latency).  Such research does not typically abandon aspirations toward
deployability or utility, but instead tries to maximize deployability and
utility subject to a certain degree of structural anonymity (structural because
usability and practicality affect usage which affects the actual anonymity
provided by the network \cite{econymics,back01}).}
%{We believe that these
%approaches can be promising and useful, but that by focusing on deploying a
%usable system in the wild, Tor helps us experiment with the actual parameters
%of what makes a system ``practical'' for volunteer operators and ``useful''
%for home users, and helps illuminate undernoticed issues which any deployed
%volunteer anonymity network will need to address.}
Because of our strategy, Tor has a weaker threat model than many designs in
the literature.  In particular, because we
support interactive communications without impractically expensive padding,
we fall prey to a variety
of intra-network~\cite{back01,attack-tor-oak05,flow-correlation04} and
end-to-end~\cite{danezis:pet2004,SS03} anonymity-breaking attacks.

Tor does not attempt to defend against a global observer.  In general, an
attacker who can measure both ends of a connection through the Tor network
% I say 'measure' rather than 'observe', to encompass murdoch-danezis
% style attacks. -RD
can correlate the timing and volume of data on that connection as it enters
and leaves the network, and so link communication partners.
Known solutions to this attack would seem to require introducing a
prohibitive degree of traffic padding between the user and the network, or
introducing an unacceptable degree of latency (but see Section
\ref{subsec:mid-latency}).  Also, it is not clear that these methods would
work at all against a minimally active adversary who could introduce timing
patterns or additional traffic.  Thus, Tor only attempts to defend against
external observers who cannot observe both sides of a user's connections.


Against internal attackers who sign up Tor nodes, the situation is more
complicated.  In the simplest case, if an adversary has compromised $c$ of
$n$ nodes on the Tor network, then the adversary will be able to compromise
a random circuit with probability $\frac{c^2}{n^2}$ (since the circuit
initiator chooses hops randomly).  But there are
complicating factors:
(1)~If the user continues to build random circuits over time, an adversary
  is pretty certain to see a statistical sample of the user's traffic, and
  thereby can build an increasingly accurate profile of her behavior.  (See
  Section~\ref{subsec:helper-nodes} for possible solutions.)
(2)~An adversary who controls a popular service outside the Tor network
  can be certain to observe all connections to that service; he
  can therefore trace connections to that service with probability
  $\frac{c}{n}$.
(3)~Users do not in fact choose nodes with uniform probability; they
  favor nodes with high bandwidth or uptime, and exit nodes that
  permit connections to their favorite services.
(See Section~\ref{subsec:routing-zones} for discussion of larger
adversaries and our dispersal goals.)

% I'm trying to make this paragraph work without reference to the
% analysis/confirmation distinction, which we haven't actually introduced
% yet, and which we realize isn't very stable anyway.  Also, I don't want to
% deprecate these attacks if we can't demonstrate that they don't work, since
% in case they *do* turn out to work well against Tor, we'll look pretty
% foolish. -NM
More powerful attacks may exist. In \cite{hintz-pet02} it was
shown that an attacker who can catalog data volumes of popular
responder destinations (say, websites with consistent data volumes) may not
need to
observe both ends of a stream to learn source-destination links for those
responders.
Similarly, latencies of going through various routes can be
cataloged~\cite{back01} to connect endpoints.
% Also, \cite{kesdogan:pet2002} takes the
% attack another level further, to narrow down where you could be
% based on an intersection attack on subpages in a website. -RD
It has not yet been shown whether these attacks will succeed or fail
in the presence of the variability and volume quantization introduced by the
Tor network, but it seems likely that these factors will at best delay
rather than halt the attacks in the cases where they succeed.
Along similar lines, the same paper suggests a ``clogging
attack'' in which the throughput on a circuit is observed to slow
down when an adversary clogs the right nodes with his own traffic.
To determine the nodes in a circuit this attack requires the ability
to continuously monitor the traffic exiting the network on a circuit
that is up long enough to probe all network nodes in binary fashion.
% Though somewhat related, clogging and interference are really different
% attacks with different assumptions about adversary distribution and
% capabilities as well as different techniques. -pfs
Murdoch and Danezis~\cite{attack-tor-oak05} show a practical
interference attack against portions of
the fifty node Tor network as deployed in mid 2004.
An outside attacker can actively trace a circuit through the Tor network
by observing changes in the latency of his
own traffic sent through various Tor nodes. This can be done
simultaneously at multiple nodes; however, like clogging,
this attack only reveals
the Tor nodes in the circuit, not initiator and responder addresses,
so it is still necessary to discover the endpoints to complete an
effective attack. Increasing the size and diversity of the Tor network may
help counter these attacks.

%discuss $\frac{c^2}{n^2}$, except how in practice the chance of owning
%the last hop is not $c/n$ since that doesn't take the destination (website)
%into account. so in cases where the adversary does not also control the
%final destination we're in good shape, but if he *does* then we'd be better
%off with a system that lets each hop choose a path.
%
%Isn't it more accurate to say ``If the adversary _always_ controls the final
% dest, we would be just as well off with such as system.'' ?  If not, why
% not? -nm
% Sure. In fact, better off, since they seem to scale more easily. -rd

%Murdoch and Danezis describe an attack
%\cite{attack-tor-oak05} that lets an attacker determine the nodes used
%in a circuit; yet s/he cannot identify the initiator or responder,
%e.g., client or web server, through this attack. So the endpoints
%remain secure, which is the goal. It is conceivable that an
%adversary could attack or set up observation of all connections
%to an arbitrary Tor node in only a few minutes.  If such an adversary
%were to exist, s/he could use this probing to remotely identify a node
%for further attack.  Of more likely immediate practical concern
%an adversary with active access to the responder traffic
%wants to keep a circuit alive long enough to attack an identified
%node. Thus it is important to prevent the responding end of the circuit
%from keeping it open indefinitely. 
%Also, someone could identify nodes in this way and if in their
%jurisdiction, immediately get a subpoena (if they even need one)
%telling the node operator(s) that she must retain all the active
%circuit data she now has.
%Further, the enclave model, which had previously looked to be the most
%generally secure, seems particularly threatened by this attack, since
%it identifies endpoints when they're also nodes in the Tor network:
%see Section~\ref{subsec:helper-nodes} for discussion of some ways to
%address this issue.

\medskip
\noindent
{\bf Distributed trust.}
In practice Tor's threat model is based on
dispersal and diversity.
Our defense lies in having a diverse enough set of nodes
to prevent most real-world
adversaries from being in the right places to attack users,
by distributing each transaction
over several nodes in the network.  This ``distributed trust'' approach
means the Tor network can be safely operated and used by a wide variety
of mutually distrustful users, providing sustainability and security.
%than some previous attempts at anonymizing networks.

No organization can achieve this security on its own.  If a single
corporation or government agency were to build a private network to
protect its operations, any connections entering or leaving that network
would be obviously linkable to the controlling organization.  The members
and operations of that agency would be easier, not harder, to distinguish.

Instead, to protect our networks from traffic analysis, we must
collaboratively blend the traffic from many organizations and private
citizens, so that an eavesdropper can't tell which users are which,
and who is looking for what information.  %By bringing more users onto
%the network, all users become more secure~\cite{econymics}.
%[XXX I feel uncomfortable saying this last sentence now. -RD]
%[So, I took it out. I think we can do without it. -PFS]
The Tor network has a broad range of users, including ordinary citizens
concerned about their privacy, corporations
who don't want to reveal information to their competitors, and law
enforcement and government intelligence agencies who need
to do operations on the Internet without being noticed.
Naturally, organizations will not want to depend on others for their
security.  If most participating providers are reliable, Tor tolerates
some hostile infiltration of the network.  For maximum protection,
the Tor design includes an enclave approach that lets data be encrypted
(and authenticated) end-to-end, so high-sensitivity users can be sure it
hasn't been read or modified.  This even works for Internet services that
don't have built-in encryption and authentication, such as unencrypted
HTTP or chat, and it requires no modification of those services.

\subsection{Related work}
Tor differs from other deployed systems for traffic analysis resistance
in its security and flexibility.  Mix networks such as
Mixmaster~\cite{mixmaster-spec} or its successor Mixminion~\cite{minion-design}
gain the highest degrees of anonymity at the expense of introducing highly
variable delays, making them unsuitable for applications such as web
browsing.  Commercial single-hop
proxies~\cite{anonymizer} can provide good performance, but
a single compromise can expose all users' traffic, and a single-point
eavesdropper can perform traffic analysis on the entire network.
%Also, their proprietary implementations place any infrastructure that
%depends on these single-hop solutions at the mercy of their providers'
%financial health as well as network security.
The Java
Anon Proxy~\cite{web-mix} provides similar functionality to Tor but
handles only web browsing rather than all TCP\@.
%Some peer-to-peer file-sharing overlay networks such as
%Freenet~\cite{freenet} and Mute~\cite{mute}
The Freedom 
network from Zero-Knowledge Systems~\cite{freedom21-security}
was even more flexible than Tor in
transporting arbitrary IP packets, and also supported
pseudonymity in addition to anonymity; but it had
a different approach to sustainability (collecting money from users
and paying ISPs to run Tor nodes), and was eventually shut down due to financial
load.  Finally, %potentially more scalable
% [I had added 'potentially' because the scalability of these designs
% is not established, and I am uncomfortable making the
% bolder unmodified assertion. Roger took 'potentially' out.
% Here's an attempt at more neutral wording -pfs]
peer-to-peer designs that are intended to be more scalable,
for example Tarzan~\cite{tarzan:ccs02} and
MorphMix~\cite{morphmix:fc04}, have been proposed in the literature but
have not been fielded. These systems differ somewhat
in threat model and presumably practical resistance to threats.
Note that MorphMix differs from Tor only in
node discovery and circuit setup; so Tor's architecture is flexible
enough to contain a MorphMix experiment.
We direct the interested reader
to~\cite{tor-design} for a more in-depth review of related work.

%XXXX six-four. crowds. i2p.

%XXXX
%have a serious discussion of morphmix's assumptions, since they would
%seem to be the direct competition. in fact tor is a flexible architecture
%that would encompass morphmix, and they're nearly identical except for
%path selection and node discovery. and the trust system morphmix has
%seems overkill (and/or insecure) based on the threat model we've picked.
% this para should probably move to the scalability / directory system. -RD
% Nope. Cut for space, except for small comment added above -PFS

\section{Social challenges}

Many of the issues the Tor project needs to address extend beyond
system design and technology development. In particular, the
Tor project's \emph{image} with respect to its users and the rest of
the Internet impacts the security it can provide.
With this image issue in mind, this section discusses the Tor user base and
Tor's interaction with other services on the Internet.

\subsection{Communicating security}

Usability for anonymity systems
contributes to their security, because usability
affects the possible anonymity set~\cite{econymics,back01}.
Conversely, an unusable system attracts few users and thus can't provide
much anonymity.

This phenomenon has a second-order effect: knowing this, users should
choose which anonymity system to use based in part on how usable
and secure
\emph{others} will find it, in order to get the protection of a larger
anonymity set. Thus we might supplement the adage ``usability is a security
parameter''~\cite{back01} with a new one: ``perceived usability is a
security parameter.'' From here we can better understand the effects
of publicity on security: the more convincing your
advertising, the more likely people will believe you have users, and thus
the more users you will attract. Perversely, over-hyped systems (if they
are not too broken) may be a better choice than modestly promoted ones,
if the hype attracts more users~\cite{usability-network-effect}.

So it follows that we should come up with ways to accurately communicate
the available security levels to the user, so she can make informed
decisions. JAP aims to do this by including a
comforting `anonymity meter' dial in the software's graphical interface,
giving the user an impression of the level of protection for her current
traffic.

However, there's a catch. For users to share the same anonymity set,
they need to act like each other. An attacker who can distinguish
a given user's traffic from the rest of the traffic will not be
distracted by anonymity set size. For high-latency systems like
Mixminion, where the threat model is based on mixing messages with each
other, there's an arms race between end-to-end statistical attacks and
counter-strategies~\cite{statistical-disclosure,minion-design,e2e-traffic,trickle02}.
But for low-latency systems like Tor, end-to-end \emph{traffic
correlation} attacks~\cite{danezis:pet2004,defensive-dropping,SS03}
allow an attacker who can observe both ends of a communication
to correlate packet timing and volume, quickly linking
the initiator to her destination.

Like Tor, the current JAP implementation does not pad connections
apart from using small fixed-size cells for transport. In fact,
JAP's cascade-based network topology may be more vulnerable to these
attacks, because its network has fewer edges. JAP was born out of
the ISDN mix design~\cite{isdn-mixes}, where padding made sense because
every user had a fixed bandwidth allocation and altering the timing
pattern of packets could be immediately detected. But in its current context
as an Internet web anonymizer, adding sufficient padding to JAP
would probably be prohibitively expensive and ineffective against a
minimally active attacker.\footnote{Even if JAP could
fund higher-capacity nodes indefinitely, our experience
suggests that many users would not accept the increased per-user
bandwidth requirements, leading to an overall much smaller user base. But
see Section~\ref{subsec:mid-latency}.} Therefore, since under this threat
model the number of concurrent users does not seem to have much impact
on the anonymity provided, we suggest that JAP's anonymity meter is not
accurately communicating security levels to its users.

On the other hand, while the number of active concurrent users may not
matter as much as we'd like, it still helps to have some other users
on the network. We investigate this issue next.

\subsection{Reputability and perceived social value}
Another factor impacting the network's security is its reputability:
the perception of its social value based on its current user base. If Alice is
the only user who has ever downloaded the software, it might be socially
accepted, but she's not getting much anonymity. Add a thousand
activists, and she's anonymous, but everyone thinks she's an activist too.
Add a thousand
diverse citizens (cancer survivors, privacy enthusiasts, and so on)
and now she's harder to profile.

Furthermore, the network's reputability affects its operator base: more people
are willing to run a service if they believe it will be used by human rights
workers than if they believe it will be used exclusively for disreputable
ends.  This effect becomes stronger if node operators themselves think they
will be associated with their users' disreputable ends.

So the more cancer survivors on Tor, the better for the human rights
activists. The more malicious hackers, the worse for the normal users. Thus,
reputability is an anonymity issue for two reasons. First, it impacts
the sustainability of the network: a network that's always about to be
shut down has difficulty attracting and keeping adequate nodes.
Second, a disreputable network is more vulnerable to legal and
political attacks, since it will attract fewer supporters.

While people therefore have an incentive for the network to be used for
``more reputable'' activities than their own, there are still trade-offs
involved when it comes to anonymity. To follow the above example, a
network used entirely by cancer survivors might welcome file sharers
onto the network, though of course they'd prefer a wider
variety of users.

Reputability becomes even more tricky in the case of privacy networks,
since the good uses of the network (such as publishing by journalists in
dangerous countries) are typically kept private, whereas network abuses
or other problems tend to be more widely publicized.

The impact of public perception on security is especially important
during the bootstrapping phase of the network, where the first few
widely publicized uses of the network can dictate the types of users it
attracts next.
As an example, some U.S.~Department of Energy
penetration testing engineers are tasked with compromising DoE computers
from the outside. They only have a limited number of ISPs from which to
launch their attacks, and they found that the defenders were recognizing
attacks because they came from the same IP space. These engineers wanted
to use Tor to hide their tracks. First, from a technical standpoint,
Tor does not support the variety of IP packets one would like to use in
such attacks (see Section~\ref{subsec:tcp-vs-ip}). But aside from this,
we also decided that it would probably be poor precedent to encourage
such use---even legal use that improves national security---and managed
to dissuade them.

%% "outside of academia, jap has just lost, permanently".  (That is,
%% even though the crime detection issues are resolved and are unlikely
%% to go down the same way again, public perception has not been kind.)

\subsection{Sustainability and incentives}
One of the unsolved problems in low-latency anonymity designs is
how to keep the nodes running.  ZKS's Freedom network
depended on paying third parties to run its servers; the JAP project's
bandwidth depends on grants to pay for its bandwidth and
administrative expenses.  In Tor, bandwidth and administrative costs are
distributed across the volunteers who run Tor nodes, so we at least have
reason to think that the Tor network could survive without continued research
funding.\footnote{It also helps that Tor is implemented with free and open
  source software that can be maintained by anybody with the ability and
  inclination.}  But why are these volunteers running nodes, and what can we
do to encourage more volunteers to do so?

We have not formally surveyed Tor node operators to learn why they are
running nodes, but
from the information they have provided, it seems that many of them run Tor
nodes for reasons of personal interest in privacy issues.  It is possible
that others are running Tor nodes to protect their own
anonymity, but of course they are
hardly likely to tell us specifics if they are.
%Significantly, Tor's threat model changes the anonymity incentives for running
%a node.  In a high-latency mix network, users can receive additional
%anonymity by running their own node, since doing so obscures when they are
%injecting messages into the network.  But, anybody observing all I/O to a Tor
%node can tell when the node is generating traffic that corresponds to
%none of its incoming traffic.
%
%I didn't buy the above for reason's subtle enough that I just cut it -PFS
Tor exit node operators do attain a degree of
``deniability'' for traffic that originates at that exit node.  For
  example, it is likely in practice that HTTP requests from a Tor node's IP
  will be assumed to be from the Tor network.
  More significantly, people and organizations who use Tor for
  anonymity depend on the
  continued existence of the Tor network to do so; running a node helps to
  keep the network operational.
%\item Local Tor entry and exit nodes allow users on a network to run in an
%  `enclave' configuration.  [XXXX need to resolve this. They would do this
%   for E2E encryption + auth?]


%We must try to make the costs of running a Tor node easily minimized.
Since Tor is run by volunteers, the most crucial software usability issue is
usability by operators: when an operator leaves, the network becomes less
usable by everybody.  To keep operators pleased, we must try to keep Tor's
resource and administrative demands as low as possible.

Because of ISP billing structures, many Tor operators have underused capacity
that they are willing to donate to the network, at no additional monetary
cost to them.  Features to limit bandwidth have been essential to adoption.
Also useful has been a ``hibernation'' feature that allows a Tor node that
wants to provide high bandwidth, but no more than a certain amount in a
giving billing cycle, to become dormant once its bandwidth is exhausted, and
to reawaken at a random offset into the next billing cycle.  This feature has
interesting policy implications, however; see
the next section below.
Exit policies help to limit administrative costs by limiting the frequency of
abuse complaints (see Section~\ref{subsec:tor-and-blacklists}). We discuss
technical incentive mechanisms in Section~\ref{subsec:incentives-by-design}.

%[XXXX say more.  Why else would you run a node? What else can we do/do we
%  already do to make running a node more attractive?]
%[We can enforce incentives; see Section 6.1. We can rate-limit clients.
%  We can put "top bandwidth nodes lists" up a la seti@home.]

\subsection{Bandwidth and file-sharing}
\label{subsec:bandwidth-and-file-sharing}
%One potentially problematical area with deploying Tor has been our response
%to file-sharing applications.
Once users have configured their applications to work with Tor, the largest
remaining usability issue is performance.  Users begin to suffer
when websites ``feel slow.''
Clients currently try to build their connections through nodes that they
guess will have enough bandwidth.  But even if capacity is allocated
optimally, it seems unlikely that the current network architecture will have
enough capacity to provide every user with as much bandwidth as she would
receive if she weren't using Tor, unless far more nodes join the network.

%Limited capacity does not destroy the network, however.  Instead, usage tends
%towards an equilibrium: when performance suffers, users who value performance
%over anonymity tend to leave the system, thus freeing capacity until the
%remaining users on the network are exactly those willing to use that capacity
%there is.

Much of Tor's recent bandwidth difficulties have come from file-sharing
applications.  These applications provide two challenges to
any anonymizing network: their intensive bandwidth requirement, and the
degree to which they are associated (correctly or not) with copyright
infringement.

High-bandwidth protocols can make the network unresponsive,
but tend to be somewhat self-correcting as lack of bandwidth drives away
users who need it.  Issues of copyright violation,
however, are more interesting.  Typical exit node operators want to help
people achieve private and anonymous speech, not to help people (say) host
Vin Diesel movies for download; and typical ISPs would rather not
deal with customers who draw menacing letters
from the MPAA\@.  While it is quite likely that the operators are doing nothing
illegal, many ISPs have policies of dropping users who get repeated legal
threats regardless of the merits of those threats, and many operators would
prefer to avoid receiving even meritless legal threats.
So when letters arrive, operators are likely to face
pressure to block file-sharing applications entirely, in order to avoid the
hassle.

But blocking file-sharing is not easy: popular
protocols have evolved to run on non-standard ports to
get around other port-based bans.  Thus, exit node operators who want to
block file-sharing would have to find some way to integrate Tor with a
protocol-aware exit filter.  This could be a technically expensive
undertaking, and one with poor prospects: it is unlikely that Tor exit nodes
would succeed where so many institutional firewalls have failed.  Another
possibility for sensitive operators is to run a restrictive node that
only permits exit connections to a restricted range of ports that are
not frequently associated with file sharing.  There are increasingly few such
ports.

Other possible approaches might include rate-limiting connections, especially
long-lived connections or connections to file-sharing ports, so that
high-bandwidth connections do not flood the network.  We might also want to
give priority to cells on low-bandwidth connections to keep them interactive,
but this could have negative anonymity implications.

For the moment, it seems that Tor's bandwidth issues have rendered it
unattractive for bulk file-sharing traffic; this may continue to be so in the
future.  Nevertheless, Tor will likely remain attractive for limited use in
file-sharing protocols that have separate control and data channels.

%[We should say more -- but what?  That we'll see a similar
%  equilibriating effect as with bandwidth, where sensitive ops switch to
%  middleman, and we become less useful for file-sharing, so the file-sharing
%  people back off, so we get more ops since there's less file-sharing, so the
%  file-sharers come back, etc.]

%XXXX
%in practice, plausible deniability is hypothetical and doesn't seem very
%convincing. if ISPs find the activity antisocial, they don't care *why*
%your computer is doing that behavior.

\subsection{Tor and blacklists}
\label{subsec:tor-and-blacklists}

It was long expected that, alongside legitimate users, Tor would also
attract troublemakers who exploit Tor to abuse services on the
Internet with vandalism, rude mail, and so on.
Our initial answer to this situation was to use ``exit policies''
to allow individual Tor nodes to block access to specific IP/port ranges.
This approach aims to make operators more willing to run Tor by allowing
them to prevent their nodes from being used for abusing particular
services.  For example, all Tor nodes currently block SMTP (port 25),
to avoid being used for spam.

Exit policies are useful, but they are insufficient: if not all nodes
block a given service, that service may try to block Tor instead.
While being blockable is important to being good netizens, we would like
to encourage services to allow anonymous access. Services should not
need to decide between blocking legitimate anonymous use and allowing
unlimited abuse.

This is potentially a bigger problem than it may appear.
On the one hand, services should be allowed to refuse connections from
sources of possible abuse.
But when a Tor node administrator decides whether he prefers to be able
to post to Wikipedia from his IP address, or to allow people to read
Wikipedia anonymously through his Tor node, he is making the decision
for others as well. (For a while, Wikipedia
blocked all posting from all Tor nodes based on IP addresses.) If
the Tor node shares an address with a campus or corporate NAT,
then the decision can prevent the entire population from posting.
This is a loss for both Tor
and Wikipedia: we don't want to compete for (or divvy up) the
NAT-protected entities of the world.

Worse, many IP blacklists are coarse-grained: they ignore Tor's exit
policies, partly because it's easier to implement and partly
so they can punish
all Tor nodes. One IP blacklist even bans
every class C network that contains a Tor node, and recommends banning SMTP
from these networks even though Tor does not allow SMTP at all.  This
strategic decision aims to discourage the
operation of anything resembling an open proxy by encouraging its neighbors
to shut it down to get unblocked themselves. This pressure even
affects Tor nodes running in middleman mode (disallowing all exits) when
those nodes are blacklisted too.

Problems of abuse occur mainly with services such as IRC networks and
Wikipedia, which rely on IP blocking to ban abusive users.  While at first
blush this practice might seem to depend on the anachronistic assumption that
each IP is an identifier for a single user, it is actually more reasonable in
practice: it assumes that non-proxy IPs are a costly resource, and that an
abuser can not change IPs at will.  By blocking IPs which are used by Tor
nodes, open proxies, and service abusers, these systems hope to make
ongoing abuse difficult.  Although the system is imperfect, it works
tolerably well for them in practice.

Of course, we would prefer that legitimate anonymous users be able to
access abuse-prone services.  One conceivable approach would require
would-be IRC users, for instance, to register accounts if they want to
access the IRC network from Tor.  In practice this would not
significantly impede abuse if creating new accounts were easily automatable;
this is why services use IP blocking.  To deter abuse, pseudonymous
identities need to require a significant switching cost in resources or human
time.  Some popular webmail applications
impose cost with Reverse Turing Tests, but this step may not deter all
abusers.  Freedom used blind signatures to limit
the number of pseudonyms for each paying account, but Tor has neither the
ability nor the desire to collect payment.

We stress that as far as we can tell, most Tor uses are not
abusive. Most services have not complained, and others are actively
working to find ways besides banning to cope with the abuse. For example,
the Freenode IRC network had a problem with a coordinated group of
abusers joining channels and subtly taking over the conversation; but
when they labelled all users coming from Tor IPs as ``anonymous users,''
removing the ability of the abusers to blend in, the abuse stopped.

%The use of squishy IP-based ``authentication'' and ``authorization''
%has not broken down even to the level that SSNs used for these
%purposes have in commercial and public record contexts. Externalities
%and misplaced incentives cause a continued focus on fighting identity
%theft by protecting SSNs rather than developing better authentication
%and incentive schemes \cite{price-privacy}. Similarly we can expect a
%continued use of identification by IP number as long as there is no
%workable alternative.

%[XXX Mention correct DNS-RBL implementation. -NM]

\section{Design choices}

In addition to social issues, Tor also faces some design trade-offs that must
be investigated as the network develops.

\subsection{Transporting the stream vs transporting the packets}
\label{subsec:stream-vs-packet}
\label{subsec:tcp-vs-ip}

Tor transports streams; it does not tunnel packets.
It has often been suggested that like the old Freedom
network~\cite{freedom21-security}, Tor should
``obviously'' anonymize IP traffic
at the IP layer. Before this could be done, many issues need to be resolved:

\begin{enumerate}
\setlength{\itemsep}{0mm}
\setlength{\parsep}{0mm}
\item \emph{IP packets reveal OS characteristics.}  We would still need to do
IP-level packet normalization, to stop things like TCP fingerprinting
attacks. %There likely exist libraries that can help with this.
This is unlikely to be a trivial task, given the diversity and complexity of
TCP stacks.
\item \emph{Application-level streams still need scrubbing.} We still need
Tor to be easy to integrate with user-level application-specific proxies
such as Privoxy. So it's not just a matter of capturing packets and
anonymizing them at the IP layer.
\item \emph{Certain protocols will still leak information.} For example, we
must rewrite DNS requests so they are delivered to an unlinkable DNS server
rather than the DNS server at a user's ISP; thus, we must understand the
protocols we are transporting.
\item \emph{The crypto is unspecified.} First we need a block-level encryption
approach that can provide security despite
packet loss and out-of-order delivery. Freedom allegedly had one, but it was
never publicly specified.
Also, TLS over UDP is not yet implemented or
specified, though some early work has begun~\cite{dtls}.
\item \emph{We'll still need to tune network parameters.} Since the above
encryption system will likely need sequence numbers (and maybe more) to do
replay detection, handle duplicate frames, and so on, we will be reimplementing
a subset of TCP anyway---a notoriously tricky path.
\item \emph{Exit policies for arbitrary IP packets mean building a secure
IDS\@.}  Our node operators tell us that exit policies are one of
the main reasons they're willing to run Tor.
Adding an Intrusion Detection System to handle exit policies would
increase the security complexity of Tor, and would likely not work anyway,
as evidenced by the entire field of IDS and counter-IDS papers. Many
potential abuse issues are resolved by the fact that Tor only transports
valid TCP streams (as opposed to arbitrary IP including malformed packets
and IP floods), so exit policies become even \emph{more} important as
we become able to transport IP packets. We also need to compactly
describe exit policies so clients can predict
which nodes will allow which packets to exit.
\item \emph{The Tor-internal name spaces would need to be redesigned.} We
support hidden service {\tt{.onion}} addresses (and other special addresses,
like {\tt{.exit}} which lets the user request a particular exit node),
by intercepting the addresses when they are passed to the Tor client.
Doing so at the IP level would require a more complex interface between
Tor and the local DNS resolver.
\end{enumerate}

This list is discouragingly long, but being able to transport more
protocols obviously has some advantages. It would be good to learn which
items are actual roadblocks and which are easier to resolve than we think.

To be fair, Tor's stream-based approach has run into
stumbling blocks as well. While Tor supports the SOCKS protocol,
which provides a standardized interface for generic TCP proxies, many
applications do not support SOCKS\@. For them we already need to
replace the networking system calls with SOCKS-aware
versions, or run a SOCKS tunnel locally, neither of which is
easy for the average user. %---even with good instructions.
Even when applications can use SOCKS, they often make DNS requests
themselves before handing an IP address to Tor, which advertises
where the user is about to connect.
We are still working on more usable solutions.

%So to actually provide good anonymity, we need to make sure that
%users have a practical way to use Tor anonymously.  Possibilities include
%writing wrappers for applications to anonymize them automatically; improving
%the applications' support for SOCKS; writing libraries to help application
%writers use Tor properly; and implementing a local DNS proxy to reroute DNS
%requests to Tor so that applications can simply point their DNS resolvers at
%localhost and continue to use SOCKS for data only.

\subsection{Mid-latency}
\label{subsec:mid-latency}

Some users need to resist traffic correlation attacks.  Higher-latency
mix-networks introduce variability into message
arrival times: as timing variance increases, timing correlation attacks
require increasingly more data~\cite{e2e-traffic}. Can we improve Tor's
resistance without losing too much usability?

We need to learn whether we can trade a small increase in latency
for a large anonymity increase, or if we'd end up trading a lot of
latency for only a minimal security gain. A trade-off might be worthwhile
even if we
could only protect certain use cases, such as infrequent short-duration
transactions. % To answer this question
We might adapt the techniques of~\cite{e2e-traffic} to a lower-latency mix
network, where the messages are batches of cells in temporally clustered
connections. These large fixed-size batches can also help resist volume
signature attacks~\cite{hintz-pet02}. We could also experiment with traffic
shaping to get a good balance of throughput and security.
%Other padding regimens might supplement the
%mid-latency option; however, we should continue the caution with which
%we have always approached padding lest the overhead cost us too much
%performance or too many volunteers.

We must keep usability in mind too. How much can latency increase
before we drive users away? We've already been forced to increase
latency slightly, as our growing network incorporates more DSL and
cable-modem nodes and more nodes in distant continents. Perhaps we can
harness this increased latency to improve anonymity rather than just
reduce usability. Further, if we let clients label certain circuits as
mid-latency as they are constructed, we could handle both types of traffic
on the same network, giving users a choice between speed and security---and
giving researchers a chance to experiment with parameters to improve the
quality of those choices.

\subsection{Enclaves and helper nodes}
\label{subsec:helper-nodes}

It has long been thought that users can improve their anonymity by
running their own node~\cite{tor-design,or-ih96,or-pet00}, and using
it in an \emph{enclave} configuration, where all their circuits begin
at the node under their control. Running Tor clients or servers at
the enclave perimeter is useful when policy or other requirements
prevent individual machines within the enclave from running Tor
clients~\cite{or-jsac98,or-discex00}.

Of course, Tor's default path length of
three is insufficient for these enclaves, since the entry and/or exit
% [edit war: without the ``and/'' the natural reading here
% is aut rather than vel. And the use of the plural verb does not work -pfs]
themselves are sensitive. Tor thus increments path length by one
for each sensitive endpoint in the circuit.
Enclaves also help to protect against end-to-end attacks, since it's
possible that traffic coming from the node has simply been relayed from
elsewhere. However, if the node has recognizable behavior patterns,
an attacker who runs nodes in the network can triangulate over time to
gain confidence that it is in fact originating the traffic. Wright et
al.~\cite{wright03} introduce the notion of a \emph{helper node}---a
single fixed entry node for each user---to combat this \emph{predecessor
attack}.

However, the attack in~\cite{attack-tor-oak05} shows that simply adding
to the path length, or using a helper node, may not protect an enclave
node. A hostile web server can send constant interference traffic to
all nodes in the network, and learn which nodes are involved in the
circuit (though at least in the current attack, he can't learn their
order). Using randomized path lengths may help some, since the attacker
will never be certain he has identified all nodes in the path unless
he probes the entire network, but as
long as the network remains small this attack will still be feasible.

Helper nodes also aim to help Tor clients, because choosing entry and exit
points
randomly and changing them frequently allows an attacker who controls
even a few nodes to eventually link some of their destinations. The goal
is to take the risk once and for all about choosing a bad entry node,
rather than taking a new risk for each new circuit. (Choosing fixed
exit nodes is less useful, since even an honest exit node still doesn't
protect against a hostile website.) But obstacles remain before
we can implement helper nodes.
For one, the literature does not describe how to choose helpers from a list
of nodes that changes over time.  If Alice is forced to choose a new entry
helper every $d$ days and $c$ of the $n$ nodes are bad, she can expect
to choose a compromised node around
every $dc/n$ days. Statistically over time this approach only helps
if she is better at choosing honest helper nodes than at choosing
honest nodes.  Worse, an attacker with the ability to DoS nodes could
force users to switch helper nodes more frequently, or remove
other candidate helpers.

%Do general DoS attacks have anonymity implications? See e.g. Adam
%Back's IH paper, but I think there's more to be pointed out here. -RD
% Not sure what you want to say here. -NM

%Game theory for helper nodes: if Alice offers a hidden service on a
%server (enclave model), and nobody ever uses helper nodes, then against
%George+Steven's attack she's totally nailed. If only Alice uses a helper
%node, then she's still identified as the source of the data. If everybody
%uses a helper node (including Alice), then the attack identifies the
%helper node and also Alice, and knows which one is which. If everybody
%uses a helper node (but not Alice), then the attacker figures the real
%source was a client that is using Alice as a helper node. [How's my
%logic here?] -RD
%
% Not sure about the logic.  For the attack to work with helper nodes, the
%attacker needs to guess that Alice is running the hidden service, right?
%Otherwise, how can he know to measure her traffic specifically? -NM
%
% In the Murdoch-Danezis attack, the adversary measures all servers. -RD

%point to routing-zones section re: helper nodes to defend against
%big stuff.

\subsection{Location-hidden services}
\label{subsec:hidden-services}

% This section is first up against the wall when the revolution comes.

Tor's \emph{rendezvous points}
let users provide TCP services to other Tor users without revealing
the service's location. Since this feature is relatively recent, we describe
here
a couple of our early observations from its deployment.

First, our implementation of hidden services seems less hidden than we'd
like, since they build a different rendezvous circuit for each user,
and an external adversary can induce them to
produce traffic. This insecurity means that they may not be suitable as
a building block for Free Haven~\cite{freehaven-berk} or other anonymous
publishing systems that aim to provide long-term security, though helper
nodes, as discussed above, would seem to help.

\emph{Hot-swap} hidden services, where more than one location can
provide the service and loss of any one location does not imply a
change in service, would help foil intersection and observation attacks
where an adversary monitors availability of a hidden service and also
monitors whether certain users or servers are online. The design
challenges in providing such services without otherwise compromising
the hidden service's anonymity remain an open problem;
however, see~\cite{move-ndss05}.

In practice, hidden services are used for more than just providing private
access to a web server or IRC server. People are using hidden services
as a poor man's VPN and firewall-buster. Many people want to be able
to connect to the computers in their private network via secure shell,
and rather than playing with dyndns and trying to pierce holes in their
firewall, they run a hidden service on the inside and then rendezvous
with that hidden service externally.

News sites like Bloggers Without Borders (www.b19s.org) are advertising
a hidden-service address on their front page. Doing this can provide
increased robustness if they use the dual-IP approach we describe
in~\cite{tor-design},
but in practice they do it to increase visibility
of the Tor project and their support for privacy, and to offer
a way for their users, using unmodified software, to get end-to-end
encryption and authentication to their website.

\subsection{Location diversity and ISP-class adversaries}
\label{subsec:routing-zones}

Anonymity networks have long relied on diversity of node location for
protection against attacks---typically an adversary who can observe a
larger fraction of the network can launch a more effective attack. One
way to achieve dispersal involves growing the network so a given adversary
sees less. Alternately, we can arrange the topology so traffic can enter
or exit at many places (for example, by using a free-route network
like Tor rather than a cascade network like JAP). Lastly, we can use
distributed trust to spread each transaction over multiple jurisdictions.
But how do we decide whether two nodes are in related locations?

Feamster and Dingledine defined a \emph{location diversity} metric
in~\cite{feamster:wpes2004}, and began investigating a variant of location
diversity based on the fact that the Internet is divided into thousands of
independently operated networks called {\em autonomous systems} (ASes).
The key insight from their paper is that while we typically think of a
connection as going directly from the Tor client to the first Tor node,
actually it traverses many different ASes on each hop. An adversary at
any of these ASes can monitor or influence traffic. Specifically, given
plausible initiators and recipients, and given random path selection,
some ASes in the simulation were able to observe 10\% to 30\% of the
transactions (that is, learn both the origin and the destination) on
the deployed Tor network (33 nodes as of June 2004).

The paper concludes that for best protection against the AS-level
adversary, nodes should be in ASes that have the most links to other ASes:
Tier-1 ISPs such as AT\&T and Abovenet. Further, a given transaction
is safest when it starts or ends in a Tier-1 ISP\@. Therefore, assuming
initiator and responder are both in the U.S., it actually \emph{hurts}
our location diversity to use far-flung nodes in
continents like Asia or South America.
% it's not just entering or exiting from them. using them as the middle
% hop reduces your effective path length, which you presumably don't
% want because you chose that path length for a reason.
%
% Not sure I buy that argument. Two end nodes in the right ASs to
% discourage linking are still not known to each other. If some
% adversary in a single AS can bridge the middle node, it shouldn't
% therefore be able to identify initiator or responder; although it could
% contribute to further attacks given more assumptions.
% Nonetheless, no change to the actual text for now.

Many open questions remain. First, it will be an immense engineering
challenge to get an entire BGP routing table to each Tor client, or to
summarize it sufficiently. Without a local copy, clients won't be
able to safely predict what ASes will be traversed on the various paths
through the Tor network to the final destination. Tarzan~\cite{tarzan:ccs02}
and MorphMix~\cite{morphmix:fc04} suggest that we compare IP prefixes to
determine location diversity; but the above paper showed that in practice
many of the Mixmaster nodes that share a single AS have entirely different
IP prefixes. When the network has scaled to thousands of nodes, does IP
prefix comparison become a more useful approximation? % Alternatively, can
%relevant parts of the routing tables be summarized centrally and delivered to
%clients in a less verbose format?
%% i already said "or to summarize is sufficiently" above. is that not
%% enough? -RD
%
Second, we can take advantage of caching certain content at the
exit nodes, to limit the number of requests that need to leave the
network at all. What about taking advantage of caches like Akamai or
Google~\cite{shsm03}? (Note that they're also well-positioned as global
adversaries.)
%
Third, if we follow the recommendations in~\cite{feamster:wpes2004}
 and tailor path selection
to avoid choosing endpoints in similar locations, how much are we hurting
anonymity against larger real-world adversaries who can take advantage
of knowing our algorithm?
%
Fourth, can we use this knowledge to figure out which gaps in our network
most affect our robustness to this class of attack, and go recruit
new nodes with those ASes in mind?

%Tor's security relies in large part on the dispersal properties of its
%network. We need to be more aware of the anonymity properties of various
%approaches so we can make better design decisions in the future.

\subsection{The Anti-censorship problem}
\label{subsec:china}

Citizens in a variety of countries, such as most recently China and
Iran, are blocked from accessing various sites outside
their country. These users try to find any tools available to allow
them to get-around these firewalls. Some anonymity networks, such as
Six-Four~\cite{six-four}, are designed specifically with this goal in
mind; others like the Anonymizer~\cite{anonymizer} are paid by sponsors
such as Voice of America to encourage Internet
freedom. Even though Tor wasn't
designed with ubiquitous access to the network in mind, thousands of
users across the world are now using it for exactly this purpose.
% Academic and NGO organizations, peacefire, \cite{berkman}, etc

Anti-censorship networks hoping to bridge country-level blocks face
a variety of challenges. One of these is that they need to find enough
exit nodes---servers on the `free' side that are willing to relay
traffic from users to their final destinations. Anonymizing
networks like Tor are well-suited to this task since we have
already gathered a set of exit nodes that are willing to tolerate some
political heat.

The other main challenge is to distribute a list of reachable relays
to the users inside the country, and give them software to use those relays,
without letting the censors also enumerate this list and block each
relay. Anonymizer solves this by buying lots of seemingly-unrelated IP
addresses (or having them donated), abandoning old addresses as they are
`used up,' and telling a few users about the new ones. Distributed
anonymizing networks again have an advantage here, in that we already
have tens of thousands of separate IP addresses whose users might
volunteer to provide this service since they've already installed and use
the software for their own privacy~\cite{koepsell:wpes2004}. Because
the Tor protocol separates routing from network discovery \cite{tor-design},
volunteers could configure their Tor clients
to generate node descriptors and send them to a special directory
server that gives them out to dissidents who need to get around blocks.

Of course, this still doesn't prevent the adversary
from enumerating and preemptively blocking the volunteer relays.
Perhaps a tiered-trust system could be built where a few individuals are
given relays' locations. They could then recommend other individuals
by telling them
those addresses, thus providing a built-in incentive to avoid letting the
adversary intercept them. Max-flow trust algorithms~\cite{advogato}
might help to bound the number of IP addresses leaked to the adversary. Groups
like the W3C are looking into using Tor as a component in an overall system to
help address censorship; we wish them success.

%\cite{infranet}

\section{Scaling}
\label{sec:scaling}

Tor is running today with hundreds of nodes and tens of thousands of
users, but it will certainly not scale to millions.
Scaling Tor involves four main challenges. First, to get a
large set of nodes, we must address incentives for
users to carry traffic for others. Next is safe node discovery, both
while bootstrapping (Tor clients must robustly find an initial
node list) and later (Tor clients must learn about a fair sample
of honest nodes and not let the adversary control circuits).
We must also detect and handle node speed and reliability as the network
becomes increasingly heterogeneous: since the speed and reliability
of a circuit is limited by its worst link, we must learn to track and
predict performance. Finally, we must stop assuming that all points on
the network can connect to all other points.

\subsection{Incentives by Design}
\label{subsec:incentives-by-design}

There are three behaviors we need to encourage for each Tor node: relaying
traffic; providing good throughput and reliability while doing it;
and allowing traffic to exit the network from that node.

We encourage these behaviors through \emph{indirect} incentives: that
is, by designing the system and educating users in such a way that users
with certain goals will choose to relay traffic.  One
main incentive for running a Tor node is social: volunteers
altruistically donate their bandwidth and time.  We encourage this with
public rankings of the throughput and reliability of nodes, much like
seti@home.  We further explain to users that they can get
deniability for any traffic emerging from the same address as a Tor
exit node, and they can use their own Tor node
as an entry or exit point with confidence that it's not run by an adversary.
Further, users may run a node simply because they need such a network
to be persistently available and usable, and the value of supporting this
exceeds any countervening costs.
Finally, we can encourage operators by improving the usability and feature
set of the software:
rate limiting support and easy packaging decrease the hassle of
maintaining a node, and our configurable exit policies allow each
operator to advertise a policy describing the hosts and ports to which
he feels comfortable connecting.

To date these incentives appear to have been adequate. As the system scales
or as new issues emerge, however, we may also need to provide
 \emph{direct} incentives:
providing payment or other resources in return for high-quality service.
Paying actual money is problematic: decentralized e-cash systems are
not yet practical, and a centralized collection system not only reduces
robustness, but also has failed in the past (the history of commercial
anonymizing networks is littered with failed attempts).  A more promising
option is to use a tit-for-tat incentive scheme, where nodes provide better
service to nodes that have provided good service for them.

Unfortunately, such an approach introduces new anonymity problems.
There are many surprising ways for nodes to game the incentive and
reputation system to undermine anonymity---such systems are typically
designed to encourage fairness in storage or bandwidth usage, not
fairness of provided anonymity. An adversary can attract more traffic
by performing well or can target individual users by selectively
performing, to undermine their anonymity. Typically a user who
chooses evenly from all nodes is most resistant to an adversary
targeting him, but that approach hampers the efficient use
of heterogeneous nodes.

%When a node (call him Steve) performs well for Alice, does Steve gain
%reputation with the entire system, or just with Alice? If the entire
%system, how does Alice tell everybody about her experience in a way that
%prevents her from lying about it yet still protects her identity? If
%Steve's behavior only affects Alice's behavior, does this allow Steve to
%selectively perform only for Alice, and then break her anonymity later
%when somebody (presumably Alice) routes through his node?

A possible solution is a simplified approach to the tit-for-tat
incentive scheme based on two rules: (1) each node should measure the
service it receives from adjacent nodes, and provide service relative
to the received service, but (2) when a node is making decisions that
affect its own security (such as building a circuit for its own
application connections), it should choose evenly from a sufficiently
large set of nodes that meet some minimum service
threshold~\cite{casc-rep}.  This approach allows us to discourage
bad service
without opening Alice up as much to attacks.  All of this requires
further study.

\subsection{Trust and discovery}
\label{subsec:trust-and-discovery}

The published Tor design is deliberately simplistic in how
new nodes are authorized and how clients are informed about Tor
nodes and their status.
All nodes periodically upload a signed description
of their locations, keys, and capabilities to each of several well-known {\it
  directory servers}.  These directory servers construct a signed summary
of all known Tor nodes (a ``directory''), and a signed statement of which
nodes they
believe to be operational then (a ``network status'').  Clients
periodically download a directory to learn the latest nodes and
keys, and more frequently download a network status to learn which nodes are
likely to be running.  Tor nodes also operate as directory caches, to
lighten the bandwidth on the directory servers.

To prevent Sybil attacks (wherein an adversary signs up many
purportedly independent nodes to increase her network view),
this design
requires the directory server operators to manually
approve new nodes.  Unapproved nodes are included in the directory,
but clients
do not use them at the start or end of their circuits.  In practice,
directory administrators perform little actual verification, and tend to
approve any Tor node whose operator can compose a coherent email.
This procedure
may prevent trivial automated Sybil attacks, but will do little
against a clever and determined attacker.

There are a number of flaws in this system that need to be addressed as we
move forward. First,
each directory server represents an independent point of failure: any
compromised directory server could start recommending only compromised
nodes.
Second, as more nodes join the network, %the more unreasonable it
%becomes to expect clients to know about them all.
directories
become infeasibly large, and downloading the list of nodes becomes
burdensome.
Third, the validation scheme may do as much harm as it does good.  It 
does not prevent clever attackers from mounting Sybil attacks,
and it may deter node operators from joining the network---if
they expect the validation process to be difficult, or they do not share
any languages in common with the directory server operators.

We could try to move the system in several directions, depending on our
choice of threat model and requirements.  If we did not need to increase
network capacity to support more users, we could simply
 adopt even stricter validation requirements, and reduce the number of
nodes in the network to a trusted minimum.  
But, we can only do that if can simultaneously make node capacity
scale much more than we anticipate to be feasible soon, and if we can find
entities willing to run such nodes, an equally daunting prospect.

In order to address the first two issues, it seems wise to move to a system
including a number of semi-trusted directory servers, no one of which can
compromise a user on its own.  Ultimately, of course, we cannot escape the
problem of a first introducer: since most users will run Tor in whatever
configuration the software ships with, the Tor distribution itself will
remain a single point of failure so long as it includes the seed
keys for directory servers, a list of directory servers, or any other means
to learn which nodes are on the network.  But omitting this information
from the Tor distribution would only delegate the trust problem to each
individual user. %, most of whom are presumably less informed about how to make
%trust decisions than the Tor developers.
A well publicized, widely available, authoritatively and independently
endorsed and signed list of initial directory servers and their keys
is a possible solution. But, setting that up properly is itself a large 
bootstrapping task.

%Network discovery, sybil, node admission, scaling. It seems that the code
%will ship with something and that's our trust root. We could try to get
%people to build a web of trust, but no. Where we go from here depends
%on what threats we have in mind. Really decentralized if your threat is
%RIAA; less so if threat is to application data or individuals or...

\subsection{Measuring performance and capacity}
\label{subsec:performance}

One of the paradoxes with engineering an anonymity network is that we'd like
to learn as much as we can about how traffic flows so we can improve the
network, but we want to prevent others from learning how traffic flows in
order to trace users' connections through the network.  Furthermore, many
mechanisms that help Tor run efficiently
require measurements about the network.

Currently, nodes try to deduce their own available bandwidth (based on how
much traffic they have been able to transfer recently) and include this
information in the descriptors they upload to the directory. Clients
choose servers weighted by their bandwidth, neglecting really slow
servers and capping the influence of really fast ones.
%
This is, of course, eminently cheatable.  A malicious node can get a
disproportionate amount of traffic simply by claiming to have more bandwidth
than it does.  But better mechanisms have their problems.  If bandwidth data
is to be measured rather than self-reported, it is usually possible for
nodes to selectively provide better service for the measuring party, or
sabotage the measured value of other nodes.  Complex solutions for
mix networks have been proposed, but do not address the issues
completely~\cite{mix-acc,casc-rep}.

Even with no cheating, network measurement is complex.  It is common
for views of a node's latency and/or bandwidth to vary wildly between
observers.  Further, it is unclear whether total bandwidth is really
the right measure; perhaps clients should instead be considering nodes
based on unused bandwidth or observed throughput.
%How to measure performance without letting people selectively deny service
%by distinguishing pings. Heck, just how to measure performance at all. In
%practice people have funny firewalls that don't match up to their exit
%policies and Tor doesn't deal.
%
%Network investigation: Is all this bandwidth publishing thing a good idea?
%How can we collect stats better? Note weasel's smokeping, at
%http://seppia.noreply.org/cgi-bin/smokeping.cgi?target=Tor
%which probably gives george and steven enough info to break tor?
%
And even if we can collect and use this network information effectively,
we must ensure
that it is not more useful to attackers than to us.  While it
seems plausible that bandwidth data alone is not enough to reveal
sender-recipient connections under most circumstances, it could certainly
reveal the path taken by large traffic flows under low-usage circumstances.

\subsection{Non-clique topologies}

Tor's comparatively weak threat model may allow easier scaling than
other
designs.  High-latency mix networks need to avoid partitioning attacks, where
network splits let an attacker distinguish users in different partitions.
Since Tor assumes the adversary cannot cheaply observe nodes at will,
a network split may not decrease protection much.
Thus, one option when the scale of a Tor network
exceeds some size is simply to split it. Nodes could be allocated into
partitions while hampering collaborating hostile nodes from taking over
a single partition~\cite{casc-rep}.
Clients could switch between
networks, even on a per-circuit basis.
%Future analysis may uncover
%other dangers beyond those affecting mix-nets.

More conservatively, we can try to scale a single Tor network. Likely
problems with adding more servers to a single Tor network include an
explosion in the number of sockets needed on each server as more servers
join, and increased coordination overhead to keep each users' view of
the network consistent. As we grow, we will also have more instances of
servers that can't reach each other simply due to Internet topology or
routing problems.

%include restricting the number of sockets and the amount of bandwidth
%used by each node.  The number of sockets is determined by the network's
%connectivity and the number of users, while bandwidth capacity is determined
%by the total bandwidth of nodes on the network.  The simplest solution to
%bandwidth capacity is to add more nodes, since adding a Tor node of any
%feasible bandwidth will increase the traffic capacity of the network.  So as
%a first step to scaling, we should focus on making the network tolerate more
%nodes, by reducing the interconnectivity of the nodes; later we can reduce
%overhead associated with directories, discovery, and so on.

We can address these points by reducing the network's connectivity.
Danezis~\cite{danezis:pet2003} considers
the anonymity implications of restricting routes on mix networks and
recommends an approach based on expander graphs (where any subgraph is likely
to have many neighbors).  It is not immediately clear that this approach will
extend to Tor, which has a weaker threat model but higher performance
requirements: instead of analyzing the
probability of an attacker's viewing whole paths, we will need to examine the
attacker's likelihood of compromising the endpoints.
%
Tor may not need an expander graph per se: it
may be enough to have a single central subnet that is highly connected, like
an Internet backbone. %  As an
%example, assume fifty nodes of relatively high traffic capacity.  This
%\emph{center} forms a clique.  Assume each center node can
%handle 200 connections to other nodes (including the other ones in the
%center). Assume every noncenter node connects to three nodes in the
%center and anyone out of the center that they want to.  Then the
%network easily scales to c. 2500 nodes with commensurate increase in
%bandwidth.
There are many open questions: how to distribute connectivity information
(presumably nodes will learn about the central nodes
when they download Tor), whether central nodes
will need to function as a `backbone', and so on. As above,
this could reduce the amount of anonymity available from a mix-net,
but for a low-latency network where anonymity derives largely from
the edges, it may be feasible.

%In a sense, Tor already has a non-clique topology.
%Individuals can set up and run Tor nodes without informing the
%directory servers. This allows groups to run a
%local Tor network of private nodes that connects to the public Tor
%network. This network is hidden behind the Tor network, and its
%only visible connection to Tor is at those points where it connects.
%As far as the public network, or anyone observing it, is concerned,
%they are running clients.

\section{The Future}
\label{sec:conclusion}

Tor is the largest and most diverse low-latency anonymity network
available, but we are still in the beginning stages of deployment. Several
major questions remain.

First, will our volunteer-based approach to sustainability work in the
long term? As we add more features and destabilize the network, the
developers spend a lot of time keeping the server operators happy. Even
though Tor is free software, the network would likely stagnate and die at
this stage if the developers stopped actively working on it. We may get
an unexpected boon from the fact that we're a general-purpose overlay
network: as Tor grows more popular, other groups who need an overlay
network on the Internet are starting to adapt Tor to their needs.
%
Second, Tor is only one of many components that preserve privacy online.
For applications where it is desirable to
keep identifying information out of application traffic, someone must build
more and better protocol-aware proxies that are usable by ordinary people.
%
Third, we need to gain a reputation for social good, and learn how to
coexist with the variety of Internet services and their established
authentication mechanisms. We can't just keep escalating the blacklist
standoff forever.
%
Fourth, the current Tor
architecture does not scale even to handle current user demand. We must
find designs and incentives to let some clients relay traffic too, without
sacrificing too much anonymity.

These are difficult and open questions. Yet choosing not to solve them
means leaving most users to a less secure network or no anonymizing
network at all.

\bibliographystyle{plain} \bibliography{tor-design}

\clearpage
\appendix

\begin{figure}[t]
%\unitlength=1in
\centering
%\begin{picture}(6.0,2.0)
%\put(3,1){\makebox(0,0)[c]{\epsfig{figure=graphnodes,width=6in}}}
%\end{picture}
\mbox{\epsfig{figure=graphnodes,width=5in}}
\caption{Number of Tor nodes over time, through January 2005. Lowest
line is number of exit
nodes that allow connections to port 80. Middle line is total number of
verified (registered) Tor nodes. The line above that represents nodes
that are running but not yet registered.}
\label{fig:graphnodes}
\end{figure}

\begin{figure}[t]
\centering
\mbox{\epsfig{figure=graphtraffic,width=5in}}
\caption{The sum of traffic reported by each node over time, through
January 2005. The bottom
pair show average throughput, and the top pair represent the largest 15
minute burst in each 4 hour period.}
\label{fig:graphtraffic}
\end{figure}

\end{document}

%Making use of nodes with little bandwidth, or high latency/packet loss.

%Running Tor nodes behind NATs, behind great-firewalls-of-China, etc.
%Restricted routes. How to propagate to everybody the topology? BGP
%style doesn't work because we don't want just *one* path. Point to
%Geoff's stuff.

