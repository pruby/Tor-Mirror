\documentclass{llncs}

\usepackage{url}
\usepackage{amsmath}
\usepackage{epsfig}

\setlength{\textwidth}{5.9in}
\setlength{\textheight}{8.4in}
\setlength{\topmargin}{.5cm}
\setlength{\oddsidemargin}{1cm}
\setlength{\evensidemargin}{1cm}

\newenvironment{tightlist}{\begin{list}{$\bullet$}{
  \setlength{\itemsep}{0mm}
    \setlength{\parsep}{0mm}
    %  \setlength{\labelsep}{0mm}
    %  \setlength{\labelwidth}{0mm}
    %  \setlength{\topsep}{0mm}
    }}{\end{list}}


\newcommand{\workingnote}[1]{}        % The version that hides the note.
%\newcommand{\workingnote}[1]{(**#1)}   % The version that makes the note visible.


\begin{document}

\title{Design challenges and social factors in deploying low-latency anonymity}
% Could still use a better title -PFS

\author{Roger Dingledine\inst{1} \and
Nick Mathewson\inst{1} \and
Paul Syverson\inst{2}}
\institute{The Tor Project \email{<\{arma,nickm\}@torproject.org>} \and
Naval Research Laboratory \email{<syverson@itd.nrl.navy.mil>}}

\maketitle
\pagestyle{plain}

\begin{abstract}
  There are many unexpected or unexpectedly difficult obstacles to
  deploying anonymous communications.  We describe Tor (\emph{the}
  onion routing), how to use it, our design philosophy, and some of
  the challenges that we have faced and continue to face in building,
  deploying, and sustaining a scalable, distributed, low-latency
  anonymity network.
\end{abstract}

\section{Introduction}
This article describes Tor, a widely-used low-latency general-purpose
anonymous communication system, and discusses some unexpected
challenges arising from our experiences deploying Tor.  We will tell
you how to use it, who uses it, how it works, why we designed it the
way we did, and why this makes it usable and stable.

Tor is an overlay network for anonymizing TCP streams over the
Internet~\cite{tor-design}.  Tor works on the real-world Internet,
requires no special privileges or kernel modifications, requires
little synchronization or coordination between nodes, and provides a
reasonable trade-off between anonymity, usability, and efficiency.

Since deployment in October 2003 the public Tor network has grown to
about a thousand volunteer-operated nodes worldwide and over 110
megabytes average traffic per second from hundreds of thousands of
concurrent users. 

\section{Tor Design and Design Philosophy: Distributed Trust and Usability}

Tor enables users to connect to Internet sites without revealing their
logical or physical locations to those sites or to observers.  It
enables hosts to be publicly accessible yet have similar protection
against location through its \emph{location-hidden services}.

To connect to a remote server via Tor the client software first learns
a %signed
list of Tor nodes from several central \emph{directory servers} via a
voting protocol (to avoid dependence on or complete trust in any one
of these servers). It then incrementally creates a private pathway or
\emph{circuit} across the network. This circuit consists of
encrypted connections through authenticated Tor nodes
whose public keys were obtained from the directory servers. The client
software negotiates a separate set of encryption keys for each hop along the
circuit. The nodes in the circuit are chosen at random by the client
subject to a preference for higher performing nodes to allocate
resources effectively and with a client-chosen preferred set of first
nodes called \emph{entry guards} to complicate profiling attacks by
internal adversaries~\cite{hs-attack}.
The circuit is extended one node at a time, tunneling extensions
through already established portions of the circuit, and each node
along the way knows only the immediately previous and following nodes
in the circuit, so no individual Tor node knows the complete path that
each fixed-sized data packet (or \emph{cell}) will take.  Thus,
neither an eavesdropper nor a compromised node can see both the
connection's source and destination.  Later requests use a new
circuit to complicate long-term linkability between different actions
by a single user.

Tor attempts to anonymize the transport layer, not the application
layer.  Thus, applications such as SSH can provide
authenticated communication that is hidden by Tor from outside observers.
When anonymity from communication partners is desired,
application-level protocols that transmit identifying
information need additional scrubbing proxies, such as
Privoxy~\cite{privoxy} for HTTP\@.  Furthermore, Tor does not relay
arbitrary IP packets; it only anonymizes TCP streams and DNS requests.

Tor, the third generation of deployed onion-routing
designs~\cite{or-ih96,or-jsac98,tor-design}, was researched, developed,
and deployed by the Naval Research Laboratory and the Free Haven
Project under ONR and DARPA funding for secure government
communications.  In 2005, continuing work by Free Haven was funded by
the Electronic Frontier Foundation for maintaining civil liberties of
ordinary citizens online. In 2006, The Tor Project incorporated as a
non-profit and has received continued funding from the Omidyar Network,
the U.S. International Broadcasting Bureau, and other groups to combat
blocking and censorship on the Internet. This diversity of funding fits
Tor's overall philosophy: a wide variety of interests helps maintain
both the stability and the security of the network.

Usability is also a central goal. Downloading and installing Tor is
easy. Simply go to\\
http://torproject.org/ and download.  Tor comes with install
wizards and a GUI for major operating systems: GNU/Linux, OS X, and
Windows. It also runs on various flavors of BSD and UNIX\@. Basic
instructions, documentation, FAQs, etc.\ are available in many
languages. The Tor GUI Vidalia makes server configuration easy, e.g.,
choosing how much bandwidth to allocate to Tor, exit policy choices,
etc. And, the GUI Torbutton allows Firefox users a one-click toggle of
whether browsing goes through Tor or not.  Tor is easily configured by
a site administrator to run at either individual desktops or just at a
site firewall or combinations of these.

The ideal Tor network would be practical, useful and anonymous. When
trade-offs arise between these properties, Tor's research strategy has
been to remain useful enough to attract many users, and practical
enough to support them.  Only subject to these constraints do we try
to maximize anonymity.  Tor thus differs from other deployed systems
for traffic analysis resistance in its security and flexibility.  Mix
networks such as
% Mixmaster~\cite{mixmaster-spec} or its successor 
Mixminion~\cite{minion-design} gain the highest degrees of practical
anonymity at the expense of introducing highly variable delays, making
them unsuitable for applications such as web browsing.  Commercial
single-hop proxies~\cite{anonymizer} can provide good performance, but
a single-point compromise can expose all users' traffic, and a
single-point eavesdropper can perform traffic analysis on the entire
network.  Also, their proprietary implementations place any
infrastructure that depends on these single-hop solutions at the mercy
of their providers' financial health as well as network security.
There are numerous other designs for distributed anonymous low-latency
communication~\cite{crowds-tissec,web-mix,freedom21-security,i2p,tarzan:ccs02,morphmix:fc04}.
Some have been deployed or even commercialized; some exist only on
paper. Though each has something unique to offer, we feel Tor has
advantages over each of them that make it a superior choice for most
users and applications. For example, unlike purely P2P designs we
neither limit ordinary users to content and services available only
within our network nor require them to take on responsibility for
connections outside the network, unless they separately choose to run
server nodes. Nonetheless because we support low-latency interactive
communications, end-to-end \emph{traffic correlation}
attacks~\cite{danezis:pet2004,defensive-dropping,SS03,hs-attack,bauer:tr2007}
allow an attacker who can observe both ends of a communication to
correlate packet timing and volume, quickly linking the initiator to
her destination.


Our defense lies in having a diverse enough set of nodes to prevent
most real-world adversaries from being in the right places to attack
users, by distributing each transaction over several nodes in the
network.  This ``distributed trust'' approach means the Tor network
can be safely operated and used by a wide variety of mutually
distrustful users, providing sustainability and security.

The Tor network has a broad range of users, making it difficult for
eavesdroppers to track them or profile interests. These include
ordinary citizens concerned about their privacy, corporations who
don't want to reveal information to their competitors, and law
enforcement and government intelligence agencies who need to do
operations on the Internet without being noticed.  Naturally,
organizations will not want to depend on others for their security.
If most participating providers are reliable, Tor tolerates some
hostile infiltration of the network.

This distribution of trust is central to the Tor philosophy and
pervades Tor at all levels: Onion routing has been open source since
the mid-nineties (mistrusting users can inspect the code themselves);
Tor is free software (anyone could take up the development of Tor from
the current team); anyone can use Tor without license or charge (which
encourages a broad user base with diverse interests); Tor is designed to be
usable (also promotes a large, diverse user base) and configurable (so
users can easily set up and run server nodes); the Tor
infrastructure is run by volunteers (it is not dependent on the
economic viability or business strategy of any company) who are
scattered around the globe (not completely under the jurisdiction of
any single country); ongoing development and deployment has been
funded by diverse sources (development does not fully depend on
funding from any one source or even funding for any one primary
purpose or sources in any one jurisdiction). All of these contribute
to Tor's resilience and sustainability.


\section{Social challenges}

Many of the issues the Tor project needs to address extend beyond
system design and technology development. In particular, the Tor
project's \emph{image} with respect to its users and the rest of the
Internet impacts the security it can provide.  With this image issue
in mind, this section discusses the Tor user base and Tor's
interaction with other services on the Internet.

\subsection{Communicating security}

Usability for anonymity systems contributes to their security, because
usability affects the possible anonymity set~\cite{econymics,back01}.
Conversely, an unusable system attracts few users and thus can't
provide much anonymity.

This phenomenon has a second-order effect: knowing this, users should
choose which anonymity system to use based in part on how usable and
secure \emph{others} will find it, in order to get the protection of a
larger anonymity set. Thus we might supplement the adage ``usability
is a security parameter''~\cite{back01} with a new one: ``perceived
usability is a security parameter.''~\cite{usability-network-effect}.



\subsection{Reputability and perceived social value}
Another factor impacting the network's security is its reputability,
the perception of its social value based on its current user base. If
Alice is the only user who has ever downloaded the software, it might
be socially accepted, but she's not getting much anonymity. Add a
thousand activists, and she's anonymous, but everyone thinks she's an
activist too.  Add a thousand diverse citizens (cancer survivors,
people concerned about identity theft, law enforcement agents, and so
on) and now she's harder to profile.

Furthermore, the network's reputability affects its operator base:
more people are willing to run a service if they believe it will be
used by human rights workers than if they believe it will be used
exclusively for disreputable ends.  This effect becomes stronger if
node operators themselves think they will be associated with their
users' ends.

So the more cancer survivors on Tor, the better for the human rights
activists. The more malicious hackers, the worse for the normal
users. Thus, reputability is an anonymity issue for two
reasons. First, it impacts the sustainability of the network: a
network that's always about to be shut down has difficulty attracting
and keeping adequate nodes.  Second, a disreputable network is more
vulnerable to legal and political attacks, since it will attract fewer
supporters.

Reputability becomes even more tricky in the case of privacy networks,
since the good uses of the network (such as publishing by journalists
in dangerous countries, protecting road warriors from profiling and
potential physical harm, tracking of criminals by law enforcement,
protecting corporate research interests, etc.) are typically kept private,
whereas network abuses or other problems tend to be more widely
publicized.


\subsection{Abuse}
\label{subsec:tor-and-blacklists}

For someone willing to be antisocial or even break the law, Tor is
usually a poor choice to hide bad behavior. For example, Tor nodes are
publicly identified, unlike the million-node botnets that are now
common on the Internet. Nonetheless, we always expected that,
alongside legitimate users, Tor would also attract troublemakers who
exploit Tor to abuse services on the Internet with vandalism, rude
mail, and so on.  \emph{Exit policies} have allowed individual nodes
to block access to specific IP/port ranges.  This approach aims to
make operators more willing to run Tor by allowing them to prevent
their nodes from being used for abusing particular services.  For
example, by default Tor nodes block SMTP (port 25), to avoid the issue
of spam.

Exit policies are useful but insufficient: if not all nodes block a
given service, that service may try to block Tor instead.  While being
blockable is important to being good netizens, we would like to
encourage services to allow anonymous access. Services should not need
to decide between blocking legitimate anonymous use and allowing
unlimited abuse.  Nonetheless, blocking IP addresses is a
course-grained solution~\cite{netauth}: entire apartment buildings,
campuses, and even countries sometimes share a single IP address.
Also, whether intended or not, such blocking supports repression of
free speech. In many locations where Internet access of various kinds
is censored or even punished by imprisonment, Tor is a path both to
the outside world and to others inside.  Blocking posts from Tor makes
the job of censoring authorities easier.  This is a loss for both Tor
and services that block, such as Wikipedia: we don't want to compete
for (or divvy up) the NAT-protected entities of the world.  This is
also unfortunate because there are relatively simple technical
solutions~\cite{nym}.  Various schemes for escrowing anonymous posts
until they are reviewed by editors would both prevent abuse and remove
incentives for attempts to abuse. Further, pseudonymous reputation
tracking of posters through Tor would allow those who establish
adequate reputation to post without escrow~\cite{nym,nymble}.

We stress that as far as we can tell, most Tor uses are not
abusive. Most services have not complained, and others are actively
working to find ways besides banning to cope with the abuse. For
example, the Freenode IRC network had a problem with a coordinated
group of abusers joining channels and subtly taking over the
conversation; but when they labelled all users coming from Tor IP
addresses as ``anonymous users,'' removing the ability of the abusers
to blend in, the abusers stopped using Tor.  This is an illustration of
how simple
technical mechanisms can remove the ability to abuse anonymously
without undermining the ability to communicate anonymously and can
thus remove the incentive to attempt abusing in this way.



\section{The Future}
\label{sec:conclusion}

Tor is the largest and most diverse low-latency anonymity network
available, but we are still in the early stages. Several major
questions remain.

First, will our volunteer-based approach to sustainability continue to
work as well in the long term as it has the first several years?
Besides node operation, Tor research, deployment, maintainance, and
development is increasingly done by volunteers: package maintenance
for various OSes, document translation, GUI design and implementation,
live CDs, specification of new design changes, etc.\
%
Second, Tor is only one of many components that preserve privacy
online.  For applications where it is desirable to keep identifying
information out of application traffic, someone must build more and
better protocol-aware proxies that are usable by ordinary people.
%
Third, we need to maintain a reputation for social good, and learn how to
coexist with the variety of Internet services and their established
authentication mechanisms. We can't just keep escalating the blacklist
standoff forever.
%
Fourth, the current Tor architecture hardly scales even to handle
current user demand. We must deploy designs and incentives to further
encourage clients to relay traffic too, without thereby trading away
too much anonymity or other properties.

These are difficult and open questions. Yet choosing not to solve them
means leaving most users to a less secure network or no anonymizing
network at all.

\bibliographystyle{plain} \bibliography{tor-design}

\end{document}

