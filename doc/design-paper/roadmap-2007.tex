\documentclass{article}

\newenvironment{tightlist}{\begin{list}{$\bullet$}{
  \setlength{\itemsep}{0mm}
    \setlength{\parsep}{0mm}
    %  \setlength{\labelsep}{0mm}
    %  \setlength{\labelwidth}{0mm}
    %  \setlength{\topsep}{0mm}
    }}{\end{list}}
\newcommand{\tmp}[1]{{\bf #1} [......] \\}

\begin{document}

\title{Tor Development Roadmap: Wishlist for Nov 2006--Dec 2007}
\author{Roger Dingledine \and Nick Mathewson \and Shava Nerad}

\maketitle
\pagestyle{plain}

% TO DO:
%   add cites
%   add time estimates


\section{Introduction}
Hi, Roger!  Hi, Shava.  This paragraph should get deleted soon.  Right now,
this document goes into about as much detail as I'd like to go into for a
technical audience, since that's the audience I know best.  It doesn't have
time estimates everywhere.  It isn't well prioritized, and it doesn't
distinguish well between things that need lots of research and things that
don't.  The breakdowns don't all make sense.  There are lots of things where
I don't make it clear how they fit into larger goals, and lots of larger
goals that don't break down into little things. It isn't all stuff we can do
for sure, and it isn't even all stuff we can do for sure in 2007.  The
tmp\{\} macro indicates stuff I haven't said enough about.  That said, here
goes...

Tor (the software) and Tor (the overall software/network/support/document
suite) are now experiencing all the crises of success.  Over the next year,
we're probably going to grow more in terms of users, developers, and funding
than before.  This gives us the opportunity to perform long-neglected
maintenance tasks.

\section{Code and design infrastructure}

\subsection{Protocol revision}
To maintain backward compatibility, we've postponed major protocol
changes and redesigns for a long time.  Because of this, there are a number
of sensible revisions we've been putting off until we could deploy several of
them at once.  To do each of these, we first need to discuss design
alternatives with other cryptographers and outside collaborators to
make sure that our choices are secure.

First of all, our protocol needs better {\bf versioning support} so that we
can make backward-incompatible changes to our core protocol.  There are
difficult anonymity issues here, since many naive designs would make it easy
to tell clients apart (and then track them) based on their supported versions.

With protocol versioning support would come the ability to {\bf future-proof
  our ciphersuites}.  For example, not only our OR protocol, but also our
directory protocol, is pretty firmly tied to the SHA-1 hash function, which
though not yet known to be insecure for our purposes, has begun to show
its age.  We should
remove assumptions thoughout our design based on the assumption that public
keys, secret keys, or digests will remain any particular size indefinitely.

A new protocol could support {\bf multiple cell sizes}.  Right now, all data
passes through the Tor network divided into 512-byte cells.  This is
efficient for high-bandwidth protocols, but inefficient for protocols
like SSH or AIM that send information in small chunks.  Of course, we need to
investigate the extent to which multiple sizes could make it easier for an
adversary to fingerprint a traffic pattern.

Our OR {\bf authentication protocol}, though provably
secure\cite{tap:pet2006}, relies more on particular aspects of RSA and our
implementation thereof than we had initially believed.  To future-proof
against changes, we should replace it with a less delicate approach.

We might design a {\bf stream migration} feature so that streams tunneled
over Tor could be more resilient to dropped connections and changed IPs.

As a part of our design, we should investigate possible {\bf cipher modes}
other than counter mode.  For example, a mode with built-in integrity
checking, error propagation, and random access could simplify our protocol
significantly.  Sadly, many of these are patented and unavailable for us.


\subsection{Scalability}

\subsubsection{Improved directory efficiency}
Right now, clients download a statement of the {\bf network status} made by
each directory authority.  We could reduce network bandwidth significantly by
having the authorities jointly sign a statement reflecting their vote on the
current network status.  This would save clients up to 160K per hour, and
make their view of the network more uniform.  Of course, we'd need to make
sure the voting process was secure and resilient to failures in the network.

We should {\bf shorten router descriptors}, since the current format includes
a great deal of information that's only of interest to the directory
authorities, and not of interest to clients.  We can do this by having each
router upload a short-form and a long-form signed descriptor, and having
clients download only the short form.  Even a naive version of this would
save about 40\% of the bandwidth currently spent by clients downloading
descriptors.

We should {\bf have routers upload their descriptors even less often}, so
that clients do not need to download replacements every 18 hours whether any
information has changed or not.  (As of Tor 0.1.2.3-alpha, clients tolerate
routers that don't upload often, but routers still upload at least every 18
hours to support older clients.)

\subsubsection{Non-clique topology}
Our current network design achieves a certain amount of its anonymity by
making clients act like each other through the simple expedient of making
sure that all clients know all servers, and that any server can talk to any
other server.  But as the number of servers increases to serve an
ever-greater number of clients, these assumptions become impractical.

At worst, if these scalability issues become troubling before a solution is
found, we can design and build a solution to {\bf split the network into
multiple slices} until a better solution comes along.  This is not ideal,
since rather than looking like all other users from a point of view of path
selection, users would ``only'' look like 200,000--300,000 other users.

We are in the process of designing {\bf improved schemes for network
  scalability}.  Some approaches focus on limiting what an adversary can know
about what a user knows; others focus on reducing the extent to which an
adversary can exploit this knowledge.  These are currently in their infancy,
and will probably not be needed in 2007, but they must be designed in 2007 if
they are to be deployed in 2008.

\subsubsection{Relay incentives}
To support more users on the network, we need to get more servers.  So far,
we've relied on volunteerism to attract server operators, and so far it's
served us well.  But in the long run, we need to {\bf design incentices for
  users to run servers} and relay traffic for others.  Most obviously, we
could try to build the network so that servers offered improved service for
other servers, but we would need to do so without weakening anonymity and
making it obvious which connections originate from users running servers.  We
have some preliminary designs here~\cite{challenges}, but need to perform
some more research to make sure they would be safe and effective.

\subsection{Portability}
Our {\bf Windows implementation}, though much improved, continues to lag
behind Unix and Mac OS X, especially when running as a server.  We hope to
merge promising patches from Mike Chiussi to address this point, and bring
Windows performance on par with other platforms.

We should have {\bf better support for portable devices}, including modes of
operation that require less RAM, and that write to disk less frequently (to
avoid wearing out flash RAM).

\subsection{Performance: resource usage}
We've been working on {\bf using less RAM}, especially on servers.  This has
paid off a lot for directory caches in the 0.1.2, which in some cases are
using 90\% less memory than they used to require.  But we can do better,
especially in the area around our buffer management algorithms, by using an
approach more like the BSD and Linux kernels use instead of our current ring
buffer approach.  (For OR connections, we can just use queues of cell-sized
chunks produced with a specialized allocator.)  This could potentially save
around 25 to 50\% of the memory currently allocated for network buffers, and
make Tor a more attractive proposition for restricted-memory environments
like old computers, mobile devices, and the like.

We should improve our {\bf bandwidth limiting}.  The current system has been
crucial in making users willing to run servers: nobody is willing to run a
server if it might use an unbounded amount of bandwidth, especially if they
are charged for their usage.  We can make our system better by letting users
configure bandwidth limits independently for their own traffic and traffic
relayed for others; and by adding write limits for users running directory
servers.

On many hosts, sockets are still in short supply, and will be until we can
migrate our protocol to UDP.  We can {\bf use fewer sockets} by making our
self-to-self connections happen internally to the code rather than involving
the operating system's socket implementation.

\subsection{Performance: network usage}
We know too little about how well our current path
selection algorithms actually spread traffic around the network in practice.
We should {\bf research the efficacy of our traffic allocation} and either
assure ourselves that it is close enough to optimal as to need no improvement
(unlikely) or {\bf identify ways to improve network usage}, and get more
users' traffic delivered faster.  Performing this research will require
careful thought about anonymity implications.

We should also {\bf examine the efficacy of our congestion control
  algorithm}, and see whether we can improve client performance in the
presence of a congested network through dynamic `sendme' window sizes or
other means.  This will have anonymity implications too if we aren't careful.

% \tmp{Tune pathgen algorithms to use it better.}
% 
% I think I've included this in the above -NM

\subsection{Performance scenario: one Tor client, many users}
We should {\bf improve Tor's performance when a single Tor handles many
  clients}.  Many organizations want to manage a single Tor client on their
firewall for many users, rather than having each user install a separate
Tor client.  We haven't optimized for this scenario, and it is likely that
there are some code paths in the current implementation that become
inefficient when a single Tor is servicing hundreds or thousands of client
connections.  (Additionally, it is likely that such clients have interesting
anonymity requirements the we should investigate.)  We should profile Tor
under appropriate loads, identify bottlenecks, and fix them.

% \tmp{Other stress-testing, and fix bottlenecks we find.}
%
% I've moved this into 'improved testing harness' below

\subsection{Tor servers on asymmetric bandwidth}

\tmp{Roger, please write? I don't know what to say here.}

\subsection{Running Tor as both client and server}

\tmp{many performance tradeoffs and balances that need more attention.
  Roger, please write.}

\subsection{Protocol redesign for UDP}
Tor has relayed only TCP traffic since its first versions, and has used
TLS-over-TCP to do so.  This approach has proved reliable and flexible, but
in the long term we will need to allow UDP traffic on the network, and switch
some or all of the network to using a UDP transport.  {\bf Supporting UDP
  traffic} will make Tor more suitable for protocols that require UDP, such
as many VOIP protocols.  {\bf Using a UDP transport} could greatly reduce
resource limitations on servers, and make the network far less interruptable
by lossy connections.  Either of these protocol changes would require a great
deal of design work, however.  We hope to be able to enlist the aid of a few
talented graduate students to assist with the initial design and
specification, but the actual implementation will require significant testing
of different reliable transport approaches.


\section{Blocking resistance}

\subsection{Design for blocking resistance}
We have written a design document explaining our general approach to blocking
resistance.  We should workshop it with other experts in the field to get
their ideas about how we can improve Tor's efficacy as an anti-censorship
tool.


\subsection{Implementation: client-side and bridges-side}
Our anticensorship design calls for some nodes to act as ``bridges'' that can
circumvent a national firewall, and others inside the firewall to act as pure
clients.  This part of the design is quite clear-cut; we're probably ready to begin
implementing it.  To implement bridges, we need only to have servers publish
themselves as limited-availability relays to a special bridge authority if
they judge they'd make good servers.  Clients need a flexible interface to
learn about bridges and to act on knowledge of bridges.

Clients also need to {\bf use the encrypted directory variant} added in Tor
0.1.2.3-alpha.  This will let them retrieve directory information over Tor
once they've got their initial bridges.

Bridges will want to be able to {\bf listen on multiple addresses and ports}
if they can, to give the adversary more ports to block.

Additionally, we should {\bf resist content-based filters}.  Though an
adversary can't see what users are saying, some aspects of our protocol are
easy to fingerprint {\em as} Tor.  We should correct this where possible.

\subsection{Implementation: bridge authorities}

The design here is also reasonably clear-cut: we need to run some
directory authorities with a slightly modified protocol that doesn't leak
the entire list of bridges. Thus users can learn up-to-date information
for bridges they already know about, but they can't learn about arbitrary
new bridges.

\subsection{Implementation: how users discover bridges}

Our design anticipates an arms race between discovery methods and censors.
We need to begin the infrastructure on our side quickly, preferably in a
flexible language like Python, so we can adapt quickly to censorship.

\subsection{Resisting censorship of the Tor website, docs, and mirrors}

We should take some effort to consider {\bf initial distribution of Tor and
  related information} in countries where the Tor website and mirrors are
censored.  (Right now, most countries that block access to Tor block only the
main website and leave mirrors and the network itself untouched.)  Falling
back on word-of-mouth is always a good last resort, but we should also take
steps to make sure it's relatively easy for users to get ahold of a copy.

\section{Security}

\subsection{Security research projects}

We should investigate approaches with some promise to help Tor resist
end-to-end traffic correlation attacks.  It's an open research question
whether (and to what extent) {\bf mixed-latency} networks, {\bf low-volume
  long-distance padding}, or other approaches can resist these attacks, which
are currently some of the most effective against careful Tor users.  We
should research these questions and perform simulations to identify
opportunities for strengthening our design without dropping performance to
unacceptable levels. %Cite something

We've got some preliminary results suggesting that {\bf a topology-aware
  routing algorithm}~\cite{routing-zones} could reduce Tor users'
vulnerability against local or ISP-level adversaries, by ensuring that they
are never in a position to watch both ends of a connection.  We need to
examine the effects of this approach in more detail and consider side-effects
on anonymity against other kinds of adversaries.  If the approach still looks
promising, we should investigate ways for clients to implement it (or an
approximation of it) without having to download routing tables for the whole
internet.

%\tmp{defenses against end-to-end correlation}  We don't expect any to work
%right now, but it would be useful to learn that one did.  Alternatively,
%proving that one didn't would free up researchers in the field to go work on
%other things.
%
% See above; I think I got this.

We should research the efficacy of {\bf website fingperprinting} attacks,
wherein an adversary tries to match the distinctive traffic and timing
pattern of the resources constituting a given website to the traffic pattern
of a user's client.  These attacks work great in simulations, but in
practice we hear they don't work nearly as well.  We should get some actual
numbers to investigte the issue, and figure out what's going on.  If we
resist these attacks, or can improve our design to resist them, we should.
% add cites

\subsection{Implementation security}
Right now, each Tor node stores its keys unencrypted.  We should {\bf encrypt
  more Tor keys} so that Tor authorities can require a startup password.  We
should look into adding intermediary medium-term ``signing keys'' between
identity keys and onion keys, so that a password could be required to replace
a signing key, but not to start Tor.  This would improve Tor's long-term
security, especially in its directory authority infrastructure.

We should also {\bf mark RAM that holds key material as non-swappable} so
that there is no risk of recovering key material from a hard disk
compromise.  This would require submitting patches upstream to OpenSSL, where
support for marking memory as sensitive is currently in a very preliminary
state.

There are numerous tools for identifying trouble spots in code (such as
Coverity or even VS2005's code analysis tool) and we should convince somebody
to run some of them against the Tor codebase.  Ideally, we could figure out a
way to get our code checked periodically rather than just once.

We should try {\bf protocol fuzzing} to identify errors in our
implementation.

Our guard nodes help prevent an attacker from being able to become a chosen
client's entry point by having each client choose a few favorite entry points
as ``guards'' and stick to them.   We should implement a {\bf directory
  guards} feature to keep adversaries from enumerating Tor users by acting as
a directory cache.

\subsection{Detect corrupt exits and other servers}
With the success of our network, we've attracted servers in many locations,
operated by many kinds of people.  Unfortunately, some of these locations
have compromised or defective networks, and some of these people are
untrustworthy or incompetent.  Our current design relies on authority
administrators to identify bad nodes and mark them as nonfunctioning.  We
should {\bf automate the process of identifying malfunctioning nodes} as
follows:

We should create a generic {\bf feedback mechanism for add-on tools} like
Mike Perry's ``Snakes on a Tor'' to report failing nodes to authorities.

We should write tools to {\bf detect more kinds of innocent node failure},
such as nodes whose network providers intercept SSL, nodes whose network
providers censor popular websites, and so on.  We should also try to detect
{\bf routers that snoop traffic}; we could do this by launching connections
to throwaway accounts, and seeing which accounts get used.

We should add {\bf an efficient way for authorities to mark a set of servers
  as probably collaborating} though not necessarily otherwise dishonest.
This happens when an administrator starts multiple routers, but doesn't mark
them as belonging to the same family.

To avoid attacks where an adversary claims good performance in order to
attract traffic, we should {\bf have authorities measure node performance}
(including stability and bandwidth) themselves, and not simply believe what
they're told.  Measuring bandwidth can be tricky, since it's hard to
distinguish between a server with low capacity, and a high-capacity server
with most of its capacity in use.

{\bf Operating a directory authority should be easier.}  We rely on authority
operators to keep the network running well, but right now their job involves
too much busywork and administrative overhead.  A better interface for them
to use could free their time to work on exception cases rather than on
adding named nodes to the network.

\subsection{Protocol security}

In addition to other protocol changes discussed above,
% And should we move somve of them down here? -NM
we should add {\bf hooks for denial-of-service resistance}; we have some
prelimiary designs, but we shouldn't postpone them until we realy need them.
If somebody tries a DDoS attack against the Tor network, we won't want to
wait for all the servers and clients to upgrade to a new version.

\section{Development infrastructure}

\subsection{Build farm}
We've begun to deploy a cross-platform distributed build farm of hosts
that build and test the Tor source every time it changes in our development
repository.

We need to {\bf get more participants}, so that we can test a larger variety
of platforms.  (Previously, we've only found out when our code had broken on
obscure platforms when somebody got around to building it.)

We need also to {\bf add our dependencies} to the build farm, so that we can
ensure that libraries we need (especially libevent) do not stop working on
any important platform between one release and the next.

\subsection{Improved testing harness}
Currently, our {\bf unit tests} cover only about XX\% of the code base.  This
is uncomfortably low; we should write more and switch to a more flexible
testing framework.

We should also write flexible {\bf automated single-host deployment tests} so
we can more easily verify that the current codebase works with the network.

We should build automated {\bf stress testing} frameworks so we can see which
realistic loads cause Tor to perform badly, and regularly profile Tor against
these loads.  This would give us {\it in vitro} performance values to
supplement our deployment experience.

\subsection{Centralized build system}
We currently rely on a separate packager to maintain the packaging system and
to build Tor on each platform for which we distribute binaries.  Separate
package maintainers is sensible, but separate package builders has meant
long turnaround times between source releases and package releases.  We
should create the necessary infrastructure for us to produce binaries for all
major packages within an hour or so of source release.

\subsection{Improved metrics}
We need a way to {\bf measure the network's health, capacity, and degree of
  utilization}.  Our current means for doing this are ad hoc and not
completely accurate.

We need better ways to {\bf tell which countries are users are coming from,
  and how many there are}.  A good perspective of the network helps us
allocate resources and identify trouble spots, but our current approaches
will work less and less well as we make it harder for adversaries to
enumerate users.  We'll probably want to shift to a smarter, statistical
approach rather than our current ``count and extrapolate'' method.

% \tmp{We'd like to know how much of the network is getting used.}
% I think this is covered above -NM

\subsection{Controller library}
We've done lots of design and development on our controller interface, which
allows UI applications and other tools to interact with Tor.  We could
encourage the development of more such tools by releasing a {\bf
  general-purpose controller library}, ideally with API support for several
popular programming languages.

\section{User experience}

\subsection{Get blocked less, get blocked less broadly}
Right now, some services block connections from the Tor network because
they don't have a better
way to keep vandals from abusing them than blocking IP addresses associated
with vandalism.  Our approach so far has been to educate them about better
solutions that currently exist, but we should also {\bf create better
solutions for limiting vandalism by anonymous users} like credential and
blind-signature based implementations, and encourage their use. Other
promising starting points including writing a patch and explanation for
Wikipedia, and helping Freenode to document, maintain, and expand its
current Tor-friendly position.

Those who do block Tor users also block overbroadly, sometimes blacklisting
operators of Tor servers that do not permit exit to their services.  We could
obviate innocent reasons for doing so by designing a {\bf narrowly-targeted Tor
  RBL service} so that those who wanted to overblock Tor clould no longer
plead incompetence.

\subsection{All-in-one bundle}
We need a well-tested, well-documented bundle of Tor and supporting
applications configured to use it correctly.  We have an intial
implementation well under way, but it will need additional work in
identifying requisite Firefox extensions, identifying security threats,
improving user experience, and so on.  This will need significantly more work
before it's ready for a general public release.

\subsection{LiveCD Tor}
We need a nice bootable livecd containing a minimal OS and a few applications
configured to use it correctly.  The Anonym.OS project demonstrated that this
is quite feasible, but their project is not currently maintained.

\subsection{A Tor client in a VM}
\tmp{a.k.a JanusVM} which is quite related to the firewall-level deployment
section below .  Roger, can you write this?

%\subsection{Interface improvements}
%\tmp{Allow controllers to manipulate server status.}
% (Why is this in the User Experience section?) -RD
% I think it's better left to a generic ``make controller iface bettter'' item.

\subsection{Firewall-level deployment}
Another useful deployment mode for some users is using {\bf Tor in a firewall
  configuration}, and directing all their traffic through Tor.  This can be a
little tricky to set up currently, but it's an effective way to make sure no
traffic leaves the host un-anonymized.  To achieve this, we need to {\bf
  improve and port our new TransPort} feature which allows Tor to be used
without SOCKS support; to {\bf add an anonymizing DNS proxy} feature to Tor;
and to {\bf construct a recommended set of firewall configurations} to redirect
traffic to Tor.

This is an area where {\bf deployment via a livecd}, or an installation
targetted at specialized home routing hardware, could be useful.

\subsection{Assess software and configurations for anonymity risks}
Right now, users and packagers are more or less on their own when selecting
firefox extensions.  We should {\bf assemble a recommended list of browser
  extensions} through experiment, and include this in the application bundles
we distribute.

We should also describe {\bf best practices for using Tor with each class of
  application}.  \tmp{Roger, say more}

The Foxtor and Torbutton extensions serve similar purposes; we should pick a
favorite, and merge in the useful features of the other.

%\tmp{clean up our own bundled software:
%E.g. Merge the good features of Foxtor into Torbutton}
%
% What else did you have in mind? -NM

\subsection{Localization}
Right now, most of our user-facing code is internationalized.  We need to
internationalize the last few hold-outs (like the Tor installer), and get
more translations for the parts that are already internationalized.

%[Do you mean the Vidalia bundle installer, or the Tor-installer-for-experts?
%-RD]
% The latter -NM

Also, we should look into a {\bf unified translator's solution}.  Currently,
since different tools have been internationalized using the
framework-appropriate method, different tools require translators to localize
them via different interfaces.  Inasmuch as possible, we should make
translators only need to use a single tool to translate the whole Tor suite.

\section{Support}

\tmp{would be nice to set up some actual user support infrastructure, especially
focusing on server operators and on coordinating volunteers.} Roger, can you
write this ?  I don't know what ``user support infrastructure'' is.


\section{Documentation}

\subsection{Unified documentation scheme}

We need to {\bf inventory our documentation.}  Our documentation so far has
been mostly produced on an {\it ad hoc} basis, in response to particular
needs and requests.  We should figure out what documentation we have, which of
it (if any) should get priority, and whether we can't put it all into a
single format.

We could {\bf unify the docs} into a single book-like thing.  This will also
help us identify what sections of the ``book'' are missing.

\subsection{Missing technical documentation}

We should {\bf revise our design paper} to reflect the new decisions and
research we've made since it was published in 2004.  This will help other
researchers evaluate and suggest improvements to Tor's current design.

Other projects sometimes implement the client side of our prototocol.  We
encourage this, but we should write {\bf a document about how to avoid
excessive resource use}, so we don't need to worry that they will do so
without regard to the effect of their choices on server resources.

\subsection{Missing user documentation}

Our documentation falls into two broad categories: some is `discoursive' and
explains in detail why users should take certain actions, and other
documenation is `comprehensive' and describes all of Tor's features.  Right
now, we have no document that is both deep, readable, and thorough.  We
should correct this by identifying missing spots in our design.

\bibliographystyle{plain} \bibliography{tor-design}

\end{document}

