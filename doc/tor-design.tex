\documentclass[times,10pt,twocolumn]{article}
\usepackage{latex8}
\usepackage{times}
\usepackage{url}
\usepackage{graphics}
\usepackage{amsmath}

\pagestyle{empty}

\renewcommand\url{\begingroup \def\UrlLeft{<}\def\UrlRight{>}\urlstyle{tt}\Url}
\newcommand\emailaddr{\begingroup \def\UrlLeft{<}\def\UrlRight{>}\urlstyle{tt}\Url}

% If an URL ends up with '%'s in it, that's because the line *in the .bib/.tex
% file* is too long, so break it there (it doesn't matter if the next line is
% indented with spaces). -DH

%\newif\ifpdf
%\ifx\pdfoutput\undefined
%   \pdffalse
%\else
%   \pdfoutput=1
%   \pdftrue
%\fi

\newenvironment{tightlist}{\begin{list}{$\bullet$}{
  \setlength{\itemsep}{0mm}
    \setlength{\parsep}{0mm}
    %  \setlength{\labelsep}{0mm}
    %  \setlength{\labelwidth}{0mm}
    %  \setlength{\topsep}{0mm}
    }}{\end{list}}

\begin{document}

%% Use dvipdfm instead. --DH
%\ifpdf
%  \pdfcompresslevel=9
%  \pdfpagewidth=\the\paperwidth
%  \pdfpageheight=\the\paperheight
%\fi

\title{Tor: The Second-Generation Onion Router}
% Putting the 'Private' back in 'Virtual Private Network'

%\author{Roger Dingledine \\ The Free Haven Project \\ arma@freehaven.net \and
%Nick Mathewson \\ The Free Haven Project \\ nickm@freehaven.net \and
%Paul Syverson \\ Naval Research Lab \\ syverson@itd.nrl.navy.mil}

\maketitle
\thispagestyle{empty}

\begin{abstract}
We present Tor, a circuit-based low-latency anonymous communication
system. This second-generation Onion Routing system addresses limitations
in the original design. We add perfect forward secrecy, congestion
control, directory servers, integrity checking, variable exit policies,
and a practical design for rendezvous points. Tor works on the real-world
Internet, requires no special privileges or kernel modifications, requires
little synchronization or coordination between nodes, and provides a
reasonable trade-off between anonymity, usability, and efficiency. We
close with a list of open problems in anonymous communication systems.
\end{abstract}

%\begin{center}
%\textbf{Keywords:} anonymity, peer-to-peer, remailer, nymserver, reply block
%\end{center}

%%%%%%%%%%%%%%%%%%%%%%%%%%%%%%%%%%%%%%%%%%%%%%%%%%%%%%%%%%%%%%%%%%%%%%%

\Section{Overview}
\label{sec:intro}

Onion Routing is a distributed overlay network designed to anonymize
low-latency TCP-based applications such as web browsing, secure shell,
and instant messaging. Clients choose a path through the network and
build a \emph{virtual circuit}, in which each node (or ``onion router'')
in the path knows its predecessor and successor, but no other nodes in
the circuit.  Traffic flowing down the circuit is sent in fixed-size
\emph{cells}, which are unwrapped by a symmetric key at each node
(like the layers of an onion) and relayed downstream. The original
Onion Routing project published several design and analysis papers
\cite{or-ih96,or-jsac98,or-discex00,or-pet00}. While a wide area Onion
Routing network was deployed for some weeks, the only long-running and
publicly accessible implementation of the original design was a fragile
proof-of-concept that ran on a single machine. Even this simple deployment
processed tens of thousands of connections daily from thousands of users
worldwide. But many critical design and deployment issues were never
resolved, and the design has not been updated in several years. Here
we describe Tor, a protocol for asynchronous, loosely federated onion
routers that provides the following improvements over the old Onion
Routing design:

\begin{tightlist}

\item \textbf{Perfect forward secrecy:} The original Onion Routing
design was vulnerable to a single hostile node recording traffic and
later compromising successive nodes in the circuit and forcing them
to decrypt it.  Rather than using a single onion to lay each circuit,
Tor now uses an incremental or \emph{telescoping} path-building design,
where the initiator negotiates session keys with each successive hop in
the circuit.  Once these keys are deleted, subsequently compromised nodes
cannot decrypt old traffic.  As a side benefit, onion replay detection
is no longer necessary, and the process of building circuits is more
reliable, since the initiator knows when a hop fails and can then try
extending to a new node.

\item \textbf{Separation of protocol cleaning from anonymity:}
The original Onion Routing design required a separate ``application
proxy'' for each supported application protocol---most of which were
never written, so many applications were never supported.  Tor uses the
standard and near-ubiquitous SOCKS \cite{socks4} proxy interface, allowing
us to support most TCP-based programs without modification.  This design
change allows Tor to use the filtering features of privacy-enhancing
application-level proxies such as Privoxy \cite{privoxy} without having
to incorporate those features itself.

\item \textbf{Many TCP streams can share one circuit:} The
original Onion Routing design built a separate circuit for each
application-level request.  This hurt performance by requiring
multiple public key operations for every request, and also presented
a threat to anonymity from building so many different circuits; see
Section~\ref{sec:maintaining-anonymity}.  Tor multiplexes multiple TCP
streams along each virtual circuit, to improve efficiency and anonymity.

\item \textbf{Leaky-pipe circuit topology:} Through in-band signalling
within the circuit, Tor initiators can direct traffic to nodes partway
down the circuit. This novel approach allows both for long-range
padding to frustrate traffic shape and volume attacks at the initiator
\cite{defensive-dropping}, and, because circuits are used by more than one
application, allows traffic to exit the circuit from the middle---thus
frustrating traffic shape and volume attacks based on observing the end
of the circuit.

\item \textbf{No mixing, padding, or traffic shaping:} The original Onion
Routing design called for batching and reordering the cells arriving from
each circuit. It also included padding between onion routers and, in a
later design, between onion proxies (that is, users) and onion routers
\cite{or-ih96,or-jsac98}.  The trade-off between padding protection
and cost was discussed, but no general padding scheme was suggested. In
\cite{or-pet00} it was theorized \emph{traffic shaping} would generally
be used, but details were not provided.  Recent research \cite{econymics}
and deployment experience \cite{freedom21-security} suggest that this
level of resource use is not practical or economical; and even full link
padding is still vulnerable \cite{defensive-dropping}. Thus, until we
have a proven and convenient design for traffic shaping or low-latency
mixing that will improve anonymity against a realistic adversary, we
leave these strategies out.

\item \textbf{Congestion control:} Earlier anonymity designs do not
address traffic bottlenecks. Unfortunately, typical approaches to
load balancing and flow control in overlay networks involve inter-node
control communication and global views of traffic. Tor's decentralized
congestion control uses end-to-end acks to maintain reasonable anonymity
while allowing nodes at the edges of the network to detect congestion
or flooding attacks and send less data until the congestion subsides.

\item \textbf{Directory servers:} The original Onion Routing design
planned to flood link-state information through the network---an approach
that can be unreliable and open to partitioning attacks or outright
deception. Tor takes a simplified view toward distributing link-state
information. Certain more trusted onion routers also act as directory
servers: they provide signed \emph{directories} that describe known
routers and their availability.  Users periodically download these
directories via HTTP.

\item \textbf{Variable exit policies:} Tor provides a consistent mechanism
for each node to specify and advertise a policy describing the hosts
and ports to which it will connect. These exit policies are critical
in a volunteer-based distributed infrastructure, because each operator
is comfortable with allowing different types of traffic to exit the Tor
network from his node.

\item \textbf{End-to-end integrity checking:} The original Onion Routing
design did no integrity checking on data. Any onion router on the
circuit could change the contents of data cells as they passed by---for
example, to alter a connection request on the fly so it would connect
to a different webserver, or to `tag' encrypted traffic and look for
corresponding corrupted traffic at the network edges \cite{minion-design}.
Tor hampers these attacks by checking data integrity before it leaves
the network.

\item \textbf{Improved robustness to failed nodes:} A failed node
in the old design meant that circuit-building failed, but thanks to
Tor's step-by-step circuit building, users can notice failed nodes
while building circuits and route around them. Additionally, liveness
information from directories allows users to avoid unreliable nodes in
the first place.

\item \textbf{Rendezvous points and location-protected servers:}
Tor provides an integrated mechanism for responder anonymity via
location-protected servers.  Previous Onion Routing designs included
long-lived ``reply onions'' that could be used to build virtual circuits
to a hidden server, but these reply onions did not provide forward
security, and would become useless if any node in the path went down
or rotated its keys.  In Tor, clients negotiate {\it rendezvous points}
to connect with hidden servers; reply onions are no longer required.
\end{tightlist}

Unlike anonymity systems like Freedom \cite{freedom2-arch}, Tor only
attempts to anonymize TCP streams. Because it does not require patches
(or built-in support) in an operating system's network stack, this
approach has proven valuable to Tor's portability and deployability.

We have implemented most of the above features. Our source code is
available under a free license, and we believe it to be unencumbered by
patents. We have recently begun deploying a widespread alpha network
to test the design in practice, to get more experience with usability
and users, and to provide a research platform for experimenting with
new ideas.

We review previous work in Section~\ref{sec:related-work}, describe
our goals and assumptions in Section~\ref{sec:assumptions},
and then address the above list of improvements in
Sections~\ref{sec:design}-\ref{sec:rendezvous}. We summarize
in Section~\ref{sec:analysis} how our design stands up to
known attacks, and conclude with a list of open problems in
Section~\ref{sec:maintaining-anonymity} and future work for the Onion
Routing project in Section~\ref{sec:conclusion}.

%%%%%%%%%%%%%%%%%%%%%%%%%%%%%%%%%%%%%%%%%%%%%%%%%%%%%%%%%%%%%%%%%%%%%%%

\Section{Related work}
\label{sec:related-work}

Modern anonymity systems date to Chaum's Mix-Net design
\cite{chaum-mix}. Chaum
proposed hiding the correspondence between sender and recipient by
wrapping messages in layers of public key cryptography, and relaying them
through a path composed of ``Mixes.''  These mixes in turn decrypt, delay,
and re-order messages, before relaying them along the sender-selected
path towards their destinations.

Subsequent relay-based anonymity designs have diverged in two
principal directions.  Some have attempted to maximize anonymity at
the cost of introducing comparatively large and variable latencies,
including Babel \cite{babel}, Mixmaster \cite{mixmaster-spec}, and
Mixminion \cite{minion-design}.  Because of this
decision, these \emph{high-latency} networks are well-suited for anonymous
email, but introduce too much lag for interactive tasks such as web browsing,
internet chat, or SSH connections.

Tor belongs to the second category: \emph{low-latency} designs that
attempt to anonymize interactive network traffic. These systems handle
a variety of bidirectional protocols. They also provide more convenient
mail delivery than the high-latency fire-and-forget anonymous email
networks, because the remote mail server provides explicit delivery
confirmation. But because these designs typically
involve many packets that must be delivered quickly, it is
difficult for them to prevent an attacker who can eavesdrop both ends of the
communication from correlating the timing and volume
of traffic entering the anonymity network with traffic leaving it.  These
protocols are also vulnerable against active attacks in which an
adversary introduces timing patterns into traffic entering the network, and 
looks
for correlated patterns among exiting traffic.
Although some work has been done to frustrate
these attacks,\footnote{
  The most common approach is to pad and limit communication to a constant
  rate, or to limit
  the variation in traffic shape.  Doing so can have prohibitive bandwidth
  costs and/or performance limitations.
} most designs protect primarily against traffic analysis rather than traffic
confirmation \cite{or-jsac98}---that is, they assume that the attacker is
attempting to learn who is talking to whom, not to confirm a prior suspicion
about who is talking to whom.

The simplest low-latency designs are single-hop proxies such as the
Anonymizer \cite{anonymizer}, wherein a single trusted server strips the
data's origin before relaying it.  These designs are easy to
analyze, but require end-users to trust the anonymizing proxy. 
Concentrating the traffic to a single point increases the anonymity set
(the set of people a given user is hiding among), but it can make traffic
analysis easier: an adversary need only eavesdrop on the proxy to observe
the entire system.

More complex are distributed-trust, circuit-based anonymizing systems.
In these designs, a user establishes one or more medium-term bidirectional
end-to-end circuits, and tunnels TCP streams in fixed-size cells.
Establishing circuits is computationally expensive and typically
requires public-key
cryptography, whereas relaying cells is comparatively inexpensive and
typically requires only symmetric encryption.
Because a circuit crosses several servers, and each server only knows
the adjacent servers in the circuit, no single server can link a
user to her communication partners.

The Java Anon Proxy (also known as JAP or Web MIXes) uses fixed shared
routes known as \emph{cascades}.  As with a single-hop proxy, this
approach aggregates users into larger anonymity sets, but again an
attacker only needs to observe both ends of the cascade to bridge all
the system's traffic.  The Java Anon Proxy's design provides
protection by padding between end users and the head of the cascade
\cite{web-mix}. However, it is not demonstrated whether the current
implementation's padding policy improves anonymity.

PipeNet \cite{back01, pipenet}, another low-latency design proposed at
about the same time as the original Onion Routing design, provided
stronger anonymity at the cost of allowing a single user to shut
down the network simply by not sending.  Low-latency anonymous
communication has also been designed for other environments such as
ISDN \cite{isdn-mixes}.

In P2P designs like Tarzan \cite{tarzan:ccs02} and MorphMix
\cite{morphmix:fc04}, all participants both generate traffic and relay
traffic for others. These systems aim to prevent a peer
or observer from knowing whether a given peer originated a request
or just relayed it from another peer. While Tarzan and MorphMix use
layered encryption as above, Crowds \cite{crowds-tissec} simply assumes
an adversary who cannot observe the initiator: it uses no public-key
encryption, so nodes on a circuit can read that circuit's traffic.

Hordes \cite{hordes-jcs} is based on Crowds but also uses multicast
responses to hide the initiator. Herbivore \cite{herbivore} and P5
\cite{p5} go even further, requiring broadcast. They make anonymity
and efficiency trade-offs to make broadcast more practical.
These systems are designed primarily for communication between peers,
although Herbivore users can make external connections by
requesting a peer to serve as a proxy.

Systems like Freedom and the original Onion Routing build the circuit
all at once, using a layered ``onion'' of public-key encrypted messages,
each layer of which provides a set of session keys and the address of the
next server in the circuit. Tor as described herein, Tarzan, MorphMix,
Cebolla \cite{cebolla}, and AnonNet \cite{anonnet} build the circuit
in stages, extending it one hop at a time.
Section~\ref{subsubsec:constructing-a-circuit} describes how this
approach makes perfect forward secrecy feasible.

Circuit-based anonymity designs must choose which protocol layer
to anonymize. They may choose to intercept IP packets directly, and
relay them whole (stripping the source address) along the circuit
\cite{freedom2-arch,tarzan:ccs02}.  Alternatively, like
Tor, they may accept TCP streams and relay the data in those streams
along the circuit, ignoring the breakdown of that data into TCP frames
\cite{morphmix:fc04,anonnet}. Finally, they may accept application-level
protocols (such as HTTP) and relay the application requests themselves
along the circuit.  
Making this protocol-layer decision requires a compromise between flexibility
and anonymity.  For example, a system that understands HTTP can strip
identifying information from those requests, can take advantage of caching
to limit the number of requests that leave the network, and can batch
or encode those requests in order to minimize the number of connections.
On the other hand, an IP-level anonymizer can handle nearly any protocol,
even ones unforeseen by their designers (though these systems require
kernel-level modifications to some operating systems, and so are more
complex and less portable). TCP-level anonymity networks like Tor present
a middle approach: they are fairly application neutral (so long as the
application supports, or can be tunneled across, TCP), but by treating
application connections as data streams rather than raw TCP packets,
they avoid the well-known inefficiencies of tunneling TCP over TCP
\cite{tcp-over-tcp-is-bad}.

Distributed-trust anonymizing systems need to prevent attackers from
adding too many servers and thus compromising too many user paths.
Tor relies on a small set of well-known directory servers, run by
independent parties, to make decisions about which nodes can
join. Tarzan and MorphMix allow unknown users to run servers, and use
a limited resource (like IP addresses) to prevent an attacker from
controlling too much of the network.  Crowds suggests requiring
written, notarized requests from potential crowd members.

Anonymous communication is essential for censorship-resistant
systems like Eternity \cite{eternity}, Free~Haven \cite{freehaven-berk},
Publius \cite{publius}, and Tangler \cite{tangler}. Tor's rendezvous
points enable connections between mutually anonymous entities; they
are a building block for location-hidden servers, which are needed by
Eternity and Free~Haven.

% didn't include rewebbers. No clear place to put them, so I'll leave
% them out for now. -RD

\Section{Design goals and assumptions}
\label{sec:assumptions}

\SubSection{Goals}
Like other low-latency anonymity designs, Tor seeks to frustrate
attackers from linking communication partners, or from linking
multiple communications to or from a single user.  Within this
main goal, however, several design considerations have directed
Tor's evolution.

\textbf{Deployability:} The design must be one that can be implemented,
deployed, and used in the real world.  This requirement precludes designs
that are expensive to run (for example, by requiring more bandwidth
than volunteers are willing to provide); designs that place a heavy
liability burden on operators (for example, by allowing attackers to
implicate onion routers in illegal activities); and designs that are
difficult or expensive to implement (for example, by requiring kernel
patches, or separate proxies for every protocol).  This requirement also
precludes systems in which non-anonymous parties (such as websites)
must run our software.  (We do not meet this goal for the current
rendezvous design,
however; see Section~\ref{sec:rendezvous}.)

\textbf{Usability:} A hard-to-use system has fewer users---and because
anonymity systems hide users among users, a system with fewer users
provides less anonymity.  Usability is thus not only a convenience for Tor:
it is a security requirement \cite{econymics,back01}. Tor should
therefore not
require modifying applications; should not introduce prohibitive delays;
and should require the user to make as few configuration decisions
as possible.

\textbf{Flexibility:} The protocol must be flexible and well-specified,
so that it can serve as a test-bed for future research in low-latency
anonymity systems.  Many of the open problems in low-latency anonymity
networks, such as generating dummy traffic or preventing Sybil attacks
\cite{sybil}, may be solvable independently from the issues solved by
Tor. Hopefully future systems will not need to reinvent Tor's design.
(But note that while a flexible design benefits researchers,
there is a danger that differing choices of extensions will make users
distinguishable. Experiments should be run on a separate network.)

\textbf{Simple design:} The protocol's design and security
parameters must be well-understood. Additional features impose implementation
and complexity costs; adding unproven techniques to the design threatens
deployability, readability, and ease of security analysis. Tor aims to
deploy a simple and stable system that integrates the best well-understood
approaches to protecting anonymity.

\SubSection{Non-goals}
\label{subsec:non-goals}
In favoring simple, deployable designs, we have explicitly deferred
several possible goals, either because they are solved elsewhere, or because
their solution is an open research problem.

\textbf{Not Peer-to-peer:} Tarzan and MorphMix aim to scale to completely
decentralized peer-to-peer environments with thousands of short-lived
servers, many of which may be controlled by an adversary.  This approach
is appealing, but still has many open problems
\cite{tarzan:ccs02,morphmix:fc04}.

\textbf{Not secure against end-to-end attacks:} Tor does not claim
to provide a definitive solution to end-to-end timing or intersection
attacks. Some approaches, such as running an onion router, may help;
see Section~\ref{sec:analysis} for more discussion.

\textbf{No protocol normalization:} Tor does not provide \emph{protocol
normalization} like Privoxy or the Anonymizer. For complex and variable
protocols such as HTTP, Tor must be layered with a filtering proxy such
as Privoxy to hide differences between clients, and expunge protocol
features that leak identity. Similarly, Tor does not currently integrate
tunneling for non-stream-based protocols like UDP; this too must be
provided by an external service.
% Actually, tunneling udp over tcp is probably horrible for some apps.
% Should this get its own non-goal bulletpoint? The motivation for
% non-goal-ness would be burden on clients / portability. -RD
% No, leave it as is. -RD

\textbf{Not steganographic:} Tor does not try to conceal which users are
sending or receiving communications; it only tries to conceal with whom
they communicate.

\SubSection{Threat Model}
\label{subsec:threat-model}

A global passive adversary is the most commonly assumed threat when
analyzing theoretical anonymity designs. But like all practical
low-latency systems, Tor does not protect against such a strong
adversary. Instead, we assume an adversary who can observe some fraction
of network traffic; who can generate, modify, delete, or delay traffic
on the network; who can operate onion routers of its own; and who can
compromise some fraction of the onion routers on the network.

In low-latency anonymity systems that use layered encryption, the
adversary's typical goal is to observe both the initiator and the
receiver. Passive attackers can confirm a suspicion that Alice is
talking to Bob if the timing and volume patterns of the traffic on the
connection are distinct enough; active attackers can induce timing
signatures on the traffic to \emph{force} distinct patterns. Tor provides
some defenses against these \emph{traffic confirmation} attacks, for
example by encouraging users to run their own onion routers, but it does
not provide complete protection. Rather, we aim to prevent \emph{traffic
analysis} attacks, where the adversary uses traffic patterns to learn
which points in the network he should attack.

Our adversary might try to link an initiator Alice with any of her
communication partners, or he might try to build a profile of Alice's
behavior. He might mount passive attacks by observing the edges of the
network and correlating traffic entering and leaving the network---either
by relationships in packet timing; relationships in the volume
of data sent; or relationships in any externally visible user-selected
options. The adversary can also mount active attacks by compromising
routers or keys; by replaying traffic; by selectively denying service
to trustworthy routers to encourage users to send their traffic through
compromised routers, or denying service to users to see if the traffic
elsewhere in the
network stops; or by introducing patterns into traffic that can later be
detected. The adversary might attack the directory servers to give users
differing views of network state. Additionally, he can try to decrease
the network's reliability by attacking nodes or by performing antisocial
activities from reliable servers and trying to get them taken down;
making the network unreliable flushes users to other less anonymous
systems, where they may be easier to attack.

We consider each of these attacks in more detail below, and summarize
in Section~\ref{sec:attacks} how well the Tor design defends against
each of them.

%%%%%%%%%%%%%%%%%%%%%%%%%%%%%%%%%%%%%%%%%%%%%%%%%%%%%%%%%%%%%%%%%%%%%%%

\Section{The Tor Design}
\label{sec:design}

The Tor network is an overlay network; each node is called an onion router
(OR). Onion routers run as normal user-level processes without needing
any special
privileges.  Currently, each OR maintains a long-term TLS \cite{TLS}
connection to every other
OR.  (We examine some ways to relax this clique-topology assumption in
Section~\ref{subsec:restricted-routes}.) A subset of the ORs also act as
directory servers, tracking which routers are in the network;
see Section~\ref{subsec:dirservers} for directory server details.
Each user
runs local software called an onion proxy (OP) to fetch directories,
establish paths (called \emph{virtual circuits}) across the network,
and handle connections from user applications.  These onion proxies accept
TCP streams and multiplex them across the virtual circuit. The onion
router on the other side 
% I don't mean other side, I mean wherever it is on the circuit. But
% don't want to introduce complexity this early? Hm. -RD
of the circuit connects to the destinations of
the TCP streams and relays data.

Each onion router uses three public keys: a long-term identity key, a
short-term onion key, and a short-term link key.  The identity
(signing) key is used to sign TLS certificates, to sign its router
descriptor (a summary of its keys, address, bandwidth, exit policy,
etc), and to sign directories if it is a directory server. Changing
the identity key of a router is considered equivalent to creating a
new router. The onion (decryption) key is used for decrypting requests
from users to set up a circuit and negotiate ephemeral keys. Finally,
link keys are used by the TLS protocol when communicating between
onion routers.  Both short-term keys are rotated periodically and
independantly, to limit the impact of compromised keys.

Section~\ref{subsec:cells} discusses the structure of the fixed-size
\emph{cells} that are the unit of communication in Tor. We describe
in Section~\ref{subsec:circuits} how virtual circuits are
built, extended, truncated, and destroyed. Section~\ref{subsec:tcp}
describes how TCP streams are routed through the network, and finally
Section~\ref{subsec:congestion} talks about congestion control and
fairness issues.

\SubSection{Cells}
\label{subsec:cells}

ORs communicate with one another, and with users' OPs, via TLS
connections with ephemeral keys.  This prevents an attacker from
impersonating an OR, conceals the contents of the connection with
perfect forward secrecy, and prevents an attacker from modifying data
on the wire.

Traffic passes along these connections in fixed-size cells.  Each cell
is 256 bytes (but see Section~\ref{sec:conclusion} for a discussion of
allowing large cells and small cells on the same network), and
consists of a header and a payload. The header includes a circuit
identifier (circID) that specifies which circuit the cell refers to
(many circuits are be multiplexed over the single TLS connection), and
a command to describe what to do with the cell's payload.  (Circuit
identifiers are connection-specific; a single circuit has a different
circID on each connection it uses.)
% XXX Say that each OR can have many circuits with same circID, so
% XXX long as they're on different connections, and that ORs know 
% XXX which circIDs/connection pairs are linked by a circuit.
Based on their command, cells are either \emph{control} cells, which are
always interpreted by the node that receives them, or \emph{relay} cells,
which carry end-to-end stream data.   The controls cells commands are:
\emph{padding} (currently used for keepalive, but also usable for link
padding); \emph{create} or \emph{created} (used to set up a new circuit);
and \emph{destroy} (to tear down a circuit).

Relay cells have an additional header (the relay header) after the
cell header, containing the stream identifier (many streams can
be multiplexed over a circuit); an end-to-end checksum for integrity
checking; the length of the relay payload; and a relay command.  
% XXX Mention _here_ that relay headers are {en|de}crypted as they
% XXX progress along the circuit.
The
relay commands are: \emph{relay
data} (for data flowing down the stream), \emph{relay begin} (to open a
stream), \emph{relay end} (to close a stream cleanly), \emph{relay
teardown} (to close a broken stream), \emph{relay connected}
(to notify the OP that a relay begin has succeeded), \emph{relay
extend} and \emph{relay extended} (to extend the circuit by a hop,
and to acknowledge), \emph{relay truncate} and \emph{relay truncated}
(to tear down only part of the circuit, and to acknowledge), \emph{relay
sendme} (used for congestion control), and \emph{relay drop} (used to
implement long-range dummies).

We describe each of these cell types and commands in more detail below.

\SubSection{Circuits and streams}
\label{subsec:circuits}

% I think when we say ``the user,'' maybe we should say ``the user's OP.''

The original Onion Routing design built one circuit for each
TCP stream.  Because building a circuit can take several tenths of a
second (due to public-key cryptography delays and network latency),
this design imposed high costs on applications like web browsing that
open many TCP streams.

In Tor, each circuit can be shared by many TCP streams.  To avoid
delays, users construct circuits preemptively.  To limit linkability
among their streams, users' OPs build a new circuit
periodically if the previous one has been used,
and expire old used circuits that no longer have any open streams.  
OPs consider making a new circuit once a minute: thus
even heavy users spend a negligible amount of time and CPU in
building circuits, but only a limited number of requests can be linked
to each other through a given exit node. Also, because circuits are built
in the background, OPs can recover from failed circuit creation
without delaying streams and thereby harming user experience.

\subsubsection{Constructing a circuit}
\label{subsubsec:constructing-a-circuit}

%XXXX Discuss what happens with circIDs here.

Users construct a circuit incrementally, negotiating a symmetric key with
each OR on the circuit, one hop at a time. To begin creating a new
circuit, the user
(call her Alice) sends a \emph{create} cell to the first node in her
chosen path. This cell's payload contains the first half of the
Diffie-Hellman handshake ($g^x$), encrypted to the onion key of the OR (call
him Bob). Bob responds with a \emph{created} cell containing the second
half of the DH handshake, along with a hash of the negotiated key
$K=g^{xy}$.

Once the circuit has been established, Alice and Bob can send one
another relay cells encrypted with the negotiated
key.\footnote{Actually, the negotiated key is used to derive two
  symmetric keys: one for each direction.}  More detail is given in
the next section.

To extend the circuit further, Alice sends a \emph{relay extend} cell
to Bob, specifying the address of the next OR (call her Carol), and
an encrypted $g^{x_2}$ for her.  Bob copies the half-handshake into a
\emph{create} cell, and passes it to Carol to extend the circuit.
When Carol responds with a \emph{created} cell, Bob wraps the payload
into a \emph{relay extended} cell and passes it back to Alice.  Now
the circuit is extended to Carol, and Alice and Carol share a common key
$K_2 = g^{x_2 y_2}$.

In order to extend the circuit to a third node or beyond, Alice
proceeds as above, always telling the last node in the circuit to
extend one hop further.
% XXX Briefly mention path selection.

This circuit-level handshake protocol achieves unilateral entity
authentication (Alice knows she's handshaking with Bob/Carol, but
Bob/Carol doesn't care who is opening the circuit---Alice has no key
and is trying to remain anonymous) and unilateral key authentication
(Alice and Bob/Carol agree on a key, and Alice knows Bob/Carol is the
only other person who should know it). It also achieves forward
secrecy and key freshness.  Formally, the protocol is as follows
(Where $E_{PK_{Bob}}(\cdot)$ is encryption with Bob's public key,
$H$ is a secure hash function, and $|$ is concatenation.)

\begin{equation}
\begin{aligned}
\mathrm{Alice} \rightarrow \mathrm{Bob}&: E_{PK_{Bob}}(g^x) \\
\mathrm{Bob} \rightarrow \mathrm{Alice}&: g^y, H(K | \mathrm{``handshake"}) \\
\end{aligned}
\end{equation}

In the second step, Bob proves that it was he who who received $g^x$,
and who came up with $y$. We use PK encryption in the first step
(rather than, say, using the first two steps of STS, which has a
signature in the second step) because a single cell is too small to
hold both a public key and a signature. Preliminary analysis with the
NRL protocol analyzer \cite{meadows96} shows the above protocol to be
secure (including providing PFS) under the traditional Dolev-Yao
model.

\subsubsection{Relay cells}
Once Alice has established the circuit (so she shares keys with each
OR on the circuit), she can send relay cells.
% XXX Describe _here_ what happens with relay cells that are not 
% XXX targeted at a given node; how they're decrypted; how they're
% XXX encrypted.  The easiest expository order should probably be: What ORs
% XXX Do With Unrecognized Streams; What Alice Does To Build Relay
% XXX Cells; What ORs Do With Streams They Recognize.
Recall that every relay header has a stream ID in the relay header
that indicates to
which stream the cell belongs.
This stream ID allows a relay cell to be addressed to any of the ORs
on the circuit. To
construct a relay cell addressed to a given OR, Alice iteratively
encrypts the cell payload (that is, the relay header and payload)
with the symmetric key of each hop up to that OR. Then, at each hop
down the circuit, the OR decrypts the cell payload and checks whether
it recognizes the stream ID.  A stream ID is recognized either if it
is an already open stream at that OR, or if it is equal to zero. The
zero stream ID is treated specially, and is used for control messages,
e.g. starting a new stream. If the stream ID is unrecognized, the OR
passes the relay cell downstream. This \emph{leaky pipe} circuit topology
allows Alice's streams to exit at different ORs on a single circuit.  
Alice may choose different exit points because of their exit policies,
or to keep the ORs from knowing that two streams
originate at the same person.

To tear down a whole circuit, Alice sends a \emph{destroy} control
cell. Each OR
in the circuit receives the \emph{destroy} cell, closes all open streams on
that circuit, and passes a new \emph{destroy} cell forward. But since circuits
can be built incrementally, they can also be torn down incrementally:
Alice can instead send a relay truncate cell to a node along the circuit. That
node will send a \emph{destroy} cell forward, and reply with an acknowledgment
(a \emph{relay truncated} cell).  Alice might truncate her circuit so
she can extend it
to different nodes without signaling to the first few nodes (or somebody
observing them) that she is changing her circuit. That is, nodes in the
middle of a truncated are not even aware when the circuit is
truncated, because they see only the encrypted relay cells.
Similarly, if a node on the circuit goes down,
the adjacent node can send a \emph{relay truncated} cell back to
Alice.  Thus the
``break a node and see which circuits go down'' attack is weakened.

\SubSection{Opening and closing streams}
\label{subsec:tcp}

When Alice's application wants to open a TCP connection to a given
address and port, it asks the OP (via SOCKS) to make the connection. The
OP chooses the newest open circuit (or creates one if none is available),
chooses a suitable OR on that circuit to be the exit node (usually the
last node, but maybe others due to exit policy conflicts; see
Section~\ref{sec:exit-policies}), chooses a new random stream ID for
this stream,
and delivers a relay begin cell to that exit node. It uses a stream ID
of zero for the begin cell (so the OR will recognize it), and the relay
payload lists the new stream ID and the destination address and port.
Once the exit node completes the connection to the remote host, it
responds with a relay connected cell through the circuit. Upon receipt,
the OP notifies the application that it can begin talking.

There's a catch to using SOCKS, though -- some applications hand the
alphanumeric address to the proxy, while others resolve it into an IP
address first and then hand the IP to the proxy. When the application
does the DNS resolution first, Alice broadcasts her destination. Common
applications like Mozilla and ssh have this flaw.

In the case of Mozilla, we're fine: the filtering web proxy called Privoxy
does the SOCKS call safely, and Mozilla talks to Privoxy safely. But a
portable general solution, such as for ssh, is an open problem. We can
modify the local nameserver, but this approach is invasive, brittle, and
not portable. We can encourage the resolver library to do resolution
via TCP rather than UDP, but this approach is hard to do right, and also
has portability problems. We can provide a tool similar to \emph{dig} that
can do a private lookup through the Tor network. Our current answer is to
encourage the use of privacy-aware proxies like Privoxy wherever possible,

Ending a Tor stream is analogous to ending a TCP stream: it uses a
two-step handshake for normal operation, or a one-step handshake for
errors. If one side of the stream closes abnormally, that node simply
sends a relay teardown cell, and tears down the stream. If one side
of the stream closes the connection normally, that node sends a relay
end cell down the circuit. When the other side has sent back its own
relay end, the stream can be torn down. This two-step handshake allows
for TCP-based applications that, for example, close a socket for writing
but are still willing to read. Remember that all relay cells use layered
encryption, so only the destination OR knows what type of relay cell
it is.

\SubSection{Integrity checking on streams}

Because the old Onion Routing design used a stream cipher, traffic was
vulnerable to a malleability attack: even though the attacker could not
decrypt cells, he could make changes to an encrypted
cell to create corresponding changes to the data leaving the network.
(Even an external adversary could do this, despite link encryption!)

This weakness allowed an adversary to change a padding cell to a destroy
cell; change the destination address in a relay begin cell to the
adversary's webserver; or change a user on an ftp connection from
typing ``dir'' to typing ``delete~*''. Any node or external adversary
along the circuit could introduce such corruption in a stream.

Tor prevents external adversaries from mounting this attack simply by
using TLS. Addressing the insider malleability attack, however, is
more complex.

We could do integrity checking of the relay cells at each hop, either
by including hashes or by using a cipher mode like EAX \cite{eax},
but we don't want the added message-expansion overhead at each hop, and
we don't want to leak the path length or pad to some max path length.
Because we've already accepted that our design is vulnerable to end-to-end
timing attacks, we can perform integrity checking only at the edges of
the circuit without introducing any new anonymity attacks. When Alice
negotiates a key
with each hop, they both start a SHA-1 with some derivative of that key,
% Not just the exit hop, but each hop: any hop can be an exit node. -RD
thus starting out with randomness that only the two of them know. From
then on they each incrementally add to the SHA-1 all the data bytes
entering or exiting from the circuit, and each such relay cell includes
the first 4 bytes of the current value of the hash.

The attacker must be able to guess all previous bytes between Alice
and Bob on that circuit (including the pseudorandomness from the key
negotiation), plus the bytes in the current cell, to remove or modify the
cell. Attacks on SHA-1 where the adversary can incrementally add to a
hash to produce a new valid hash don't work,
because all hashes are end-to-end encrypted across the circuit.
The computational overhead isn't so bad, compared to doing an AES
crypt at each hop in the circuit. We use only four bytes per cell to
minimize overhead; the chance that an adversary will correctly guess a
valid hash, plus the payload the current cell, is acceptly low, given
that Alice or Bob tear down the circuit if they receive a bad hash.

\SubSection{Rate limiting and fairness}

Volunteers are generally more willing to run services that can limit
their bandwidth usage.  To accomodate them, Tor servers use a token
bucket approach to limit the number of bytes they
% XXX cite token bucket?
receive. Tokens are added to the bucket each second (when the bucket is
full, new tokens are discarded.) Each token represents permission to
receive one byte from the network---to receive a byte, the connection
must remove a token from the bucket. Thus if the bucket is empty, that
connection must wait until more tokens arrive. The number of tokens we
add enforces a long-term average rate of incoming bytes, while still
permitting short-term bursts above the allowed bandwidth. Current bucket
sizes are set to ten seconds worth of traffic.

Further, we want to avoid starving any Tor streams. Entire circuits
could starve if we read greedily from connections and one connection
uses all the remaining bandwidth. We solve this by dividing the number
of tokens in the bucket by the number of connections that want to read,
and reading at most that number of bytes from each connection. We iterate
this procedure until the number of tokens in the bucket is under some
threshold (eg 10KB), at which point we greedily read from connections.

Because the Tor protocol generates roughly the same number of outgoing
bytes as incoming bytes, it is sufficient in practice to rate-limit
incoming bytes.
% Is it?  Fun attack: I send you lots of 1-byte-at-a-time TCP frames.
% In response, you send lots of 256 byte cells.  Can I use this to 
% make you exceed your outgoing bandwidth limit by a factor of 256? -NM
% Can we resolve this by, when reading from edge connections, rounding up
% the bytes read (wrt buckets) to the nearest multiple of 256? -RD

Further, inspired by Rennhard et al's design in \cite{anonnet}, a
circuit's edges heuristically distinguish interactive streams from bulk
streams by comparing the frequency with which they supply cells.  We can
provide good latency for interactive streams by giving them preferential
service, while still getting good overall throughput to the bulk
streams. Such preferential treatment presents a possible end-to-end
attack, but an adversary who can observe both
ends of the stream can already learn this information through timing
attacks.

\SubSection{Congestion control}
\label{subsec:congestion}

Even with bandwidth rate limiting, we still need to worry about
congestion, either accidental or intentional. If enough users choose the
same OR-to-OR connection for their circuits, that connection can become
saturated. For example, an adversary could make a large HTTP PUT request
through the onion routing network to a webserver he runs, and then
refuse to read any of the bytes at the webserver end of the
circuit. Without some congestion control mechanism, these bottlenecks
can propagate back through the entire network.  We describe our
responses below.

\subsubsection{Circuit-level}

To control a circuit's bandwidth usage, each OR keeps track of two
windows. The \emph{package window} tracks how many relay data cells the OR is
allowed to package (from outside streams) for transmission back to the OP,
and the \emph{deliver window} tracks how many relay data cells it is willing
to deliver to streams outside the network. Each window is initialized
(say, to 1000 data cells). When a data cell is packaged or delivered,
the appropriate window is decremented. When an OR has received enough
data cells (currently 100), it sends a relay sendme cell towards the OP,
with stream ID zero. When an OR receives a relay sendme cell with stream
ID zero, it increments its packaging window. Either of these cells
increments the corresponding window by 100. If the packaging window
reaches 0, the OR stops reading from TCP connections for all streams
on the corresponding circuit, and sends no more relay data cells until
receiving a relay sendme cell.

The OP behaves identically, except that it must track a packaging window
and a delivery window for every OR in the circuit. If a packaging window
reaches 0, it stops reading from streams destined for that OR.

\subsubsection{Stream-level}

The stream-level congestion control mechanism is similar to the
circuit-level mechanism above. ORs and OPs use relay sendme cells
to implement end-to-end flow control for individual streams across
circuits. Each stream begins with a package window (e.g. 500 cells),
and increments the window by a fixed value (50) upon receiving a relay
sendme cell. Rather than always returning a relay sendme cell as soon
as enough cells have arrived, the stream-level congestion control also
has to check whether data has been successfully flushed onto the TCP
stream; it sends a relay sendme only when the number of bytes pending
to be flushed is under some threshold (currently 10 cells worth).

Currently, non-data relay cells do not affect the windows. Thus we
avoid potential deadlock issues, e.g. because a stream can't send a
relay sendme cell because its packaging window is empty.

% XXX Bad heading
\subsubsection{Needs more research}

We don't need to reimplement full TCP windows (with sequence numbers,
the ability to drop cells when we're full and retransmit later, etc),
because the TCP streams already guarantee in-order delivery of each
cell. But we need to investigate further the effects of the current
parameters on throughput and latency, while also keeping privacy in mind;
see Section~\ref{sec:maintaining-anonymity} for more discussion.

\Section{Other design decisions}

\SubSection{Resource management and denial-of-service}
\label{subsec:dos}

Providing Tor as a public service provides many opportunities for an
attacker to mount denial-of-service attacks against the network.  While
flow control and rate limiting (discussed in
Section~\ref{subsec:congestion}) prevent users from consuming more
bandwidth than routers are willing to provide, opportunities remain for
users to
consume more network resources than their fair share, or to render the
network unusable for other users.

First of all, there are several CPU-consuming denial-of-service
attacks wherein an attacker can force an OR to perform expensive
cryptographic operations.  For example, an attacker who sends a
\emph{create} cell full of junk bytes can force an OR to perform an RSA
decrypt.  Similarly, an attacker can
fake the start of a TLS handshake, forcing the OR to carry out its
(comparatively expensive) half of the handshake at no real computational
cost to the attacker.

Several approaches exist to address these attacks. First, ORs may
require clients to solve a puzzle \cite{puzzles-tls} while beginning new
TLS handshakes or accepting \emph{create} cells.  So long as these
tokens are easy to verify and computationally expensive to produce, this
approach limits the attack multiplier.  Additionally, ORs may limit
the rate at which they accept create cells and TLS connections, so that
the computational work of processing them does not drown out the (comparatively
inexpensive) work of symmetric cryptography needed to keep cells
flowing.  This rate limiting could, however, allow an attacker
to slow down other users when they build new circuits.

% What about link-to-link rate limiting?

Attackers also have an opportunity to attack the Tor network by mounting
attacks on its hosts and network links. Disrupting a single circuit or
link breaks all currently open streams passing along that part of the
circuit. Indeed, this same loss of service occurs when a router crashes
or its operator restarts it. The current Tor design treats such attacks
as intermittent network failures, and depends on users and applications
to respond or recover as appropriate. A future design could use an
end-to-end TCP-like acknowledgment protocol, so that no streams are
lost unless the entry or exit point itself is disrupted. This solution
would require more buffering at the network edges, however, and the
performance and anonymity implications from this extra complexity still
require investigation.

\SubSection{Exit policies and abuse}
\label{subsec:exitpolicies}

Exit abuse is a serious barrier to wide-scale Tor deployment. Anonymity
presents would-be vandals and abusers with an opportunity to hide
the origins of their activities. Attackers can harm the Tor network by
implicating exit servers for their abuse. Also, applications that commonly
use IP-based authentication (such as institutional mail or web servers)
can be fooled by the fact that anonymous connections appear to originate
at the exit OR.

We stress that Tor does not enable any new class of abuse. Spammers
and other attackers already have access to thousands of misconfigured
systems worldwide, and the Tor network is far from the easiest way
to launch these antisocial or illegal attacks.
%Indeed, because of its limited
%anonymity, Tor is probably not a good way to commit crimes.
But because the
onion routers can easily be mistaken for the originators of the abuse,
and the volunteers who run them may not want to deal with the hassle of
repeatedly explaining anonymity networks, we must block or limit attacks
and other abuse that travel through the Tor network.

To mitigate abuse issues, in Tor, each onion router's \emph{exit policy}
describes to which external addresses and ports the router will permit
stream connections. On one end of the spectrum are \emph{open exit}
nodes that will connect anywhere. On the other end are \emph{middleman}
nodes that only relay traffic to other Tor nodes, and \emph{private exit}
nodes that only connect to a local host or network.  Using a private
exit (if one exists) is a more secure way for a client to connect to a
given host or network---an external adversary cannot eavesdrop traffic
between the private exit and the final destination, and so is less sure of
Alice's destination and activities. Most onion routers will function as
\emph{restricted exits} that permit connections to the world at large,
but prevent access to certain abuse-prone addresses and services. In
general, nodes can require a variety of forms of traffic authentication
\cite{or-discex00}.

%The abuse issues on closed (e.g. military) networks are different
%from the abuse on open networks like the Internet. While these IP-based
%access controls are still commonplace on the Internet, on closed networks,
%nearly all participants will be honest, and end-to-end authentication
%can be assumed for important traffic.

Many administrators will use port restrictions to support only a
limited set of well-known services, such as HTTP, SSH, or AIM.
This is not a complete solution, since abuse opportunities for these
protocols are still well known. Nonetheless, the benefits are real,
since administrators seem used to the concept of port 80 abuse not
coming from the machine's owner.

A further solution may be to use proxies to clean traffic for certain
protocols as it leaves the network.  For example, much abusive HTTP
behavior (such as exploiting buffer overflows or well-known script
vulnerabilities) can be detected in a straightforward manner.
Similarly, one could run automatic spam filtering software (such as
SpamAssassin) on email exiting the OR network.

ORs may also choose to rewrite exiting traffic in order to append
headers or other information to indicate that the traffic has passed
through an anonymity service.  This approach is commonly used
by email-only anonymity systems.  When possible, ORs can also
run on servers with hostnames such as {\it anonymous}, to further
alert abuse targets to the nature of the anonymous traffic.

A mixture of open and restricted exit nodes will allow the most
flexibility for volunteers running servers. But while many
middleman nodes help provide a large and robust network,
having only a few exit nodes reduces the number of points
an adversary needs to monitor for traffic analysis, and places a
greater burden on the exit nodes.  This tension can be seen in the
Java Anon Proxy
cascade model, wherein only one node in each cascade needs to handle
abuse complaints---but an adversary only needs to observe the entry
and exit of a cascade to perform traffic analysis on all that
cascade's users. The Hydra model (many entries, few exits) presents a
different compromise: only a few exit nodes are needed, but an
adversary needs to work harder to watch all the clients; see
Section~\ref{sec:conclusion}.

Finally, we note that exit abuse must not be dismissed as a peripheral
issue: when a system's public image suffers, it can reduce the number
and diversity of that system's users, and thereby reduce the anonymity
of the system itself.  Like usability, public perception is also a
security parameter.  Sadly, preventing abuse of open exit nodes is an
unsolved problem, and will probably remain an arms race for the
forseeable future.  The abuse problems faced by Princeton's CoDeeN
project \cite{darkside} give us a glimpse of likely issues.

\SubSection{Directory Servers}
\label{subsec:dirservers}

First-generation Onion Routing designs \cite{freedom2-arch,or-jsac98} used
in-band network status updates: each router flooded a signed statement
to its neighbors, which propagated it onward. But anonymizing networks
have different security goals than typical link-state routing protocols.
For example, delays (accidental or intentional)
that can cause different parts of the network to have different pictures
of link-state and topology are not only inconvenient---they give
attackers an opportunity to exploit differences in client knowledge.
We also worry about attacks to deceive a
client about the router membership list, topology, or current network
state. Such \emph{partitioning attacks} on client knowledge help an
adversary to efficiently deploy resources
when attacking a target.

Tor uses a small group of redundant, well-known onion routers to
track changes in network topology and node state, including keys and
exit policies.  Each such \emph{directory server} also acts as an HTTP
server, so participants can fetch current network state and router
lists (a \emph{directory}), and so other onion routers can upload
their router descriptors.  Onion routers periodically publish signed
statements of their state to each directory server, which combines this
state information with its own view of network liveness, and generates
a signed description of the entire network state. Client software is
pre-loaded with a list of the directory servers and their keys; it uses
this information to bootstrap each client's view of the network.

When a directory server receives a signed statement from an onion
router, it recognizes the onion router by its identity key. Directory
servers do not automatically advertise unrecognized ORs. (If they did,
an adversary could take over the network by creating many servers
\cite{sybil}.) Instead, new nodes must be approved by the directory
server administrator before they are included. Mechanisms for automated
node approval are an area of active research, and are discussed more
in Section~\ref{sec:maintaining-anonymity}.
  
Of course, a variety of attacks remain. An adversary who controls
a directory server can track certain clients by providing different
information---perhaps by listing only nodes under its control, or by
informing only certain clients about a given node. Even an external
adversary can exploit differences in client knowledge: clients who use
a node listed on one directory server but not the others are vulnerable.

Thus these directory servers must be synchronized and redundant.
Valid directories are those signed by a threshold of the directory
servers.

The directory servers in Tor are modeled after those in Mixminion
\cite{minion-design}, but our situation is easier. First, we make the
simplifying assumption that all participants agree on the set of
directory servers. Second, while Mixminion needs to predict node
behavior, Tor only needs a threshold consensus of the current
state of the network.

Tor directory servers build a consensus directory through a simple
four-round broadcast protocol.  In round one, each server dates and
signs its current opinion, and broadcasts it to the other directory
servers; then in round two, each server rebroadcasts all the signed
opinions it has received.  At this point all directory servers check
to see whether any server has signed multiple opinions in the same
period. Such a server is either broken or cheating, so the protocol
stops and notifies the administrators, who either remove the cheater
or wait for the broken server to be fixed.  If there are no
discrepancies, each directory server then locally computes an algorithm
(described below)
on the set of opinions, resulting in a uniform shared directory. In
round three servers sign this directory and broadcast it; and finally
in round four the servers rebroadcast the directory and all the
signatures.  If any directory server drops out of the network, its
signature is not included on the final directory.

The rebroadcast steps ensure that a directory server is heard by
either all of the other servers or none of them, even when some links
are down (assuming that any two directory servers can talk directly or
via a third). Broadcasts are feasible because there are relatively few
directory servers (currently 3, but we expect as many as 9 as the network
scales). Computing the shared directory locally is a straightforward
threshold voting process: we include an OR if a majority of directory
servers believe it to be good.

To avoid attacks where a router connects to all the directory servers
but refuses to relay traffic from other routers, the directory servers
must build circuits and use them to anonymously test router reliability
\cite{mix-acc}.

Using directory servers is simpler and more flexible than flooding.
For example, flooding complicates the analysis when we
start experimenting with non-clique network topologies. And because
the directories are signed, they can be cached by other onion routers.
Thus directory servers are not a performance
bottleneck when we have many users, and do not aid traffic analysis by
forcing clients to periodically announce their existence to any
central point.

\Section{Rendezvous points and location privacy}
\label{sec:rendezvous}

Rendezvous points are a building block for \emph{location-hidden
services} (also known as \emph{responder anonymity}) in the Tor
network.  Location-hidden services allow Bob to offer a TCP
service, such as a webserver, without revealing its IP.
This type of anonymity protects against distributed DoS attacks:
attackers are forced to attack the onion routing network as a whole
rather than just Bob's IP.

Our design for location-hidden servers has the following goals.
\textbf{Flood-proof:} Bob needs a way to filter incoming requests,
so an attacker cannot flood Bob simply by sending many requests.
\textbf{Robust:} Bob should be able to maintain a long-term pseudonymous
identity even in the presence of router failure. Bob's service must
not be tied to a single OR, and Bob must be able to tie his service
to new ORs. \textbf{Smear-resistant:} if a social attacker offers a
location-hidden service that is illegal or disreputable, it should not
appear---even to a casual observer---that a rendezvous router is hosting
that service. \textbf{Application-transparent:} Although we require users
to run special software to access location-hidden servers, we must not
require them to modify their applications.

We provide location-hiding for Bob by allowing him to advertise
several onion routers (his \emph{introduction points}) as contact
points. He may do this on any robust efficient
key-value lookup system with authenticated updates, such as a
distributed hash table (DHT) like CFS \cite{cfs:sosp01}\footnote{
Rather than rely on an external infrastructure, the Onion Routing network
can run the DHT; to begin, we can run a simple lookup system on the
directory servers.} Alice, the client, chooses an OR as her
\emph{rendezvous point}. She connects to one of Bob's introduction
points, informs him about her rendezvous point, and then waits for him
to connect to the rendezvous point. This extra level of indirection
helps Bob's introduction points avoid problems associated with serving
unpopular files directly (for example, if Bob chooses
an introduction point in Texas to serve anti-ranching propaganda,
or if Bob's service tends to get attacked by network vandals).
The extra level of indirection also allows Bob to respond to some requests
and ignore others.

We give an overview of the steps of a rendezvous. These steps are
performed on behalf of Alice and Bob by their local onion proxies;
application integration is described more fully below.
\begin{tightlist}
\item Bob chooses some introduction points, and advertises them on
      the DHT.
\item Bob establishes a Tor circuit to each of his introduction points,
      and waits.
\item Alice learns about Bob's service out of band (perhaps Bob told her,
      or she found it on a website). She retrieves the details of Bob's
      service from the DHT.
\item Alice chooses an OR to serve as the rendezvous point (RP) for this
      transaction. She establishes a circuit to RP, and gives it a
      rendezvous cookie, which it will use to recognize Bob.
\item Alice opens an anonymous stream to one of Bob's introduction
      points, and gives it a message (encrypted for Bob) which tells him
      about herself, her chosen RP and the rendezvous cookie, and the
      first half of an ephemeral
      key handshake. The introduction point sends the message to Bob.
\item If Bob wants to talk to Alice, he builds a new circuit to Alice's
      RP and provides the rendezvous cookie and the second half of the DH
      handshake (along with a hash of the session key they now share).
\item The RP connects Alice's circuit to Bob's. Note that RP can't
      recognize Alice, Bob, or the data they transmit.
\item Alice now sends a \emph{relay begin} cell along the circuit. It
      arrives at Bob's onion proxy. Bob's onion proxy connects to Bob's
      webserver.
\item An anonymous stream has been established, and Alice and Bob
      communicate as normal.
\end{tightlist}

When establishing an introduction point, Bob provides the onion router
with a public ``introduction'' key. The hash of this public key
identifies a unique service, and (since Bob is required to sign his
messages) prevents anybody else from usurping Bob's introduction point
in the future. Bob uses the same public key when establishing the other
introduction points for that service. Bob periodically refreshes his
entry in the DHT.

The message that Alice gives
the introduction point includes a hash of Bob's public key to identify
the service, along with an optional initial authentication token (the
introduction point can do prescreening, for example to block replays). Her
message to Bob may include an end-to-end authentication token so Bob
can choose whether to respond.
The authentication tokens can be used to provide selective access:
important users get tokens to ensure uninterrupted access to the
service. During normal situations, Bob's service might simply be offered
directly from mirrors, and Bob gives out tokens to high-priority users. If
the mirrors are knocked down by distributed DoS attacks, those users
can switch to accessing Bob's service via the Tor rendezvous system.

\SubSection{Integration with user applications}

Bob configures his onion proxy to know the local IP and port of his
service, a strategy for authorizing clients, and a public key. Bob
publishes the public key, an expiration time (``not valid after''), and
the current introduction points for his service into the DHT, all indexed
by the hash of the public key. Note that Bob's webserver is unmodified,
and doesn't even know that it's hidden behind the Tor network.

Alice's applications also work unchanged---her client interface
remains a SOCKS proxy. We encode all of the necessary information
into the fully qualified domain name Alice uses when establishing her
connection. Location-hidden services use a virtual top level domain
called `.onion': thus hostnames take the form x.y.onion where x is the
authentication cookie, and y encodes the hash of PK. Alice's onion proxy
examines addresses; if they're destined for a hidden server, it decodes
the PK and starts the rendezvous as described in the table above.

\subsection{Previous rendezvous work}

Ian Goldberg developed a similar notion of rendezvous points for
low-latency anonymity systems \cite{ian-thesis}. His design differs from
ours in three ways. First, Goldberg suggests that Alice should manually
hunt down a current location of the service via Gnutella; whereas our
use of CFS makes lookup faster, more robust, and transparent to the
user. Second, in Tor the client and server negotiate ephemeral keys
via Diffie-Hellman, so plaintext is not exposed at any point. Third,
our design tries to minimize the exposure associated with running the
service, to encourage volunteers to offer introduction and rendezvous
point services. Tor's introduction points do not output any bytes to the
clients, and the rendezvous points don't know the client or the server,
and can't read the data being transmitted. The indirection scheme is
also designed to include authentication/authorization---if Alice doesn't
include the right cookie with her request for service, Bob need not even
acknowledge his existence.

\Section{Analysis}
\label{sec:analysis}

In this section, we discuss how well Tor meets our stated design goals
and its resistance to attacks.

\SubSection{Meeting Basic Goals}
% None of these seem to say very much.  Should this subsection be removed?
\begin{tightlist}
\item [Basic Anonymity:] Because traffic is encrypted, changing in
  appearance, and can flow from anywhere to anywhere within the
  network, a simple observer that cannot see both the initiator
  activity and the corresponding activity where the responder talks to
  the network will not be able to link the initiator and responder.
  Nor is it possible to directly correlate any two communication
  sessions as coming from a single source without additional
  information. Resistance to more sophisticated anonymity threats is
  discussed below.
\item[Deployability:] Tor requires no specialized hardware. Tor
  requires no kernel modifications; it runs in user space (currently
  on Linux, various BSDs, and Windows). All of these imply a low
  technical barrier to running a Tor node. There is an assumption that
  Tor nodes have good relatively persistent net connectivity
  (currently T1 or better);
% Is that reasonable to say? We haven't really discussed it -P.S.
% Roger thinks otherwise; he will fix this. -NM
  however, there is no padding overhead, and operators can limit
  bandwidth on any link.  Tor is freely available under the modified
  BSD license, and operators are able to choose their own exit
  policies, thus reducing legal and social barriers to
  running a node.
  
\item[Usability:] As noted, Tor runs in user space. So does the onion
  proxy, which is comparatively easy to install and run. SOCKS-aware
  applications require nothing more than to be pointed at the onion
  proxy; other applications can be redirected to use SOCKS for their
  outgoing TCP connections by drop-in libraries such as tsocks.
  
\item[Flexibility:] Tor's design and implementation is fairly modular,
  so that, for example, a scalable P2P replacement for the directory
  servers would not substantially impact other aspects of the system.
  Tor runs on top of TCP, so design options that could not easily do
  so would be difficult to test on the current network. However, most
  low-latency protocols are designed to run over TCP. We are currently
  working with the designers of MorphMix to render our two systems
  interoperable. So for, this seems to be relatively straightforward.
  Interoperability will allow testing and direct comparison of the two
  rather different designs.
  
\item[Simple design:] Tor opts for practicality when there is no
  clear resolution of anonymity trade-offs or practical means to
  achieve resolution. Thus, we do not currently pad or mix; although
  it would be easy to add either of these. Indeed, our system allows
  long-range and variable padding if this should ever be shown to have
  a clear advantage.  Similarly, we do not currently attempt to
  resolve such issues as Sybil attacks to dominate the network except
  by such direct means as personal familiarity of director operators
  with all node operators.
\end{tightlist}

\SubSection{Attacks and Defenses}
\label{sec:attacks}

Below we summarize a variety of attacks, and discuss how well our
design withstands them.

\subsubsection*{Passive attacks}
\begin{tightlist}
\item \emph{Observing user traffic patterns.} Observations of connection
  between an end user and a first onion router will not reveal to whom
  the user is connecting or what information is being sent. It will
  reveal patterns of user traffic (both sent and received). Simple
  profiling of user connection patterns is not generally possible,
  however, because multiple application connections (streams) may be
  operating simultaneously or in series over a single circuit. Thus,
  further processing is necessary to try to discern even these usage
  patterns.
  
\item \emph{Observing user content.} At the user end, content is
  encrypted; however, connections from the network to arbitrary
  websites may not be. Further, a responding website may itself be
  considered an adversary. Filtering content is not a primary goal of
  Onion Routing; nonetheless, Tor can directly make use of Privoxy and
  related filtering services via SOCKS and thus anonymize their
  application data streams.

\item \emph{Option distinguishability.} Configuration options can be a
  source of distinguishable patterns. In general there is economic
  incentive to allow preferential services \cite{econymics}, and some
  degree of configuration choice can be a factor in attracting many users
  to provide anonymity.  So far, however, we have
  not found a compelling use case in Tor for any client-configurable
  options.  Thus, clients are currently distinguishable only by their
  behavior.
%Actually, circuitrebuildperiod is such an option. -RD
  
\item \emph{End-to-end Timing correlation.}  Tor only minimally hides
  end-to-end timing correlations. If an attacker can watch patterns of
  traffic at the initiator end and the responder end, then he will be
  able to confirm the correspondence with high probability. The
  greatest protection currently against such confirmation is if the
  connection between the onion proxy and the first Tor node is hidden,
  possibly because it is local or behind a firewall.  This approach
  requires an observer to separate traffic originating the onion
  router from traffic passes through it.  We still do not, however,
  predict this approach to be a large problem for an attacker who can
  observe traffic at both ends of an application connection.
  
\item \emph{End-to-end Size correlation.} Simple packet counting
  without timing consideration will also be effective in confirming
  endpoints of a connection through Onion Routing; although slightly
  less so. This is because, even without padding, the leaky pipe
  topology means different numbers of packets may enter one end of a
  circuit than exit at the other.
  
\item \emph{Website fingerprinting.} All the above passive
  attacks that are at all effective are traffic confirmation attacks.
  This puts them outside our general design goals. There is also
  a passive traffic analysis attack that is potentially effective.
  Instead of searching exit connections for timing and volume
  correlations it is possible to build up a database of
  ``fingerprints'' containing file sizes and access patterns for many
  interesting websites. If one now wants to
  monitor the activity of a user, it may be possible to confirm a
  connection to a site simply by consulting the database. This attack has
  been shown to be effective against SafeWeb \cite{hintz-pet02}. Onion
  Routing is not as vulnerable as SafeWeb to this attack: There is the
  possibility that multiple streams are exiting the circuit at
  different places concurrently.  Also, fingerprinting will be limited to
  the granularity of cells, currently 256 bytes. Larger cell sizes
  and/or minimal padding schemes that group websites into large sets
  are possible responses.  But this remains an open problem.  Link
  padding or long-range dummies may also make fingerprints harder to
  detect. (Note that
  such fingerprinting should not be confused with the latency attacks
  of \cite{back01}. Those require a fingerprint of the latencies of
  all circuits through the network, combined with those from the
  network edges to the targeted user and the responder website. While
  these are in principal feasible and surprises are always possible,
  these constitute a much more complicated attack, and there is no
  current evidence of their practicality.)

\item \emph{Content analysis.}  Tor explicitly provides no content
  rewriting for any protocol at a higher level than TCP.  When
  protocol cleaners are available, however (as Privoxy is for HTTP),
  Tor can integrate them in order to address these attacks.

\end{tightlist}

\subsubsection*{Active attacks}
\begin{tightlist}
\item \emph{Key compromise.}  We consider the impact of a compromise
  for each type of key in turn, from the shortest- to the
  longest-lived.  If a circuit session key is compromised, the
  attacker can unwrap a single layer of encryption from the relay
  cells traveling along that circuit.  (Only nodes on the circuit can
  see these cells.)  If a TLS session key is compromised, an attacker
  can view all the cells on TLS connection until the key is
  renegotiated.  (These cells are themselves encrypted.)  If a TLS
  private key is compromised, the attacker can fool others into
  thinking that he is the affected OR, but still cannot accept any
  connections.  If an onion private key is compromised, the attacker
  can impersonate the OR in circuits, but only if the attacker has
  also compromised the OR's TLS private key, or is running the
  previous OR in the circuit.  (This compromise affects newly created
  circuits, but because of perfect forward secrecy, the attacker
  cannot hijack old circuits without compromising their session keys.)
  In any case, an attacker can only take advantage of a compromise in
  these mid-term private keys until they expire.  Only by
  compromising a node's identity key can an attacker replace that
  node indefinitely, by sending new forged mid-term keys to the
  directories.  Finally, an attacker who can compromise a
  \emph{directory's} identity key can influence every client's view
  of the network---but only to the degree made possible by gaining a
  vote with the rest of the the directory servers.

\item \emph{Iterated compromise.} A roving adversary who can
  compromise ORs (by system intrusion, legal coersion, or extralegal
  coersion) could march down length of a circuit compromising the
  nodes until he reaches the end.  Unless the adversary can complete
  this attack within the lifetime of the circuit, however, the ORs
  will have discarded the necessary information before the attack can
  be completed.  (Thanks to the perfect forward secrecy of session
  keys, the attacker cannot cannot force nodes to decrypt recorded
  traffic once the circuits have been closed.)  Additionally, building
  circuits that cross jurisdictions can make legal coercion
  harder---this phenomenon is commonly called ``jurisdictional
  arbitrage.'' The Java Anon Proxy project recently experienced this
  issue, when
  the German government successfully ordered them to add a backdoor to
  all of their nodes \cite{jap-backdoor}.
  
\item \emph{Run a recipient.} By running a Web server, an adversary
  trivially learns the timing patterns of those connecting to it, and
  can introduce arbitrary patterns in its responses.  This can greatly
  facilitate end-to-end attacks: If the adversary can induce certain
  users to connect to connect to his webserver (perhaps by providing
  content targeted at those users), she now holds one end of their
  connection.  Additonally, here is a danger that the application
  protocols and associated programs can be induced to reveal
  information about the initiator.  This is not directly in Onion
  Routing's protection area, so we are dependent on Privoxy and
  similar protocol cleaners to solve the problem.
  
\item \emph{Run an onion proxy.} It is expected that end users will
  nearly always run their own local onion proxy. However, in some
  settings, it may be necessary for the proxy to run
  remotely---typically, in an institutional setting where it was
  necessary to monitor the activity of those connecting to the proxy.
  The drawback, of course, is that if the onion proxy is compromised,
  then all future connections through it are completely compromised.

\item \emph{DoS non-observed nodes.} An observer who can observe some
  of the Tor network can increase the value of this traffic analysis
  if it can attack non-observed nodes to shut them down, reduce
  their reliability, or persuade users that they are not trustworthy.
  The best defense here is robustness.
  
\item \emph{Run a hostile node.}  In addition to the abilties of a
  local observer, an isolated hostile node can create circuits through
  itself, or alter traffic patterns, in order to affect traffic at
  other nodes. Its ability to directly DoS a neighbor is now limited
  by bandwidth throttling. Nonetheless, in order to compromise the
  anonymity of the endpoints of a circuit by its observations, a
  hostile node is only significant if it is immediately adjacent to
  that endpoint. 
  
\item \emph{Run multiple hostile nodes.}  If an adversary is able to
  run multiple ORs, and is able to persuade the directory servers
  that those ORs are trustworthy and independant, then occasionally
  some user will choose one of those ORs for the start and another of
  those ORs as the end of a circuit.  When this happens, the user's
  anonymity is compromised for those circuits.  If an adversary can
  control $m$ out of $N$ nodes, he should be able to correlate at most 
  $\frac{m}{N}$ of the traffic in this way---although an adersary
  could possibly attract a disproportionately large amount of traffic
  by running an exit node with an unusually permisssive exit policy.

\item \emph{Compromise entire path.} Anyone compromising both
  endpoints of a circuit can confirm this with high probability. If
  the entire path is compromised, this becomes a certainty; however,
  the added benefit to the adversary of such an attack is small in
  relation to the difficulty.
  
\item \emph{Run a hostile directory server.} Directory servers control
  admission to the network. However, because the network directory
  must be signed by a majority of servers, the threat of a single
  hostile server is minimized.
  
\item \emph{Selectively DoS a Tor node.} As noted, neighbors are
  bandwidth limited; however, it is possible to open up sufficient
  circuits that converge at a single onion router to
  overwhelm its network connection, its ability to process new
  circuits, or both.

\item \emph{Introduce timing into messages.} This is simply a stronger
  version of passive timing attacks already discussed above.
  
\item \emph{Tagging attacks.} A hostile node could try to ``tag'' a
  cell by altering it. This would render it unreadable, but if the
  connection is, for example, an unencrypted request to a Web site,
  the garbled content coming out at the appropriate time could confirm
  the association. However, integrity checks on cells prevent
  this attack from succeeding.

\item \emph{Replace contents of unauthenticated protocols.}  When a
  relaying an unauthenticated protocol like HTTP, a hostile exit node 
  can impersonate the target server.  Thus, whenever possible, clients
  should prefer protocols with end-to-end authentication.

\item \emph{Replay attacks.} Some anonymity protocols are vulnerable
  to replay attacks.  Tor is not; replaying one side of a handshake
  will result in a different negotiated session key, and so the rest
  of the recorded session can't be used.  
  % ``NonSSL Anonymizer''?

\item \emph{Smear attacks.} An attacker could use the Tor network to
  engage in socially dissapproved acts, so as to try to bring the
  entire network into disrepute and get its operators to shut it down.
  Exit policies can help reduce the possibilities for abuse, but
  ultimately, the network will require volunteers who can tolerate
  some political heat.

\item \emph{Distribute hostile code.} An attacker could trick users
  into running subverted Tor software that did not, in fact, anonymize
  their connections---or worse, trick ORs into running weakened
  software that provided users with less anonymity.  We address this
  problem (but do not solve it completely) by signing all Tor releases
  with an official public key, and including an entry the directory
  describing which versions are currently believed to be secure.  To
  prevent an attacker from subverting the official release itself
  (through threats, bribery, or insider attacks), we provide all
  releases in source code form, encourage source audits, and
  frequently warn our users never to trust any software (even from
  us!) that comes without source.
\end{tightlist}

\subsubsection*{Directory attacks}
\begin{tightlist}
\item \emph{Destroy directory servers.}  If a single directory
  server drops out of operation, the others still arrive at a final
  directory.  So long as any directory servers remain in operation,
  they will still broadcast their views of the network and generate a
  consensus directory.  (If more than half are destroyed, this
  directory will not, however, have enough signatures for clients to
  use it automatically; human intervention will be necessary for
  clients to decide whether to trust the resulting directory.)

\item \emph{Subvert a directory server.}  By taking over a directory
  server, an attacker can influence (but not control) the final
  directory.  Since ORs are included or excluded by majority vote,
  the corrupt directory can at worst cast a tie-breaking vote to
  decide whether to include marginal ORs.  How often such marginal
  cases will occur in practice, however, remains to be seen.

\item \emph{Subvert a majority of directory servers.}  If the
  adversary controls more than half of the directory servers, he can
  decide on a final directory, and thus can include as many
  compromised ORs in the final directory as he wishes.  Other than
  trying to ensure that directory server operators are truly
  independent and resistant to attack, Tor does not address this
  possibility.

\item \emph{Encourage directory server dissent.}  The directory
  agreement protocol requires that directory server operators agree on 
  the list of directory servers.  An adversary who can persuade some
  of the directory server operators to distrust one another could
  split the quorum into mutually hostile camps, thus partitioning
  users based on which directory they used.  Tor does not address
  this attack.

\item \emph{Trick the directory servers into listing a hostile OR.}
  Our threat model explicitly assumes directory server operators will
  be able to filter out most hostile ORs.  If this is not true, an
  attacker can flood the directory with compromised servers.

\item \emph{Convince the directories that a malfunctioning OR is
  working.}  In the current Tor implementation, directory servers
  assume that if they can start a TLS connection to an an OR, that OR
  must be running correctly.  It would be easy for a hostile OR to
  subvert this test by only accepting TLS connections from ORs, and
  ignoring all cells. Thus, directory servers must actively test ORs
  by building circuits and streams as appropriate.  The benefits and
  hazards of a similar approach are discussed in \cite{mix-acc}.
  
\end{tightlist}

\subsubsection*{Attacks against rendezvous points}
\begin{tightlist}
\item \emph{Make many introduction requests.}  An attacker could
  attempt to deny Bob service by flooding his Introduction Point with
  requests.  Because the introduction point can block requests that
  lack authentication tokens, however, Bob can restrict the volume of
  requests he receives, or require a certain amount of computation for
  every request he receives.
  
\item \emph{Attack an introduction point.} An attacker could try to
  disrupt a location-hidden service by disabling its introduction
  point.  But because a service's identity is attached to its public
  key, not its introduction point, the service can simply re-advertise
  itself at a different introduction point.

\item \emph{Compromise an introduction point.} If an attacker controls
  an introduction point for a service, it can flood the service with
  introduction requests, or prevent valid introduction requests from
  reaching the hidden server.  The server will notice a flooding
  attempt if it receives many introduction requests.  To notice
  blocking of valid requests, however, the hidden server should
  periodically test the introduction point by sending its introduction
  requests, and making sure it receives them.

\item \emph{Compromise a rendezvous point.}  Controlling a rendezvous
  point gains an attacker no more than controlling any other OR along
  a circuit, since all data passing along the rendezvous is protected
  by the session key shared by the client and server.

\end{tightlist}


\Section{Open Questions in Low-latency Anonymity}
\label{sec:maintaining-anonymity}
 
% There must be a better intro than this! -NM
In addition to the open problems discussed in
Section~\ref{subsec:non-goals}, many other questions remain to be
solved by future research before we can be truly confident that we
have built a secure low-latency anonymity service.

Many of these open issues are questions of balance.  For example,
how often should users rotate to fresh circuits?  Too-frequent
rotation is inefficient and expensive, but too-infrequent rotation
makes the user's traffic linkable.   Instead of opening a fresh
circuit; clients can also limit linkability exit from a middle point
of the circuit, or by truncating and re-extending the circuit, but
more analysis is needed to determine the proper trade-off.
%[XXX mention predecessor attacks?]

A similar question surrounds timing of directory operations:
how often should directories be updated?  With too-infrequent
updates clients receive an inaccurate picture of the network; with
too-frequent updates the directory servers are overloaded.

%do different exit policies at different exit nodes trash anonymity sets,
%or not mess with them much?
%
%% Why would they?  By routing traffic to certain nodes preferentially?

%[XXX Choosing paths and path lengths: I'm not writing this bit till
%  Arma's pathselection stuff is in. -NM]

%%%% Roger said that he'd put a path selection paragraph into section
%%%% 4 that would replace this.
%
%I probably should have noted that this means loops will be on at least
%five hop routes, which should be rare given the distribution.  I'm    
%realizing that this is reproducing some of the thought that led to a  
%default of five hops in the original onion routing design.  There were
%some different assumptions, which I won't spell out now.  Note that   
%enclave level protections really change these assumptions.  If most   
%circuits are just two hops, then just a single link observer will be  
%able to tell that two enclaves are communicating with high probability.
%So, it would seem that enclaves should have a four node minimum circuit
%to prevent trivial circuit insider identification of the whole circuit,
%and three hop minimum for circuits from an enclave to some nonclave    
%responder. But then... we would have to make everyone obey these rules 
%or a node that through timing inferred it was on a four hop circuit    
%would know that it was probably carrying enclave to enclave traffic.   
%Which... if there were even a moderate number of bad nodes in the      
%network would make it advantageous to break the connection to conduct  
%a reformation intersection attack. Ahhh! I gotta stop thinking         
%about this and work on the paper some before the family wakes up.  
%On Sat, Oct 25, 2003 at 06:57:12AM -0400, Paul Syverson wrote:
%> Which... if there were even a moderate number of bad nodes in the
%> network would make it advantageous to break the connection to conduct
%> a reformation intersection attack. Ahhh! I gotta stop thinking
%> about this and work on the paper some before the family wakes up. 
%This is the sort of issue that should go in the 'maintaining anonymity
%with tor' section towards the end. :)
%Email from between roger and me to beginning of section above. Fix and move.

Throughout this paper, we have assumed that end-to-end traffic
analysis will immediately and automatically defeat a low-latency
anonymity system. Even high-latency anonymity
systems can be vulnerable to end-to-end traffic analysis, if the
traffic volumes are high enough, and if users' habits are sufficiently
distinct \cite{limits-open,statistical-disclosure}.  \emph{Can
  anything be done to make low-latency systems resist these attacks as
  well as high-latency systems?}
Tor already makes some effort to conceal the starts and
ends of streams by wrapping all long-range control commands in
identical-looking relay cells, but more analysis is needed.  Link
padding could frustrate passive observers who count packets; long-range
padding could work against observers who own the first hop in a
circuit.  But more research needs to be done in order to find an
efficient and practical approach.  Volunteers prefer not to run
constant-bandwidth padding; but more sophisticated traffic shaping
approaches remain somewhat unanalyzed. 
%[XXX is this so?] 
Recent work
on long-range padding \cite{defensive-dropping} shows promise.  One
could also try to reduce correlation in packet timing by batching and
re-ordering packets, but it is unclear whether this could improve
anonymity without introducing so much latency as to render the
network unusable.

Even if passive timing attacks were wholly solved, active timing
attacks would remain.  \emph{What can
  be done to address attackers who can introduce timing patterns into
  a user's traffic?}  % [XXX mention likely approaches]

%%% I think we cover this by framing the problem as ``Can we make 
%%% end-to-end characteristics of low-latency systems as good as
%%% those of high-latency systems?''  Eliminating long-term
%%% intersection is a hard problem.
%
%Even regardless of link padding from Alice to the cloud, there will be
%times when Alice is simply not online. Link padding, at the edges or
%inside the cloud, does not help for this.

In order to scale to large numbers of users, and to prevent an
attacker from observing the whole network at once, it may be necessary
for low-latency anonymity systems to support far more servers than Tor
currently anticipates.  This introduces several issues.  First, if
approval by a centralized set of directory servers is no longer
feasible, what mechanism should be used to prevent adversaries from
signing up many spurious servers? 
Second, if clients can no longer have a complete
picture of the network at all times, how can should they perform
discovery while preventing attackers from manipulating or exploiting
gaps in client knowledge?  Third, if there are too many servers
for every server to constantly communicate with every other, what kind
of non-clique topology should the network use?   Restricted-route
topologies promise comparable anonymity with better scalability
\cite{danezis-pets03}, but whatever topology we choose, we need some
way to keep attackers from manipulating their position within it.
Fourth, since no centralized authority is tracking server reliability,
How do we prevent unreliable servers from rendering the network
unusable?  Fifth, do clients receive so much anonymity benefit from
running their own servers that we should expect them all to do so, or
do we need to find another incentive structure to motivate them?
(Tarzan and MorphMix present possible solutions.)

% [[ XXX how to approve new nodes (advogato, sybil, captcha (RTT));]

Alternatively, it may be the case that one of these problems proves
intractable, or that the drawbacks to many-server systems prove
greater than the benefits.  Nevertheless, we may still do well to
consider non-clique topologies.  A cascade topology may provide more
defense against traffic confirmation.
% XXX Why would it?   Cite.  -NM
Does the hydra (many inputs, few outputs) topology work
better? Are we going to get a hydra anyway because most nodes will be
middleman nodes?

As mentioned in Section~\ref{subsec:dos}, Tor could improve its
robustness against node failure by buffering transmitted stream data
at the network's edges until the data has been acknowledged by the
other end of the stream.  The efficacy of this approach remains to be
tested, however, and there may be more effective means for ensuring
reliable connections in the presence of unreliable nodes.

%%% Keeping this original paragraph for a little while, since it 
%%% is not the same as what's written there now.
%
%Because Tor depends on TLS and TCP to provide a reliable transport,
%when one of the servers goes down, all the circuits (and thus streams)
%traveling over that server must break.  This reduces anonymity because
%everybody needs to reconnect right then (does it? how much?)  and
%because exit connections all break at the same time, and it also harms
%usability. It seems the problem is even worse in a peer-to-peer
%environment, because so far such systems don't really provide an
%incentive for nodes to stay connected when they're done browsing, so
%we would expect a much higher churn rate than for onion routing.
%there ways of allowing streams to survive the loss of a node in the
%path?

% Roger or Paul suggested that we say something about incentives,
% too, but I think that's a better candidate for our future work
% section.  After all, we will doubtlessly learn very much about why
% people do or don't run and use Tor in the near future. -NM

%We should run a squid at each exit node, to provide comparable anonymity
%to private exit nodes for cache hits, to speed everything up, and to
%have a buffer for funny stuff coming out of port 80.
% on the other hand, it hampers PFS, because ORs have pages in the cache.
%I previously elsewhere suggested bulk transfer proxies to carve
%up big things so that they could be downloaded in less noticeable
%pieces over several normal looking connections. We could suggest
%similarly one or a handful of squid nodes that might serve up
%some of the more sensitive but common material, especially if
%the relevant sites didn't want to or couldn't run their own OR.
%This would be better than having everyone run a squid which would
%just help identify after the fact the different history of that
%node's activity. All this kind of speculation needs to move to
%future work section I guess. -PS]

%%%%%%%%%%%%%%%%%%%%%%%%%%%%%%%%%%%%%%%%%%%%%%%%%%%%%%%%%%%%%%%%%%%%%%%

\Section{Future Directions}
\label{sec:conclusion}

Tor brings together many innovations into
a unified deployable system. But there are still several attacks that
work quite well, as well as a number of sustainability and run-time
issues remaining to be ironed out. In particular:

% Many of these (Scalability, cover traffic, morphmix) 
% are duplicates from open problems.
%
\begin{tightlist}
\item \emph{Scalability:} Tor's emphasis on design simplicity and
  deployability has led us to adopt a clique topology, a
  semi-centralized model for directories and trusts, and a
  full-network-visibility model for client knowledge.  None of these
  properties will scale to more than a few hundred servers, at most.
  Promising approaches to better scalability exist (see
  Section~\ref{sec:maintaining-anonymity}), but more deployment
  experience would be helpful in learning the relative importance of
  these bottlenecks.
\item \emph{Cover traffic:} Currently we avoid cover traffic because
  of its clear costs in performance and bandwidth, and because its
  security benefits have not well understood. With more research
  \cite{SS03,defensive-dropping}, the price/value ratio may change,
  both for link-level cover traffic and also long-range cover traffic.
\item \emph{Better directory distribution:} Even with the threshold
  directory agreement algorithm described in Section~\ref{subsec:dirservers},
  the directory servers are still trust bottlenecks. We must find more
  decentralized yet practical ways to distribute up-to-date snapshots of
  network status without introducing new attacks.  Also, directory
  retrieval presents a scaling problem, since clients currently
  download a description of the entire network state every 15
  minutes.  As the state grows larger and clients more numerous, we
  may need to move to a solution in which clients only receive
  incremental updates to directory state, or where directories are
  cached at the ORs to avoid high loads on the directory servers.
\item \emph{Implementing location-hidden servers:} While
  Section~\ref{sec:rendezvous} describes a design for rendezvous
  points and location-hidden servers, these feature has not yet been
  implemented.  While doing so, will likely encounter additional
  issues, both in terms of usability and anonymity, that must be
  resolved.
\item \emph{Further specification review:} Although we have a public,
  byte-level specification for the Tor protocols, this protocol has
  not received extensive external review.  We hope that as Tor
  becomes more widely deployed, more people will become interested in
  examining our specification.
\item \emph{Wider-scale deployment:} The original goal of Tor was to
  gain experience in deploying an anonymizing overlay network, and
  learn from having actual users.  We are now at the point in design
  and development where we can start deploying a wider network.  Once
  we have are ready for actual users, we will doubtlessly be better
  able to evaluate some of our design decisions, including our
  robustness/latency trade-offs, our performance trade-offs (including
  cell size), our abuse-prevention mechanisms, and
  our overall usability.
% XXX large and small cells on same network.
% XXX work with morphmix spec
\end{tightlist}

%%%%%%%%%%%%%%%%%%%%%%%%%%%%%%%%%%%%%%%%%%%%%%%%%%%%%%%%%%%%%%%%%%%%%%%

%% commented out for anonymous submission
%\Section{Acknowledgments}
% Peter Palfrader, Geoff Goodell, Adam Shostack, Joseph Sokol-Margolis,
%   John Bashinski
%   for editing and comments
% Matej Pfajfar, Andrei Serjantov, Marc Rennhard for design discussions
% Bram Cohen for congestion control discussions
% Adam Back for suggesting telescoping circuits

%%%%%%%%%%%%%%%%%%%%%%%%%%%%%%%%%%%%%%%%%%%%%%%%%%%%%%%%%%%%%%%%%%%%%%%

\bibliographystyle{latex8}
\bibliography{tor-design}

\end{document}

% Style guide:
%     U.S. spelling
%     avoid contractions (it's, can't, etc.)
%     prefer ``for example'' or ``such as'' to e.g.
%     prefer ``that is'' to i.e.
%     'mix', 'mixes' (as noun)
%     'mix-net'
%     'mix', 'mixing' (as verb)
%     'middleman'  [Not with a hyphen; the hyphen has been optional
%         since Middle English.]
%     'nymserver'
%     'Cypherpunk', 'Cypherpunks', 'Cypherpunk remailer'
%     'Onion Routing design', 'onion router' [note capitalization]
%     'SOCKS'
%     Try not to use \cite as a noun.  
%     'Authorizating' sounds great, but it isn't a word.
%     'First, second, third', not 'Firstly, secondly, thirdly'.
%     'circuit', not 'channel'
%     Typography: no space on either side of an em dash---ever.
%     Hyphens are for multi-part words; en dashs imply movement or
%        opposition (The Alice--Bob connection); and em dashes are
%        for punctuation---like that.
%     A relay cell; a control cell; a \emph{create} cell; a
%     \emph{relay truncated} cell.  Never ``a \emph{relay truncated}.''
%
%     'Substitute ``Damn'' every time you're inclined to write ``very;'' your
%     editor will delete it and the writing will be just as it should be.'
%     -- Mark Twain
