\documentclass[times,10pt,twocolumn]{article}
\usepackage{latex8}
%\usepackage{times}
\usepackage{url}
\usepackage{graphics}
\usepackage{amsmath}

\pagestyle{empty}

\renewcommand\url{\begingroup \def\UrlLeft{<}\def\UrlRight{>}\urlstyle{tt}\Url}
\newcommand\emailaddr{\begingroup \def\UrlLeft{<}\def\UrlRight{>}\urlstyle{tt}\Url}

% If an URL ends up with '%'s in it, that's because the line *in the .bib/.tex
% file* is too long, so break it there (it doesn't matter if the next line is
% indented with spaces). -DH

%\newif\ifpdf
%\ifx\pdfoutput\undefined
%   \pdffalse
%\else
%   \pdfoutput=1
%   \pdftrue
%\fi

\newenvironment{tightlist}{\begin{list}{$\bullet$}{
  \setlength{\itemsep}{0mm}
    \setlength{\parsep}{0mm}
    %  \setlength{\labelsep}{0mm}
    %  \setlength{\labelwidth}{0mm}
    %  \setlength{\topsep}{0mm}
    }}{\end{list}}

\begin{document}

%% Use dvipdfm instead. --DH
%\ifpdf
%  \pdfcompresslevel=9
%  \pdfpagewidth=\the\paperwidth
%  \pdfpageheight=\the\paperheight
%\fi

\title{Tor: Design of a Next-Generation Onion Router}

\author{Anonymous}
%\author{Roger Dingledine \\ The Free Haven Project \\ arma@freehaven.net \and
%Nick Mathewson \\ The Free Haven Project \\ nickm@freehaven.net \and
%Paul Syverson \\ Naval Research Lab \\ syverson@itd.nrl.navy.mil}

\maketitle
\thispagestyle{empty}

\begin{abstract}
We present Tor, a connection-based low-latency anonymous communication
system. It is intended as an update and replacement for onion routing
and addresses many limitations in the original onion routing design.
Tor works in a real-world Internet environment,
requires little synchronization or coordination between nodes, and
protects against known anonymity-breaking attacks as well
as or better than other systems with similar design parameters.
\end{abstract}

%\begin{center}
%\textbf{Keywords:} anonymity, peer-to-peer, remailer, nymserver, reply block
%\end{center}

%%%%%%%%%%%%%%%%%%%%%%%%%%%%%%%%%%%%%%%%%%%%%%%%%%%%%%%%%%%%%%%%%%%%%%%

\Section{Overview}
\label{sec:intro}

Onion routing is a distributed overlay network designed to anonymize
low-latency TCP-based applications such as web browsing, secure shell,
and instant messaging. Users choose a path through the network and
build a \emph{virtual circuit}, in which each node in the path knows its
predecessor and successor, but no others. Traffic flowing down the circuit
is sent in fixed-size \emph{cells}, which are unwrapped by a symmetric key
at each node, revealing the downstream node. The original onion routing
project published several design and analysis papers
\cite{or-jsac98,or-discex00,or-ih96,or-pet02}. While there was briefly
a wide area onion routing network,
the only long-running and publicly accessible
implementation was a fragile proof-of-concept that ran on a single
machine. Many critical design and deployment issues were never implemented,
and the design has not been updated in several years.
Here we describe Tor, a protocol for asynchronous, loosely
federated onion routers that provides the following improvements over
the old onion routing design:

\begin{tightlist}

\item \textbf{Perfect forward secrecy:} The original onion routing
design is vulnerable to a single hostile node recording traffic and later
forcing successive nodes in the circuit to decrypt it. Rather than using
onions to lay the circuits, Tor uses an incremental or \emph{telescoping}
path-building design, where the initiator negotiates session keys with
each successive hop in the circuit. Onion replay detection is no longer
necessary, and the network as a whole is more reliable to boot, since
the initiator knows which hop failed and can try extending to a new node.

\item \textbf{Applications talk to the onion proxy via Socks:}
The original onion routing design required a separate proxy for each
supported application protocol, resulting in a lot of extra code (most
of which was never written) and also meaning that a lot of TCP-based
applications were not supported. Tor uses the unified and standard Socks
\cite{socks4,socks5} interface, allowing us to support any TCP-based
program without modification.

\item \textbf{Many applications can share one circuit:} The original
onion routing design built one circuit for each request. Aside from the
performance issues of doing public key operations for every request, it
also turns out that regular communications patterns mean building lots
of circuits, which can endanger anonymity \cite{wright03}.
%[XXX Was this
%supposed to be Wright02 or Wright03. In any case I am hesitant to cite
%that work in this context. While the point is valid in general, that
%work is predicated on assumptions that I don't think typically apply
%to onion routing (whether old or new design). -PS]
%[I had meant wright03, but I guess wright02 could work as well.
% If you don't think these attacks work on onion routing, you need to
% write up a convincing argument of this. Such an argument would
% be very worthwhile to include in this paper. -RD]
Tor multiplexes many
connections down each circuit, but still rotates the circuit periodically
to avoid too much linkability.

\item \textbf{No mixing or traffic shaping:} The original onion routing
design called for full link padding both between onion routers and between
onion proxies (that is, users) and onion routers \cite{or-jsac98}. The
later analysis paper \cite{or-pet02} suggested \emph{traffic shaping}
to provide similar protection but use less bandwidth, but did not go
into detail. However, recent research \cite{econymics} and deployment
experience \cite{freedom} indicate that this level of resource
use is not practical or economical; and even full link padding is still
vulnerable to active attacks \cite{defensive-dropping}.
% [XXX what is being referenced here, Dogan? -PS]
%[An upcoming FC04 paper. I'll add a cite when it's out. -RD]

\item \textbf{Leaky pipes:} Through in-band signalling within the circuit,
Tor initiators can direct traffic to nodes partway down the circuit. This
allows for long-range padding to frustrate timing attacks at the initiator
\cite{defensive-dropping}, but because circuits are used by more than
one application, it also allows traffic to exit the circuit from the
middle -- thus frustrating timing attacks based on observing exit points.
%Or something like that. hm. Tone this down maybe? Or support it. -RD

\item \textbf{Congestion control:} Earlier anonymity designs do not
address traffic bottlenecks. Unfortunately, typical approaches to load
balancing and flow control in overlay networks involve inter-node control
communication and global views of traffic. Our decentralized ack-based
congestion control maintains reasonable anonymity while allowing nodes
at the edges of the network to detect congestion or flooding attacks
and send less data until the congestion subsides.

\item \textbf{Directory servers:} Rather than attempting to flood
link-state information through the network, which can be unreliable and
open to partitioning attacks or outright deception, Tor takes a simplified
view towards distributing link-state information. Certain more trusted
onion routers also serve as directory servers; they provide signed
\emph{directories} describing all routers they know about, and which
are currently up. Users periodically download these directories via HTTP.

\item \textbf{End-to-end integrity checking:} Without integrity checking
on traffic going through the network, an onion router can change the
contents of cells as they pass by, e.g. by redirecting a connection on
the fly so it connects to a different webserver, or by tagging encrypted
traffic and looking for traffic at the network edges that has been
tagged \cite{minion-design}.

\item \textbf{Robustness to node failure:} router twins

\item \textbf{Exit policies:}
Tor provides a consistent mechanism for each node to specify and
advertise an exit policy.

\item \textbf{Rendezvous points:}
location-protected servers

\end{tightlist}

We review previous work in Section \ref{sec:background}, describe
our goals and assumptions in Section \ref{sec:assumptions},
and then address the above list of improvements in Sections
\ref{sec:design}-\ref{sec:maintaining-anonymity}. We then summarize
how our design stands up to known attacks, and conclude with a list of
open problems.

%%%%%%%%%%%%%%%%%%%%%%%%%%%%%%%%%%%%%%%%%%%%%%%%%%%%%%%%%%%%%%%%%%%%%%%

\Section{Background and threat model}
\label{sec:background}

\SubSection{Related work}
\label{sec:related-work}
Modern anonymity designs date to Chaum's Mix-Net\cite{chaum-mix} design of
1981.  Chaum proposed hiding sender-recipient connections by wrapping
messages in several layers of public key cryptography, and relaying them
through a path composed of Mix servers.  Mix servers in turn decrypt, delay,
and re-order messages, before relay them along the path towards their
destinations.

Subsequent relay-based anonymity designs have diverged in two
principal directions.  Some have attempted to maximize anonymity at
the cost of introducing comparatively large and variable latencies,
for example, Babel\cite{babel}, Mixmaster\cite{mixmaster-spec}, and
Mixminion\cite{minion-design}.  Because of this
decision, such \emph{high-latency} networks are well-suited for anonymous
email, but introduce too much lag for interactive tasks such as web browsing,
internet chat, or SSH connections.

Tor belongs to the second category: \emph{low-latency} designs that
attempt to anonymize interactive network traffic.  Because such
traffic tends to involve a relatively large numbers of packets, it is
difficult to prevent an attacker who can eavesdrop entry and exit
points from correlating packets entering the anonymity network with
packets leaving it. Although some work has been done to frustrate
these attacks, most designs protect primarily against traffic analysis
rather than traffic confirmation \cite{or-jsac98}.  One can pad and
limit communication to a constant rate or at least to control the
variation in traffic shape. This can have prohibitive bandwidth costs
and/or performance limitations. One can also use a cascade (fixed
shared route) with a relatively fixed set of users. This assumes a
degree of agreement and provides an easier target for an active
attacker since the endpoints are generally known. However, a practical
network with both of these features has been run for many years
\cite{web-mix}.

they still...  
[XXX go on to explain how the design choices implied in low-latency result in
significantly different designs.]

The simplest low-latency designs are single-hop proxies such as the
Anonymizer \cite{anonymizer}, wherein a single trusted server removes
identifying users' data before relaying it.  These designs are easy to
analyze, but require end-users to trust the anonymizing proxy.

More complex are distributed-trust, channel-based anonymizing systems.  In
these designs, a user establishes one or more medium-term bidirectional
end-to-end tunnels to exit servers, and uses those tunnels to deliver a
number of low-latency packets to and from one or more destinations per
tunnel.  Establishing tunnels is comparatively expensive and typically
requires public-key cryptography, whereas relaying packets along a tunnel is
comparatively inexpensive.  Because a tunnel crosses several servers, no
single server can learn the user's communication partners.

Systems such as earlier versions of Freedom and onion routing
build the anonymous channel all at once (using an onion). Later
designs of each of these build the channel in stages as does AnonNet
\cite{anonnet}. Amongst other things, this makes perfect forward
secrecy feasible.

Some systems, such as Crowds \cite{crowds-tissec}, do not rely on the
changing appearance of packets to hide the path; rather they employ
mechanisms so that an intermediary cannot be sure when it is
receiving/sending to the ultimate initiator. There is no public-key
encryption needed for Crowds, but the responder and all data are
visible to all nodes on the path so that anonymity of connection
initiator depends on filtering all identifying information from the
data stream. Crowds is also designed only for HTTP traffic.

Hordes \cite{hordes-jcs} is based on Crowds but also uses multicast
responses to hide the initiator. Some systems go even further
requiring broadcast \cite{herbivore,p5} although tradeoffs are made to
make this more practical. Both Herbivore and P5 are designed primarily
for communication between communicating peers, although Herbivore
permits external connections by requesting a peer to serve as a proxy.
Allowing easy connections to nonparticipating responders or recipients
is a practical requirement for many users, e.g., to visit
nonparticipating Web sites or to send mail to nonparticipating
recipients.

Distributed-trust anonymizing systems differ in how they prevent attackers
from controlling too many servers and thus compromising too many user paths.
Some protocols rely on a centrally maintained set of well-known anonymizing
servers.  Others (such as Tarzan and MorphMix) allow unknown users to run
servers, while using a limited resource (DHT space for Tarzan; IP space for
MorphMix) to prevent an attacker from owning too much of the network.
[XXX what else?  What does (say) crowds do?]

All of the above systems  Several systems with varying design goals
and capabilities but all of which require that communicants be
intentionally participating are mentioned here.

Some involve multicast or more to work
herbivore

There are also many systems which are intended for anonymous
and/or censorship resistant file sharing. [XXX Should we list all these
or just say it's out of scope for the paper?
eternity, gnunet, freenet, freehaven, publius, tangler, taz/rewebber]



[XXX Should we add a paragraph dividing servers by all-at-once approach to
  tunnel-building (OR1,Freedom1) versus piecemeal approach
  (OR2,Anonnet?,Freedom2) ?]



Channel-based anonymizing systems also differ in their use of dummy traffic.
[XXX]

Finally, several systems provide low-latency anonymity without channel-based
communication.  Crowds and [XXX] provide anonymity for HTTP requests; [...]

[XXX Mention error recovery?]

Web-MIXes \cite{web-mix} (also known as the Java Anon Proxy or JAP)
use a cascade architecture with relatively constant groups of users
sending and receiving at a constant rate.

Some, such as Crowds \cite{crowds-tissec}, do nothing against such
confirmation but still make it difficult for nodes along a connection to
perform timing confirmations that would more easily identify when
the immediate predecessor is the initiator of a connection, which in
Crowds would reveal both initiator and responder to the attacker.


anonymizer
pipenet
freedom v1
freedom v2
onion routing v1
isdn-mixes
crowds
real-time mixes, web mixes
anonnet (marc rennhard's stuff)
morphmix
P5
gnunet
rewebbers
tarzan
herbivore
hordes
cebolla (?)

[XXX Close by mentioning where Tor fits.]

\SubSection{Our threat model}
\label{subsec:threat-model}

\SubSection{Known attacks against low-latency anonymity systems}
\label{subsec:known-attacks}

We discuss each of these attacks in more detail below, along with the
aspects of the Tor design that provide defense. We provide a summary
of the attacks and our defenses against them in Section \ref{sec:attacks}.

Passive attacks:
simple observation,
timing correlation,
size correlation,
option distinguishability,

Active attacks:
key compromise,
iterated subpoena,
run recipient,
run a hostile node,
compromise entire path,
selectively DOS servers,
introduce timing into messages,
directory attacks,
tagging attacks

\Section{Design goals and assumptions}
\label{sec:assumptions}

[XXX Perhaps the threat model belongs here.]

%%%%%%%%%%%%%%%%%%%%%%%%%%%%%%%%%%%%%%%%%%%%%%%%%%%%%%%%%%%%%%%%%%%%%%%

\Section{The Tor Design}
\label{sec:design}


\Section{Other design decisions}

\SubSection{Exit policies and abuse}
\label{subsec:exitpolicies}

\SubSection{Directory Servers}
\label{subsec:dir-servers}

\Section{Rendezvous points for location privacy}
\label{sec:rendezvous}

Rendezvous points are a building block for \emph{location-hidden services}
(that is, responder anonymity) in the Tor network. Location-hidden
services means Bob can offer a tcp service, such as an Apache webserver,
without revealing the IP of that service.

We provide censorship resistance for Bob by allowing him to advertise
several onion routers (nodes known as his Introduction Points, see
Section \ref{subsec:intro-point}) as his public location. Alice,
the client, chooses a node known as a Meeting Point (see Section
\ref{subsec:meeting-point}). She connects to one of Bob's introduction
points, informs him about her meeting point, and then waits for him to
connect to her meeting point. This extra level of indirection is needed
so Bob's introduction points don't serve files directly (else they open
themselves up to abuse, eg from serving Nazi propaganda in France). The
extra level of indirection also allows Bob to choose which requests to
respond to, and which to ignore.

We provide the necessary glue code so that Alice can view
webpages on a location-hidden webserver, and Bob can run a
location-hidden server, with minimal invasive changes (see Section
\ref{subsec:client-rendezvous}). Both Alice and Bob must run local
onion proxies (OPs) -- software that knows how to talk to the onion
routing network.

The steps of a rendezvous:
\begin{tightlist}
\item Bob chooses some Introduction Points, and advertises them on a
      Distributed Hash Table (DHT).
\item Bob establishes onion routing connections to each of his
      Introduction Points, and waits.
\item Alice learns about Bob's service out of band (perhaps Bob gave her
      a pointer, or she found it on a website). She looks up the details
      of Bob's service from the DHT.
\item Alice chooses and establishes a Meeting Point (MP) for this
      transaction.
\item Alice goes to one of Bob's Introduction Points, and gives it a blob
      (encrypted for Bob) which tells him about herself and the Meeting
      Point she chose. The Introduction Point sends the blob to Bob.
\item Bob chooses whether to ignore the blob, or to onion route to MP.
      Let's assume the latter.
\item MP plugs together Alice and Bob. Note that MP doesn't know (or care)
      who Alice is, or who Bob is; and it can't read anything they
      transmit either, because they share a session key.
\item Alice sends a 'begin' cell along the circuit. It makes its way
      to Bob's onion proxy. Bob's onion proxy connects to Bob's webserver.
\item Data goes back and forth as usual.
\end{tightlist}

Ian Goldberg developed a similar notion of rendezvous points for
low-latency anonymity systems \cite{goldberg-thesis}. His ``service tag''
is the same concept as our ``hash of service's public key''. We make it
a hash of the public key so it can be self-authenticating, and so the
client can recognize the same service with confidence later on.
The main differences are:
* We force the client to use our software. This means
   - the client can get anonymity for himself pretty easily, since he's
     already running our onion proxy.
   - the client can literally just click on a url in his Mozilla, paste it
     into wget, etc, and it will just work. (The url is a long-term
     service tag; like Ian's, it doesn't expire as the server changes
     public addresses. But in Ian's scheme it seems the client must
     manually hunt down a current location of the service via gnutella?)
   - the client and server can share ephemeral DH keys, so at no point
     in the path is the plaintext exposed.
* I fear that we would get *no* volunteers to run Ian's rendezvous points,
  because they have to actually serve the Nazi propaganda (or whatever)
  in plaintext. So we add another layer of indirection to the system:
  the rendezvous service is divided into Introduction Points and
  Meeting Points. The introduction points (the ones that the server is
  publically associated with) do not output any bytes to the clients. And
  the meeting points don't know the client, the server, or the stuff
  being transmitted. The indirection scheme is also designed with
  authentication/authorization in mind -- if the client doesn't include
  the right cookie with its request for service, the server doesn't even
  acknowledge its existence.


\subsubsection{Integration with user applications}

\Section{Maintaining anonymity sets}
\label{sec:maintaining-anonymity}

\SubSection{Using a circuit many times}
\label{subsec:many-messages}

%%%%%%%%%%%%%%%%%%%%%%%%%%%%%%%%%%%%%%%%%%%%%%%%%%%%%%%%%%%%%%%%%%%%%%%

\Section{Attacks and Defenses}
\label{sec:attacks}

Below we summarize a variety of attacks and how well our design withstands
them.

%%%%%%%%%%%%%%%%%%%%%%%%%%%%%%%%%%%%%%%%%%%%%%%%%%%%%%%%%%%%%%%%%%%%%%%

\Section{Future Directions and Open Problems}
\label{sec:conclusion}

Tor brings together many innovations from many different projects into
a unified deployable system. But there are still several attacks that
work quite well, as well as a number of sustainability and run-time
issues remaining to be ironed out. In particular:

\begin{itemize}
\item foo
\end{itemize}

%%%%%%%%%%%%%%%%%%%%%%%%%%%%%%%%%%%%%%%%%%%%%%%%%%%%%%%%%%%%%%%%%%%%%%%

\Section{Acknowledgments}

%%%%%%%%%%%%%%%%%%%%%%%%%%%%%%%%%%%%%%%%%%%%%%%%%%%%%%%%%%%%%%%%%%%%%%%

\bibliographystyle{latex8}
\bibliography{tor-design}

\end{document}

% Style guide:
%     U.S. spelling
%     avoid contractions (it's, can't, etc.)
%     'mix', 'mixes' (as noun)
%     'mix-net'
%     'mix', 'mixing' (as verb)
%     'Mixminion Project'
%     'Mixminion' (meaning the protocol suite or the network)
%     'Mixmaster' (meaning the protocol suite or the network)
%     'middleman'  [Not with a hyphen; the hyphen has been optional
%         since Middle English.]
%     'nymserver'
%     'Cypherpunk', 'Cypherpunks', 'Cypherpunk remailer'
%
%     'Whenever you are tempted to write 'Very', write 'Damn' instead, so
%     your editor will take it out for you.'  -- Misquoted from Mark Twain

