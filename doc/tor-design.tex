\documentclass[times,10pt,twocolumn]{article}
\usepackage{latex8}
%\usepackage{times}
\usepackage{url}
\usepackage{graphics}
\usepackage{amsmath}

\pagestyle{empty}

\renewcommand\url{\begingroup \def\UrlLeft{<}\def\UrlRight{>}\urlstyle{tt}\Url}
\newcommand\emailaddr{\begingroup \def\UrlLeft{<}\def\UrlRight{>}\urlstyle{tt}\Url}

% If an URL ends up with '%'s in it, that's because the line *in the .bib/.tex
% file* is too long, so break it there (it doesn't matter if the next line is
% indented with spaces). -DH

%\newif\ifpdf
%\ifx\pdfoutput\undefined
%   \pdffalse
%\else
%   \pdfoutput=1
%   \pdftrue
%\fi

\newenvironment{tightlist}{\begin{list}{$\bullet$}{
  \setlength{\itemsep}{0mm}
    \setlength{\parsep}{0mm}
    %  \setlength{\labelsep}{0mm}
    %  \setlength{\labelwidth}{0mm}
    %  \setlength{\topsep}{0mm}
    }}{\end{list}}

\begin{document}

%% Use dvipdfm instead. --DH
%\ifpdf
%  \pdfcompresslevel=9
%  \pdfpagewidth=\the\paperwidth
%  \pdfpageheight=\the\paperheight
%\fi

\title{Tor: Design of a Next-Generation Onion Router}

%\author{Roger Dingledine \\ The Free Haven Project \\ arma@freehaven.net \and
%Nick Mathewson \\ The Free Haven Project \\ nickm@freehaven.net \and
%Paul Syverson \\ Naval Research Lab \\ syverson@itd.nrl.navy.mil}

\maketitle
\thispagestyle{empty}

\begin{abstract}
We present Tor, a connection-based low-latency anonymous communication
system. It is intended as an update and replacement for onion routing
and addresses many limitations in the original onion routing design.
Tor works in a real-world Internet environment,
requires little synchronization or coordination between nodes, and
protects against known anonymity-breaking attacks as well
as or better than other systems with similar design parameters.
\end{abstract}

%\begin{center}
%\textbf{Keywords:} anonymity, peer-to-peer, remailer, nymserver, reply block
%\end{center}

%%%%%%%%%%%%%%%%%%%%%%%%%%%%%%%%%%%%%%%%%%%%%%%%%%%%%%%%%%%%%%%%%%%%%%%

\Section{Overview}
\label{sec:intro}

Onion routing is a distributed overlay network designed to anonymize
low-latency TCP-based applications such as web browsing, secure shell,
and instant messaging. Users choose a path through the network and
build a \emph{virtual circuit}, in which each node in the path knows its
predecessor and successor, but no others. Traffic flowing down the circuit
is sent in fixed-size \emph{cells}, which are unwrapped by a symmetric key
at each node, revealing the downstream node. The original onion routing
project published several design and analysis papers
\cite{or-jsac98,or-discex00,or-ih96,or-pet00}. While there was briefly
a wide area onion routing network,
% how long is briefly? a day, a month? -RD
the only long-running and publicly accessible
implementation was a fragile proof-of-concept that ran on a single
machine. Many critical design and deployment issues were never implemented,
and the design has not been updated in several years.
Here we describe Tor, a protocol for asynchronous, loosely
federated onion routers that provides the following improvements over
the old onion routing design:

\begin{tightlist}

\item \textbf{Perfect forward secrecy:} The original onion routing
design is vulnerable to a single hostile node recording traffic and later
forcing successive nodes in the circuit to decrypt it. Rather than using
onions to lay the circuits, Tor uses an incremental or \emph{telescoping}
path-building design, where the initiator negotiates session keys with
each successive hop in the circuit. Onion replay detection is no longer
necessary, and the process of building circuits is more reliable, since
the initiator knows which hop failed and can try extending to a new node.

\item \textbf{Applications talk to the onion proxy via Socks:}
The original onion routing design required a separate proxy for each
supported application protocol, resulting in a lot of extra code --- most
of which was never written, so most applications were not supported.
Tor uses the unified and standard Socks
\cite{socks4,socks5} interface, allowing us to support any TCP-based
program without modification.

\item \textbf{Many applications can share one circuit:} The original
onion routing design built one circuit for each request. Aside from the
performance issues of doing public key operations for every request, it
also turns out that regular communications patterns mean building lots
of circuits, which can endanger anonymity.
The very first onion routing design \cite{or-ih96} protected against
this to some extent by hiding network access behind an onion
router/firewall that was also forwarding traffic from other nodes.
However, even if this meant complete protection, many users can
benefit from onion routing for which neither running one's own node
nor such firewall configurations are adequately convenient to be
feasible. Those users, especially if they engage in certain unusual
communication behaviors, may be identifiable \cite{wright03}. To
complicate the possibility of such attacks Tor multiplexes many
connections down each circuit, but still rotates the circuit
periodically to avoid too much linkability from requests on a single
circuit.

\item \textbf{No mixing or traffic shaping:} The original onion routing
design called for full link padding both between onion routers and between
onion proxies (that is, users) and onion routers \cite{or-jsac98}. The
later analysis paper \cite{or-pet00} suggested \emph{traffic shaping}
to provide similar protection but use less bandwidth, but did not go
into detail. However, recent research \cite{econymics} and deployment
experience \cite{freedom21-security} indicate that this level of resource
use is not practical or economical; and even full link padding is still
vulnerable to active attacks \cite{defensive-dropping}.
%[An upcoming FC04 paper. I'll add a cite when it's out. -RD]

\item \textbf{Leaky pipes:} Through in-band signalling within the
  circuit, Tor initiators can direct traffic to nodes partway down the
  circuit. This allows for long-range padding to frustrate traffic
  shape and volume attacks at the initiator \cite{defensive-dropping},
  but because circuits are used by more than one application, it also
  allows traffic to exit the circuit from the middle -- thus
  frustrating traffic shape and volume attacks based on observing exit
  points.
%Or something like that. hm. Tone this down maybe? Or support it. -RD
%How's that? -PS

\item \textbf{Congestion control:} Earlier anonymity designs do not
address traffic bottlenecks. Unfortunately, typical approaches to load
balancing and flow control in overlay networks involve inter-node control
communication and global views of traffic. Our decentralized ack-based
congestion control maintains reasonable anonymity while allowing nodes
at the edges of the network to detect congestion or flooding attacks
and send less data until the congestion subsides.

\item \textbf{Directory servers:} Rather than attempting to flood
link-state information through the network, which can be unreliable and
open to partitioning attacks or outright deception, Tor takes a simplified
view towards distributing link-state information. Certain more trusted
onion routers also serve as directory servers; they provide signed
\emph{directories} describing all routers they know about, and which
are currently up. Users periodically download these directories via HTTP.

\item \textbf{End-to-end integrity checking:} Without integrity checking
on traffic going through the network, any onion router on the path
can change the contents of cells as they pass by, e.g. to redirect a
connection on the fly so it connects to a different webserver, or to
tag encrypted traffic and look for the tagged traffic at the network
edges \cite{minion-design}.

\item \textbf{Robustness to failed nodes:} A failed node in a traditional
mix network means lost messages, but in Tor, users can notice failed
nodes while building circuits and route around them. We further provide a
simple mechanism that allows connections to be established despite recent
node failure or slightly dated information from a directory server. Tor
permits onion routers to have \emph{router twins} --- nodes that share
the same private decryption key. Note that because connections now have
perfect forward secrecy, an onion router still cannot read the traffic
on a connection established through its twin even while that connection
is active. Also, which nodes are twins can change dynamically depending
on current circumstances, and twins may or may not be under the same
administrative authority.

\item \textbf{Exit policies:} Tor provides a consistent mechanism for
each node to specify and advertise its own exit policy. Exit policies
are critical in a volunteer-based distributed infrastructure, because
each operator is comfortable with allowing different types of traffic
to exit the Tor network from his node.

\item \textbf{Rendezvous points and location-protected servers:} Tor
provides an integrated mechanism for responder-anonymity
location-protected servers

\end{tightlist}

We review previous work in Section \ref{sec:background}, describe
our goals and assumptions in Section \ref{sec:assumptions},
and then address the above list of improvements in Sections
\ref{sec:design}-\ref{sec:maintaining-anonymity}. We then summarize
how our design stands up to known attacks, and conclude with a list of
open problems.

%%%%%%%%%%%%%%%%%%%%%%%%%%%%%%%%%%%%%%%%%%%%%%%%%%%%%%%%%%%%%%%%%%%%%%%

\Section{Background and threat model}
\label{sec:background}

\SubSection{Related work}
\label{sec:related-work}
Modern anonymity designs date to Chaum's Mix-Net\cite{chaum-mix} design of
1981.  Chaum proposed hiding sender-recipient connections by wrapping
messages in several layers of public key cryptography, and relaying them
through a path composed of Mix servers.  Mix servers in turn decrypt, delay,
and re-order messages, before relay them along the path towards their
destinations.

Subsequent relay-based anonymity designs have diverged in two
principal directions.  Some have attempted to maximize anonymity at
the cost of introducing comparatively large and variable latencies,
for example, Babel\cite{babel}, Mixmaster\cite{mixmaster-spec}, and
Mixminion\cite{minion-design}.  Because of this
decision, such \emph{high-latency} networks are well-suited for anonymous
email, but introduce too much lag for interactive tasks such as web browsing,
internet chat, or SSH connections.

Tor belongs to the second category: \emph{low-latency} designs that
attempt to anonymize interactive network traffic.  Because such
traffic tends to involve a relatively large numbers of packets, it is
difficult to prevent an attacker who can eavesdrop entry and exit
points from correlating packets entering the anonymity network with
packets leaving it. Although some work has been done to frustrate
these attacks, most designs protect primarily against traffic analysis
rather than traffic confirmation \cite{or-jsac98}.  One can pad and
limit communication to a constant rate or at least to control the
variation in traffic shape. This can have prohibitive bandwidth costs
and/or performance limitations. One can also use a cascade (fixed
shared route) with a relatively fixed set of users. This assumes a
significant degree of agreement and provides an easier target for an active
attacker since the endpoints are generally known. However, a practical
network with both of these features and thousands of active users has
been run for many years (the Java Anon Proxy, aka Web MIXes,
\cite{web-mix}).

Another low latency design that was proposed independently and at
about the same time as onion routing was PipeNet \cite{pipenet}.
This provided anonymity protections that were stronger than onion routing's,
but at the cost of allowing a single user to shut down the network simply
by not sending. It was also never implemented or formally published.

The simplest low-latency designs are single-hop proxies such as the
Anonymizer \cite{anonymizer}, wherein a single trusted server removes
identifying users' data before relaying it.  These designs are easy to
analyze, but require end-users to trust the anonymizing proxy.

More complex are distributed-trust, channel-based anonymizing systems.  In
these designs, a user establishes one or more medium-term bidirectional
end-to-end tunnels to exit servers, and uses those tunnels to deliver a
number of low-latency packets to and from one or more destinations per
tunnel.  Establishing tunnels is comparatively expensive and typically
requires public-key cryptography, whereas relaying packets along a tunnel is
comparatively inexpensive.  Because a tunnel crosses several servers, no
single server can learn the user's communication partners.

Systems such as earlier versions of Freedom and onion routing
build the anonymous channel all at once (using an onion). Later
designs of Freedom and onion routing as described herein build
the channel in stages as does AnonNet
\cite{anonnet}. Amongst other things, this makes perfect forward
secrecy feasible.

Some systems, such as Crowds \cite{crowds-tissec}, do not rely on the
changing appearance of packets to hide the path; rather they employ
mechanisms so that an intermediary cannot be sure when it is
receiving from/sending to the ultimate initiator. There is no public-key
encryption needed for Crowds, but the responder and all data are
visible to all nodes on the path so that anonymity of connection
initiator depends on filtering all identifying information from the
data stream. Crowds is also designed only for HTTP traffic.

Hordes \cite{hordes-jcs} is based on Crowds but also uses multicast
responses to hide the initiator. Herbivore \cite{herbivore} and
P5 \cite{p5} go even further requiring broadcast.
They each use broadcast in very different ways, and tradeoffs are made to
make broadcast more practical. Both Herbivore and P5 are designed primarily
for communication between communicating peers, although Herbivore
permits external connections by requesting a peer to serve as a proxy.
Allowing easy connections to nonparticipating responders or recipients
is a practical requirement for many users, e.g., to visit
nonparticipating Web sites or to exchange mail with nonparticipating
recipients.

Distributed-trust anonymizing systems differ in how they prevent attackers
from controlling too many servers and thus compromising too many user paths.
Some protocols rely on a centrally maintained set of well-known anonymizing
servers.  Current Tor design falls into this category.
Others (such as Tarzan and MorphMix) allow unknown users to run
servers, while using a limited resource (DHT space for Tarzan; IP space for
MorphMix) to prevent an attacker from owning too much of the network.
Crowds uses a centralized ``blender'' to enforce Crowd membership
policy. For small crowds it is suggested that familiarity with all
members is adequate. For large diverse crowds, limiting accounts in
control of any one party is more difficult: 
``(e.g., the blender administrator sets up an account for a user only
after receiving a written, notarized request from that user) and each
account to one jondo, and by monitoring and limiting the number of
jondos on any one net- work (using IP address), the attacker would be
forced to launch jondos using many different identities and on many
different networks to succeed'' \cite{crowds-tissec}.


Many systems have been designed for censorship resistant publishing.
The first of these was the Eternity Service \cite{eternity}. Since
then, there have been many alternatives and refinements, of which we note
but a few
\cite{eternity,gap-pets03,freenet-pets00,freehaven-berk,publius,tangler,taz}.
From the beginning, traffic analysis resistant communication has been
recognized as an important element of censorship resistance because of
the relation between the ability to censor material and the ability to
find its distribution source.

Tor is not primarily for censorship resistance but for anonymous
communication. However, Tor's rendezvous points, which enable
connections between mutually anonymous entities, also facilitate
connections to hidden servers.  These building blocks to censorship
resistance and other capabilities are described in
Section~\ref{sec:rendezvous}.


[XXX I'm considering the subsection as ended here for now. I'm leaving the
following notes in case we want to revisit any of them. -PS]


Channel-based anonymizing systems also differ in their use of dummy traffic.
[XXX]

Finally, several systems provide low-latency anonymity without channel-based
communication.  Crowds and [XXX] provide anonymity for HTTP requests; [...]

[XXX Mention error recovery?]



anonymizer\\
pipenet\\
freedom v1\\
freedom v2\\
onion routing v1\\
isdn-mixes\\
crowds\\
real-time mixes, web mixes\\
anonnet (marc rennhard's stuff)\\
morphmix\\
P5\\
gnunet\\
rewebbers\\
tarzan\\
herbivore\\
hordes\\
cebolla (?)\\

[XXX Close by mentioning where Tor fits.]

\Section{Design goals and assumptions}
\label{sec:assumptions}


\subsection{Goals}
% Are these really our goals? ;) -NM
Like other low-latency anonymity designs, Tor seeks to frustrate
attackers from linking communication partners, or from linking
multiple communications to or from a single point.  Within this
main goal, however, several design considerations have directed
Tor's evolution.

First, we have tried to build a {\bf deployable} system.  [XXX why?]
This requirement precludes designs that are expensive to run (for
example, by requiring more bandwidth than volunteers will easily
provide); designs that place a heavy liability burden on operators
(for example, by allowing attackers to implicate operators in illegal
activities); and designs that are difficult or expensive to implement
(for example, by requiring kernel patches to many operating systems,
or ).

Second, the system must be {\bf usable}.  A hard-to-use system has
fewer users --- and because anonymity systems hide users among users, a
system with fewer users provides less anonymity.  Thus, usability is
not only a convenience, but is a security requirement for anonymity
systems.

Third, the protocol must be {\bf extensible}, so that it can serve as
a test-bed for future research in low-latency anonymity systems.
(Note that while an extensible protocol benefits researchers, there is
a danger that differing choices of extensions will render users
distinguishable.  Thus, implementations should not permit different
protocol extensions to coexist in a single deployed network.)

The protocol's design and security parameters must be {\bf
conservative}.  Additional features impose implementation and
complexity costs. [XXX Say that we don't want to try to come up with
speculative solutions to problems we don't KNOW how to solve? -NM]

[XXX mention something about robustness?  But we really aren't that
  robust.  We just assume that tunneled protocols tolerate connection
  loss. -NM]

\subsection{Non-goals}
In favoring conservative, deployable designs, we have explicitly
deferred a number of goals --- not because they are not desirable in
anonymity systems --- but because solving them is either solved
elsewhere, or an area of active research without a generally accepted
solution.

Unlike Tarzan or Morphmix, Tor does not attempt to scale to completely
decentralized peer-to-peer environments with thousands of short-lived
servers. 

Tor does not claim to provide a definitive solution to end-to-end
timing or intersection attacks for users who do not run their own
Onion Routers.
% Does that mean we do claim to solve intersection attack for
% the enclave-firewall model? -RD

Tor does not provide \emph{protocol normalization} like the Anonymizer or
Privoxy.  In order to provide client indistinguishibility for
complex and variable protocols such as HTTP, Tor must be layered with
a filtering proxy such as Privoxy.  Similarly, Tor does not currently
integrate tunneling for non-stream-based protocols; this too must be
provided by an external service.

Tor is not steganographic: it doesn't try to conceal which users are
sending or receiving communications.


\SubSection{Adversary Model}
\label{subsec:adversary-model}

Like all practical low-latency systems, Tor is not secure against a
global passive adversary, which is the most commonly assumed adversary
for analysis of theoretical anonymous communication designs. The adversary
we assume
is weaker than global with respect to distribution, but it is not
merely passive.
We assume a threat model that expands on that from \cite{or-pet00}.


The basic adversary components we consider are:
\begin{description}
\item[Observer:] can observe a connection (e.g., a sniffer on an
  Internet router), but cannot initiate connections. Observations may
  include timing and/or volume of packets as well as appearance of
  individual packets (including headers and content).
\item[Disrupter:] can delay (indefinitely) or corrupt traffic on a
  link. Can change all those things that an observer can observe up to
  the limits of computational ability (e.g., cannot forge signatures
  unless a key is compromised).
\item[Hostile initiator:] can initiate (or destroy) connections with
  specific routes as well as vary the timing and content of traffic
  on the connections it creates. A special case of the disrupter with
  additional abilities appropriate to its role in forming connections.
\item[Hostile responder:] can vary the traffic on the connections made
  to it including refusing them entirely, intentionally modifying what
  it sends and at what rate, and selectively closing them. Also a
  special case of the disrupter.
\item[Key breaker:] can break the key used to encrypt connection
  initiation requests sent to a Tor-node.
% Er, there are no long-term private decryption keys. They have
% long-term private signing keys, and medium-term onion (decryption)
% keys. Plus short-term link keys. Should we lump them together or
% separate them out? -RD
%
%  Hmmm, I was talking about the keys used to encrypt the onion skin
%  that contains the public DH key from the initiator. Is that what you
%  mean by medium-term onion key? (``Onion key'' used to mean the
%  session keys distributed in the onion, back when there were onions.)
%  Also, why are link keys short-term? By link keys I assume you mean
%  keys that neighbor nodes use to superencrypt all the stuff they send
%  to each other on a link.  Did you mean the session keys? I had been
%  calling session keys short-term and everything else long-term. I
%  know I was being sloppy. (I _have_ written papers formalizing
%  concepts of relative freshness.) But, there's some questions lurking
%  here. First up, I don't see why the onion-skin encryption key should
%  be any shorter term than the signature key in terms of threat
%  resistance. I understand that how we update onion-skin encryption
%  keys makes them depend on the signature keys. But, this is not the
%  basis on which we should be deciding about key rotation. Another
%  question is whether we want to bother with someone who breaks a
%  signature key as a particular adversary. He should be able to do
%  nearly the same as a compromised tor-node, although they're not the
%  same. I reworded above, I'm thinking we should leave other concerns
%  for later. -PS

  
\item[Compromised Tor-node:] can arbitrarily manipulate the
  connections under its control, as well as creating new connections
  (that pass through itself).
\end{description}


All feasible adversaries can be composed out of these basic
adversaries. This includes combinations such as one or more
compromised Tor-nodes cooperating with disrupters of links on which
those nodes are not adjacent, or such as combinations of hostile
outsiders and link observers (who watch links between adjacent
Tor-nodes).  Note that one type of observer might be a Tor-node. This
is sometimes called an honest-but-curious adversary. While an observer
Tor-node will perform only correct protocol interactions, it might
share information about connections and cannot be assumed to destroy
session keys at end of a session.  Note that a compromised Tor-node is
stronger than any other adversary component in the sense that
replacing a component of any adversary with a compromised Tor-node
results in a stronger overall adversary (assuming that the compromised
Tor-node retains the same signature keys and other private
state-information as the component it replaces).


In general we are more focused on traffic analysis attacks than
traffic confirmation attacks. A user who runs a Tor proxy on his own
machine, connects to some remote Tor-node and makes a connection to an
open Internet site, such as a public web server, is vulnerable to
traffic confirmation. That is, an active attacker who suspects that
the particular client is communicating with the particular server will
be able to confirm this if she can attack and observe both the
connection between the Tor network and the client and that between the
Tor network and the server. Even a purely passive attacker will be
able to confirm if the timing and volume properties of the traffic on
the connnection are unique enough.  This is not to say that Tor offers
no resistance to traffic confirmation; it does.  We defer discussion
of this point and of particular attacks until Section~\ref{sec:attacks},
after we have described Tor in more detail. However, we note here some
basic assumptions that affect the threat model.

[XXX I think this next subsection should be cut, leaving its points
for the attacks section. But I'm leaving it here for now. The above
line refers to the immediately following SubSection.-PS]


\SubSection{Known attacks against low-latency anonymity systems}
\label{subsec:known-attacks}

We discuss each of these attacks in more detail below, along with the
aspects of the Tor design that provide defense. We provide a summary
of the attacks and our defenses against them in Section~\ref{sec:attacks}.

Passive attacks:
simple observation,
timing correlation,
size correlation,
option distinguishability,

Active attacks:
key compromise,
iterated subpoena,
run recipient,
run a hostile node,
compromise entire path,
selectively DOS servers,
introduce timing into messages,
directory attacks,
tagging attacks


\SubSection{Assumptions}

For purposes of this paper, we assume all directory servers are honest
% No longer true, see subsec:dirservers below -RD
and trusted. Perhaps more accurately, we assume that all users and
nodes can perform their own periodic checks on information they have
from directory servers and that all will always have access to at
least one directory server that they trust and from which they obtain
all directory information. Future work may include robustness
techniques to cope with a minority dishonest servers.

Somewhere between ten percent and twenty percent of nodes are assumed
to be compromised. In some circumstances, e.g., if the Tor network is
running on a hardened network where all operators have had
background checks, the percent of compromised nodes might be much
lower. It may be worthwhile to consider cases where many of the `bad'
nodes are not fully compromised but simply (passive) observing
adversaries or that some nodes have only had compromise of the keys
that decrypt connection initiation requests. But, we assume for
simplicity that `bad' nodes are compromised in the sense spelled out
above. We assume that all adversary components, regardless of their
capabilities are collaborating and are connected in an offline clique.

We do not assume any hostile users, except in the context of
% This sounds horrible. What do you mean we don't assume any hostile
% users? Surely we can tolerate some? -RD
rendezvous points. Nonetheless, we assume that users vary widely in
both the duration and number of times they are connected to the Tor
network. They can also be assumed to vary widely in the volume and
shape of the traffic they send and receive.


[XXX what else?]


%%%%%%%%%%%%%%%%%%%%%%%%%%%%%%%%%%%%%%%%%%%%%%%%%%%%%%%%%%%%%%%%%%%%%%%

\Section{The Tor Design}
\label{sec:design}


\Section{Other design decisions}

\SubSection{Exit policies and abuse}
\label{subsec:exitpolicies}

Exit abuse is a serious barrier to wide-scale Tor deployment --- we
must block or limit attacks and other abuse that users can do through
the Tor network.

Each onion router's \emph{exit policy} describes to which external
addresses and ports the router will permit stream connections. On one end
of the spectrum are \emph{open exit} nodes that will connect anywhere;
on the other end are \emph{middleman} nodes that only relay traffic to
other Tor nodes, and \emph{private exit} nodes that only connect locally
or to addresses internal to that node's organization. This private exit
node configuration is more secure for clients --- the adversary cannot
see plaintext traffic leaving the network (e.g. to a webserver), so he
is less sure of Alice's destination. More generally, nodes can require
a variety of forms of traffic authentication \cite{onion-discex00}.

Tor offers more reliability than the high-latency fire-and-forget
anonymous email networks, because the sender opens a TCP stream
with the remote mail server and receives an explicit confirmation of
acceptance. But ironically, the private exit node model works poorly for
email, when Tor nodes are run on volunteer machines that also do other
things, because it's quite hard to configure mail transport agents so
normal users can send mail normally, but the Tor process can only deliver
mail locally. Further, most organizations have specific hosts that will
deliver mail on behalf of certain IP ranges; Tor operators must be aware
of these hosts and consider putting them in the Tor exit policy.

The abuse issues on closed (e.g. military) networks are very different
from the abuse on open networks like the Internet. While these IP-based
access controls are still commonplace on the Internet, on closed networks,
nearly all participants will be honest, and end-to-end authentication
can be assumed for anything important.

Tor is harder than minion because tcp doesn't include an abuse
address. you could reach inside the http stream and change the agent
or something, but that's a very specific case and probably won't help
much anyway.
And volunteer nodes don't resolve to anonymizer.mit.edu so it never
even occurs to people that it wasn't you.

Preventing abuse of open exit nodes is an unsolved problem. Princeton's
CoDeeN project \cite{darkside} gives us a glimpse of what we're in for.

but their solutions, which mainly involve rate limiting and blacklisting
nodes which do bad things, don't translate directly to Tor. Rate limiting
still works great, but Tor intentionally separates sender from recipient,
so it's hard to know which sender was the one who did the bad thing,
without just making the whole network wide open.

even limiting most nodes to allow http, ssh, and aim to exit and reject
all other stuff is sketchy, because plenty of abuse can happen over
port 80. but it's a very good start, because it blocks most things,
and because people are more used to the concept of port 80 abuse not
coming from the machine's owner.

we could also run intrusion detection system (IDS) modules at each tor
node, to dynamically monitor traffic streams for attack signatures. it
can even react when it sees a signature by closing the stream. but IDS's
don't actually work most of the time, and besides, how do you write a
signature for "is sending a mean mail"?

we should run a squid at each exit node, to provide comparable anonymity
to private exit nodes for cache hits, to speed everything up, and to
have a buffer for funny stuff coming out of port 80. we could similarly
have other exit proxies for other protocols, like mail, to check
delivered mail for being spam.

A mixture of open and restricted exit nodes will allow the most
flexibility for volunteers running servers. But while a large number
of middleman nodes is useful to provide a large and robust network,
a small number of exit nodes still simplifies traffic analysis because
there are fewer nodes the adversary needs to monitor, and also puts a
greater burden on the exit nodes.
The JAP cascade model is really nice because they only need one node to
take the heat per cascade. On the other hand, a hydra scheme could work
better (it's still hard to watch all the clients).

\SubSection{Directory Servers}
\label{subsec:dirservers}

First-generation Onion Routing designs \cite{or-jsac98,freedom2-arch} did
% is or-jsac98 the right cite here? what's our stock OR cite? -RD
in-band network status updates: each router flooded a signed statement
to its neighbors, which propagated it onward. But anonymizing networks
have different security goals than typical link-state routing protocols.
For example, we worry more about delays (accidental or intentional)
which cause different parts of the network to have different pictures
of link-state and topology. We also worry about attacks to deceive a
client about the router membership list, topology, or current network
state. Such \emph{partitioning attacks} on client knowledge help an
adversary with limited resources to efficiently deploy those resources
when attacking a target.

Instead, Tor uses a small group of redundant directory servers to
track network topology and node state such as current keys and exit
policies. The directory servers are normal onion routers, but there are
only a few of them and they are more trusted. They listen on a separate
port as an HTTP server, both so participants can fetch current network
state and router lists (a \emph{directory}), and so other onion routers
can upload a signed summary of their keys, address, bandwidth, exit
policy, etc (\emph{server descriptors}.

Of course, a variety of attacks remain. An adversary who controls a
directory server can track certain clients by providing different
information --- perhaps by listing only nodes under its control
as working, or by informing only certain clients about a given
node. Moreover, an adversary without control of a directory server can
still exploit differences among client knowledge. If Eve knows that
node $M$ is listed on server $D_1$ but not on $D_2$, she can use this
knowledge to link traffic through $M$ to clients who have queried $D_1$.

Thus these directory servers must be synchronized and redundant. The
software is distributed with the signature public key of each directory
server, and directories must be signed by a threshold of these keys.

The directory servers in Tor are modeled after those in Mixminion
\cite{minion-design}, but our situation is easier. Firstly, we make the
simplifying assumption that all participants agree on who the directory
servers are. Secondly, Mixminion needs to predict node behavior ---
that is, build a reputation system for guessing future performance of
nodes based on past performance, and then figure out a way to build
a threshold consensus of these predictions. Tor just needs to get a
threshold consensus of the current state of the network.

The threshold consensus can be reached with standard Byzantine agreement
techniques \cite{castro-liskov}.
% Should I just stop the section here? Is the rest crap? -RD
But this library, while more efficient than previous Byzantine agreement
systems, is still complex and heavyweight for our purposes: we only need
to compute a single algorithm, and we do not require strict in-order
computation steps. The Tor directory servers build a consensus directory
through a simple four-round broadcast protocol. First, each server signs
and broadcasts its current opinion to the other directory servers; each
server then rebroadcasts all the signed opinions it has received. At this
point all directory servers check to see if anybody's cheating. If so,
directory service stops, the humans are notified, and that directory
server is permanently removed from the network. Assuming no cheating,
each directory server then computes a local algorithm on the set of
opinions, resulting in a uniform shared directory. Then the servers sign
this directory and broadcast it; and finally all servers rebroadcast
the directory and all the signatures.

The rebroadcast steps ensure that a directory server is heard by either
all of the other servers or none of them (some of the links between
directory servers may be down). Broadcasts are feasible because there
are so few directory servers (currently 3, but we expect to use as many
as 9 as the network scales). The actual local algorithm for computing
the shared directory is straightforward, and is described in the Tor
specification \cite{tor-spec}.
% we should, uh, add this to the spec. oh, and write it. -RD

Using directory servers rather than flooding approaches provides
simplicity and flexibility. For example, they don't complicate
the analysis when we start experimenting with non-clique network
topologies. And because the directories are signed, they can be cached at
all the other onion routers (or even elsewhere). Thus directory servers
are not a performance bottleneck when we have many users, and also they
won't aid traffic analysis by forcing clients to periodically announce
their existence to any central point.

\Section{Rendezvous points: location privacy}
\label{sec:rendezvous}

Rendezvous points are a building block for \emph{location-hidden services}
(aka responder anonymity) in the Tor network. Location-hidden
services means Bob can offer a tcp service, such as a webserver,
without revealing the IP of that service.

We provide this censorship resistance for Bob by allowing him to
advertise several onion routers (his \emph{Introduction Points}) as his
public location. Alice, the client, chooses a node for her \emph{Meeting
Point}. She connects to one of Bob's introduction points, informs him
about her meeting point, and then waits for him to connect to the meeting
point. This extra level of indirection means Bob's introduction points
don't open themselves up to abuse by serving files directly, eg if Bob
chooses a node in France to serve material distateful to the French. The
extra level of indirection also allows Bob to respond to some requests
and ignore others.

We provide the necessary glue so that Alice can view webpages from Bob's
location-hidden webserver with minimal invasive changes. Both Alice and
Bob must run local onion proxies.

The steps of a rendezvous:
\begin{tightlist}
\item Bob chooses some Introduction Points, and advertises them on a
      Distributed Hash Table (DHT).
\item Bob establishes onion routing connections to each of his
      Introduction Points, and waits.
\item Alice learns about Bob's service out of band (perhaps Bob told her,
      or she found it on a website). She looks up the details of Bob's
      service from the DHT.
\item Alice chooses and establishes a Meeting Point (MP) for this
      transaction.
\item Alice goes to one of Bob's Introduction Points, and gives it a blob
      (encrypted for Bob) which tells him about herself, the Meeting Point
      she chose, and the first half of an ephemeral key handshake. The
      Introduction Point sends the blob to Bob.
\item Bob chooses whether to ignore the blob, or to onion route to MP.
      Let's assume the latter.
\item MP plugs together Alice and Bob. Note that MP can't recognize Alice,
      Bob, or the data they transmit (they share a session key).
\item Alice sends a Begin cell along the circuit. It arrives at Bob's
      onion proxy. Bob's onion proxy connects to Bob's webserver.
\item Data goes back and forth as usual.
\end{tightlist}

When establishing an introduction point, Bob provides the onion router
with a public ``introduction'' key.  The hash of this public key
identifies a unique service, and (since Bob is required to sign his
messages) prevents anybody else from usurping Bob's introduction point
in the future. Bob uses the same public key when establish the other
introduction points for that service.

The blob that Alice gives the introduction point includes a hash of Bob's
public key to identify the service, an optional initial authentication
token (the introduction point can do prescreening, eg to block replays),
and (encrypted to Bob's public key) the location of the meeting point,
a meeting cookie Bob should tell the meeting point so he gets connected to
Alice, an optional authentication token so Bob can choose whether to respond,
and the first half of a DH key exchange. When Bob connects to the meeting
place and gets connected to Alice's pipe, his first cell contains the
other half of the DH key exchange.

% briefly talk about our notion of giving cookies to people proportional
% to how important they are, for location-protected servers hardened
% against DDoS threat? -RD

\subsection{Integration with user applications}

For each service Bob offers, he configures his local onion proxy to know
the local IP and port of the server, a strategy for authorizating Alices,
and a public key. We assume the existence of a robust decentralized
efficient lookup system which allows authenticated updates, eg
\cite{cfs:sosp01}. (Each onion router could run a node in this lookup
system; also note that as a stopgap measure, we can just run a simple
lookup system on the directory servers.)  Bob publishes into the DHT
(indexed by the hash of the public key) the public key, an expiration
time (``not valid after''), and the current introduction points for that
service. Note that Bob's webserver is completely oblivious to the fact
that it's hidden behind the Tor network.

As far as Alice's experience goes, we require that her client interface
remain a SOCKS proxy, and we require that she shouldn't have to modify
her applications. Thus we encode all of the necessary information into
the hostname (more correctly, fully qualified domain name) that Alice
uses, eg when clicking on a url in her browser. Location-hidden services
use the special top level domain called `.onion': thus hostnames take the
form x.y.onion where x encodes the hash of PK, and y is the authentication
cookie. Alice's onion proxy examines hostnames and recognizes when they're
destined for a hidden server. If so, it decodes the PK and starts the
rendezvous as described in the table above.

\subsection{Previous rendezvous work}

Ian Goldberg developed a similar notion of rendezvous points for
low-latency anonymity systems \cite{ian-thesis}. His ``service tag''
is the same concept as our ``hash of service's public key''. We make it
a hash of the public key so it can be self-authenticating, and so the
client can recognize the same service with confidence later on. His
design differs from ours in the following ways though. Firstly, Ian
suggests that the client should manually hunt down a current location of
the service via Gnutella; whereas our use of the DHT makes lookup faster,
more robust, and transparent to the user. Secondly, the client and server
can share ephemeral DH keys, so at no point in the path is the plaintext
exposed. Thirdly, our design is much more practical for deployment in a
volunteer network, in terms of getting volunteers to offer introduction
and meeting point services. The introduction points do not output any
bytes to the clients. And the meeting points don't know the client,
the server, or the stuff being transmitted. The indirection scheme
is also designed with authentication/authorization in mind -- if the
client doesn't include the right cookie with its request for service,
the server doesn't even acknowledge its existence.

\Section{Maintaining anonymity sets}
\label{sec:maintaining-anonymity}

packet counting attacks work great against initiators. need to do some
level of obfuscation for that. standard link padding for passive link
observers. long-range padding for people who own the first hop. are
we just screwed against people who insert timing signatures into your
traffic?

Even regardless of link padding from Alice to the cloud, there will be
times when Alice is simply not online. Link padding, at the edges or
inside the cloud, does not help for this.

how often should we pull down directories? how often send updated
server descs?

when we start up the client, should we build a circuit immediately,
or should the default be to build a circuit only on demand? should we
fetch a directory immediately?

would we benefit from greater synchronization, to blend with the other
users? would the reduced speed hurt us more?

does the "you can't see when i'm starting or ending a stream because
you can't tell what sort of relay cell it is" idea work, or is just
a distraction?

does running a server actually get you better protection, because traffic
coming from your node could plausibly have come from elsewhere? how
much mixing do you need before this is actually plausible, or is it
immediately beneficial because many adversary can't see your node?

do different exit policies at different exit nodes trash anonymity sets,
or not mess with them much?

do we get better protection against a realistic adversary by having as
many nodes as possible, so he probably can't see the whole network,
or by having a small number of nodes that mix traffic well? is a
cascade topology a more realistic way to get defenses against traffic
confirmation? does the hydra (many inputs, few outputs) topology work
better? are we going to get a hydra anyway because most nodes will be
middleman nodes?

using a circuit many times is good because it's less cpu work
  good because of predecessor attacks with path rebuilding
  bad because predecessor attacks can be more likely to link you with a
    previous circuit since you're so verbose
  bad because each thing you do on that circuit is linked to the other
    things you do on that circuit

Because Tor runs over TCP, when one of the servers goes down it seems
that all the circuits (and thus streams) going over that server must
break. This reduces anonymity because everybody needs to reconnect
right then (does it? how much?) and because exit connections all break
at the same time, and it also reduces usability. It seems the problem
is even worse in a p2p environment, because so far such systems don't
really provide an incentive for nodes to stay connected when they're
done browsing, so we would expect a much higher churn rate than for
onion routing. Are there ways of allowing streams to survive the loss
of a node in the path?


%%%%%%%%%%%%%%%%%%%%%%%%%%%%%%%%%%%%%%%%%%%%%%%%%%%%%%%%%%%%%%%%%%%%%%%

\Section{Attacks and Defenses}
\label{sec:attacks}

Below we summarize a variety of attacks and how well our design withstands
them.

\begin{enumerate}
\item \textbf{Passive attacks}
\begin{itemize}
\item \emph{Simple observation.}
\item \emph{Timing correlation.}
\item \emph{Size correlation.}
\item \emph{Option distinguishability.}
\end{itemize}

\item \textbf{Active attacks}
\begin{itemize}
\item \emph{Key compromise.}
\item \emph{Iterated subpoena.}
\item \emph{Run recipient.}
\item \emph{Run a hostile node.}
\item \emph{Compromise entire path.}
\item \emph{Selectively DoS servers.}
\item \emph{Introduce timing into messages.}
\item \emph{Tagging attacks.}
the exit node can change the content you're getting to try to
trick you. similarly, when it rejects you due to exit policy,
it could give you a bad IP that sends you somewhere else.
\end{itemize}

\item \textbf{Directory attacks}
\begin{itemize}
\item foo
\end{itemize}

\end{enumerate}

%%%%%%%%%%%%%%%%%%%%%%%%%%%%%%%%%%%%%%%%%%%%%%%%%%%%%%%%%%%%%%%%%%%%%%%

\Section{Future Directions and Open Problems}
\label{sec:conclusion}

Tor brings together many innovations into
a unified deployable system. But there are still several attacks that
work quite well, as well as a number of sustainability and run-time
issues remaining to be ironed out. In particular:

\begin{itemize}
\item \emph{Scalability:} Since Tor's emphasis currently is on simplicity
of design and deployment, the current design won't easily handle more
than a few hundred servers, because of its clique topology. Restricted
route topologies \cite{danezis-pets03} promise comparable anonymity
with much better scaling properties, but we must solve problems like
how to randomly form the network without introducing net attacks.
% [cascades are a restricted route topology too. we must mention
% earlier why we're not satisfied with the cascade approach.]-RD
% [We do. At least 
\item \emph{Cover traffic:} Currently we avoid cover traffic because
it introduces clear performance and bandwidth costs, but and its
security properties are not well understood. With more research
\cite{SS03,defensive-dropping}, the price/value ratio may change, both for
link-level cover traffic and also long-range cover traffic. In particular,
we expect restricted route topologies to reduce the cost of cover traffic
because there are fewer links to cover.
\item \emph{Better directory distribution:} Even with the threshold
directory agreement algorithm described in \ref{subsec:dirservers},
the directory servers are still trust bottlenecks. We must find more
decentralized yet practical ways to distribute up-to-date snapshots of
network status without introducing new attacks.
\item \emph{Implementing location-hidden servers:} While Section
\ref{sec:rendezvous} provides a design for rendezvous points and
location-hidden servers, this feature has not yet been implemented.
We will likely encounter additional issues, both in terms of usability
and anonymity, that must be resolved.
\item \emph{Wider-scale deployment:} The original goal of Tor was to
gain experience in deploying an anonymizing overlay network, and learn
from having actual users. We are now at the point where we can start
deploying a wider network. We will see what happens!
% ok, so that's hokey. fix it. -RD
\end{itemize}

%%%%%%%%%%%%%%%%%%%%%%%%%%%%%%%%%%%%%%%%%%%%%%%%%%%%%%%%%%%%%%%%%%%%%%%

%\Section{Acknowledgments}
%% commented out for anonymous submission

%%%%%%%%%%%%%%%%%%%%%%%%%%%%%%%%%%%%%%%%%%%%%%%%%%%%%%%%%%%%%%%%%%%%%%%

\bibliographystyle{latex8}
\bibliography{tor-design}

\end{document}

% Style guide:
%     U.S. spelling
%     avoid contractions (it's, can't, etc.)
%     'mix', 'mixes' (as noun)
%     'mix-net'
%     'mix', 'mixing' (as verb)
%     'Mixminion Project'
%     'Mixminion' (meaning the protocol suite or the network)
%     'Mixmaster' (meaning the protocol suite or the network)
%     'middleman'  [Not with a hyphen; the hyphen has been optional
%         since Middle English.]
%     'nymserver'
%     'Cypherpunk', 'Cypherpunks', 'Cypherpunk remailer'
%
%     'Whenever you are tempted to write 'Very', write 'Damn' instead, so
%     your editor will take it out for you.'  -- Misquoted from Mark Twain

